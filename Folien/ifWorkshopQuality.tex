\documentclass[10pt,slidestop,compress,mathserif, aspectratio = 169]{beamer}
\usepackage{etex}
%\usepackage[bars]{beamerthemetree} % Beamer theme v 2.2
%\usetheme{Singapore} % Beamer theme v 3.0
\usecolortheme{seagull} % Beamer color theme
%\usepackage{etex}
%\usepackage[round]{natbib}
%\useoutertheme[subsection = false]{smoothbars}
\useinnertheme{rectangles}
%\usepackage{bibentry}
\usepackage{graphicx}
\usepackage{xparse}
\usepackage{epigraph}
\usepackage{makeidx}
\usepackage[]{amsmath}
\usepackage[]{amssymb}
\usepackage{color}
\usepackage{pict2e}
\usepackage{algorithm2e}
\usepackage[ngerman]{babel}
\usepackage{tikz}
\usetikzlibrary{mindmap}
\usepackage{multirow}
\usepackage[utf8]{inputenc}
\usepackage{multimedia}
\usepackage{xcolor}
\usepackage{keyval}
\usepgfmodule{shapes}
\usepackage{pgfplots}
\usepackage{sansmath}
\usepackage{pgfgantt}
\usetikzlibrary{arrows}
\usetikzlibrary{graphs, automata, spy, positioning}
\usetikzlibrary{shapes.gates.logic.US,trees}
\usetikzlibrary{shadows}
\usepackage{colortbl}
\usepackage{array}
\usepackage{listings}
\usepackage{pdfpages}
\pgfplotsset{width=6cm}

%Bunt und in Farbe!
\definecolor{RYB1}{RGB}{240,249,232}
\definecolor{RYB2}{RGB}{186,228,188}
\definecolor{RYB3}{RGB}{123,204,196}
\definecolor{RYB4}{RGB}{67,162,202}
\definecolor{RYB5}{RGB}{18,104,172}

\definecolor{incolor}{rgb}{0.0, 0.0, 0.5}
\definecolor{outcolor}{rgb}{0.545, 0.0, 0.0}

\definecolor{blue}{rgb}{0,0,1}
\definecolor{mint}{cmyk}{75,0,40,0}
\definecolor{mint}{rgb}{32,178,170}
\definecolor{myyellow}{RGB}{242,226,149}

\makeatletter
\pgfdeclareshape{document}{
\inheritsavedanchors[from=rectangle] % this is nearly a rectangle
\inheritanchorborder[from=rectangle]
\inheritanchor[from=rectangle]{center}
\inheritanchor[from=rectangle]{north}
\inheritanchor[from=rectangle]{south}
\inheritanchor[from=rectangle]{west}
\inheritanchor[from=rectangle]{east}
% ... and possibly more
\backgroundpath{% this is new
% store lower right in xa/ya and upper right in xb/yb
\southwest \pgf@xa=\pgf@x \pgf@ya=\pgf@y
\northeast \pgf@xb=\pgf@x \pgf@yb=\pgf@y
% compute corner of ‘‘flipped page’’
\pgf@xc=\pgf@xb \advance\pgf@xc by-10pt % this should be a parameter
\pgf@yc=\pgf@yb \advance\pgf@yc by-10pt
% construct main path
\pgfpathmoveto{\pgfpoint{\pgf@xa}{\pgf@ya}}
\pgfpathlineto{\pgfpoint{\pgf@xa}{\pgf@yb}}
\pgfpathlineto{\pgfpoint{\pgf@xc}{\pgf@yb}}
\pgfpathlineto{\pgfpoint{\pgf@xb}{\pgf@yc}}
\pgfpathlineto{\pgfpoint{\pgf@xb}{\pgf@ya}}
\pgfpathclose
% add little corner
\pgfpathmoveto{\pgfpoint{\pgf@xc}{\pgf@yb}}
\pgfpathlineto{\pgfpoint{\pgf@xc}{\pgf@yc}}
\pgfpathlineto{\pgfpoint{\pgf@xb}{\pgf@yc}}
\pgfpathlineto{\pgfpoint{\pgf@xc}{\pgf@yc}}
}
}
\makeatother

\NewDocumentCommand\StickyNote{O{4.5cm}mO{4.5cm}}{%
\begin{tikzpicture}
\node[
document,
draw,
drop shadow={
  shadow xshift=2pt,
  shadow yshift=-4pt
},
inner xsep=7pt,
fill=myyellow,
xslant=-0.1,
yslant=0.1,
inner ysep=10pt
] {\parbox[t][#1][t]{#3}{#2}};
\end{tikzpicture}%
}

\NewDocumentCommand\StickyNotePi{O{4.5cm}mO{4.5cm}}{%
\begin{tikzpicture}
\node[
document,
draw,
fill=myyellow,
inner xsep=10pt,
xslant=-0.1,
yslant=0.1,
inner ysep=0pt,
text depth=\the\dimexpr#1+2.5ex\relax
] {\parbox[t][#1][c]{#3}{#2}};
\end{tikzpicture}%
}


\graphicspath{{../../../figures/}}

\mode<handout>{
\usepackage{pgfpages}
\pgfpagesuselayout{2 on 1}[a4paper,border shrink=5mm]
}

\AtBeginSection{\frame{\sectionpage}}
\AtBeginSubsection{\frame{\subsectionpage}}
%\newtranslation[to=ngerman]{Section}{Teil}
%\newtranslation[to=ngerman]{Subsection}{Abschnitt}

\setbeamersize{text margin left=.8cm, text margin right=.8cm}

%\setbeamertemplate{frametitle}{%
%    \nointerlineskip%
%    \begin{beamercolorbox}[wd=\paperwidth,ht=2.0ex,dp=0.6ex]{frametitle}
%        \hspace*{1ex}\insertframetitle%
%    \end{beamercolorbox}%
%}

\newcommand\MyBox[2]{
  \fbox{\lower0.75cm
    \vbox to 1.7cm{\vfil
      \hbox to 1.7cm{\hfil\parbox{1.4cm}{#1\\#2}\hfil}
      \vfil}%
  }%
}

\pgfmathdeclarefunction{norm}{3}{%
                      \pgfmathparse{sqrt(0.5*#3/pi)*exp(-0.5*#3*(#1-#2)^2)}%
                    }
\begin{document}

\setbeamertemplate{bibliography item}{\insertbiblabel}
\tikzstyle{element} = [draw,rectangle, align = center, text width = 2.5 cm, node distance = 1.25 cm]

% Formalia
\newcommand{\Var}{\operatorname{Var}}
\newcommand{\mum}{\operatorname{\mu m}}
\newcommand{\E}{\operatorname{E}}
\newcommand{\offslide}[2]{\frame{\frametitle{ \includegraphics[scale=0.01] {Off}\hspace{1.5mm} #1}
\framesubtitle{#2}
}}
\newcommand{\source}[1]{\rotatebox{90}{\tiny \color{gray} #1}}

\title{Rail-Data}
%\subtitle{im Produktlebenszyklus mechatronischer Produkte}
\author[R. Pfaff]{Prof. Dr. Raphael Pfaff}
\institute[\jobname]{Fachhochschule Aachen}
\logo{\put(-18, 170){\includegraphics[width=1.4cm]{logoR}}}
\date{\today}
\begin{frame} % Cover slide
\titlepage
\end{frame}

%Pr\"aliminarien
%% !TEX root = ../17-MdQM-Vorlesung.tex
%\section*{Einführung}
\label{Sec:Einfuehrung}

\frame{\frametitle{Prof. Dr. Raphael Pfaff}
\framesubtitle{Lehr- und Forschungsgebiet Schienenfahrzeugtechnik}
\begin{columns}[t] 
     \begin{column}[T]{7cm} 
     	\begin{itemize}
		\item[] \includegraphics[width=0.4cm]{Email} \hspace{.1cm} pfaff@fh-aachen.de
		\item[] \includegraphics[width=0.4cm]{Twitter} \hspace{.1cm} @RailProfAC
		\item[] \includegraphics[width=0.4cm]{Wordpress} \hspace{.1cm} www.raphaelpfaff.net
		\item[] Prezume: \texttt{\url{http://goo.gl/iq6lhh}}
		\vspace{1cm}
		\item Raum 02305
		\item Sprechstunde nach Vereinbarung
     	\end{itemize}
	
     \end{column}
     	\begin{column}[T]{5cm} 
         	\begin{center}
            		\includegraphics[width=0.8\textwidth]{Profilklein}
        		\end{center}
     \end{column}
 \end{columns}
}

\frame{\frametitle{Curriculum Vitae}
\framesubtitle{}
\begin{center}
\includegraphics[width = 12 cm]{CVGraphic}
\end{center}
}

\frame{\frametitle{Selected publications}
\framesubtitle{}
\scriptsize
\begin{itemize}
%\item Enning, M.; Pfaff, R.: Güterwagen 4.0 -- Ausgewählte technologische Ansätze für einen wettbewerbsfähigen Schienengüterverkehr. Mai 2016: Bahntechnik-Symposium 2016 der IFV-Bahntechnik. 
\item Pfaff, R.; Schmidt, B.D.: \textbf{Daten in der Cloud und dann?} - Vorhersage der Subsystemverfügbarkeit aus beobachteten Fehlerraten der Teilfunktionen. Deine Bahn 5/2016, Bahnfachverlag, Berlin.
\item Pfaff, R.: \textbf{Aktive, mitdenkende Güterwagenbremse.} VDI-Expertenforum Automatisierung für Schienenverkehrssysteme – Der Weg zum Güterwagen 4.0. Aachen, 01./02.09.2016
%\item Enning, M.; Pfaff, R.: Titel noch offen. Combinet-Konferenz. Wien, 10.11.2016
%\item Pfaff, R.; Enning, M.: Zeitgemäße Automatisierung am Güterwagen: das Potential des Güterwagen 4.0 in der Logistik der Industrie 4.0. November 2016: Keynote zum Fachsymposium Rail-IT des IFV-Bahntechnik.
\item Enning, M.; Pfaff, R.: \textbf{Digitalisierung bringt mehr Güter auf die Schiene}. ATZ Automobiltechnische Zeitschrift, Dezember 2016, Springer, Wiesbaden.
\item Moshiri, A.; Pfaff, R.; Reich, A.; Gäbel, M.: \textbf{Modellierung der Adhäsionsfläche im Rad-Schiene-Kontakt unter Einsatz von reibwertverbessernden Mitteln.} 15. International Schienenfahrzeugtagung, März 2017, Dresden. 
\item Pfaff, R.; Enning, M.: \textbf{Towards Inclusion of the Freight Rail System in the Industrial Internet of Things - Wagon 4.0.} The Stephenson Conference 2017: Research for Railways, April 2017, London. 
\item Shahidi, P; Pfaff, R.; Enning, M.: \textbf{The connected wagon - a concept for the integration of vehicle side sensors and actors with cyber physical representation for condition based maintenance.} First International Conference on Rail Transportation, July 2017, Chengdu.
\item Pfaff, R.; Moshiri, A.; Reich, A.; Gäbel, M.: \textbf{Modelling of the effect of sanding on the wheel rail contact area.} First International Conference on Rail Transportation, July 2017, Chengdu. 
\item Pfaff, R.: \textbf{Analysis of Big Data Streams to obtain Braking Reliability Information for Train Protection systems.} Asia-Pacific Conference of the Prognostics and Health Management Society, July 2017, Jeju, Korea
\end{itemize}
}

\frame{\frametitle{Vorstellungsrunde und Erwartungen}
\framesubtitle{}
\begin{columns}[t] 
     \begin{column}[T]{6cm} 
     	\begin{itemize}
     		\item Ausbildung?
		\item Berufserfahrung?
		\item Undergraduate?
		\item Warum MPE?
		\item QM-Vorwissen?
		\item Was muss passieren, damit ich MdQM hassen werde?
		\item Was muss passieren, damit ich MdQM lieben werde?
     	\end{itemize}
     \end{column}
     	\begin{column}[T]{6cm} 
         	\begin{center}
            		\includegraphics[width=0.95\textwidth]{HelloSticker}\source{}
        		\end{center}
     \end{column}
 \end{columns}
}

\frame{\frametitle{Anforderungen ``Second Cycle'' - Master}
\framesubtitle{Anforderungen gem\"a{\ss} Dublin Descriptors}
\begin{columns}[t] 
     \begin{column}[T]{8cm} 
     	\begin{itemize}
     		\item Knowledge and understanding founded upon and extends or enhances that typically associated with Bachelor's level:
		\begin{itemize}
		\item Provides basis or opportunity for originality in developing and applying ideas
		\item Often within a research context
		\end{itemize}
		\item Apply their knowledge and understanding and problem solving abilities in new or unfamiliar environments 
		\item Have the ability to integrate knowledge and handle complexity and formulate judgements with incomplete or limited information
		\item Have the learning skills to allow them to continue to study in a manner that may be largely self-directed or autonomous
     	\end{itemize}
     \end{column}
     	\begin{column}[T]{4cm} 
         	\begin{center}
	\vspace{1.5cm}
            		\includegraphics[width=0.8\textwidth]{GraduationHat}
        		\end{center}
     \end{column}
 \end{columns}
}

\frame{\frametitle{Anforderungen ``Niveau 7'' - Master}
\framesubtitle{Anforderungen gem\"a{\ss} Deutschem Qualifizierungsrahmen}
\begin{columns}[t] 
     \begin{column}[T]{7cm} 
     	\begin{itemize}
     		\item Umfassendes, detailliertes und spezialisiertes Wissen
		\begin{itemize}
		\item Auf dem neuesten Erkenntnisstand
		\item Erweitertes Wissen in angrenzenden Bereichen 
		\end{itemize}
		\item Spezialisierte fachliche oder konzeptionelle Fertigkeiten zur Lösung auch strategischer Probleme
		\begin{itemize}
		\item Neue L\"osungen erarbeiten und bewerten
		\end{itemize}
		\item Gruppen oder Organisationen im Rahmen komplexer Aufgabenstellungen verantwortlich leiten und ihre Arbeitsergebnisse vertreten.
		\item Für neue anwendungs- oder forschungsorientierte Aufgaben Ziele definieren
		\item Wissen eigenst\"andig erschlie{\ss}en
     	\end{itemize}
     \end{column}
     	\begin{column}[T]{5cm} 
         	\begin{center}
	\vspace{1cm}
            		\includegraphics[width=0.8\textwidth]{GraduationHat}
        		\end{center}
     \end{column}
 \end{columns}}
\frame{\frametitle{Rolle des Lehrenden}
\framesubtitle{}
\begin{columns}[t] 
     \begin{column}[T]{6cm} 
     \vspace{1cm}
     	\begin{quote}
     		A teacher is never a giver of truth; he is a guide, a pointer to the truth that each student must find for himself.
     	\end{quote}
	\flushright Bruce Lee
	
     \end{column}
     	\begin{column}[T]{6cm} 
         	\begin{center}
            		\includegraphics[width=0.8\textwidth]{Bruce}
        		\end{center}
     \end{column}
 \end{columns}
}

\frame{\frametitle{Was bedeutet Methoden des QM?}
\framesubtitle{}
\centering
\tikz [small mindmap, every node/.style=concept, concept color=black!20, grow cyclic,
	level 1/.append style={level distance=3cm,sibling angle=72},
	level 2/.append style={level distance=3cm,sibling angle=45}] 
	\node [root concept] {Methoden des Qualit\"ats-managements} % root
	child { node {Lebenszyklus}}
	child { node {QM-Systeme}} 
	child { node {Mechatronik}}
	child { node {Methoden}}
	child { node {Data Science}}
	;
}


\frame{\frametitle{Themenplan}
\framesubtitle{Die angegebenen Kapitel sind dringend empfohlene Begleitlekt\"ure. Alle sind als E-Book verf\"ugbar.}
\begin{center}
\footnotesize
\vspace{-.8cm}
\begin{tabular}{|p{1cm}|p{6cm}|p{3cm}|l|}
\hline
Datum & Thema & Literatur \\ \hline
4. 10 & Einf\"uhrung, Einf\"uhrung Data Science & \cite{tebeka2017python}\\ \hline
5. 11.  & Recap Fertigungsmesstechnik  & \cite[Kap. 1.1, 2.1, 2.2]{keferstein} \\ 
  & Pr\"ufdatenerfassung und -auswertung, Messunsicherheit &  \cite[Kap. 1.1, 2.1, 2.2, 4.1, 4.2, 5.5.1]{keferstein} \\ \hline
 12. 11 & [V+P] Data Science: Tools and Methods &  \cite{tebeka2017python}\\ \hline
19. 11.& [V+P] Data Science: Tools and Methods &  \cite{tebeka2017python}\\ \hline
26. 11. & \hyperref[Sec:Zuverlaessig]{Einführung Zuverlässigkeitstechnik} & \cite[Kap. 2, 3, 10.5]{eberlin} \\ \hline
3. 12. & {Prüfdatenerfassung und -auswertung elektrischer Komponenten} & \cite[Kap. 3, 6.1]{muehl}   \\  \hline
17. 12. & \hyperref[Sec:Software]{Software-QS} & \cite[Kap. 1, 2, 4]{hoffmann} \\ \hline
7. 1. & Requirements Engineering, QM-Systeme & \cite[Kap. 1, 5]{bruggemann2012grundlagen} \\ \hline
 14. 1. & QFD, FTA, FMEA & \cite[Kap. 3.1, 3.2, 3.3]{bruggemann2012grundlagen} \\ \hline
%& Reserve / Q\&A & \\ \hline
21. 1. & Pitch-Event  &  \\ \hline
\end{tabular}
\end{center}
}


\frame[allowframebreaks]{\frametitle{Themen\"ubersicht}
\framesubtitle{}
%\begin{itemize}
%\item Grundbegriffe der Fertigungsmesstechnik 
%\item Grundlagen
%\begin{itemize}
%\item Ma{\ss}verk\"orperungen 
%\item Messunsicherheit und Ma{\ss}abweichung 
%\item Zeichnungseintragungen und Tolerierungen 
%\end{itemize}
%\item Pr\"ufdatenerfassung
%\begin{itemize}
%\item \"Ubersicht der Verfahren 
%\item Werkstattpr\"ufmittel 
%\item Messwertaufnehmer 
%\item Lehrende Pr\"ufung
%\end{itemize}
%\item Pr\"ufdatenauswertung
%\begin{itemize}
%\item Statistische Grundlagen 
%\item Pr\"ufprozesseignung 
%\item Erforderliche Messgenauigkeit 
%\end{itemize}
%\item R\"uckf\"uhrbarkeit 
%\item Monte-Carlo-Simulation und Faltungsintegrale
%\end{itemize}
\tableofcontents
}

\frame{\frametitle{Lernziele}
\framesubtitle{}
Die Studierenden k\"onnen nach erfolgreicher Bearbeitung des Moduls:
\begin{itemize}
\item QM-Methoden im Produktlebenszyklus anwenden und bewerten, um sie im Rahmen der Produktentwicklung einzusetzen
\item Produkte innerhalb eines QM-Systems entwickeln
\item Mess- und Pr\"ufverfahren anwenden und die Ergebnisse bewerten und kommunizieren
\item Angemessene Verfahren der Pr\"ufdatenerfassung ausw\"ahlen und bestehende beurteilen
\item Daten visualisieren und interpretieren, um Entscheidungen zu untermauern
\item Methoden und Werkzeuge der Datenwissenschaft nutzen, um Vorhersagen zu treffen
\item Produktdaten nutzen, um Fehlerbilder zu identifizieren
\item Grundlegende Pr\"ufdaten elektrischer Komponenten bestimmen und auswerten
\item Zuverl\"assigkeit von Systemen und Komponenten bewerten
\item Verfahren zur Qualit\"atssicherung in Software-Komponenten bewerten und anwenden
\end{itemize}
}

%\frame{\frametitle{Praktika und Pr\"ufung}
%\framesubtitle{Zwei verpflichtende Praktikumstermine [V+P], semesterbegleitende Coursework}
%\begin{columns}[t] 
%     \begin{column}[T]{6cm} 
%     	\begin{enumerate}
%     		\item Werkstattpr\"ufmittel (ca. KW 45)
%		\begin{itemize}
%		\item Erstellung eines Berichts (15\% Modulnote)
%		\item Formular zum Download auf meiner Homepage
%		\end{itemize}
%		\item Messen Ohmscher Widerst\"ande (ca. KW 47)
%		\begin{itemize}
%		\item Erstellung eines Berichts (15\% Modulnote)
%		\item Formular zum Download auf meiner Homepage
%		\end{itemize}
%		\item Coursework inkl. Pitch (70\% Modulnote)
%     	\end{enumerate}
%     \end{column}
%     	\begin{column}[T]{6cm} 
%         	\begin{center}
%            		\includegraphics[width=0.8\textwidth]{TechReport}\source{}
%        		\end{center}
%     \end{column}
% \end{columns}
%}



\frame{\frametitle{Praktikum/Coursework}
\framesubtitle{Zwei verpflichtende Praktikumstermine [V+P], semesterbegleitende Coursework}
\begin{columns}[t] 
     \begin{column}[T]{7cm} 
\begin{itemize}
\item Gewichtung: 100\% der Modulnote durch Seminararbeit
\end{itemize}
\begin{itemize}
\item Recommendations:
\begin{itemize}
	\item Start early!
	\item Join the slack group at mdqmws1819.slack.com
\end{itemize}
\item Requirements:
\begin{itemize}
	\item Use Jupyter and hand in a Jupyter Notebook for each question, containing fully readable documents.
	\item Exchange with your fellow students, however hand in an individual solution.
	\item Deliver a pitch
	\begin{itemize}
		\item Not more than five minutes
		\item Why do you provide the best solution?
		\end{itemize} 
\end{itemize}
\end{itemize}
\end{column}
     	\begin{column}[T]{6cm} 
	\begin{center}
            		\includegraphics[width=0.8\textwidth]{MdQMSlack1819} 
        		\end{center}
     \end{column}
 \end{columns}
}

\frame{\frametitle{Keine Angst!}
\framesubtitle{Evaluationsergebnisse WS 17/18}
         	\begin{center}
            		\includegraphics[width=0.95\textwidth]{EvaMdQM17}\source{}
        		\end{center}
}


%\section{Data Science als Methode des QM}
% !TEX root = ../17-MdQM-Vorlesung.tex
%\section*{Einführung}
\label{Sec:introDataScience}
\subsection{Brief Introduction to Data Science}

\frame{\frametitle{What is Data Science?}
\framesubtitle{Data science is our means of taming unstructured information and gathering insight. - Matthew Mayo, KDnuggets}
\begin{itemize}
\item Interdisciplinary field of
\begin{itemize}
		\item Systems,
		\item Methods and
		\item Processs to extract insight or knowledge from data.
\end{itemize}
\item Term coined in 2001, gained popularity in 2010
\item Integrates:
\begin{itemize}
		\item Data Engineering
		\item Scientific Method
		\item Mathematics
		\item Statistics
		\item Advanced Computational Methods
		\item Visualisation
		\item Hacker Mindset
		\item Domain Expertise
		\end{itemize}
\end{itemize}
}


\frame{\frametitle{How does Data Science integrate to Mechanical Engineering?}
\framesubtitle{}
\begin{center}
\begin{tikzpicture}
\begin{scope}[blend group=soft light]
\fill[red!30!white] ( 90:1.2) circle (2); 
\fill[green!30!white] (210:1.2) circle (2); 
\fill[blue!30!white] (330:1.2) circle (2); 
\end{scope}
%\node at ( 90:2.6) {Railway}; 
\node at ( 90:2.4) {Domain}; 
\node at ( 90:2.0) {expertise}; 
\node at (205:2.1) {Maths \&};
\node at (220:2.1) {Statistics}; 
\node at (335:2.1) {Computer}; 
\node at (320:2.1) {science}; 
%\node at (0,.4) {Rail};
\node at (0,0.2) {Data};
\node at (0,-.2) {Science};
\node at (270:1.1) {\small Machine};
\node at (270:1.5) {\small Learning};
\node at (145:1.4) {\small Research};
\node at (35:1.4) {\small Danger};
\node at (20:1.4) {\small zone};
\end{tikzpicture}
\end{center}
}

\frame{\frametitle{Why Data Sciene?}
\framesubtitle{Applying the right tools and techniques, you bring more value! }
\begin{columns}[t] 
     \begin{column}[T]{6cm} 
     	\begin{itemize}
     		\item Turn data to information
		\begin{itemize}
		\item Inform decisions
		\item Increase insight
		\end{itemize}
		\item Companies:
		\begin{itemize}
		\item Collect large amounts of data
		\item Do rarely integrate them
		\item Frequently decide based on the ``gut''
		\end{itemize}
     	\end{itemize}
     \end{column}
     	\begin{column}[T]{6cm} 
         	\begin{center}
            		\begin{tikzpicture}
                            \begin{scope}[blend group=soft light]
                            \fill[red!30!white] ( 90:1.2) circle (2); 
                            \fill[green!30!white] (210:1.2) circle (2); 
                            \fill[blue!30!white] (330:1.2) circle (2); 
                            \end{scope}
                            %\node at ( 90:2.6) {Railway}; 
                            \node at ( 90:2.4) {Domain}; 
                            \node at ( 90:2.0) {expertise}; 
                            \node at (205:2.1) {Maths \&};
                            \node at (220:2.1) {Statistics}; 
                            \node at (335:2.1) {Computer}; 
                            \node at (320:2.1) {science}; 
                            %\node at (0,.4) {Rail};
                            \node at (0,0.2) {Data};
                            \node at (0,-.2) {Science};
                            \node at (270:1.1) {\small Machine};
                            \node at (270:1.5) {\small Learning};
                            \node at (145:1.4) {\small Research};
                            \node at (35:1.4) {\small Danger};
                            \node at (20:1.4) {\small zone};
                            \end{tikzpicture}

        		\end{center}
     \end{column}
 \end{columns}
}

\frame{\frametitle{How do you incrase Value with Data Science?}
\framesubtitle{ }
\begin{columns}[t] 
     \begin{column}[T]{6cm} 
     	\begin{itemize}
     		\item Improve decision making
		\begin{itemize}
		\item Empower management
		\item Supply data driven evidence
		\end{itemize}
		\item Identify trends and bring to action
		\item Challenge your colleagues
		\item Find opportunities for improvement
		\item Test decisions
		\item Understand customers
     	\end{itemize}
     \end{column}
     	\begin{column}[T]{6cm} 
         	\begin{center}
            		\begin{tikzpicture}
                            \begin{scope}[blend group=soft light]
                            \fill[red!30!white] ( 90:1.2) circle (2); 
                            \fill[green!30!white] (210:1.2) circle (2); 
                            \fill[blue!30!white] (330:1.2) circle (2); 
                            \end{scope}
                            %\node at ( 90:2.6) {Railway}; 
                            \node at ( 90:2.4) {Domain}; 
                            \node at ( 90:2.0) {expertise}; 
                            \node at (205:2.1) {Maths \&};
                            \node at (220:2.1) {Statistics}; 
                            \node at (335:2.1) {Computer}; 
                            \node at (320:2.1) {science}; 
                            %\node at (0,.4) {Rail};
                            \node at (0,0.2) {Data};
                            \node at (0,-.2) {Science};
                            \node at (270:1.1) {\small Machine};
                            \node at (270:1.5) {\small Learning};
                            \node at (145:1.4) {\small Research};
                            \node at (35:1.4) {\small Danger};
                            \node at (20:1.4) {\small zone};
                            \end{tikzpicture}

        		\end{center}
     \end{column}
 \end{columns}
}

\frame{\frametitle{The data science process}
\framesubtitle{}
         	\begin{center}
            		\includegraphics[width=0.8\textwidth]{DataScienceProcess}\source{Source: Farcaster/English Wikipedia}
        		\end{center}
   }

\frame{\frametitle{How to get started}
\framesubtitle{}
\begin{itemize}
\item Set up your system. 
\begin{itemize}
		\item Install Anaconda to obtain Python/Jupyter
		\item Set up a free education account with github.com
		\item Install the app for your operating system
		\end{itemize}
\item Acquire data: start with popular open data sets. Get your company to make data accessible.
\item Ingest and transform: figure out the formats and sizes of your data. Find appropriate ways to import or access them.
\item Explore the data. Do you already find patterns from just plotting them?
\item Try your ``toolbox'' of methods (or a subset of it that sounds promising).
\item Visualise the results. Make your findings convincing to others: colleagues, managers, customers etc.
\end{itemize}
}




% !TEX root = ../17-MdQM-Vorlesung.tex
\subsection{Data Science: Tools and Techniques}
\frame{\frametitle{Tools for Data Scientist}
\framesubtitle{}
\begin{columns}[t] 
     \begin{column}[T]{6cm} 
     	\begin{itemize}
     		\item Programming languages:
		\begin{itemize}
		\item R
		\item Python
		\item SAS
		\item ...
		\end{itemize}
		\item Visualisation app:
		\begin{itemize}
		\item Tableau
		\end{itemize}
		\item Development Environment (IDE):
		\begin{itemize}
		\item Jupyter
		\item Spyder
		\end{itemize}
		\item Also potentially:
		\begin{itemize}
		\item Matlab
		\item Scilab
		\item ...
		\end{itemize}
     	\end{itemize}
  For installation, visit: https://www.anaconda.com/ \\
  Data folder:\\
  http://bit.ly/ifv122019
     \end{column}
     	\begin{column}[T]{6cm} 
         	\begin{center}
            		\includegraphics[width=0.95\textwidth]{Jupyter}\source{}
        		\end{center}
     \end{column}
 \end{columns}
}

\frame{\frametitle{Python Introduction}
	\begin{itemize}
		\item Python 
		\begin{itemize}
			\item is interpreted
			\item can be run on terminal or IDE
		\end{itemize}
		\item Indentation is strict, defines program structures (no braces)
		\item Dynamically typed
		\begin{itemize}
			\item No variable declaration required
		\end{itemize}
		\item Syntax
		\begin{itemize}
			\item Refer to example worksheets
		\end{itemize}
		\item Data structures
		\begin{itemize}
			\item List: \texttt{l = [1, 2, "a"]}
			\item Tuples: \texttt{t = (1, 2, "a")}
			\item Dictionaries: \texttt{d = \{"a":1, "b":2\}}
			\item Sets: \texttt{s = set(\[1, 2, 3, 4\])}
		\end{itemize}
	\end{itemize}
}

\frame{\frametitle{Pandas Introduction}
\begin{itemize}
	\item Pandas is a framework for manipulation and handling of series and rectangular data
	\item Syntax
		\begin{itemize}
			\item Refer to example worksheets
		\end{itemize}
	\item Data structures:
	\begin{itemize}
		\item Series (\texttt{pd.Series(\textsf{from e.g. array})})
		\item DataFrame (\texttt{pd.DataFrame(\textsf{from e.g. dicts})})
	\end{itemize}
\end{itemize}
}

\frame{\frametitle{Get to work!}
\framesubtitle{Load and try \texttt{0-Basics.ipynb (http://bit.ly/ifv122019
)} }
\begin{center}
\includegraphics[width = .3\textwidth]{Wagenmeister.png}
\end{center}
}

\frame{\frametitle{Selected Techniques applied in Data Science}
\framesubtitle{}
\begin{itemize}
\item \textbf{Visualisation}
\item Regression: Linear, Logistic
\item \textbf{Density Estimation}
\item Confidence Intervals
\item Test of Hypotheses
\item Pattern Recognition
\item {Time Series}
\item \textbf{Unsupervised Learning (Clustering)}
\item Supervised Learning
\item Decision Trees
\item \textbf{Monte-Carlo-Simulation}
\item Bayesian Statistics
\item \textbf{Principal Component Analysis}
\item \textbf{Support Vector Machines}
\end{itemize}
}

\subsubsection{Visualisation}
\frame{\frametitle{Visualisation approaches}
\framesubtitle{Likely, you will all know Excel plots - there is much more to it...}
\begin{columns}[t] 
     \begin{column}[T]{6cm} 
     	\begin{itemize}
     		\item Line plot with / w/o confidence
		\item Bar plots (stacked/grouped)
		\item Histogram
		\item Scatter plot
		\begin{itemize}
		\item Blob sizes
		\item Colors
		\end{itemize}
		\item Box whisker
		\item Bubble chart
		\item Geospatial plots
		\item Surface plot
		\item Heat map
		\item Tree map
     	\end{itemize}
     \end{column}
     	\begin{column}[T]{6cm} 
         	\begin{center}
            		\only<1>{\includegraphics[width=0.9\textwidth]{SingularValues}\source{}}
			\only<2>{
			\begin{tikzpicture}[scale = 0.9]
		\begin{axis}[ybar stacked, 
		colormap/bluered,
		ylabel={m EUR/a},
		legend style={at={(0.5,-0.40)},
		anchor=north, legend columns=3},
		symbolic x coords={EMEA, ASEAN, CIS, E. Europe, NAFTA, RoA, W. Europe},
		xtick=data,
		x tick label style={rotate=45,anchor=east},
		title={Urban Vehicle Market Size}]]
		\addplot[fill = RYB1, draw = RYB1!50!black] coordinates
			{(EMEA,.054) (ASEAN,.152) (CIS, .082) (E. Europe, .392) (NAFTA, .563) (RoA, .043) (W. Europe, 1.309)};
		\addplot[fill = RYB2, draw = RYB2!50!black] coordinates
		{(EMEA,.257) (ASEAN,1.638) (CIS, .052) (E. Europe, .215) (NAFTA, 1.131) (RoA, .607) (W. Europe, .661)};
		\addplot[fill = RYB3, draw = RYB3!50!black] coordinates
		{(EMEA,.142) (ASEAN, .094) (CIS, 0) (E. Europe, 0) (NAFTA, .085) (RoA, .151) (W. Europe, .104)};
		\legend{LRV, Metro, APM};
		\end{axis}
		\end{tikzpicture}}
			\only<3>{\includegraphics[width=0.95\textwidth]{MuHist}\source{}}
			\only<4>{\includegraphics[width=0.95\textwidth]{Scatter}\source{}}
		\only<5>{Intentionally left blank}
			\only<6>{\includegraphics[width=0.95\textwidth]{ReichweiteFilm}\source{}}
			\only<7>{\includegraphics[width=0.95\textwidth]{VTPowerConsumption}\source{}}
			\only<8>{\includegraphics[width=0.95\textwidth]{fxfyr}\source{}}
%			\only<9>{\includegraphics[width=0.8\textwidth]{}\source{}}
%			\only<10>{\includegraphics[width=0.8\textwidth]{}\source{}}
        		\end{center}
     \end{column}
 \end{columns}
}

\subsubsection{Density Estimation}
\frame{\frametitle{Density estimation}
\framesubtitle{Estimate underlying probability density function from observed data}
\begin{columns}[t] 
     \begin{column}[T]{6cm} 
     	\begin{itemize}
     		\item Analysis of distribution properties from sampled data
		\item Fit an appropriate kernel to the samples
     	\end{itemize}
     \end{column}
     	\begin{column}[T]{6cm} 
         	\begin{center}
		\pgfplotsset{compat=newest,
                      tick label style={font=\sansmath\sffamily},
                      axis line style={draw=black!80, line width=0.1875ex},
                      y tick label style={/pgf/number format/fixed},
                      tick style={major tick length=0.0ex},
                      major grid style={thick, dash pattern=on 0pt off 2pt, black!50},
                      height = 6cm
                    }
                    
            		\begin{tikzpicture}[line cap=round, line join=round]
                    \begin{axis}[ymajorgrids, xmajorgrids]
                    \addplot [ybar, domain=0:15, samples=16, fill=blue!50!cyan, draw=none] 
                      (x, {0.6*norm(x, 4, 0.1) + 0.4*norm(x, 12, .05)});
                    \addplot [very thick, draw=orange,  domain=0:15, samples=100, smooth]
                      (x, {0.6*norm(x, 4, 0.1) + 0.4*norm(x, 12, .05) });
                    \end{axis}
                    \end{tikzpicture}
        		\end{center}
     \end{column}
 \end{columns}
}
%\subsubsection{Time Series}

\frame{\frametitle{Get to work!}
\framesubtitle{Load and try \texttt{1-Visualisation.ipynb}  (http://bit.ly/ifv122019
)}
\begin{center}
\includegraphics[width = .3\textwidth]{Wagenmeister.png}
\end{center}
}

\subsubsection{Unsupervised Learning (Clustering)}
\frame{\frametitle{Machine Learning}
\framesubtitle{Using MIT OpenCourseware Slides}
Source:
\begin{center}
Eric Grimson, John Guttag, and Ana Bell. 6.0002 Introduction to Computational Thinking and Data Science. Fall 2016. Massachusetts Institute of Technology: MIT OpenCourseWare, https://ocw.mit.edu. License: Creative Commons BY-NC-SA.
\end{center}
}
\setbeamercolor{background canvas}{bg=}
%\frame{
\includepdf[page = 4-6]{OCW/ML.pdf}%}
\frame{\frametitle{Clustering}
\framesubtitle{Using MIT OpenCourseware Slides}
Source:
\begin{center}
Eric Grimson, John Guttag, and Ana Bell. 6.0002 Introduction to Computational Thinking and Data Science. Fall 2016. Massachusetts Institute of Technology: MIT OpenCourseWare, https://ocw.mit.edu. License: Creative Commons BY-NC-SA.
\end{center}
}
\includepdf[page = 3-21]{OCW/Clustering.pdf}%}

\subsubsection{Monte-Carlo-Simulation}
\frame{\frametitle{Monte-Carlo-Simulation (MC-Simulation) recap}
\framesubtitle{}
\begin{itemize}
\item Method of estimating value of unknown quantity using inferential statistics
\item Inferential statistics terms:
\begin{itemize}
		\item Population: set of examples
		\item Sample: Proper subset of population
\end{itemize}
\item Random sample tends to exhibit same properties as population it is drawn from
\end{itemize}
\begin{example}[Flip coins]
 Let's estimate the probabilities of heads vs. tails for an infinite number of coin flips:
 \begin{itemize}
		\item One flip (heads): 100\% heads? 
		\item Two flips (h, h): still 100 \% heads? Confidence level?
		\item 100 flips (52 h, 48 t): Probability of next coin coming up heads 52/100.
		\end{itemize}
\end{example}
}

\frame{\frametitle{Key aspects to be asked in MC-Simulations}
\framesubtitle{}
\begin{itemize}
\item Never possible to guarantee perfect accuracy through sampling
\item Never assume that an estimate is precisely correct
\item How many samples do we need to look at before we can have justified confidence in our answers?
\begin{itemize}
	\item Answer depends on underlying distribution
	\item Especially hard for defects ``Rare event simulation''
\end{itemize}
\end{itemize}
}

\subsubsection{Monte-Carlo-Simulation Application Example}
\frame{\frametitle{MC-Simulation application example: Braking curve}
\framesubtitle{CCS systems rely on braking curves to describe the train's braking capability.}
\begin{columns}[t] 
     \begin{column}[T]{6cm} 
     	\begin{itemize}
     		\item To supervise train velocity, CCS systems predict the future braking capability of the train
		\item However, there is not \textit{the} braking capability
		\item Braking curves exhibit a randomised behaviour
     	\end{itemize}
     \end{column}
     	\begin{column}[T]{6cm} 
         	\begin{center}
			\includegraphics[width=\textwidth]{brakingcurvesND}
        		\end{center}
     \end{column}
 \end{columns}
}

% \frame{\frametitle{White Box Modelling of the braking system}
% \framesubtitle{Which parameters can be identified and which effect do they have on the braking distance?}
% \begin{itemize}
% \item Brake pipe: propagation velocity, flow resistances, train length
% \item Distributor valve: Filling time, brake cylinder pressure
% \item Braking force generation: efficiency, brake radius (for disc brakes), pad/block friction coefficient
% \item Wheel/rail contact: rail surface, contaminants, slip, ...
% \end{itemize}
% \begin{center}
% \begin{center}
%           \includegraphics[width=\textwidth]{TrainBPDiagram}
% \end{center}
% \end{center}
% \begin{itemize}
% 	\item Also discrete failure events need to be considered 
% \end{itemize}
% }

% \frame{\frametitle{Why chose MC-Simulation? What are the challenges?}
% \framesubtitle{}
% \begin{itemize}
% \item Error-propagation:
% \begin{itemize}
% 		\item Conservative: assumes normal distribution for all parameters
% 		\item Complex: requires explicit function formulation and partial differentiation
% \end{itemize}
% \item (Standard) Monte-Carlo-Simulation:
% \begin{itemize}
% 		\item Efficient (in terms of confidence): returns shortest (also asymmetric) confidence interval
% 		\item Inefficient (in terms of computational effort):
% 		\begin{itemize}
% 		\item For rare event $\varepsilon \ll 1$, $N \approx \frac{100}{\varepsilon}$ trials required
% 		\item Typical according to CSM: $\varepsilon \in \left[10^{-7} \ldots  10^{-9}\right] \, \Rightarrow \, N \approx 10^{11}$
% 		\end{itemize}
% \end{itemize}
% \item ERA proposes to precalculate braking curves for limited number of train formations
% \begin{itemize}
% 		\item Freight trains to be handled using braked weight and correction factor
% 		\end{itemize}
% \end{itemize}
% }

% \frame{\frametitle{Importance sampling for MC-Simulations}
% \framesubtitle{Importance sampling (IS) increases the probability of ``desired'' outcomes in Monte-Carlo-Simulations.}
% \begin{itemize}
% \item Typical IS approaches:
% \begin{itemize}
% 		\item Stratification: select only relevant strata of the sampling range
% 		\item Scaling: Scale random variable
% 		\item Translation: Move random variable to more relevant part of sampling space
% 		\item Change of random variable: Replace random variable by one more likely to produce outcomes in the relevant range
% 		\item Adaptive approaches
% 		\end{itemize}
% \item Effect: higher number of samples in region of interest
% \item Correction factor: Likelihood ratio $L(y) = \frac{f(y)}{\tilde{f}(y)}$
% \end{itemize}
% }

% \frame{\frametitle{Application of IS to braking curves}
% \framesubtitle{Select relevant variables for IS.}
% \begin{center}
% \includegraphics[width = .9\textwidth]{mhist}
% \end{center}
% }

% \frame{\frametitle{Application of IS to braking curves}
% \framesubtitle{Change identified random variables, in the case at hand $\mu_{B}$}
% \begin{center}
% \includegraphics[width = .9\textwidth]{BChist}
% \end{center}
% }

% \frame{\frametitle{Application of IS to braking curves}
% \framesubtitle{Analyse for rare events, here braking distances in excess of 1100 m. $N = 5 \cdot 10^7$}
% \only<1>{\begin{center}
% \includegraphics[width = .9\textwidth]{rare}
% \end{center}}
% \only<2>{
% %\hspace{-.62cm}
% \begin{tabular}{|c|c|c|c|c|c|c|}
% \hline
% $s$ & $n_{\mathcal{U}}$ & $p_{\mathcal{U}}$ & $n_{IS, 1}$ & $p_{IS, 1}$ & $n_{IS, 2}$ & $p_{IS, 2}$ \\ \hline
% 1000 & 24400 & 4.89$\cdot 10^{-3}$ & 2.27$\cdot 10^6$ & 1.14$\cdot 10^{-2}$ & 3.11$\cdot 10^5$ & 1.77$\cdot 10^{-3}$ \\ \hline
% 1050 & 2 & 4$\cdot 10^{-7}$ & 6.66$\cdot 10^4$ & 2.02$\cdot 10^{-4}$ & 1.48$\cdot 10^4$ & 1.59$\cdot 10^{-4}$ \\ \hline
% 1100 & 0 & 0 & 115 & 2.04$\cdot 10^{-7}$ & 419 & 7.50$\cdot 10^{-6}$ \\ \hline
% 1150 & 0 & 0 & 0 & 0 & 15 & 3.88$\cdot 10^{-7}$ \\ \hline
% 1160 & 0 & 0 & 0 & 0 & 7 & 2.05$\cdot 10^{-7}$ \\ \hline
% 1170 & 0 & 0 & 0 & 0 & 5 & 1.46$\cdot 10^{-7}$ \\ \hline
% 1180 & 0 & 0 & 0 & 0 & 4 & 1.16$\cdot 10^{-7}$ \\ \hline
% 1190 & 0 & 0 & 0 & 0 & 1 & 2.90$\cdot 10^{-8}$ \\ \hline
% \end{tabular}
% }
% }

\subsubsection{Principal Component Analysis}
\frame{\frametitle{Principal Component Analysis (PCA)}
\framesubtitle{Using MIT OpenCourseware Slides}
Source:
\begin{center}
Philippe Rigollet. 18.650 Statistics for Applications . Fall 2016. Massachusetts Institute of Technology: MIT OpenCourseWare, https://ocw.mit.edu. License: Creative Commons BY-NC-SA.
\end{center}
}

\setbeamercolor{background canvas}{bg=}
%\frame{
\includepdf[page = 2-16]{OCW/PCA.pdf}%}




\subsubsection{Support Vector Machines}

\frame{\frametitle{Support Vector Machine}
\framesubtitle{Separate data into subsets according to their $nD$-coordinates.}
\begin{columns}[t] 
     \begin{column}[T]{5cm} 
     	\begin{itemize}
		\item Idea:
		\begin{itemize}
		\item Find separating hyperplane maximising the distance between borderline instances
		\item For data mixed in feature space, slack parameter (and penalty $C$) is used
		\item If no separating hyperplane can be found in feature space, dimension is increased ``Kernel trick''
		\end{itemize}
     		\item Robust:
		\begin{itemize}
		\item High dimensionality
		\item Small datasets
		\end{itemize}
		\item Simple to complex models
     	\end{itemize}
     \end{column}
     	\begin{column}[T]{7cm} 
         	\begin{center}
		\only<1>{
            		\begin{tikzpicture}[>=stealth', scale = 0.9]
                      % Draw axes
                      \draw [<->,thick] (0,5) node (yaxis) [above] {$y$}
                            |- (5,0) node (xaxis) [right] {$x$};
                      % draw line
                      \draw (0,-1) -- (5,4); % y=x-1
                      \draw[dashed] (-1,0) -- (4,5); % y=x+1
                      \draw[dashed] (2,-1) -- (6,3); % y=x-3
                      % \draw labels
                      \draw (3.5,3) node[rotate=45,font=\small] 
                            {$\mathbf{w}\cdot \mathbf{x} + b = 0$};
                      \draw (2.5,4) node[rotate=45,font=\small] 
                            {$\mathbf{w}\cdot \mathbf{x} + b = 1$};
                      \draw (4.5,2) node[rotate=45,font=\small] 
                            {$\mathbf{w}\cdot \mathbf{x} + b = -1$};
                      % draw distance
                      \draw[dotted] (4,5) -- (6,3);
                      \draw (5.25,4.25) node[rotate=-45] {$\frac{2}{\Vert \mathbf{w} \Vert}$};
                      \draw[dotted] (0,0) -- (0.5,-0.5);
                      \draw (0,-0.5) node[rotate=-45] {$\frac{b}{\Vert \mathbf{w} \Vert}$};
                      \draw[->] (2,1) -- (1.5,1.5);
                      \draw (1.85,1.35) node[rotate=-45] {$\mathbf{w}$};
                      % draw negative dots
                      \fill[red] (0.5,1.5) circle (3pt);
                      \fill[red]   (1.5,2.5)   circle (3pt);
                      \fill[black] (1,2.5)     circle (3pt);
                      \fill[black] (0.75,2)    circle (3pt);
                      \fill[black] (0.6,1.9)   circle (3pt);
                      \fill[black] (0.77, 2.5) circle (3pt);
                      \fill[black] (1.5,3)     circle (3pt);
                      \fill[black] (1.3,3.3)   circle (3pt);
                      \fill[black] (0.6,3.2)   circle (3pt);
                      % draw positive dots
                      \draw[red,thick] (4,1)     circle (3pt); 
                      \draw[red,thick] (3.3,.3)  circle (3pt); 
                      \draw[black]     (4.5,1.2) circle (3pt); 
                      \draw[black]     (4.5,.5)  circle (3pt); 
                      \draw[black]     (3.9,.7)  circle (3pt); 
                      \draw[black]     (5,1)     circle (3pt); 
                      \draw[black]     (3.5,.2)  circle (3pt); 
                      \draw[black]     (4,.3)    circle (3pt); 
                    \end{tikzpicture}}
 %%%%%%%%%%%%%                  
                    \only<2>{
                    \vspace{1cm}
                    \begin{tikzpicture}[>=stealth',x=1cm,y=1cm, scale = .5]
 
                        %draw[color=gray] (0,0) grid (6,6);
                        \draw (0,0) rectangle (6,6);
                        % \draw line
                        \draw[color=red,line width=2pt]
                          (2,6) .. controls (3,5.5) and (3,5) .. 
                          (3,5) .. controls (3,4) and (2,2.5) .. 
                          (2,2) .. controls (2,1) and (2.8,1) .. 
                          (3,1) .. controls (3.5,1) and (3.5,2) .. 
                          (4,2) .. controls (4.5,2) and (6,0) .. 
                          (6,0);
                        % \draw left dashed line
                        \draw[dashed] 
                          (1.5,6) .. controls (2.5,5.5) and (2.5,5) .. 
                          (2.5,5) .. controls (2.5,4) and (1.5,2.5) .. 
                          (1.5,2) .. controls (1.5,.5) and (2.8,.5) .. 
                          (3,.5) .. controls (3.75,.5) and (3.5,1.5) .. 
                          (4,1.5) .. controls (4.5,1.5) and (5.5,0) .. 
                          (5.5,0);
                        % \draw right dashed line
                        \draw[dashed] 
                          (2.5,6) .. controls (3.5,5.5) and (3.5,5) .. 
                          (3.5,5) .. controls (3.5,4) and (2.5,2.5) .. 
                          (2.5,2) .. controls (2.5,1.5) and (2.8,1.5) .. 
                          (3,1.5) .. controls (3.25,1.5) and (3.5,2.5) .. 
                          (4,2.5) .. controls (4.5,2.5) and (6,0.5) .. 
                          (6,0.5);
                        %\draw[color=gray] (2,6) -- (3,5) -- (2,2) -- (3,1) -- (4,2) -- (6,0);
                        %\draw[color=gray] (1.5,6) -- (2.5,5) -- (1.5,2) -- (3,.5)-- (4,1.5)-- (5.5,0);
                        %\draw[color=gray] (2.5,6) -- (3.5,5) -- (2.5,2) -- (3,1.5)-- (4,2.5)-- (6,0.5);
                         
                        %\draw[color=gray] (7,0) grid (13,6);
                        \draw (7,0) rectangle (13,6);
                        % \draw line
                        \draw[color=red,line width=2pt] (8.5,6) -- (12,0);
                        % \draw dashed line
                        \draw[dashed]  (8,6) -- (11.5,0);
                        \draw[dashed]  (9,6) -- (12.5,0);
                         
                        \draw[->,thick] (5,3) -- (8,3) node [above,pos=.5] {$\phi$};
                         
                        \def\positive{{%
                        {2.3,5.3},
                        {3.5,.7},
                        {1.5,2},
                        {1.2,2.1},
                        {1.8,.8},
                        {1,5.5},
                        {1.2,5.8},
                        {.75,.2},
                        {2,4},
                        {5, 0.5},
                        {1.5,3},
                        {2.3,.5},
                        %
                        {9.3,3.3},
                        {11,.8},
                        {8.5,2},
                        {7.2,4.1},
                        {8.8,.8},
                        {8,5.5},
                        {8.2,5},
                        {7.75,.2},
                        {9,4.2},
                        {12, 0.5},
                        {8.5,3},
                        {9.3,.5},
                        }}
                         
                        % \draw positive dots
                        \foreach \i in {0,...,20} {
                          \pgfmathparse{\positive[\i][0]}\let \x \pgfmathresult;
                          \pgfmathparse{\positive[\i][1]}\let \y \pgfmathresult;
                          \fill[black] (\x,\y) circle (2pt);
                        }
                         
                        \def\negative{{%
                        {4,2.5},
                        {3.5,5},
                        {2.6,1.6},
                        {4.5,5.2},
                        {5.5,3.7},
                        {3.9,4.7},
                        {5,2.7},
                        {3.5,4.2},
                        {5.8,.9},
                        %
                        {10.75,3},
                        {10.5,5},
                        {11.6,1.6},
                        {11.5,5.2},
                        {12.5,3.7},
                        {10.9,4.7},
                        {12,2.7},
                        {10.5,4.2},
                        {12.8,.9},
                        }}
                         
                        % \draw negative dots
                        \foreach \i in {0,...,16} {
                          \pgfmathparse{\negative[\i][0]}\let \x \pgfmathresult;
                          \pgfmathparse{\negative[\i][1]}\let \y \pgfmathresult;
                          \draw[black] (\x,\y) circle (3pt);
                        }
                         
                        \end{tikzpicture}
                    }
        		\end{center}
     \end{column}
 \end{columns}
}

%\subsubsection{Further Outlook}

\subsubsection{Practical hints}
\frame{\frametitle{Practical hints for application of data science methods}
\framesubtitle{}
\begin{columns}[t] 
     \begin{column}[T]{6cm} 
     	\begin{itemize}
     		\item Get to know your data
		\item Use training and validation sets
		\item For clustering and ML: 
		\begin{itemize}
		\item Regularise
		\item Remove outliers, NaN
		\end{itemize} 
		\item When building models:
		\begin{itemize}
		\item Don't expect perfect fit
		\item Inspect confusion matrix
		\end{itemize}
     	\end{itemize}
     \end{column}
     	\begin{column}[T]{6cm} 
         	\begin{center}
            		\renewcommand\arraystretch{1.5}
                            \setlength\tabcolsep{0pt}
                            \begin{tabular}{c >{\bfseries}r @{\hspace{0.7em}}c @{\hspace{0.4em}}c @{\hspace{0.7em}}l}
                              \multirow{10}{*}{\parbox{1.1cm}{\bfseries\raggedleft actual\\ value}} & 
                                & \multicolumn{2}{c}{\bfseries Prediction outcome} & \\
                              & & \bfseries p & \bfseries n & \bfseries total \\
                              & p$'$ & \MyBox{True}{Positive} & \MyBox{False}{Negative} & P$'$ \\[2.4em]
                              & n$'$ & \MyBox{False}{Positive} & \MyBox{True}{Negative} & N$'$ \\
                              & total & P & N &
                            \end{tabular}        		
                            \end{center}
     \end{column}
 \end{columns}
}

\frame{\frametitle{Get to work!}
\framesubtitle{Load and try \texttt{2-Container-SVM.ipynb}  (http://bit.ly/ifv122019
)}
\begin{center}
\includegraphics[width = .3\textwidth]{Wagenmeister.png}
\end{center}
}



%\section{Lebenszyklus}
%%Zuverl\"assigkeitstechnik
%% !TEX root = ../SFV15001_MdQM_Fertigungsmesstechnik_Rev04.tex

\label{Sec:Zuverlaessig}

\frame{\frametitle{Motivation}
\framesubtitle{Ein Gro{\ss}teil der Kosten technischer Systeme entsteht h\"aufig erst im Betrieb.}
\begin{itemize}
\item Kunde:
\begin{itemize}
	\item Verf\"ugbarkeit des Systems
	\item Ersatzteilbedarf
	\item Wartungsplanung
\end{itemize}
\item Lieferant:
\begin{itemize}
	\item Erwartungswert Gew\"ahrleistungskosten
	\item Absch\"atzung Obsolenszenz 
	\item Erf\"ullung vertraglicher Forderungen
\end{itemize}
\end{itemize}
}

\frame{\frametitle{Zuverl\"assigkeit}
\framesubtitle{}
\begin{itemize}
\item Zuverl\"assigkeit ist die F\"ahigkeit eines Objekts, unter gegebenen Bedingungen f\"ur eine bestimmte Zeit eine geforderte Funktion auszu\"uben.
\begin{itemize}
	\item Anwendbar f\"ur Systeme ohne Reparatur oder Wartung 
\end{itemize}
\item Verf\"ugbarkeit ist die Wahrscheinlichkeit, dass sich ein System zu einem beliebigen Zeitpunkt in einem Zustand befindet, in dem es seine Funktion erf\"ullen kann.
\begin{itemize}
		\item Auch anwendbar f\"ur reparierte und gewartete Systeme
		\end{itemize}
\end{itemize}
}


\frame{\frametitle{Fehler, Fehlertypen und Fehlerrate}
\framesubtitle{}
\begin{itemize}
\item Fehler:
\begin{itemize}
	\item System kann seine geforderten Eigenschaften nicht erf\"ullen
	\item Trotz ordnungsgem\"a{\ss}em Betrieb
\end{itemize}
\item Fehlertypen:
\begin{itemize}
	\item Fr\"uhe Fehler: nahezu ausschlie{\ss}lich am Anfang des Lebenszyklus auftretend
	\item Zuf\"allige Fehler: Auftreten unerwartet und unabh\"angig von Alter oder Nutzung einer Komponente
	\item Verschlei{\ss}fehler: zunehmende Fehler mit steigendem Alter der Komponente
\end{itemize}
\item Fehlerrate $\lambda$:
\begin{itemize}
	\item Fehler pro Zeiteinheit
	\item Basis: tats\"achliche Betriebsdauer
	\item Einheit f\"ur Elektrokomponenten h\"aufig FIT $\left(1 \mathrm{FIT} = 10^{-9} \mathrm{h}^{-1} \right)$ 
\end{itemize}
\end{itemize}
}

 \frame{\frametitle{Praktisches Verhalten der Ausfallrate}
\framesubtitle{}
\begin{center}
			 \pgfplotsset{width=12cm, height = 7cm}
            		\begin{tikzpicture}
			\only<3->{\fill[fill = yellow!50!black, opacity = 0.3] (0,0) rectangle (1.4,5.4);
			\node[rotate = 90] at (0.7,3.5) {\Large Fr\"uhausf\"alle};}
			\only<4->{\fill[fill = green!70!black, opacity = 0.3] (1.4,0) rectangle (6.8,5.4);
			\node[rotate = 90] at (4,3.5) {\Large Nutzbare Zeit};}
			\only<5->{\fill[fill = red!70!black, opacity = 0.3] (6.8,0) rectangle (10.4,5.4);
			\node[rotate = 90] at (9,3.5) {\Large Unzuverl\"assigkeit};}
			\only<6-6>{\draw[thick, draw = red!90!black] (6.8,0.4) circle [radius=0.2];
			\node[align = center] at (6,1){Wie l\"asst sich der \\ \"Ubergang erkennen?};}
			 
			\begin{scope}
				[spy using outlines={circle, black,
				magnification=2, size=3.5cm, connect spies}]
			 
				\begin{axis}[stack plots=y, no marks,
				xmax=14, xmin = 0,
				 ymin = 0, ymax = 1, %grid=both,
				 xtick = \empty, ytick = \empty,
				%axis lines=middle, 
				%ylabel = {$p$},
				%xlabel = {$t/\mathrm{a}$},
				%legend entries={Zufallssch\"aden, Zeitabh\"angigkeit},
				legend style = {at={(0.5,0.8)},
      				  anchor=north}]
  
  				\only<1->{\addplot[fill = red!70!black, opacity = 0.5, thick, draw = red!70!black] table[x=t,y=h, %row sep=crcr
				] {Exponential.dat} \closedcycle; }
				\only<1-2>{\addlegendentry{Zufallssch\"aden};}
  
  				\only<2->{\addplot[fill = blue!70!black, opacity = 0.5, thick, draw = blue!70!black] table[x=t,y=h, %row sep=crcr
				] {Weibull.dat} 
				coordinate [pos=.6] (A)
   				coordinate [pos=.7]  (B)
				\closedcycle ; }
				
				\only<1-2>{\addlegendentry{Zeitabh\"angigkeit};}
				\only<7->{\fill[thick, draw = black, fill = black] (A) -| (B)
				 node [align = center, pos=2.5,anchor=west]
    				 {Wie entwickelt \\ sich die \\Zuverl\"assigkeit?};}

	\end{axis}
	\only<7->{ \spy on (6.8,0.5) in node (zoom) [left] at (8.5,3.5);}
	\end{scope}
	
	\end{tikzpicture}
\end{center}
}


\frame{\frametitle{Mathematische Beschreibung des Ausfallverhaltens}
\framesubtitle{Durch mathematische Beschreibung lassen sich zu verschiedenen Zwecken Daten gewinnen.}
\begin{columns}[t] 
     \begin{column}[T]{6cm} 
     	\begin{itemize}
		\only<1-3>{\item Zuverl\"assigkeitsfunktion $R(t)$
		\begin{itemize}
		\item Anteil funktionsf\"ahiger Komponenten zum Zeitpunkt $t$
		\end{itemize}
		\item Hazard rate $h$
		\begin{itemize}
		\item Anteil ausfallender Komponenten zum Zeitpunkt $t$
		\end{itemize}}
     		\only<2-2>{\item Konstante Fehlerrate:
		\begin{itemize}
			\item Exponentialverteilung
			\begin{eqnarray*}
			R(t) &=& e^{-\left(\lambda t \right)}\\
			h &=& \lambda
			\end{eqnarray*}
		\end{itemize}}
		\only<3-3>{\item Zeitabh\"angige Fehlerraten:
		\begin{itemize}
			\item (Misch-)Weibull-Verteilung
			\begin{eqnarray*}
			R(t) &=& e^{-\left(\lambda t \right)^{\beta}}\\
			h&=& \beta \lambda \left(\lambda t \right)^{\beta-1}
			\end{eqnarray*}
			\end{itemize}}
     	\end{itemize}
     \end{column}
     	\begin{column}[T]{5cm} 
         	\begin{center}
			% Pgf plots
            		\begin{tikzpicture}
				\begin{axis}[no marks,
				xmax=14, xmin = 0,
				 ymin = 0, ymax = 1, grid=both,
				%axis lines=middle, 
				%ylabel = {$p$},
				xlabel = {$t/\mathrm{a}$},
				legend entries={\small Survival rate $\bar{F}$, \small Hazard rate $h$, PDF $f$},
				legend style = {at={(0.5,1.28)},
      				  anchor=north}]
				
				\only<1-2>{\addplot[thick, draw = blue!70!black] table[x=t,y=Fq, %row sep=crcr
				] {Exponential.dat}; 
  
  				\addplot[thick, draw = red!70!black] table[x=t,y=h, %row sep=crcr
				] {Exponential.dat}; }

 				 \only<3-3>{\addplot[thick, draw = blue!70!black] table[x=t,y=Fq, %row sep=crcr
				 ] {Weibull.dat}; 
  
  				\addplot[thick, draw = red!70!black] table[x=t,y=h, %row sep=crcr
				] {Weibull.dat}; }
  
  %\addplot[thick, draw = green!70!black] table[x=t,y=f, row sep=crcr] {Weibull.dat}; 
  
\end{axis}
\end{tikzpicture}

        		\end{center}
     \end{column}
 \end{columns}
}

\frame{\frametitle{Zuverl\"assigkeitswerte bei bekanntem Ausfallverhalten}
\framesubtitle{Definition von Vergleichsgr\"o{\ss}en}
\begin{itemize}
\item Mittlere Lebensdauer $T$:
\begin{itemize}
	\item Erwartungswert der Zeit bis zum Ausfall
\end{itemize}
\item MTBF (Mean Time Between Failures) (auch MDBF)
\begin{itemize}
	\item Erwartungswert der Zeit zwischen zwei Ausf\"allen
\end{itemize}
\end{itemize}
}

%\offslide{Herleitung mittlere Lebensdauer und MTBF}

%\offslide{\small Fehlerrate und MTBF f\"ur Systeme aus mehreren Komponenten}

\frame{\frametitle{Beispielsystem f\"ur Zuverl\"assigkeitsanalyse}
\framesubtitle{}
		\begin{itemize}
		\item Elektronische F\"uhrerbremsventilanlage
		\begin{itemize}
		\item Bremssteuerger\"at (BSG) $R_{B}(t), \lambda_{B}$
		\item F\"uhrerbremsventil (FbrV) $R_{F}(t), \lambda_{F}$
		\item Elektropneumatische R\"uckfallebene (BU) $R_{R}(t), \lambda_{R}$
		\end{itemize}
		\item Aufgabe: Umwandlung Sollwert (SW) in HL-Druck (HL)
		\end{itemize}
         	\begin{center}
            		\tikz 	{
			\node (sw) [rectangle, fill = gray!30, minimum width = 1.2cm] at (-1,0) {SW};
			\node[minimum width = 0cm] (a) at (0.5,1) {};
			\node (bsg) [rectangle, fill = gray!30, minimum width = 1.2cm] at (2,1) {BSG}; 
			\node (fbrv) [rectangle, fill = gray!30, minimum width = 1.2cm] at (4,1) {FbrV}; 
			\node (b) at (5.5,1) {};
			\node (bu) [rectangle, fill = gray!30, minimum width = 1.2cm] at (3,-1) {BU}; 
			\node (hl) [rectangle, fill = gray!30, minimum width = 1.2cm] at (7,0) {HL};
 			 %\graph { (sw) -> (bsg) -> (fbrv) -> (hl) 
			%(sw) -> (bu) -> (hl)};
			\draw[thick] (sw) -| (0.5,1);
			\draw[thick, ->] (0.5,1) -> (bsg);
			\draw[thick, ->] (bsg) -> (fbrv);
			\draw[thick]  (fbrv) -| (5.5,0);
			\draw[thick, ->] (5.5,0) -> (hl);
			\draw[thick] (sw) -| (0.5,-1);
			\draw[thick, ->] (0.5,-1) -> (bu);
			\draw[thick]  (bu) -| (5.5,0);
			}
        		\end{center}
}

%\offslide{Analytische Bestimmung der Zuverl\"assigkeit}

\subsection{Verfahren mittels Weibull-Verteilung und MCMC-Simulation}
\frame{\frametitle{Betriebszust\"ande der elektronischen F\"uhrerbremsventilanlage}
\framesubtitle{Die elektronische F\"uhrerbremsventilanlage kann verschiedene Betriebszust\"ande annehmen.}
\begin{columns}[t] 
     \begin{column}[T]{6cm} 
     	\begin{itemize}
     		\item $S_{0}$: Ausgeschaltet
		\begin{itemize}
		\item Ca. 3600 Betriebsstunden/Jahr
		\end{itemize}
		\item $S_{1}$: Normaler Betrieb (BSG und FbrV)
		\begin{itemize}
		\item  \"Ubergang zur Nutzung R\"uckfallebene: Wahrscheinlichkeit $p_{1}$
		\end{itemize}
		\item $S_{2}$: Betrieb mit R\"uckfallebene
		\begin{itemize}
		\item \"Ubergang zur St\"orung: Wahrscheinlichkeit $p_{2}$
		\end{itemize}
		\item $S_{3}$: St\"orung
		\begin{itemize}
		\item Fahrzeugst\"orung ``Stop on line''
		\end{itemize}
     	\end{itemize}
     \end{column}
     	\begin{column}[T]{6cm} 
         	\begin{center}
            		\only<1>{\begin{tikzpicture}[->,>=stealth',shorten >=1pt,auto,node distance=2.8cm,
                    semithick]
 		 \tikzstyle{every state}=[fill=gray!30,draw=none]

 		 \node[state] (s0)    at (0,0)                {$S_{0}$};
 		 \node[state]         (s1)  at (-2,-2) {$S_{1}$};
		  \node[state]         (s2) at (2,-2) {$S_{2}$};
		  \node[state]         (s3) at (0,-4)     {$S_{3}$};

 		 \path (s0) edge [bend left]  node {$p_{01}$} (s1)
		 edge [loop right] node{$p_{00}$} (s0)
           	(s1) edge [bend right]             node {$p_{12}$} (s2)
		edge [bend left] node{$p_{10}$} (s0)
		 edge [loop below] node{$p_{11}$} (s1)
		%(s2) edge              node {$p_{3}$} (s0)
        		(s2) edge              node {$p_{23}$} (s3)
		edge[loop above] node{$p_{22}$} (s1)
		edge[bend right] node{$p_{20}$} (s0)
		(s3) edge[loop left] node{$p_{33}$} (s3);
		\end{tikzpicture}}
		\only<2>{
		\begin{equation*}
		P = 
		\left( \begin{matrix}
 		p_{00} & p_{01} & 0 &0 \\
		p_{10} & p_{11} & p_{12} & 0 \\
		p_{20} & 0 & p_{22} & p_{23} \\
		0 & 0 & 0 & p_{33} \\
		\end{matrix} \right)
		\end{equation*}
		}
        		\end{center}
     \end{column}
 \end{columns}
}

\frame{\frametitle{Ausfallzeitpunkte Bremssteuerger\"at}
\framesubtitle{}
\begin{columns}[t] 
     \begin{column}[T]{6cm} 
     	\begin{itemize}
     		\item Annahme:
		\begin{itemize}
		\item Elektronikkomponente
		\item Weibullverteilt:
		\begin{itemize}
		\item $\beta = 5$
		\item $\lambda = 12$
		\end{itemize}
		\item Flottengr\"o{\ss}e $N = 100$
		\end{itemize}
		\item Ergebnis:
		\begin{itemize}
		\item $\hat{\beta} = 4.5$
		\item $\hat{\lambda} = 12.2$
		\end{itemize}
     	\end{itemize}
     \end{column}
     	\begin{column}[T]{6cm} 
	\vspace{-.8cm}
         	\begin{center}
            		\begin{tabular}{|c|c|}
		\hline
			Nr. & TTF \\ \hline
    			1 & 4.4 a \\ \hline  
    			2 & 5.6 a  \\ \hline
    			3 & 5.7 a \\ \hline  
   			4 & 5.7 a \\ \hline 
    			5 & 6.3 a \\ \hline 
    			6 & 6.3 a \\ \hline 
    			7 & 6.7 a \\ \hline 
    			8 & 6.9 a \\ \hline 
    			9 & 7.1 a \\ \hline 
    			10 & 7.6 a \\ \hline 
    			11 & 7.7 a \\ \hline
    			12 & 7.8 a \\ \hline 
    			13 & 7.8 a \\ \hline 
    			14 & 7.9 a \\ \hline 
    			16 & 8.0 a \\ \hline 
			\end{tabular}
        		\end{center}
     \end{column}
 \end{columns}
}

\frame{\frametitle{Ausfallzeitpunkte F\"uhrerbremsventilanlage}
\framesubtitle{}
\begin{columns}[t] 
     \begin{column}[T]{6cm} 
     	\begin{itemize}
     		\item Annahme:
		\begin{itemize}
		\item Elektropneumatik-Komponente
		\item Exponentialverteilt:
		\begin{itemize}
		\item $\lambda = 50$
		\end{itemize}
		\item Flottengr\"o{\ss}e $N = 100$
		\end{itemize}
		\item Ergebnis:
		\begin{itemize}
		\item $\hat{\lambda} = 55.6$
		\end{itemize}
     	\end{itemize}
     \end{column}
     	\begin{column}[T]{6cm} 
	%\vspace{-.8cm}
         	\begin{center}
            		\begin{tabular}{|c|c|}
			\hline
			Nr. & TTF \\ \hline
    			1 & 0.24 a \\ \hline  
    			2 & 0.30 a  \\ \hline
    			3 & 0.35 a \\ \hline  
   			4 & 0.53 a \\ \hline 
    			5 & 2.71 a \\ \hline 
    			6 & 3.08 a \\ \hline 
    			7 & 3.44 a \\ \hline 
    			8 & 3.46 a \\ \hline 
    			9 & 5.05 a \\ \hline 
    			10 & 7.44 a \\ \hline 
    			\end{tabular}
        		\end{center}
     \end{column}
 \end{columns}
}

\frame{\frametitle{Identifikation Verschleissdaten Bremssteuerger\"at}
\framesubtitle{}
\begin{center}
 \pgfplotsset{width=12cm, height = 7cm}
\begin{tikzpicture}
\begin{axis}[no marks,
xmax=15, xmin = 0,
 ymin = 0, ymax = 1, grid=both,
%axis lines=middle, 
%ylabel = {$p$},
xlabel = {$t/\mathrm{a}$},
%legend entries={$\bar{F}$, $h$, $\bar{F}_{\text{sim}}$, $\hat{F}$, $\hat{h}$}, 
legend style = {at={(0.05,0.5)},
        anchor=west}]

  \only<1,5>{\addplot[draw = blue!70!black] table[x=t,y=Fq, %row sep=crcr
  ] {BSG.dat}; \addlegendentry{$\bar{F}$}

 \addplot[draw = orange!70!black] table[x=t,y=h, %row sep=crcr
 ] {BSG.dat}; 
 \addlegendentry{$h$}
 }

\only<2-4>{ \addplot[thick, draw = red!70!black] table[x=t,y=Fqest, %row sep=crcr
] {BSG2.dat}; \addlegendentry{$\bar{F}_{\text{sim}}$}}
    
  
\only<3-5>{  \addplot[thick, draw = green!70!black] table[x=t,y=Fqsim,% row sep=crcr
] {BSG.dat}; \addlegendentry{$\hat{F}$}}
  
  
 \only<4-5>{ \addplot[thick, draw = gray] table[x=t,y=hest, %row sep=crcr
 ] {BSG.dat}
 coordinate [pos=0.53] (A)
   coordinate [pos=0.6]  (B)
 ; \addlegendentry{$\hat{h}$}
 \fill[fill = gray!70, opacity = 0.8] (A) -| (B)
 node [pos=0.75,anchor=west]
     {$\triangle h = 0.048$};}
 \end{axis}
\end{tikzpicture}
%%
\end{center}
}



\frame{\frametitle{Identifikation Verschleissdaten F\"uhrerbremsventilanlage}
\framesubtitle{}
\begin{center}
 \pgfplotsset{width=12cm, height = 7cm}
\begin{tikzpicture}
\begin{axis}[no marks,
xmax=15, xmin = 0,
 ymin = 0, ymax = 1, grid=both,
%axis lines=middle, 
%ylabel = {$p$},
xlabel = {$t/\mathrm{a}$},
%legend entries={$\bar{F}$, $h$, $\bar{F}_{\text{sim}}$, $\hat{F}$, $\hat{h}$}, 
legend style = {at={(0.05,0.5)},
        anchor=west}]

  \only<1,5>{\addplot[draw = blue!70!black] table[x=t,y=Fq, %row sep=crcr
  ] {FbrVA.dat}; \addlegendentry{$\bar{F}$}

 \addplot[draw = orange!70!black] table[x=t,y=h, %row sep=crcr
 ] {FbrVA.dat}; 
 \addlegendentry{$h$}
 }

\only<2-4>{ \addplot[thick, draw = red!70!black] table[x=t,y=Fqest, %row sep=crcr
] {FbrVA2.dat}; \addlegendentry{$\bar{F}_{\text{sim}}$}}
    
  
\only<3-5>{  \addplot[thick, draw = green!70!black] table[x=t,y=Fqsim, %row sep=crcr
] {FbrVA.dat}; \addlegendentry{$\hat{F}$}}
  
  
 \only<4-5>{ \addplot[thick, draw = gray] table[x=t,y=hest, %row sep=crcr
 ] {FbrVA.dat}
 coordinate [pos=0.53] (A)
   coordinate [pos=0.6]  (B)
 ; \addlegendentry{$\hat{h}$}}
 \end{axis}
\end{tikzpicture}
%%
\end{center}
}

\frame{\frametitle{Simulationsergebnisse}
\framesubtitle{Die wiederholte Simulation der Markov-Kette eines Betriebstages liefert die Wahrscheinlichkeitsdichte der Betriebszust\"ande.}
\begin{columns}[t] 
     \begin{column}[T]{6cm} 
     	\begin{itemize}
     		\item Markov-Chain-Monte-Carlo-Simulation
		\begin{itemize}
		\item Wiederholte Ausf\"uhrung einer Markov Kette
		\item $N = 10^6$ Betriebstage
		\end{itemize}
		\item Zust\"ande:
		\begin{itemize}
		\item $S_{0}$: 39{,}8\%
		\item $S_{1}$: 60{,}2\%
		\item $S_{2}$: 0{,}014\%
		\item $S_{3}$: 0\%
		\end{itemize}
     	\end{itemize}
     \end{column}
     	\begin{column}[T]{6cm} 
         	\begin{center}
            		\begin{tikzpicture}[->,>=stealth',shorten >=1pt,auto,node distance=2.8cm,
                    semithick]
 		 \tikzstyle{every state}=[fill=gray!30,draw=none]

 		 \node[state] (s0)    at (0,0)                {$S_{0}$};
 		 \node[state]         (s1)  at (-2,-2) {$S_{1}$};
		  \node[state]         (s2) at (2,-2) {$S_{2}$};
		  \node[state]         (s3) at (0,-4)     {$S_{3}$};
		  
		  \node [red!70!black,above] at (s0.north) {39{,}8\%};
		  \node [red!70!black,below] at (s1.south) {60{,}2\%};
		  \node [red!70!black,below] at (s2.south) {0{,}014\%};
		  \node [red!70!black,below] at (s3.south) {0\%}; 

 		 \path (s0) edge [bend left]  (s1)
		 edge [loop right]  (s0)
           	(s1) edge [bend right]  (s2)
		edge [bend left]  (s0)
		 edge [loop above]  (s1)
        		(s2) edge (s3)
		edge[loop above]  (s1)
		edge[bend right] (s0)
		(s3) edge[loop left]  (s3);
		\end{tikzpicture}
        		\end{center}
     \end{column}
 \end{columns}
}

\frame{\frametitle{Nutzung der Zuverl\"assigkeitsdaten}
\framesubtitle{Erkenntnisse \"uber das Lebensdauerverhalten lassen sich zur B\"undelung von Aktivit\"aten oder zur Fristverl\"angerung nutzen.}
\begin{columns}[t] 
     \begin{column}[T]{6cm} 
     	\begin{itemize}
     		\item B\"undelung von Aktivit\"aten
		\begin{itemize}
		\item Vorhersage \"uber Ausfallwahrscheinlichkeit im n\"achsten Intervall
		\item Reduzierung von reaktiven T\"atigkeiten
		\end{itemize}
		\item Verl\"angerung von Fristen
		\begin{itemize}
		\item Eskalation \"uber Instandhaltungsentwicklung
		\item Wirtschaftlichkeitspr\"ufung
		\item \"Anderung Wartungsanweisung
		\end{itemize}
		\item Vorhersage Ersatzteilbedarf
		\item Unterst\"utzung Fehlerdiagnose
     	\end{itemize}
     \end{column}
     	\begin{column}[T]{6cm} 
         	\begin{center}
		\tikzstyle{decision} = [diamond, fill=blue!20, 
	    text width=6em, text badly centered, inner sep=0pt, yshift = 1.5cm]
                    \tikzstyle{block} = [rectangle, fill=blue!20, 
                        text width=8em, text centered, rounded corners, minimum height=4em, yshift = 1.5cm]
                    \tikzstyle{line} = [thick, draw, -latex']
                    \tikzstyle{cloud} = [fill=red!20,  text width=8em, node distance=5cm, text centered,
                        minimum height=2em]
                     \vspace{-1cm}
                     \resizebox{!}{7cm}{   
                    \begin{tikzpicture}[node distance =  4cm, auto]%[node distance = 2cm,auto]
                        % Place nodes
                        \node [block] (init) {Komponente w\"ahlen};
                        \node [cloud, right of=init] (expert) {Expertenwissen};
                        \node [decision, below of=init] (decide) {Optimierung};
                        \node [block, below of = decide] (daten1) {Lebensdauerkurve identifizieren};
                        \node [block, below of = daten1] (predict1) {Vorhersage Zuverl\"assigkeit};
                        \node [block, below of = predict1] (decide2) {Entscheidung Eingriff};
                        
                       \node [block, right of = daten1] (daten2) {Lebensdauerkurve identifizieren};
                       \node [block, below of = daten2] (predict2) {Vorhersage Risiko};
                       \node [block, below of = predict2] (assess) {Bewertung Risiko};
                       \node [block, below of = assess] (do2) {Umsetzung};
                       
                        % Draw edges
                        \path [line] (init) -- (decide);
                        \path [line,dashed] (expert) -- (init);
                        \path [line] (decide) -| node [above, near start] {Verl\"angerung} (daten2);
                        \path [line] (decide) -- node {B\"undelung} (daten1);
                         \path [line] (daten1) -- (predict1);
                          \path [line] (predict1) -- (decide2);
                          
                           \path [line] (daten2) -- (predict2);
                            \path [line] (predict2) -- (assess);
                             \path [line] (assess) -- (do2);
                    \end{tikzpicture}}
        		\end{center}
     \end{column}
 \end{columns}
}

%\subsection{Importance Sampling}
%\frame{\frametitle{Importance sampling}
%\framesubtitle{Importance sampling (IS) increases the probability of ``desired'' outcomes in Monte-Carlo-Simulations.}
%\begin{itemize}
%\item Typical IS approaches:
%\begin{itemize}
%		\item Stratification: select only relevant strata of the sampling range
%		\item Scaling: Scale random variable
%		\item Translation: Move random variable to more relevant part of sampling space
%		\item Change of random variable: Replace random variable by one more likely to produce outcomes in the relevant range
%		\item Adaptive approaches
%		\end{itemize}
%\item Effect: higher number of samples in region of interest
%\item Correction factor: Likelihood ratio $L(y) = \frac{f(y)}{\tilde{f}(y)}$
%\end{itemize}
%}
%
%\frame{\frametitle{Example: Application of IS to braking curves}
%\framesubtitle{Change identified random variables, in the case at hand $\mu_{B}$}
%\begin{center}
%\includegraphics[width = .9\textwidth]{hist}
%\end{center}
%}
%
%\frame{\frametitle{Example: Application of IS to braking curves}
%\framesubtitle{Step 3: Analyse for rare events, here braking distances in excess of 1100 m. $N = 5 \cdot 10^7$}
%\only<1>{\begin{center}
%\includegraphics[width = .9\textwidth]{rare}
%\end{center}}
%\only<2>{
%%\hspace{-.62cm}
%\begin{tabular}{|c|c|c|c|c|c|c|}
%\hline
%$s$ & $n_{\mathcal{U}}$ & $p_{\mathcal{U}}$ & $n_{IS, 1}$ & $p_{IS, 1}$ & $n_{IS, 2}$ & $p_{IS, 2}$ \\ \hline
%1000 & 24400 & 4.89$\cdot 10^{-3}$ & 2.27$\cdot 10^6$ & 1.14$\cdot 10^{-2}$ & 3.11$\cdot 10^5$ & 1.77$\cdot 10^{-3}$ \\ \hline
%1050 & 2 & 4$\cdot 10^{-7}$ & 6.66$\cdot 10^4$ & 2.02$\cdot 10^{-4}$ & 1.48$\cdot 10^4$ & 1.59$\cdot 10^{-4}$ \\ \hline
%1100 & 0 & 0 & 115 & 2.04$\cdot 10^{-7}$ & 419 & 7.50$\cdot 10^{-6}$ \\ \hline
%1150 & 0 & 0 & 0 & 0 & 15 & 3.88$\cdot 10^{-7}$ \\ \hline
%1160 & 0 & 0 & 0 & 0 & 7 & 2.05$\cdot 10^{-7}$ \\ \hline
%1170 & 0 & 0 & 0 & 0 & 5 & 1.46$\cdot 10^{-7}$ \\ \hline
%1180 & 0 & 0 & 0 & 0 & 4 & 1.16$\cdot 10^{-7}$ \\ \hline
%1190 & 0 & 0 & 0 & 0 & 1 & 2.90$\cdot 10^{-8}$ \\ \hline
%\end{tabular}
%}
%}


\subsection{Weibull-Verteilung und Analyse}
\frame{\frametitle{Formulierung der Weibull-Verteilung}
\framesubtitle{}
\begin{columns}[t] 
     \begin{column}[T]{6cm} 
     \textbf{Fr\"uhausf\"alle}
     	\begin{itemize}
     		\item $R(t) = e^{- \left( \frac{t}{\eta}\right)^\beta}$
		\item $\lambda(t) = \frac{\beta}{\eta} \left( \frac{t}{\eta} \right)^{\beta-1}$
		\item $T = \eta \Gamma\left( 1 + \frac{1}{\beta} \right) $
		\item $\beta > 1$, $t > 0$
     	\end{itemize}
     \end{column}
     	\begin{column}[T]{6cm} 
	\textbf{Alterung}
         \begin{itemize}
     		\item $R(t) = e^{- \left( \frac{t-t_{2}}{\eta}\right)^\beta}$
		\item $\lambda(t) = \frac{\beta}{\eta} \left( \frac{t - t_{2}}{\eta} \right)^{\beta-1}$
		\item $T = \eta \Gamma\left( 1 + \frac{1}{\beta} \right)  $
		\item $\beta > 1$, $t > t_{2}$
     	\end{itemize}
     \end{column}
 \end{columns}
}

%\begin{frame}[fragile]
%\frametitle{Weibull-Plot}
%\framesubtitle{}
%\begin{center}
%\begin{tikzpicture}
%\begin{axis}[scale only axis,
%        only marks,
%         xmin=10,xmax = 1000,
%         ymin=0.001, ymax=0.999,
%         grid=major,
%         xtick={10, 20, 50, 100, 200, 500, 1000},
%         ytick={0.001,0.05,0.4,0.99},
%         yticklabels={0.1,5,40,99},
%        xmode=log,
%        log ticks with fixed point,
%y coord trafo/.code=\pgfmathparse{ln(-ln(1-#1)))},
%y coord inv trafo/.code=\pgfmathparse{exp(-exp(-#1-1))}
%         ]
%%
%\addplot [domain=10:1000, sharp plot] {1-exp(-(x/820)^1.96)};
%\addplot [mark=o, mark size=1.5, black] table {
%96  0.02
%147 0.04
%202 0.0604
%232 0.0813
%277 0.1022
%278 0.1231
%325 0.1439
%346 0.1648
%354 0.1857
%367 0.2066
%414 0.228
%440 0.2501
%471 0.2721
%525 0.2942
%541 0.3163
%542 0.3383
%574 0.3604
%608 0.3824
%631 0.4045
%643 0.4265
%661 0.4504
%669 0.4743
%706 0.4982
%720 0.5221
%722 0.546
%738 0.5712
%747 0.598
%770 0.6248
%802 0.6537
%841 0.6825
%850 0.7114
%884 0.7403
%893 0.7691
%901 0.798
%};
%\end{axis} 
%\end{tikzpicture}
%
%\end{center}
%\end{frame}


%\section{Mechatronik}
%%Elektrische Komponenten
%% !TEX root = ../SFV15001_MdQM_Fertigungsmesstechnik_Rev03.tex
% \section{Prüfdatenerfassung und -auswertung elektrischer Komponenten}
\label{Sec:Elektrisch}

\frame{\frametitle{Motivation}
\framesubtitle{Warum elektrische Komponenten in MdQM aufnehmen?}
\begin{itemize}
\item Zunahme mechatronischer Komponenten
	\begin{itemize}
		\item Integrationsleistung steigt
		\item Engineering/Fertigung h\"aufig extern
		\item Aber Qualit\"atsmanagement?
	\end{itemize}
\item Fehlereingrenzung
	\begin{itemize}
		\item H\"aufig im elektrischen Subsystem
	\end{itemize}
\item Indirekte Messung
	\begin{itemize}
		\item Kraft, Weg \"uber Widerstand
		\item Schichtdicke \"uber Induktivit\"at
		\item ...
	\end{itemize}
\end{itemize}
}


\subsection{Eigenschaften elektrischer Messger\"ate}
\frame{\subsectionpage}

\frame{\frametitle{Statisches Verhalten}
\framesubtitle{}
\begin{columns}[t] 
     \begin{column}[T]{6cm} 
     	\begin{itemize}
     		\item Statischer Zustand:
		\begin{itemize}
		\item Eingangsgr\"o{\ss}e konstant
		\item Ausgangsgr\"o{\ss}e eingeschwungen
		\end{itemize}
		\item Kennlinie:
		\begin{itemize}
		\item Eindeutige Zuordung Eingang zu Ausgang
		\item Ideal: Linear
		\item Real: Nichtlinear
		\end{itemize}
		\item Empfindlichkeit:
		\begin{itemize}
		\item $E = \frac{\delta x_{a}}{\delta x_{a}} \approx \frac{x_{a}}{x_{e}}$
		\end{itemize}
     	\end{itemize}
     \end{column}
     	\begin{column}[T]{6cm} 
         	\begin{center}	
		\begin{tikzpicture}
 			\node[rectangle, draw, thick, align= center] (m) at (2,5) {Mess-\\ger\"at};
			\draw[->, thick] (0,5) -- (m) node[pos = 0.2, above] {$x_{e}$};
			\draw[->, thick] (m) -- (4,5)node[pos = 0.8, above] {$x_{a}$};
			
			\begin{axis}[
			axis lines=left,
			xmin = 0, xmax = 1, ymin = 0, ymax = 1,
			ytick = \empty, xtick = \empty,
			xlabel = $x_{e}$, ylabel = $x_{a}$,
			every axis x label/.style={
  				  at={(ticklabel* cs:1.05)},
 				   anchor=west,
				},
					every axis y label/.style={
				    at={(ticklabel* cs:1.05)},
				    anchor=south,
				},]
                            \addplot[smooth, blue] coordinates {
                                (0,   0)
                                (0.7,  0.75)
                                (0.9,  0.85)
                                (1,  0.87)
                                };
				\addplot[smooth, red] coordinates {
                                (0,   0)
                                (1,  1)
                                };
                            \end{axis}
		\end{tikzpicture}
        		\end{center}
     \end{column}
 \end{columns}
}

\frame{\frametitle{Anzeige- und Messbereich}
\framesubtitle{}
     	\begin{itemize}
     		\item Anzeigebereich
		\begin{itemize}
		\item Werte werden angezeigt
		\end{itemize}
		\item Messbereich
		\begin{itemize}
		\item Werte werden erfasst
		\item Genauigkeitsangabe gilt
		\end{itemize}
		\item Aufl\"osung
		\begin{itemize}
		\item Kleinste darstellbare \"Anderung von $x_{a}$
		\item Nicht mit Genauigkeit verwechseln!
		\end{itemize}
     	\end{itemize}
}

\frame{\frametitle{Dynamisches Verhalten}
\framesubtitle{}
\begin{columns}[t] 
     \begin{column}[T]{6cm} 
     	\begin{itemize}
     		\item Messeinrichtung folgt nicht beliebig schnell der Eingangsgr\"o{\ss}e
		\item Dynamische Messabweichung:
		\begin{equation*}
			e_{dyn} = x(t) - x_{w}(t)
		\end{equation*}
		\item Einschwingzeit
		\begin{itemize}
		\item Zeit, die nach Sprung auf $x_{e}$ gewartet werden muss
		\end{itemize}
     	\end{itemize}
     \end{column}
     	\begin{column}[T]{6cm} 
         	\begin{center}
            		\begin{tikzpicture}
 			\node[rectangle, draw, thick, align= center] (m) at (2,5) {Mess-\\ger\"at};
			\draw[->, thick] (0,5) -- (m) node[pos = 0.2, above] {$x_{e}$};
			\draw[->, thick] (m) -- (4,5)node[pos = 0.8, above] {$x_{a}$};
			
			\begin{axis}[
			axis lines=left,
			%xmin = 0, xmax = 1, ymin = 0, ymax = 1,
			ytick = \empty, xtick = \empty,
			xlabel = $t$, ylabel = $x$,
			legend entries = {$x_{e}$, $x_{a}$},
			every axis x label/.style={
  				  at={(ticklabel* cs:1.05)},
 				   anchor=west,
				},
					every axis y label/.style={
				    at={(ticklabel* cs:1.05)},
				    anchor=south,
				},
				every axis legend/.append style={
        				at={(0.7,0.8)},
        				anchor=south west}]
				  \addplot[thick, draw = blue!70!black] table[x=t,y=u, 
				  ] {Output.dat}; 
  
				  \addplot[thick, draw = red!70!black] table[x=t,y=y, 
				  ] {Output.dat}; 
    
\end{axis}
		\end{tikzpicture}
        		\end{center}
     \end{column}
 \end{columns}
}

\frame{\frametitle{Genauigkeitsangaben elektrischer Messger\"ate}
\framesubtitle{}
\begin{itemize}
\item Fehlergrenze:
\begin{itemize}
		\item Maximal zul\"assige Messabweichung $G = \left\| e \right\|_{max} = \left\| x-x_{w} \right\|_{max}$
		\item Interpretation: F\"ur Messwert $x_{m}$ gilt: $x_{w} \in \left[ x_{w}-G; x_{w} + G \right]$
		\end{itemize}
		\item Eigenabweichung:
		\begin{itemize}
		\item Grunds\"atzliche Messabweichung unter Normbedingungen, an einem Punkt im Messbereich
		\end{itemize}
		\item Betriebsmessabweichung:
		\begin{itemize}
		\item Im Betrieb erzielte Messabweichung, unter tats\"achlichen Bedingungen
		\end{itemize}
		\item Vereinbart: Abweichungen mit $U_{95\%}$-Konfidenz
		\item Angabe Fehlergrenzen:
		\begin{itemize}
		\item Absolut: $\pm 1 \mathrm{V}$
		\item Relativ: $\pm 0{,}5 \%$ des Anzeigewerts
		\item Kombiniert: $\pm\left(0{,}5 \%+ 1 \mathrm{V}\right)$
		\end{itemize}
\end{itemize}
}

\frame{\frametitle{Grunds\"atzliche Bauarten}
\framesubtitle{}
\begin{columns}[t] 
     \begin{column}[T]{6cm} 
     	\begin{itemize}
     		\item Elektromechanisch:
		\begin{itemize}
		\item Drehspulmesswerk
		\item Dreheisenmesswerk
		\item Elektrodynamisches Messwerk
		\item Elektrostatisches Messwerk
		\item und weitere
		\end{itemize}
		\item Digitale Messger\"ate
     	\end{itemize}
     \end{column}
     	\begin{column}[T]{6cm} 
         	\begin{center}
            		\includegraphics[width=0.5\textwidth]{Drehspul}\source{Quelle: wikimedia/S\o ren Peo Pedersen}
        		\end{center}
     \end{column}
 \end{columns}
}

\frame{\frametitle{Effekte digitaler Messger\"ate}
\framesubtitle{}
\begin{columns}[t] 
     \begin{column}[T]{6cm} 
     	\begin{itemize}
     		\item Abtastung:
		\begin{itemize}
		\item Zeitdiskretion der Anzeigewerte
		\item Abtastrate $f_{a} = \frac{1}{T_{a}}$
		\item Sampling-Theorem (Shannon-Nyquist): 
		\begin{itemize}
		\item $f_{a} > 2 f_{max}$
		\item i.d.R. h\"oher
		\end{itemize}
		\end{itemize}
		\item Quantisierung:
		\begin{itemize}
		\item Umwandlung in Wert mit endlicher Aufl\"osung
		\item F\"ur $N$ Bit:
			$\triangle U = \frac{U_{max}}{2^N-1}$
		\item Abweichung durch Quantisierung $\pm \frac{\triangle U}{2}$
		\end{itemize}
     	\end{itemize}
     \end{column}
     	\begin{column}[T]{6cm} 
         	\begin{center}
            		\only<1>{\includegraphics[width=0.6\textwidth]{Multimeter}\source{Quelle: wikimedia/Harke}}
			\only<2>{\includegraphics[width=0.75\textwidth]{ShannonNyquist}\source{Quelle: wikimedia/Peterpall}}
        		\end{center}
     \end{column}
 \end{columns}
}

\subsubsection{Strom- und Spannungsmessung}
\frame{\subsectionpage}
\offslide{Gleichstrommessung}

\offslide{Gleichspannungsmessung}

\subsubsection{Messung Ohmscher Widerst\"ande}
\frame{\subsectionpage}

\frame{\frametitle{Stromrichtige Messung}
\framesubtitle{}
\begin{columns}[t] 
     \begin{column}[T]{6cm} 
     	\begin{equation*}
		R = \frac{U}{I} = \frac{I \left(R_{x} + R_{I} \right)}{I} = R_{X} + R_{I}
	\end{equation*}
	\begin{equation*}
		R_{korr} = \frac{U}{I} - R_{I}
	\end{equation*}
     \end{column}
     	\begin{column}[T]{6cm} 
         	\begin{center}
            		\includegraphics[width=0.9\textwidth]{Stromrichtig}\source{}
        		\end{center}
     \end{column}
 \end{columns}
}

\frame{\frametitle{Spannungsrichtige Messung}
\framesubtitle{}
\begin{columns}[t] 
     \begin{column}[T]{6cm} 
     	\begin{equation*}
		R = \frac{U}{I} = R_{X} \| R_{U} = \frac{R_{X} R_{U}}{R_{X}+R_{U}}
	\end{equation*}
	\begin{equation*}
		R_{korr} = \frac{U}{I - \frac{U}{R_{U}}}
	\end{equation*}
     \end{column}
     	\begin{column}[T]{6cm} 
         	\begin{center}
            		\includegraphics[width=0.9\textwidth]{Spannungsrichtig}\source{}
        		\end{center}
     \end{column}
 \end{columns}
}

\frame{\frametitle{Wheatstone-Messbr\"ucke}
\framesubtitle{}
\begin{columns}[t] 
     \begin{column}[T]{6cm} 
     Spannungsteiler:
     	\begin{equation*}
		U_{1} = U_{0}\frac{R_{1}}{R_{1} + R_{2}}\quad U_{3} = U_{0}\frac{R_{3}}{R_{3} + R_{4}}
	\end{equation*}
	Maschenumlauf oben:
	\begin{equation*}
	\begin{split}
	U_{B} &= U_{3} - U_{1} = U_{0}\left(\frac{R_{3}}{R_{3} + R_{4}} - \frac{R_{1}}{R_{1} + R_{2}} \right)\\
	&= U_{0} \left(\frac{R_{2}R_{3} - R_{1}R_{4}}{\left(R_{1} + R_{2} \right)\left(R_{3} + R_{4} \right)} \right)
	\end{split}
	\end{equation*}
	Abgleich: $U_{B} = 0$
	\begin{equation*}
	R_{2} = R_{1}\frac{R_{4}}{R_{3}}
	\end{equation*}
     \end{column}
     	\begin{column}[T]{6cm} 
         	\begin{center}
            		\includegraphics[width=0.9\textwidth]{Wheatstone}\source{}
        		\end{center}
     \end{column}
 \end{columns}
}

%%Software-Qualit\"at
%% !TEX root = ../SFV15001_MdQM_Fertigungsmesstechnik_Rev05.tex
 \subsection{Software-Qualität}
\label{Sec:Software}
\subsubsection{Einf\"uhrung}
\frame{\subsectionpage}
\frame{\frametitle{Motivation}
\framesubtitle{Zunehmende Allgegenw\"artigkeit von Computersystemen in Alltag und Industrie erh\"oht die Bedeutung von qualitativ hochwertiger Software.}
\begin{itemize}
\item Trends:
\begin{itemize}
	\item Industrie 4.0 / Internet of Things: 
	\begin{itemize}
		\item Selbststeuerung
		\item M2M-Kommunikation
		\item Embedded Systems
	\end{itemize}
	\item Zunehmende Wertsch\"opfung durch Software:
	\begin{itemize}
		\item Siemens einer der gr\"o{\ss}ten Software-Hersteller der Welt (nach Code-Menge)
		\item H\"aufig wird Software-Anteil im Maschinen- und Anlagenbau vorausgesetzt
		\item Aufstieg zur h\"aufigsten Fehlerursache
	\end{itemize}
	\item Code-Gr\"o{\ss}e je SW-Produkt steigt exponentiell	
\end{itemize}
	\item Software-Engineering junge Ingenieursdisziplin
	\begin{itemize}
		\item Detail-Spezifikationen n\"otig
		\item Experimentierfreudige Entwickler
		\item Wachsende Teamgr\"o{\ss}en
	\end{itemize}
\end{itemize}
}

\frame{\frametitle{Defintion und Aspekte von Software-Qualit\"at}
\framesubtitle{}
\begin{definition}[Software-Qualit\"at]
Software-Qualit\"at ist die Gesamtheit der Merkmale und Merkmalswerte eines Software-Produkts, die sich auf die Eignung beziehen, festgelegt Erfordernisse zu erf\"ullen.
\end{definition}
\begin{itemize}
\item Funktionalit\"at \textit{Functionality, Capability}
\item Laufzeit \textit{Performance}
\item Zuverl\"assigkeit \textit{Reliability}
\item Benutzbarkeit \textit{Usability}
\item Wartbarkeit \textit{Maintainability}
\item Transparenz \textit{Transparency}
\item \"Ubertragbarkeit \textit{Portability}
\item Testbarkeit \textit{Testability}
\end{itemize}}

\frame{\frametitle{Gr\"unde f\"ur mangelnde SW-Qualit\"at}
\framesubtitle{}
\begin{itemize}
\item H\"andisch erstellter Code: Fehlerdichte $>0$
\item Komplexit\"atswachstum exponentiell
\item Spezialisierung der SW-Ingenieure in junger Disziplin
\begin{itemize}
	\item Vergleichbar mit fr\"uher Phase des Maschinenbaus
\end{itemize}
\item Dokumentation des Codes im Code oder im Mitarbeiter
\begin{itemize}
	\item Fluktuation f\"uhrt zu Problemen
\end{itemize}
\item Inkrementelles Wachstum des SW-Produkts
\begin{itemize}
	\item Kompromissl\"osungen f\"ur Abw\"artskompatibilit\"at n\"otig
\end{itemize}
\item Unzureichende Code-Metriken
\end{itemize}
}

\frame{\frametitle{Ma{\ss}nahmen zur Quali\"atssicherung von Software}
\framesubtitle{}
     	\begin{itemize}
     		\item Produktqualit\"at
		\begin{itemize}
		\item Konstruktive Qualit\"atssicherung
		\only<1>{
		\begin{itemize}
		\item Software-Richtlinien
		\item Typisierung
		\item Vertragsbasierte Programmierung
		\item Fehlertolerante Programmierung
		\item Portabilit\"at
		\item Dokumentation
		\end{itemize}}
		\item Analytische Qualit\"atssicherung
		\only<2>{
		\begin{itemize}
		\item Software-Test: Black-Box, White-Box, Test-Metriken
		\item Statistische Analyse: Software-Metriken, Konformit\"atspr\"ufung, Exploit-Analyse, Anomalienanalyse, Manuelle Software-Pr\"ufung
		\item Software-Verifikation
		\end{itemize}}
		\end{itemize}
		\item Prozessqualit\"at
		\only<3>{
		\begin{itemize}
		\item Software-Infrastruktur
		\begin{itemize}
		\item Konfigurationsmanagement
		\item Build-Automatisierung
		\item Test-Automatisierung
		\item Defektmanagement
		\end{itemize}
		\item Managementprozess
		\begin{itemize}
		\item Vorgehensmodelle
		\item Reifegradmodelle
		\end{itemize}
		\end{itemize}
		}
     	\end{itemize}
}

\frame{\frametitle{Arten von Software-Fehlern}
\framesubtitle{}
\begin{columns}[t] 
     \begin{column}[T]{6cm} 
     	\begin{itemize}
     		\item Lexikalische und syntaktische Fehlerquellen
		\only<1>{\begin{itemize}
		\item Nutzung reservierter Ausdr\"ucke
		\item Fehlerhafte Syntax
		\end{itemize}}
		\item Semantische Fehlerquellen
		\only<2>{\begin{itemize}
		\item Fehlerhafte Nutzung der Bedeutung der Befehle
		\item Auch bei Kommunikation, z.B. \"uber Bus
		\end{itemize}}
		\item Parallelit\"at
		\only<3>{\begin{itemize}
		\item Bei Multitasking scheduling der Prozesse abh\"angig von Last
		\end{itemize}}
		\item Numerische Fehlerquellen:
		\only<4>{\begin{itemize}
		\item \"Uberlauf
		\item Rundung
		\end{itemize}}
		\item Portabilit\"at
		\only<5>{\begin{itemize}
		\item HW-abh\"angiges Verhalten
		\end{itemize}}
		\item Spezifikationsfehler
		\only<6>{\begin{itemize}
		\item Gefordertes Verhalten fehlerhaft
		\end{itemize}}
     	\end{itemize}
     \end{column}
     	\begin{column}[T]{6cm} 
		\scriptsize
         	\lstinputlisting[language = Fortran]{Mercury.txt}
     \end{column}
 \end{columns}
}
\subsubsection{Beispiel konstruktiver Software-QS}
\frame{\subsectionpage}
\frame{\frametitle{Coding conventions}
\framesubtitle{Mittels coding conventions erh\"alt der Quellcode ein standardisiertes Erscheinungsbild.}
\begin{columns}[t] 
     \begin{column}[T]{6cm} 
     	\begin{itemize}
     		\item Definition:
		\begin{itemize}
		\item Darstellun Schachtelung
		\item Einr\"uckung
		\end{itemize}
		\item Einschr\"ankung Sprachkonstrukte
		\item etc.
		\item Beispiel: MISRA-C
		\begin{itemize}
		\item 127 Regeln
		\end{itemize}
		\item Nutzung automatischer Pr\"ufwerkzeuge
     	\end{itemize}
     \end{column}
     	\begin{column}[T]{6cm} 
         	\begin{center}
			\vspace{-1cm}
            		\scriptsize
         	\lstinputlisting[language = C]{CodingConventions.txt}
        		\end{center}
     \end{column}
 \end{columns}
}


\subsubsection{Analytische Qualit\"atssicherung}
\frame{\subsectionpage}

%\offslide{V-Modell in der Software-Entwicklung}

\offslide{Branch-Modell in der Software-Entwicklung}

\frame{\frametitle{Integrationsstrategien}
\framesubtitle{}
\begin{itemize}
\item Big-Bang-Integration: gleichzeitige Zusammenf\"uhrung aller Module
\item Strukturorientiert:
\begin{itemize}
		\item Bottom-Up: beginnend mit Basiskomponenten
		\item Top-Down: beginneng mit oberster SW-Schicht
		\item Outside-In: Integration gleichzeitig von Top und Bottom
		\item Inside-Out: Integration beginnend mit der Zwischenschicht
\end{itemize}
\item Funktionsgetrieben:
\begin{itemize}
		\item Termingetrieben
		\item Risikogetrieben
		\item Testgetrieben
		\item Anwendungsgetrieben
		\end{itemize}
\end{itemize}
}


\frame{\frametitle{Klassifizierung von Tests}
\framesubtitle{}
\begin{itemize}
\item Pr\"ufebene:
\begin{itemize}
		\item Unit-Test
		\item Integrationstest
		\item Systemtest
		\item Abnahmetest
		\end{itemize}
\item Pr\"ufkriterien:
	\begin{itemize}
		\item Funktionale Tests
		\item Operationale Tests
		\item Temporale Tests
		\end{itemize}
\item Pr\"uftechniken:
\begin{itemize}
		\item Black-Box-Test
		\item White-Box-Test
		\item Grey-Box-Test
		\end{itemize}
\end{itemize}
}

\frame{\frametitle{Tests nach Pr\"ufebene}
\framesubtitle{}
\begin{itemize}
\item Unit-Test:
	\begin{itemize}
		\item Kleinste testbare Einheit
		\item z.B. Funktion, Klasse, etc.
		\item Fokus auf Schnittstellen
	\end{itemize}
\item Integrationstest:
	\begin{itemize}
		\item Zusammenwirken der einzelnen Module wird gepr\"uft
		\item Abh\"angig von Integrationsstrategie
	\end{itemize}
\item Systemtest:
\begin{itemize}
		\item Nach vollst\"andiger Integration
		\item Funktionalit\"at nach Pflichtenheft
		\end{itemize}
\item Abnahmetest:
\begin{itemize}
		\item Unter Kundenbeteiligung
		\item Nach Kundenvorgaben
		\item In Einsatzumgebung
		\end{itemize}
\end{itemize}
}

\frame{\frametitle{Tests nach Pr\"ufkriterium}
\framesubtitle{}
\begin{itemize}
\item Funktionale Tests:
	\begin{itemize}
		\item Funktionstest
		\item Trivialtest
		\item Crashtest
		\item Kompatibilit\"atstests
		\item Zufallstests
	\end{itemize}
\item Operationale Tests:
\begin{itemize}
		\item Installationstests
		\item Ergonomietests
		\item Sicherheitstests
		\end{itemize}
\item Temporale Tests:
\begin{itemize}
		\item Komplexit\"atstests
		\item Laufzeittests
		\item Lasttests
		\item Stresstest (\"Uberlasttest)
		\end{itemize}
\end{itemize}
}

\frame[allowframebreaks]{
\frametitle{Black-Box-Testtechniken}
\framesubtitle{Testen nur durch Bewertung des Eingabe- und Ausgabeverhaltens.}
\begin{itemize}
\item \"Aquivalenzklassentest:
\begin{itemize}
	\item Testen aller Zahlenwerte zu aufwendig
	\item \"Aquivalenzklassen bilden Wertebereich ab, f\"ur die gleiches Verhalten erwartet wird
	\item Test mit Zufallszahl aus \"Aquivalenzklasse
	\item Bei mehrdimensionalen Funktionen Kombination der individuellen \"Aquivalenzklassen
	\item Einschlie{\ss}lich oder ausschlie{\ss}lich ung\"ultiger Kombinationen
\end{itemize}
\item Grenzwertbetrachtung:
\begin{itemize}
		\item Nutzung von unterem und oberen Randwert, je innerhalb und au{\ss}erhalb
		\item Komination f\"ur mehrdimensionale Funktionen
		\end{itemize}
		\newpage
\item Zustandsbasierter Softwaretest
\begin{itemize}
		\item F\"ur ged\"achtnisbehaftete Funktionen
		\item Nutzung endlicher Automaten
		\item Pr\"ufung jedes \"Ubergangs 
		\end{itemize}
\item Use-Case-Test: 
\begin{itemize}
		\item \"Uberpr\"ufung einzelner Anwendungsf\"alle
		\end{itemize}
\item Entscheidungstabellenbasierter Test
\begin{itemize}
		\item Pr\"ufung aller Kombination (insbesonderer diskrete) Werte
		\item Reduzierung der Tabelle m\"oglich
		\end{itemize}
\item Paarweises Testen:
\begin{itemize}
		\item Reduzierung auf Paare fehlerhafter Eingaben
		\end{itemize}
\item Diversifizierende Verfahren:
\begin{itemize}
		\item Back-to-Back-Test
		\item Regressionstest
		\end{itemize}
\end{itemize}
}


\frame{\frametitle{ \includegraphics[scale=0.01] {Off}\hspace{1.5mm} Entwicklung Kontrollflussgraph}
\framesubtitle{Entwicklung am Beispiel der Manhattan-Distanz (Taxi-Cab-Norm): $d\left( \left( x_1, y_1 \right), \left( x_2, y_2 \right) \right) = \left| x_1 - x_2 \right| + \left| y_1 - y_2 \right|$}
\begin{center}
         	\lstinputlisting[language = C]{Manhattan.txt}
\end{center}

}


\frame[allowframebreaks]{
\frametitle{White-Box-Testtechniken}
\framesubtitle{Testen unter Kenntnis der Quellodes der Software.}
\begin{itemize}
\item Kontrollflussorientierte Tests
\begin{itemize}
		\item Modellierung des Kontrolflusses \"uber gerichtete Graphen 
		\item Anweisungs\"uberdeckung ($C_{0}$-Test): alle Knoten des Kontrollflussgraphen einmal durchlaufen
		\item Zweig\"uberdeckung ($C_{1}$-Test): alle Kanten des Kontrollflussgraphen einmal durchlaufen
		\item Pfad\"uberdeckung: alle Pfade des Kontrollflussgraphen einmal durchlaufen
		\item Bedingungs\"uberdeckung:
		\begin{itemize}
		\item Einfach: alle atomaren Pr\"adikate m\"ussen beide Wahrheitswerte annehmen
		\item Minimal Mehrfach: alle atomaren und zusammengesetzten Pr\"adikate m\"ussen beide Wahrheitswerte annehmen
		\item Mehrfach: alle Kombinationen der atomaren Pr\"adikate wurden getestet
		\end{itemize}
		\item McCabe-\"Uberdeckung: Reduzierung auf Elementarpfade
		\end{itemize}
\item Datenflussorientierte Tests
\begin{itemize}
		\item Defs-Uses-\"Uberdeckung
		\item Required-$k$-Tupel
		\end{itemize}
\item \"Ublicher Ablauf:
\begin{itemize}
		\item Strukturanalyse: Erstellung Kontrollflussgraph
		\item Testkonstruktion
		\item Testdurchf\"uhrung
		\end{itemize}
\end{itemize}
}

\frame{\frametitle{Testmetriken}
\framesubtitle{}
\begin{itemize}
\item \"Uberdeckungsmetrik:
\begin{itemize}
		\item Verh\"altnis \"uberdeckter Knoten zu Anzahl Knoten
		\item Ab 80\% gilt Software als gut getestet
		\end{itemize}
\end{itemize}
}

\frame{\frametitle{Grenzen des Software-Tests}
\framesubtitle{}
\begin{itemize}
\item Unklare oder fehlende Anforderungen
\begin{itemize}
		\item Insbesondere bei funktionalen Tests
		\end{itemize}
\item Programmkomplexit\"at
\begin{itemize}
		\item Starkes Wachstum der n\"otigen Testf\"alle
		\end{itemize}
\item Fehlende Unterst\"utzung durch das Management
\begin{itemize}
		\item Fehlende Ressourcen oder Zeit
		\end{itemize}
\item Ausbildungs- und Fortbildungsdefizite
\begin{itemize}
		\item Thema Testing vernachl\"assigt
		\end{itemize}
\end{itemize}
}



% !TEX root = ../16_MdQM_Vorlesung RevC.tex
%\frame{\frametitle{Beispiel komplexe Komponente}
%\label{Sec:Grundlagen}
%\framesubtitle{}
%\begin{center}
%\includegraphics[width=.7\textwidth]{WSP} \source{ „189 038-3 C-AKv-Kupplung“ von Christian Liebscher}
%\end{center}
%}
\subsection{Recap Fertigungsmesstechnik}
\frame{\frametitle{Definition Fertigungsmesstechnik}
\framesubtitle{}
\begin{columns}[t] 
\begin{column}[T]{6cm} 
            		\begin{quotation}
				[...] Oberbegriff f\"ur alle mit Mess- und Pr\"ufaufgaben verbundenen T\"atigkeiten, die beim industriellen Entstehungsprozess eines Produktes zu erbringen sind. \cite[S. 1]{pfeifer10}
			\end{quotation} 
     \end{column}
     \begin{column}[T]{5cm} 
     	\begin{itemize}
	 
     		\item Im Entstehungsprozess
		 
		\item Messen und Pr\"ufen
		 
		\item Aspekte der Definition:
		\begin{itemize}
			\item Geringe Fertigungstiefe
			\item Automatisierung
			\item gestiegene Qualit\"atsforderungen
		\end{itemize}
		 
		\item Von Kontrollinstanz zu Komponente des QM
     	\end{itemize}
     \end{column}
     	 \end{columns}
}

\frame{\frametitle{Fr\"uhe Entwicklung und Rolle der Fertigungsmesstechnik}
\framesubtitle{}
\begin{itemize}
\item Messtechnik seit ca. 4000 v. Chr.: Vergleich mit nat\"urlichen Ma{\ss}en
\item Festlegung Urmeter 1799
\item Beschleunigt durch Austauschbau und Massenfertigung
\item Elektronische Messtechnik ca. seit 1970
\end{itemize}
\begin{columns}[t] 
     \begin{column}[T]{5cm} 
\begin{itemize}
\item Rolle der Messtechnik:
\begin{itemize}
\item 20er Jahre: Sortierung
\item 30er Jahre: Prozess\"uberwachung und Regelung
\item 80er Jahre: Planung zur Fehlervermeidung
\item 90er Jahre: Gesamtheitliches Qualit\"atsdenken
\end{itemize}
\end{itemize}
\end{column}
     	\begin{column}[T]{6cm} 
		\includegraphics[width=.9\textwidth]{Fehlerbehebung}
          \end{column}
 \end{columns}

}


\frame{\frametitle{Aufgaben und Ziele der Fertigungsmesstechnik}
\framesubtitle{}
\begin{itemize}
\item Erfassung von Qualit\"atsmerkmalen an Messobjekt
 
	\begin{itemize}
	\item Werkstoffeigenschaften (z.B. Gef\"uge, H\"arte) 
	\begin{itemize}
		\item Als Eingangspr\"ufung
		\item Nach thermischer Behandlung
		\item auch f\"ur Schwei{\ss}nahtg\"ute
	\end{itemize}
	\item Geometrie (z.B. Ma{\ss}, Form, Lage)  
	\begin{itemize}
		\item Dominierende Pr\"ufung
		\item Gestalt des Werkst\"ucks
		\item Oberfl\"acheneigenschaft
	\end{itemize}
	\item Funktion (z.B. Kraft, Geschwindigkeit)  
	\begin{itemize}
		\item Produkt oder Baugruppe
		\item Von Sichtpr\"ufung bis zur vollautomatisierten Funktionspr\"ufung
	\end{itemize}
	\end{itemize}
\end{itemize}
}

%\frame{\frametitle{Aufgaben im Entwicklungs- und Fertigungsprozess}
%\framesubtitle{}
%\begin{center}
%\includegraphics[width=.9\textwidth]{VModell}
%\end{center}
%}

%\frame{\frametitle{Geometrische Eigenschaften und Toleranzen}
%\framesubtitle{Das exakte Herstellen eines bestimmten Ma{\ss}es ist nicht m\"oglich.}
%\begin{columns}[t] 
%     \begin{column}[T]{6cm} 
%     	\begin{itemize}
%     		\item Abma{\ss}e
%		\item Form
%		\item Lage
%		\item Explizit oder \"uber Allgemeintoleranzen
%		\item Abgeleitet aus technischen Anforderungen
%		\item i.A. sind tolerierte Ma{\ss}e die intensiv \"uberpr\"uften
%     	\end{itemize}
%     \end{column}
%     	\begin{column}[T]{5cm} 
%         	\begin{center}
%            		\includegraphics[width=0.95\textwidth]{Toleranzen}
%        		\end{center}
%     \end{column}
% \end{columns}
%}
%
%\frame{\frametitle{Geometrische Eigenschaften und Toleranzen}
%\framesubtitle{Das exakte Herstellen eines bestimmten Ma{\ss}es ist nicht m\"oglich.}
%\begin{columns}[t] 
%     	\begin{column}[T]{5cm} 
%         	\begin{center}
%            		\includegraphics[width=0.95\textwidth]{Toleranzen}
%        		\end{center}
%     \end{column}
%     \begin{column}[T]{6cm} 
%           \begin{itemize}
%     		\item Abma{\ss}e
%		\begin{itemize}
%			\item[{\color{red}\put(0,2){\vector(-3,0){120}}}] Allgemeintoleranz (Schriftfeld, ugs. Freima{\ss}) 
%			\item[{\color{red}\put(0,2){\vector(-3,0){110}}}] Passungen 
%			\item[{\color{red}\put(0,2){\vector(-5,-2){62}}}] Spezifische Toleranzen
%		\end{itemize}
%		\item[{\color{red}\put(0,2){\vector(-5,-2){60}}}] Form
%		\item[{\color{red}\put(0,2){\vector(-4,2){85}}}] Pr\"ufma{\ss}e
%		\item Lagetoleranzen
%     	\end{itemize}
%     \end{column}
% \end{columns}
%}



%\frame{\frametitle{SI-Einheitensystem}
%\framesubtitle{Internationales Einheitensystem erm\"oglicht Vergleich der Messgr\"o{\ss}en}
%\begin{columns}[t] 
%     \begin{column}[T]{6cm} 
%     	\begin{itemize}
%     		\item International anerkannt
%		\item Beruht auf 7 physikalischen Gr\"o{\ss}en
%		\item Weitere Gr\"o{\ss}en abgeleitet, z.B.
%		\begin{equation*}
%			1 \mathrm{N} = 1 \frac{\mathrm{kg \, m}}{\mathrm{s^{2}}}
%		\end{equation*}
%		\item Koh\"arent: auschlie{\ss}lich Faktor 1
%     	\end{itemize}
%     \end{column}
%     	\begin{column}[T]{5cm} 
%         	\begin{center}
%	\vspace{0.5cm}
%            		\begin{table}
%%\begin{center}
%\scriptsize 
%\begin{tabular}{|c|c|c|}
%\hline
%Gr\"o{\ss}e & SI-Einheit & Einheiten-\\
% & & zeichen \\ \hline
% L\"ange & Meter & m \\ \hline
% Masse & Kilogramm & kg \\ \hline
% Zeit & Sekunde & s\\ \hline
% Temperatur & Kelvin & K \\ \hline
% Stromst\"arke & Ampere & A \\ \hline
% Stoffmenge & Mol & mol \\ \hline
% Lichtst\"arke & Candela & cd \\ \hline
%\end{tabular}
% \caption{Basiseinheiten des SI Systems}
%\end{center}
%\label{Tab:SIUnits}
%\end{table}%
%
%        		\end{center}
%     \end{column}
% \end{columns}
% \normalsize
%}
\frame{\frametitle{Gr\"o{\ss}enordnungen und Definitionen}
\framesubtitle{}
\begin{columns}[t] 
     \begin{column}[T]{.4cm} 
     	    \end{column}
     	\begin{column}[T]{11cm} 
         	\begin{itemize}
     		\item Messgr\"o{\ss}e L\"ange: \"ublicherweise $\left(10^{-9}...10^{2}\right) \mathrm{m}$
		 \item Toleranzen zunehmend enger
     	\end{itemize}
	 
	\begin{definition}[Messgr\"o{\ss}e] Die Messgr\"o{\ss}e ist die physikalische Gr\"o{\ss}e, der die Messung gilt.
	\end{definition}
	 
	\begin{definition}[Messen] Messen ist das Ausf\"uhren von geplanten T\"atigkeiten zum Vergleich der Messgr\"o{\ss}e mit einer Einheit.
	\end{definition}
	 
	\begin{definition}[Pr\"ufen] Pr\"ufen hei{\ss}t feststellen, inwieweit ein Pr\"ufobjekt eine Forderung erf\"ullt.
	\end{definition}
     \end{column}
 \end{columns}
}
% \frame{\frametitle{Pr\"ufplanung}
% \framesubtitle{}
% \begin{columns}[t] 
%      \begin{column}[T]{6cm} 
%      	\begin{itemize}
%      		\item Festlegung von Art und Umfang der Qualit\"atspr\"ufung
% 		\item M\"oglichst fr\"uhzeitig
% 		\item Pr\"ufdaten gem\"a{\ss}:
% 		\begin{itemize}
% 			\item Technik
% 			\item Prozess-FMEA
% 			\item Fertigungsunsicherheiten
% 			\item Vertragsforderungen
% 			\item ... 
% 		\end{itemize}
% 		\item Trade-Off: Kosten vs. Qualit\"at
% 		\item Pr\"ufzeitpunkt: Ausschuss nicht weiterverarbeiten
% 		\item Pr\"ufart: Lehren oder Messen
%      	\end{itemize}
%      \end{column}
%      	\begin{column}[T]{5cm} 
%          	\begin{center}
%             		\includegraphics[width=0.95\textwidth]{Pruefplan}
%         		\end{center}
%      \end{column}
%  \end{columns}
% }

%\frame{\frametitle{\"Ubung 2: Erstellen eines Pr\"ufplans}
%\framesubtitle{}
%\begin{columns}[t] 
%     \begin{column}[T]{6cm} 
%     	\begin{enumerate}
%     		\item Erstellen Sie Pr\"ufpl\"ane f\"ur die Komponenten des Gelenks gem\"a{\ss} SFU-14008 und SFU-14009 sowie f\"ur die Lagerscheibe (Zukaufteil).
%		\item Schlagen Sie f\"ur alle drei Pr\"ufpl\"ane Pr\"ufzeitpunkte vor. Gehen Sie dazu von Eigenfertigung basierend auf
%		\begin{enumerate}
%		\item Halbzeugen
%		\item Stahlgussteilen aus.
%	\end{enumerate} 
%     	\end{enumerate}
%     \end{column}
%     	\begin{column}[T]{5cm} 
%         	\begin{center}
%            		\includegraphics[width=0.8\textwidth]{SFU-14008}\\
%			\includegraphics[width=0.8\textwidth]{SFU-14009}
%        		\end{center}
%     \end{column}
% \end{columns}
%}

% \frame{\frametitle{Umgang mit Abweichungen}
% \framesubtitle{Abh\"angig von St\"uckzahl und Marktposition sind Abweichungen durch die Lieferanten mehr oder weniger h\"aufig der Fall.}
% \begin{columns}[t] 
%      \begin{column}[T]{6cm} 
%      	\begin{itemize}
%      		\item Abweichungen werden dem Kunden vom Lieferanten angezeigt.
% 		\item Verwendbarkeit der fehlerhaften Teile wird von betroffenen Abteilungen gepr\"uft.
% 		\item Akzeptanz und Ablehnung bis auf Bauteilebene, d.h. einzelne Teile einer Serie werden betrachtet. 
%      	\end{itemize}
%      \end{column}
%      	\begin{column}[T]{5cm} 
%          	\begin{center}
%             		\includegraphics[width=0.95\textwidth]{Abweichungsantrag}
%         		\end{center}
%      \end{column}
%  \end{columns}
% }


%%Pr\"ufdatenerfassung und -auswertung
% !TEX root = ../SFV15001_MdQM_Fertigungsmesstechnik_Rev02.tex

%\label{Sec:Pruefdaten}
%\subsection{Anzeigende Messger\"ate}
%\frame{\frametitle{Werkstattpr\"ufmittel}
%\framesubtitle{}
%\begin{itemize}
%\item Werkstattpr\"ufmittel dienen der Aufnahme eindimensionaler Merkmale:
%	\begin{itemize}
%		\item Au{\ss}en-, Innen- und Absatzma{\ss}e
%		\item Durchmesser, Breite und Dicke
%		\item H\"ohe und Tiefe
%		\item Winkel
%	\end{itemize}
%\item Auch als \textit{Messzeuge} bezeichnet
%\item Gro{\ss}e Bedeutung f\"ur fertigungsnahes Messen
%\begin{itemize}
%		\item Universell einsetzbar
%		\item Leicht handhabbar
%	\end{itemize}
%	\item Heute h\"aufig elektronische Anzeigen
%	\begin{itemize}
%		\item Vereinfachte Ablesbarkeit
%		\item Erweiterte Funktionalit\"at
%	\end{itemize}
%	\item Mechanische Messzeuge robust und kosteng\"unstig 
%\end{itemize}
%}
%
%\frame{\frametitle{Abbesches Komparatorprinzip}
%\framesubtitle{}
%\begin{columns}[t] 
%     \begin{column}[T]{6cm} 
%     	\begin{itemize}
%     		\item Fluchtende Ausrichtung von Ma{\ss}verk\"orperung und Messgr\"o{\ss}e
%		\item Vermeidung von Fehlern durch Kippen des Messger\"ates zum Messobjekt $\rightarrow$ Fehler 1. Ordnung
%		\item Bei Ber\"ucksichtigung nur Verkippen des Messobjektes zur Ebene m\"oglich $\rightarrow$ Fehler 2. Ordnung
%     	\end{itemize}
%     \end{column}
%     	\begin{column}[T]{6cm} 
%         	\begin{center}
%            		\includegraphics[width=0.5\textwidth]{Abbe}\source{Quelle: wikimedia/ArtMechanic}
%        		\end{center}
%     \end{column}
% \end{columns}
%}
%
%\frame{\frametitle{Messschieber}
%\framesubtitle{}
%\begin{columns}[t] 
%     \begin{column}[T]{6cm} 
%     	\begin{itemize}
%     		\item Gebr\"auchlichstes Handmessmittel
%		\begin{itemize}
%		\item Au{\ss}enmessung (1)
%		\item Innenmessung (2)
%		\item Tiefenmessung (3)
%		\end{itemize}
%		\item Ausf\"uhrung nach DIN 862
%     	\end{itemize}
%     \end{column}
%     	\begin{column}[T]{6cm} 
%         	\begin{itemize}
%		\item Ausf\"uhrungen bis 3000 mm (2000 mm nach Norm)
%		\item Ablesung am Nonius
%		\item Alternativ:
%		\begin{itemize}
%		\item Digital
%		\item Rundskala
%		\end{itemize}
%		\end{itemize}
%     \end{column}
% \end{columns}
% \begin{center}
%            		\includegraphics[width=0.85\textwidth]{Messchieber}\\
%		\begin{itemize}
%		\item {\color{red!80!black}Abbesches Komparatorprinzip nicht eingehalten!}
%		\end{itemize}
%        		\end{center}
%}
%
%\frame{\frametitle{H\"ohenmessger\"ate}
%\framesubtitle{}
%     	\begin{itemize}
%     		\item Verwandt mit Messschieber
%		\item 1-Punkt-Antastung
%		\begin{itemize}
%		\item Verwendung i.d.R. auf Messtisch
%		\end{itemize}
%     	\end{itemize}
%         	\begin{center}
%            		\includegraphics[width=0.8\textwidth]{Hohenmess}
%        		\end{center}
%}
%
%\frame{\frametitle{Messschrauben}
%\framesubtitle{}
%\begin{columns}[t] 
%     \begin{column}[T]{6cm} 
%     	\begin{itemize}
%     		\item Ma{\ss}verk\"orperung \begin{itemize}
%		\item fluchtend mit Messstrecke
%		\item Gewindesteigung
%		\end{itemize}
%		\item B\"ugelmessschraube: DIN 863
%		\item Rutschkupplung zur Begrenzung der Messkraft
%     	\end{itemize}
%     \end{column}
%     	\begin{column}[T]{6cm} 
%        	\begin{itemize}
%		\item G\"angige Ausf\"uhrungen:
%		\begin{itemize}
%		\item Au{\ss}en-,
%		\item Innen-,
%		\item Tiefenmessschraube
%		\end{itemize}
%		\item Angepasste Messfl\"achen, z.B. f\"ur Gewinde
%		\end{itemize}
%     \end{column}
% \end{columns}
% \begin{center}
%            		\includegraphics[width=0.8\textwidth]{Micrometer}
%        		\end{center}
%}
%
%\frame{\frametitle{Anzeigende Aufnehmer mit mechanischer \"Ubersetzung}
%\framesubtitle{}
%\begin{columns}[t] 
%     \begin{column}[T]{6cm} 
%     	\begin{itemize}
%     		\item 1-Punkt-Antastung
%		\begin{itemize}
%		\item Messuhren
%		\item Feinzeiger
%		\item F\"uhlhebel
%		\end{itemize}
%		\item Messuhr: 
%		\begin{itemize}
%		\item Anzeigebereich 3 oder 10 mm
%		\item Skalenteilung $10 \mum$ 
%		\end{itemize} 
%		\item Feinzeiger:
%		\begin{itemize}
%		\item Anzeigendes L\"angenmessger\"at
%		\item Winkelausschlag $< 360^{\circ}$
%		\end{itemize}
%		\item F\"uhlhebelmessger\"at:
%		\begin{itemize}
%		\item Winkelausschlag
%		\item Kleine Messspanne
%		\end{itemize}
%     	\end{itemize}
%     \end{column}
%     	\begin{column}[T]{5cm} 
%         	\begin{center}
%            		\includegraphics[width=0.7\textwidth]{Messuhr}
%        		\end{center}
%     \end{column}
% \end{columns}
%}
%
%\subsection{Messwertaufnehmer}
%\frame{\frametitle{Potenziometer}
%\framesubtitle{}
%\begin{columns}[t] 
%     \begin{column}[T]{6cm} 
%     	\begin{itemize}
%     		\item Einfach
%		\item Kosteng\"unstig
%		\item F\"ur Widerstand $R_{0}$ und Gesamtl\"ange $S_{max}$
%		\end{itemize}
%		\begin{equation*}
%		R = \frac{s}{s_{max}} R_{0}
%		\end{equation*}
%     	%\end{itemize}
%     \end{column}
%     	\begin{column}[T]{5cm} 
%	 \begin{itemize}
%		\item Betrieb als Spannungsteiler
%		\end{itemize}
%		\begin{equation*}
%		\frac{U_{A}}{U_{0}} = \frac{\frac{s}{s_{max}}}{1+\frac{R_{0} s}{R_{B}s_{max}} \left(1 - \frac{s}{s_{max}}\right)}
%		\end{equation*}
%         	     \end{column}
% \end{columns}
% \begin{center}
%            		\includegraphics[width=0.95\textwidth]{Potentiometer}
%        	\end{center}
%
%}
%\frame{\frametitle{Induktive Sensoren}
%\framesubtitle{}
%\begin{columns}[t] 
%     \begin{column}[T]{6cm} 
%     	\begin{itemize}
%     		\item Grundlage:
%		\begin{equation*}
%			U = -L \dot{I}
%		\end{equation*}
%			\item Bauarten:
%			\begin{itemize}
%		\item Tauchankeraufnehmer
%		\item Querankeraufnehmer
%		\item Wirbelstromaufnehmer
%		\end{itemize}
%		\item Kleine bis mittlere Wege
%		\item Dynamisch (Weg\"anderungen)
%		\item Nichtlinearer Zusammenhang:
%		\begin{equation*}
%			\frac{L}{L_{max}} = \frac{1}{1+ \frac{\frac{s_{Luft}}{\mu_{Luft}}}{\frac{s_{Eisen}}{\mu_{Eisen}}}}
%		\end{equation*}
%     	\end{itemize}
%     \end{column}
%     	\begin{column}[T]{6cm} 
%         	\begin{center}
%            		\includegraphics[width=0.9\textwidth]{InduktiverTaster}
%        		\end{center}
%     \end{column}
% \end{columns}
%}
%
%\frame{\frametitle{Lasertriangulation}
%\framesubtitle{}
%\begin{columns}[t] 
%     \begin{column}[T]{6cm} 
%     	\begin{itemize}
%		\item Abstandsmessung
%     		\item Weit verbreitet
%		\item Verschiebung Antastpunkt f\"uhrt zu Verschiebung Bildpunkt auf Sensor
%		\item Messunsicherheit h\"angt ab von Oberfl\"ache des Messobjekts (Streuung) 
%     	\end{itemize}
%	\begin{equation*}
%		\triangle h = b_{0} \frac{\sin\left(\phi\right)}{\sin\left(\delta -\phi\right)}
%	\end{equation*}
%
%     \end{column}
%     	\begin{column}[T]{5cm} 
%         	\begin{center}
%            		\includegraphics[width=0.8\textwidth]{Triangulation}
%        		\end{center}
%     \end{column}
% \end{columns}
%}
%
%\frame{\frametitle{Streifenprojektionsverfahren}
%\framesubtitle{}
%\begin{picture}(500,300)(-250,-150)
%\thicklines
%\put(-230,-80){\includegraphics[width=0.4\textwidth]{Zylinder}}
%\put(-150,10){{\color{red!80!black}\vector(1,1){40}}}
%\put(-150,45){\includegraphics[width=0.5\textwidth]{Projektion}}
%\put(-100,-80){\includegraphics[width=0.5\textwidth]{Streifen}}
%\put(-30,50){{\color{red!80!black}\vector(1,-1){40}}}
%\end{picture}
%}
%
%\frame{\frametitle{Streifenprojektionsverfahren}
%\framesubtitle{}
%\begin{columns}[t] 
%     \begin{column}[T]{6cm} 
%     	\begin{itemize}
%     		\item Fl\"achenerfassungsverfahren
%		\item Statisch: eine Aufnahme
%		\begin{itemize}
%		\item Aufl\"osung ca. 1/20 Streifenbreite
%		\end{itemize}
%		\item Dynamisch: Bilderserie
%		\begin{itemize}
%		\item Aufl\"osung bis zu 1/100 Streifenbreite
%		\end{itemize}
%		\item Streifenverschiebung nicht eindeutig
%		\begin{itemize}
%		\item Kontinuierliche Formen g\"unstiger
%		\end{itemize}
%		\item Phasenlage durch Sch\"atzen der Sinusfunktion
%     	\end{itemize}
%     \end{column}
%     	\begin{column}[T]{6cm} 
%         	\begin{center}
%            		\includegraphics[width=1.0\textwidth]{Streifen}
%        		\end{center}
%     \end{column}
% \end{columns}
%}
%
%
%
%
%\subsection{Ma{\ss}verk\"orperungen}
%
%\frame{\frametitle{Ma{\ss}verk\"orperungen}
%\framesubtitle{}
%\begin{itemize}
%\item Unterscheidung:  
%\begin{itemize}
%\item Materiell:   Stellt Ma{\ss} durch geometrische Gestalt dar, z.B. Endma{\ss}  
%\item Immateriell:   Stellt Ma{\ss} durch eigenes Merkmal dar, z.B. Lichtgeschwindigkeit als L\"angennormal  
%\end{itemize}
%\begin{definition}[Ma{\ss}verk\"orperung, \cite{pfeifer10}]
%Unter einer Ma{\ss}verk\"orperung ist allgemein ein fa{\ss}bares Objekt oder auch ein Naturph\"anomen zu verstehen, das durch ein bestimmtes unver\"anderliches Merkmal das zu verk\"orpernde Ma{\ss} darstellt. Ma{\ss}verk\"orperungen haben keine w\"ahrend der Messung beweglichen Teile.
%\end{definition} 
%\item Alternative Definition: Ger\"at, ``mit dem in gleichbleibender Weise w\"arend seines Gebrauchs ein oder mehrere Werte einer Gr\"o{\ss}e wiedergegeben oder geliefert werden sollen. \cite{Din94}''
%\end{itemize}
%}
%
%\frame{\frametitle{Parallelendma{\ss}e}
%\framesubtitle{Praxisnahe Grundlage f\"ur industrielles Messen und Pr\"ufen}
%\begin{columns}[t] 
%     \begin{column}[T]{6cm} 
%     	\begin{itemize}
%     		\item L\"angenverk\"orperung durch Abstand zwei paralleler Fl\"achen
%		\item Bereich $\left(0{,}5 \ldots 3000\right) \mathsf{mm}$
%		\item Verschlei{\ss}fest, formbest\"andiger Werkstoff: Hartmetall, Keramik
%		\item Oberfl\"ache fehlerfrei
%		\item Verbindung durch Ansprengen m\"oglich: Viele Kombinationen mit wenigen Elementen m\"oglich.
%		\item Zusammenstellung des Endma{\ss}satzes bestimmt Me{\ss}bereich und kleinste Abstufung
%     	\end{itemize}
%     \end{column}
%     	\begin{column}[T]{5cm} 
%         	\begin{center}
%            		\includegraphics[width=0.95\textwidth]{Endmass}\\ \vspace{0.2cm}
%		\scriptsize
%		\begin{tabular}{c|c}
%		Klasse & M\"oglicher Einsatz \\ \hline
%		 K & Kalibriergrad \\ \hline
%		0 & Betriebliche Kontrolle \\
%		& von Endma{\ss}en \\ \hline
%		I & Lehrenkontrolle, Einstellung \\
%		& von Me{\ss}gera\"aten \\ \hline
%		II &  Messen und Pr\"ufen 
%		\end{tabular}
%		\end{center}
%     \end{column}
% \end{columns}
%}
%
%\frame{\frametitle{Besondere Geometrieverk\"orperungen}
%\framesubtitle{Vorwiegend zu Kalibrierzwecken eingesetzte Geometrieverk\"orperungen}
%\begin{columns}[t] 
%     \begin{column}[T]{6cm} 
%     	\begin{itemize}
%     		\item Bidirektionales Stufenendma{\ss}:
%		\begin{itemize}
%		\item Verwirklicht Innen-, Au{\ss}en-, vorderes und hinteres Stufenma{\ss}, Mittelpunktsabst\"ande
%		\item \"Uberwachung von Koordinatenmessger\"aten
%		\item Stufen zur Entdeckung kurz- und langperiodischer Fehler
%		\end{itemize}
%		\item Sinuslineal
%		\begin{itemize}
%		\item Feste L\"ange
%		\item H\"ohenunterschied durch Endma{\ss}e
%		\item Winkel \"uber Sinusbeziehung
%		\end{itemize}
%     	\end{itemize}
%     \end{column}
%     	\begin{column}[T]{5cm} 
%         	\begin{center}
%			%Bidirektionales Stufenendma{\ss}
%            		\includegraphics[width=0.45\textwidth]{Stufenendmass}\\
%			\includegraphics[width=0.6\textwidth]{Sinuslineal}		
%			%Sinuslineal
%        		\end{center}
%     \end{column}
% \end{columns}
%}
%
%\frame{\frametitle{Inkrementale Ma{\ss}verk\"orperungen}
%\framesubtitle{}
%\begin{itemize}
%\item L\"angenerfassung durch Schrittz\"ahlung
%	\begin{itemize}
%		\item Multiplikation mit Schrittweite ergibt Gesamtverschiebung
%		\item Richtungssignal ben\"otigt
%		\item Absolutwert durch Referenzmarke 
%	\end{itemize}
%\item Vorteil: Ben\"otigt nur eine Spur zur Codierung der L\"ange
%\item H\"aufiger Einsatz: NC-Maschinen und Koordinatenmessger\"ate
%\item Beispiele:
%	\begin{itemize}
%		\item Strichma{\ss}st\"abe
%		\item Polygonspiegel
%		\item Interferometer
%	\end{itemize}
%\end{itemize}
%}
%
%\frame{\frametitle{Strichma{\ss}st\"abe}
%\framesubtitle{}
%\begin{columns}[t] 
%     \begin{column}[T]{6cm} 
%     	\begin{itemize}
%     		\item Optisch: Glasma{\ss}stab
%		\begin{itemize}
%		\item Feste und bewegliche Glasplatte
%		\item Verk\"orperung durch Streifen
%		\item Verschiebungsmessung durch Lichtimpulse
%		\item ggf. Moir\'e-Verfahren
%		\end{itemize}
%		\item Elektrisch:
%		\begin{itemize}
%		\item Induktiv:
%		\begin{itemize}
%		\item Inductosyn: Verschiebung Prim\"ar- gegen Sekund\"arspule
%		\item Accupin: Verschiebung ferromagnetischer Zapfen im $b$-Feld
%		\end{itemize}
%		\item Magnetisch
%		\item Kapazitiv
%		\end{itemize}
%     	\end{itemize}
%     \end{column}
%     	\begin{column}[T]{5cm} 
%         	\begin{center}
%            		\includegraphics[width=0.95\textwidth]{Glasmass}\rotatebox{90}{{\tiny \color{gray} Quelle: wikimedia/MatthiasDD}} \vspace{.2cm}
%			%\includegraphics[width=0.9\textwidth]{ElektrInk}\hspace{.1cm}\rotatebox{90}{{\tiny \color{gray} \cite[S. 42]{pfeifer10}}}
%        		\end{center}
%     \end{column}
% \end{columns}
%}
%
%
%
%\frame{\frametitle{Inkrementale Ma{\ss}verk\"orperung, optisch}
%\framesubtitle{}
%\begin{columns}[t] 
%     \begin{column}[T]{6.2cm} 
%		\begin{itemize}
%		\item Polygonspiegel:
%		\begin{itemize}
%		\item Winkelma{\ss}verk\"orperung, $(4 \ldots 72)$ Fl\"achen
%		\item Antastung mittel Autokollimationsfernrohr
%		\end{itemize}
%		\item Interferenz
%		\begin{itemize}
%		\item Immaterielle Ma{\ss}verk\"orperung: Einsatz der Lichtwellenl\"ange als Ma{\ss}verk\"orperung
%		\item Auswertung mittels Michelson-Interferometer
%		\item Messung geringer Unterschiede zwischen $l_{1}$ und $l_{2}$
%		\item Aufl\"osung: $\frac{\lambda}{2}$
%		\end{itemize}
%		\end{itemize}
%     \end{column}
%     	\begin{column}[T]{5cm} 
%         	\begin{center}
%	\setlength{\unitlength}{0.75pt}
%            		\begin{picture}(300, 150)(-100,-60)
%\thicklines
%\scriptsize
%\put(-100,3){\vector(1,0){95}}
%\put(-100,-3){\vector(1,0){95}}
%% Prolong here
%\put(5,3){\vector(1,0){75}}
%\put(80,-3){\vector(-1,0){75}}
%%Til here
%\put(3,5){\vector(0,1){75}}
%\put(-3,80){\vector(0,-1){75}}
%\put(-20,-20){\framebox(40,40){}}
%\put(-20,-20){\line(1,1){40}}
%\put(-3,-5){\vector(0,-1){35}}
%\put(3,-5){\vector(0,-1){35}}
%%Mirrors
%\put(-30,80){\framebox(60,5){}}
%\put(80,-30){\framebox(5,60){}}
%\put(-30,-50){\framebox(60,10){Beobachter}}
%\put(40,80){Spiegel $S_1$}
%\put(60,-40){Spiegel $S_2$}
%\put(-90,10){Strahlteiler}
%\put(0, 30){\vector(1,0){80}}
%\put(10, 30){\vector(-1,0){10}}
%\put(55, 33){$l_2$}
%\put(30, 0){\vector(0,1){80}}
%\put(30, 10){\vector(0,-1){10}}
%\put(33, 55){$l_1$}
%\end{picture}
%\scriptsize Michelson-Interferometer zur L\"angendifferenzmessung
%        		\end{center}
%     \end{column}
% \end{columns}
%}
%
%\frame{\frametitle{Inkrementale Ma{\ss}verk\"orperungen, mechanisch}
%\framesubtitle{}
%\begin{columns}[t] 
%     \begin{column}[T]{5cm} 
%     	\begin{itemize}
%     		\item Teilung von Ritzel oder Zahnstange
%		\item Funktionsintegration: Ma{\ss}verk\"orperung und Kraft\"ubertragung
%		\item z.B. Portalfr\"asmaschine, St\"anderbohrmaschine
%		\item Antastung durch Abrollen des Ritzels bzw. der Zahnstange 
%     	\end{itemize}
%     \end{column}
%     	\begin{column}[T]{7cm} 
%         	\begin{center}
%            		\includegraphics[width=0.9\textwidth]{Mechanisch} \hspace{.1cm}\rotatebox{90}{{\tiny \color{gray} \cite[S. 44]{pfeifer10}}}
%        		\end{center}
%     \end{column}
% \end{columns}
%}
%
%\frame{\frametitle{Absolut codierte Ma{\ss}verk\"orperungen}
%\framesubtitle{}
%\begin{columns}[t] 
%     \begin{column}[T]{5.5cm} 
%     	\begin{itemize}
%     		\item Eindeutige Zuordnung zwischen Position und Anzeige
%		\item Einsetzbar f\"ur L\"angen und Winkel
%		\item H\"ohere Anzahl Spuren und Detektoren n\"otig:
%		\begin{equation*}
%			n = \log_{2}\left(\frac{L}{a}\right)
%		\end{equation*}
%		\begin{itemize}
%		\item[] $L$: L\"ange des Ma{\ss}stabs
%		\item[] $a$: Aufl\"osung der Ma{\ss}verk\"orperung 
%		\end{itemize}
%		\item Keine Interpolation m\"oglich
%     	\end{itemize}
%     \end{column}
%     	\begin{column}[T]{6cm} 
%	\begin{center}
%		\setlength{\unitlength}{2pt}
%         	\begin{picture}(200,200)(0,-200)
%		%Inkremental
%		\multiput(0,0)(10,0){8}{
%			\put(0,0){\line(1,0){10}}
%			\put(0,-5){\rule{10pt}{10pt}}
%			\put(0,-5){\line(1,0){10}}}
%			\put(80,0){\line(0,-1){5}}
%			\put(10,-10){Inkrementaler Ma{\ss}stab}
%            	%Bin\"ar
%		\multiput(0,-20)(10,0){8}{
%			\put(0,0){\line(0,-1){5}}
%			\put(0,0){\line(1,0){10}}
%			\put(0,-5){\rule{10pt}{10pt}}
%			\put(0,-5){\line(1,0){10}}
%			\put(10,0){\line(0,-1){5}}}
%		\multiput(0,-25)(20,0){4}{
%			\put(0,0){\line(1,0){20}}
%			\put(0,-5){\line(1,0){20}}
%			\multiput(0,0)(5,0){5}{
%			\put(0,0){\line(0,-1){5}}}
%			\put(10,-5){\rule{20pt}{10pt}}
%			}
%		\multiput(0,-30)(40,0){2}{
%			\put(0,0){\line(1,0){40}}
%			\put(0,-5){\line(1,0){40}}
%			\multiput(0,0)(5,0){9}{
%			\put(0,0){\line(0,-1){5}}}
%			\put(20,-5){\rule{40pt}{10pt}}
%			}
%		\multiput(0,-35)(80,0){1}{
%			\put(0,0){\line(1,0){80}}
%			\put(0,-5){\line(1,0){80}}
%			\multiput(0,0)(5,0){17}{
%			\put(0,0){\line(0,-1){5}}}
%			\put(40,-5){\rule{80pt}{10pt}}
%			}		
%			\put(10,-45){Bin\"ar codierter Ma{\ss}stab}		
%		\multiput(0,-55)(80,0){1}{
%			\put(0,0){\line(1,0){80}}
%			\put(0,-5){\line(1,0){80}}
%			\multiput(0,0)(5,0){17}{
%			\put(0,0){\line(0,-1){5}}}
%			\multiput(0,0)(20,0){4}{
%			\put(5,-5){\rule{20pt}{10pt}}}
%			}
%		\multiput(0,-60)(80,0){1}{
%			\put(0,0){\line(1,0){80}}
%			\put(0,-5){\line(1,0){80}}
%			\multiput(0,0)(5,0){17}{
%			\put(0,0){\line(0,-1){5}}}
%			\multiput(0,0)(40,0){2}{
%			\put(10,-5){\rule{40pt}{10pt}}}
%			}
%		\multiput(0,-65)(80,0){1}{
%			\put(0,0){\line(1,0){80}}
%			\put(0,-5){\line(1,0){80}}
%			\multiput(0,0)(5,0){17}{
%			\put(0,0){\line(0,-1){5}}}
%			\put(20,-5){\rule{80pt}{10pt}}
%			}
%		\multiput(0,-70)(80,0){1}{
%			\put(0,0){\line(1,0){80}}
%			\put(0,-5){\line(1,0){80}}
%			\multiput(0,0)(5,0){17}{
%			\put(0,0){\line(0,-1){5}}}
%			\put(40,-5){\rule{80pt}{10pt}}
%			}
%		\put(10,-80){Gray codierter Ma{\ss}stab}	
%        		\end{picture}
%		\end{center}
%     \end{column}
% \end{columns}
%}
%
%\frame{\frametitle{Pr\"ufdatenauswertung - Darstellung}
%\framesubtitle{Quantitative Messdaten k\"onnen als Verteilung dargestellt werden.}
%\begin{columns}[t] 
%     \begin{column}[T]{6cm} 
%     	\begin{itemize}
%		\item Diskrete Verteilungen:
%			\begin{itemize}
%				\item H\"aufigkeiten der Merkmalsauspr\"agung 
%			\end{itemize}
%     		\item Kontinuierliche Verteilungen:
%		\begin{itemize}
%			\item Einteilung in Klassen
%			\item G\"unstig: $\sqrt{n}$ Klassen
%			\item H\"aufigkeit der Klassenzugeh\"origkeit
%			\item Darstellung im Histogramm
%		\end{itemize}
%     	\end{itemize}
%     \end{column}
%     	\begin{column}[T]{5cm} 
%         	\begin{center}
%            		\includegraphics[width=0.95\textwidth]{hist}\\
%		\tiny Pseudo-Zufallsverteilung eines Ma{\ss}es $x = 50$
%        		\end{center}
%     \end{column}
% \end{columns}
%}

\subsection{Pr\"ufdatenauswertung}
\frame{\frametitle{Beschreibung}
\framesubtitle{Beschreibung der Verteilung durch Lage und Streuungsparameter verdichtet die information.}
\begin{columns}[t] 
	\begin{column}[T]{5cm} 
         	\begin{center}
            		\includegraphics[width=0.9\textwidth]{hist2}
		% Spannweite
		{\thicklines \color{black} \put(-96,9){\vector(1,0){93}}\put(-96,9){\vector(-1,0){20}} \put(-80, 13){Spannweite}} 
		% Quartilsabstand
		{\thicklines \color{black} \put(-80,59){\vector(1,0){46}}\put(-80,59){\vector(-1,0){4}} \put(-94, 63){Quartilsabstand}}
		% Standardabweichung
		{\thicklines \color{black} \put(-63,39){\vector(1,0){30}}\put(-63,39){\vector(-1,0){30}} \put(-64, 43){s} \put(-33,0){\line(0,1){120}} \put(-93,0){\line(0,1){120}}}
		%Arithmetisches Mittel
		{\thicklines \color{green!50!black}\put(-66,170){\line(0,-1){160}} \put(-137,160){$\bar{x} = \frac{1}{n} \sum_{i=1}^{n}x_{i}$}} 
		% Modalwert
		{\thicklines \color{blue}\put(-90,130){\vector(1,1){20}} \put(-140,128){Modalwert}}
        		\end{center}
     \end{column}
     \begin{column}[T]{6cm} 
     	\begin{itemize}
		\item Lageparameter:  
		\begin{itemize}
			\item Arithmetisches Mittel $\bar{x}$  
			\item Modalwert: am h\"aufigsten angenommene Klasse 			\item Median: je 50\% der Messwerte gr\"o{\ss}er bzw. kleiner
		\end{itemize}
		\item Streuungsparameter
		\begin{itemize}
			\item Spannweite  
			\item Quartilsabstand: Spannweite der zentralen 50\% der Messwerte 
			\item Standardabweichung $s$
			\item Quantile: Merkmalsauspr\"agung, f\"ur die ein Anteil  $\alpha$ kleiner ist 
		\end{itemize}
	\end{itemize}
     \end{column}
 \end{columns}
}

\frame{\frametitle{Verteilungen}
\framesubtitle{Zufallsmodelle k\"onnen helfen, beobachtete Ph\"anomene zu beschreiben und \"uber die Modellbildung vorhersagen zu treffen.}
\begin{columns}[t] 
     \begin{column}[T]{6cm} 
     	\begin{itemize}
     		\item N\"aherungsweise Beschreibung 
		\item I.d.R. Konvergenz f\"ur gro{\ss}e Stichproben
		\item Wahrscheinlichkeit kann als relative H\"aufigkeit interpretiert werden
		\item Verteilung stetiger und diskreter Merkmale unterschiedlich zu modellieren
     	\end{itemize}
     \end{column}
     	\begin{column}[T]{5cm} 
         	\begin{center}
            		\includegraphics[width=0.95\textwidth]{cont}
        		\end{center}
     \end{column}
 \end{columns}
}

\frame{\frametitle{Verteilungen diskreter Merkmale}
\framesubtitle{}
\begin{columns}[t] 
     \begin{column}[T]{6cm} 
     	\begin{itemize}
     		\item Poisson-Verteilung:
		\begin{equation*}
			p_{P}(k) = \frac{\lambda^{k}}{k!} \exp\left(-\lambda\right)
		\end{equation*}
		\item[$\rightarrow$] Geringe Wahrscheinlichkeiten, z.B. $p\leq0,1$
		\item Binomial-Verteilung:
		\begin{equation*}
			p_{B}(k) = \binom{n}{k} p^{k} \left(1-p\right)^{n-k}
		\end{equation*}
		\item[$\rightarrow$] H\"ohere Wahrscheinlichkeiten, z.B. $p>0,1$
     	\end{itemize}
     \end{column}
     	\begin{column}[T]{5cm} 
         	\begin{center}
            		\includegraphics[width=0.95\textwidth]{poisson}
        		\end{center}
     \end{column}
 \end{columns}
}

\frame{\frametitle{Verteilungen kontinuierlicher Merkmale}
\framesubtitle{}
\begin{columns}[t] 
     \begin{column}[T]{6cm} 
     	\begin{itemize}
     		\item Normalverteilung:
		\begin{equation*}
			p_{N}(x) = \frac{1}{\sigma \sqrt{2\pi}}\exp\left(- \frac{\left(x-\mu\right)^{2}}{2\sigma^{2}} \right)
		\end{equation*}
		\item[$\rightarrow$] Faltung vieler Verteilungen
		\item Gleichverteilung
		\item Dreieckverteilung
		\item[$\rightarrow$] Faltung zweier Rechteckverteilungen
     	\end{itemize}
     \end{column}
     	\begin{column}[T]{5cm} 
         	\begin{center}
            		\includegraphics[width=0.95\textwidth]{cont}
        		\end{center}
     \end{column}
 \end{columns}
}

\frame{\frametitle{Zentraler Grenzwertsatz}
\framesubtitle{Konvergenz der Summe von Zufallsvariablen gegen die Standardnormalverteilung.}
\begin{columns}[t] 
     \begin{column}[T]{6cm} 
     	\begin{itemize}
	\scriptsize
     		\item Sei $X_{1}, X_{2}, X_{3},\ldots$ eine Folge von Zufallsvariable, die auf demselben Wahrscheinlichkeitsraum unabh\"angig und identisch verteilt sind.
		\item Sei weiterhin $S_{n} = X_{1} + X_{2} + \ldots + X_{n}$ die $n$-te Teilsumme, eine Zufallsvariable mit $E\left(S_{n}\right) = n \mu$ und $\Var\left(S_{n}\right) = n \sigma^{2}$
		\item Dann konvergiert die Verteilungsfunktion der standardisierten Zufallsvariablen
		\begin{equation*}
		Z_{n} = \frac{S_{n} - n\mu}{\sigma \sqrt{n}}
		\end{equation*}
		f\"ur $n \rightarrow \infty$ punktweise gegen die Standardnormalverteilung $\mathcal{N}(0,1)$.
     	\end{itemize}
     \end{column}
     	\begin{column}[T]{6.5cm} 
         	\begin{center}
            		\setlength{\unitlength}{0.0500bp}%
  \begin{picture}(3800.00,3000.00)%
  	\scriptsize
    { \put(400,2127){\makebox(0,0)[r]{\strut{}0}}%
    %\put(400,2235){\makebox(0,0)[r]{\strut{}20}}%
      
      \put(400,2343){\makebox(0,0)[r]{\strut{}40}}%
      
      %\put(400,2451){\makebox(0,0)[r]{\strut{}60}}%
      
      \put(400,2558){\makebox(0,0)[r]{\strut{}80}}%
      
      %\put(400,2666){\makebox(0,0)[r]{\strut{}100}}%
      
      \put(400,2774){\makebox(0,0)[r]{\strut{}120}}%
      
      %\put(520,2000){\makebox(0,0){\strut{}0}}%
      
      %\put(788,2000){\makebox(0,0){\strut{}0.2}}%
      
      %\put(1055,2000){\makebox(0,0){\strut{}0.4}}%
      
      %\put(1323,2000){\makebox(0,0){\strut{}0.6}}%
      
      %\put(1590,2000){\makebox(0,0){\strut{}0.8}}%
      
      %\put(1858,1927){\makebox(0,0){\strut{}1}}%
      \csname LTb\endcsname%
      \put(1189,2900){\makebox(0,0){\strut{}$n = 1$}}%
    }%
    {%
      
      \put(2161,2127){\makebox(0,0)[r]{\strut{}0}}%
      
      %\put(2161,2235){\makebox(0,0)[r]{\strut{}20}}%
      
      \put(2161,2343){\makebox(0,0)[r]{\strut{}40}}%
      
      %\put(2161,2451){\makebox(0,0)[r]{\strut{}60}}%
      
      \put(2161,2558){\makebox(0,0)[r]{\strut{}80}}%
      
      %\put(2161,2666){\makebox(0,0)[r]{\strut{}100}}%
      
      \put(2161,2774){\makebox(0,0)[r]{\strut{}120}}%
      
      \put(2950,2900){\makebox(0,0){\strut{}$n = 5$}}%
    }%
    {%
      
      \put(400,1228){\makebox(0,0)[r]{\strut{}0}}%
      
      %\put(400,1336){\makebox(0,0)[r]{\strut{}20}}%
      
      \put(400,1444){\makebox(0,0)[r]{\strut{}40}}%
      
      %\put(400,1552){\makebox(0,0)[r]{\strut{}60}}%
      
      \put(400,1659){\makebox(0,0)[r]{\strut{}80}}%
      
      %\put(400,1767){\makebox(0,0)[r]{\strut{}100}}%
      
      \put(400,1875){\makebox(0,0)[r]{\strut{}120}}%
      
      %\put(520,1028){\makebox(0,0){\strut{}0}}%
      
      %\put(788,1028){\makebox(0,0){\strut{}0.2}}%
      
      %\put(1055,1028){\makebox(0,0){\strut{}0.4}}%
      
      %\put(1323,1028){\makebox(0,0){\strut{}0.6}}%
      
      %\put(1590,1028){\makebox(0,0){\strut{}0.8}}%
      
      %\put(1858,1028){\makebox(0,0){\strut{}1}}%
      %\csname LTb\endcsname%
      \put(1189,2000){\makebox(0,0){\strut{}$n = 9$}}%
    }%
    {%
      
      \put(2161,1228){\makebox(0,0)[r]{\strut{}0}}%
      
      %\put(2161,1336){\makebox(0,0)[r]{\strut{}20}}%
      
      \put(2161,1444){\makebox(0,0)[r]{\strut{}40}}%
      
      %\put(2161,1552){\makebox(0,0)[r]{\strut{}60}}%
      
      \put(2161,1659){\makebox(0,0)[r]{\strut{}80}}%
      
      %\put(2161,1767){\makebox(0,0)[r]{\strut{}100}}%
      
      \put(2161,1875){\makebox(0,0)[r]{\strut{}120}}%
       \put(2950,2000){\makebox(0,0){\strut{}$n = 13$}}%
    }%
    {%
      
      \put(400,330){\makebox(0,0)[r]{\strut{}0}}%
      
      %\put(400,438){\makebox(0,0)[r]{\strut{}20}}%
      
      \put(400,545){\makebox(0,0)[r]{\strut{}40}}%
      
      %\put(400,653){\makebox(0,0)[r]{\strut{}60}}%
      
      \put(400,761){\makebox(0,0)[r]{\strut{}80}}%
      
      %\put(400,868){\makebox(0,0)[r]{\strut{}100}}%
      
      \put(400,976){\makebox(0,0)[r]{\strut{}120}}%
      
      \put(520,130){\makebox(0,0){\strut{}0}}%
      
      \put(788,130){\makebox(0,0){\strut{}0.2}}%
      
      \put(1055,130){\makebox(0,0){\strut{}0.4}}%
      
      \put(1323,130){\makebox(0,0){\strut{}0.6}}%
      
      \put(1590,130){\makebox(0,0){\strut{}0.8}}%
      
      \put(1858,130){\makebox(0,0){\strut{}1}}%
      %\csname LTb\endcsname%
      \put(1189,1100){\makebox(0,0){\strut{}$n = 17$}}%
    }{%
      
      \put(2161,330){\makebox(0,0)[r]{\strut{}0}}%
      
      %\put(2161,438){\makebox(0,0)[r]{\strut{}20}}%
      
      \put(2161,545){\makebox(0,0)[r]{\strut{}40}}%
      
      %\put(2161,653){\makebox(0,0)[r]{\strut{}60}}%
      
      \put(2161,761){\makebox(0,0)[r]{\strut{}80}}%
      
      %\put(2161,868){\makebox(0,0)[r]{\strut{}100}}%
      
      \put(2161,976){\makebox(0,0)[r]{\strut{}120}}%
      
      \put(2281,130){\makebox(0,0){\strut{}0}}%
      
      \put(2549,130){\makebox(0,0){\strut{}0.2}}%
      
      \put(2816,130){\makebox(0,0){\strut{}0.4}}%
      
      \put(3084,130){\makebox(0,0){\strut{}0.6}}%
      
      \put(3351,130){\makebox(0,0){\strut{}0.8}}%
      
      \put(3619,130){\makebox(0,0){\strut{}1}}%
      \csname LTb\endcsname%
      \put(2950,1100){\makebox(0,0){\strut{}$n = 21$}}%
    }
    \put(0,0){\includegraphics{ZentralerGWS}}%
  \end{picture}
  \tiny Summe aus $n$ gleichverteilten Zufallsvariablen
        		\end{center}
     \end{column}
 \end{columns}
}

\frame{\frametitle{Eigenschaften der Normalverteilung}
\framesubtitle{Viele Ph\"anomene in Technik und Naturwissenschaft lassen sich durch eine Normalverteilung ann\"ahern.}
\begin{columns}[t]
	\begin{column}[T]{5cm} 
         	\begin{center}
            		\includegraphics[width=0.95\textwidth]{gaussian}
%		\begin{tikzpicture} 
%		\begin{axis}[xmin = -5, xmax = 5, ymin = 0,
%                    ]
%                    % density of Normal distribution:
%                    \addplot+[name path = A,
%                       red,
%                       domain=-5:5,
%                       samples=201,
%                    ]
%                       {exp(-x^2 / (2)) / (1 * sqrt(2*pi))};
%		
%		\path[name path=B] (-5,0) -- (5,0);
%		\addplot[red!30] fill between[of=A and B,
%        			soft clip= {domain=0:3}];
%                    \end{axis}
%                   \end{tikzpicture}
        		\end{center}
     \end{column} 
     \begin{column}[T]{6cm} 
     	\begin{itemize}
     		\item Mittelwert $\mu$
		\item Standardabweichung $\sigma$
		\item $\sigma$-Umgebungen:\\
		\begin{tabular}{|c|c|}
		\hline
		$k$ & \% der Realisierungen \\ \hline
		1 & 68{,}3 \\ \hline
		2 & 95{,}5 \\ \hline
		3 & 99{,}4 \\ \hline
		%6 & 99{,}99966 \\ \hline
		\end{tabular}
		\item Quantile:\\
		\begin{tabular}{|c|c|}
		\hline
		\% der Realisierungen & $k$  \\ \hline
		50 & 0{,}675 \\ \hline
		90 & 1{,}65 \\ \hline
		95 & 1{,}96 \\ \hline
		99 & 2{,}58 \\ \hline
		\end{tabular}
     	\end{itemize}
     \end{column}
 \end{columns}
}


\frame{\frametitle{Vertiefung: Konfidenzintervall, Bestimmung $u$}
\framesubtitle{Konfidenzintervalle sind Intervalle in denen der wahre Wert einer Merkmalsauspr\"agung mit Wahrscheinlichkeit $p$ liegt.}
%\only<1> {\begin{center}\includegraphics[scale=0.15]{Off} \end{center}}
%\only<2>
{
\begin{columns}[t] 
     \begin{column}[T]{6cm} 
     	\begin{itemize}
     		\item Gr\"o{\ss}e und Lage abh\"abh\"angig von
		\begin{itemize}
     		\item Verteilung
		\item Konfidenzniveau (z.B. 95\%)
		\end{itemize}
		\item F\"ur Normalverteilung:\\
		\scriptsize
		\begin{tabular}{|c|c|}
		\hline 
		$k$ & \% der Realisierungen \\ \hline
		1 & 68{,}3 \\ \hline
		2 & 95{,}5 \\ \hline
		3 & 99{,}4 \\ \hline
		%6 & 99{,}99966 \\ \hline
		\end{tabular}
		\normalsize
		\item Fertigungsmesstechnik: 95\%-Konfidenzintervall $U_{95\%}$ f\"ur $k=2$
     	\end{itemize}
     \end{column}
     	\begin{column}[T]{5cm} 
         	  \begin{picture}(80,60)(30,15)
    \put(0,0){\includegraphics[scale=0.401]{cont2}}
    \thicklines
    \scriptsize
    \put(85,11){{\line(0,1){55}}}
    \put(85,49){\color{red}{\vector(1,0){32}}}
    \put(117,49){\color{red}{\vector(-1,0){32}}}
    \put(117,11){\color{red}{\line(0,1){40}}}
    \put(88,65){$\mu$}
    \put(95,52){\color{red}$a_{\triangle}$}
    %%
    \put(62,15){\color{green!60!black}{\vector(1,0){22}}}
    \put(84,15){\color{green!60!black}{\vector(-1,0){22}}}
    \put(67,18){\color{green!60!black}$a_{\square}$}
    %%
    \put(90,34){\color{blue!60!black}$a_{\mathcal{N}}$}
    \put(112,2){\color{blue!60!black}{\line(0,1){35}}}
    \put(58,2){\color{blue!60!black}{\line(0,1){15}}}
    \put(85,30){\color{blue!60!black}{\vector(1,0){27}}}
    \put(85,2){\color{blue!60!black}{\vector(1,0){27}}}
    \put(85,2){\color{blue!60!black}{\vector(-1,0){27}}}
    \put(85,5){\color{blue!60!black}{$U_{95\%}$}}
  \end{picture}
  \vspace{.5cm}
			\begin{equation*}
		u = \frac{a}{K}
		\end{equation*}
		Normalverteilung: $k = \sqrt{4}$, $a$  Abma{\ss}e von $U_{95\%}$\\
		Gleichverteilung: $k = \sqrt{3}$\\
		Dreiecksverteilung: $k = \sqrt{6}$ 
     \end{column}
 \end{columns}}
}

\frame{\frametitle{Vertiefung: Konfidenzintervall, Bestimmung $u$}
\framesubtitle{}
\begin{picture}(160,160)(00,0)
	\setlength{\unitlength}{2pt}
    \put(0,0){\includegraphics[scale=0.8]{cont2}}
    \thicklines
    %\scriptsize
    \put(85,11){{\line(0,1){55}}}
    \put(85,49){\color{red}{\vector(1,0){32}}}
    \put(117,49){\color{red}{\vector(-1,0){32}}}
    \put(117,11){\color{red}{\line(0,1){40}}}
    \put(88,65){$\mu$}
    \put(95,52){\color{red}$a_{\triangle}$}
    %%
    \put(62,15){\color{green!60!black}{\vector(1,0){22}}}
    \put(84,15){\color{green!60!black}{\vector(-1,0){22}}}
    \put(67,18){\color{green!60!black}$a_{\square}$}
    %%
    \put(90,34){\color{blue!60!black}$a_{\mathcal{N}}$}
    \put(112,2){\color{blue!60!black}{\line(0,1){35}}}
    \put(58,2){\color{blue!60!black}{\line(0,1){15}}}
    \put(85,30){\color{blue!60!black}{\vector(1,0){27}}}
    \put(85,2){\color{blue!60!black}{\vector(1,0){27}}}
    \put(85,2){\color{blue!60!black}{\vector(-1,0){27}}}
    \put(85,5){\color{blue!60!black}{$U_{95\%}$}}
    \put(110,60){\framebox{\huge $u = \frac{a}{K}$}}
  \end{picture}
		\begin{itemize}
		\item Normalverteilung: $k = \sqrt{4}$, $a$  Abma{\ss}e von $U_{95\%}$\\
		\item Gleichverteilung: $k = \sqrt{3}$\\
		\item Dreiecksverteilung: $k = \sqrt{6}$ 
		\end{itemize}
}

\frame{\frametitle{Student'sche t-Verteilung}
%\framesubtitle{Bei kleinem Stichprobenumfang}
%\only<1> {\begin{center}\includegraphics[scale=0.15]{Off} \end{center}}
%\only<2>
{
\begin{columns}[t] 
     \begin{column}[T]{6cm} 
     	\begin{itemize}
     		\item Normalverteilung (Hypothese):
		\begin{itemize}
		\item Endliche Stichprobe
		\item Wahrscheinlichkeit, ``seltene'' Werte zu realisieren?
		\end{itemize}
		\item Stichprobenvarianz f\"allt zu klein aus
		\item ``schlanke'' Normalverteilung nicht konservativ
		\item Abhilfe: Student'sche t-Verteilung:
		\begin{equation*}
		p_{n}(x) = \frac{\Gamma\left(\frac{n+1}{2} \right)}{\sqrt{n \pi}\Gamma\left(\frac{n}{2} \right)} \left( 1 + \frac{x^{2}}{2}\right)^{\frac{-n+1}{2}} 
		\end{equation*}
		\begin{itemize}
		\item f\"ur $n$ Freiheitsgrade
		\end{itemize}
     	\end{itemize}
     \end{column}
     	\begin{column}[T]{5cm} 
         	\begin{center}
            		\setlength{\unitlength}{0.0280bp}%
		 
  \begin{picture}(6000.00,8000.00)(500,-1000)%
  	\scriptsize
  	\put(0,0){\includegraphics[scale=0.56]{cont3}}
      \put(792,480){\makebox(0,0)[r]{\strut{}0}}%
      \put(792,1384){\makebox(0,0)[r]{\strut{}0.05}}
      \put(792,2288){\makebox(0,0)[r]{\strut{}0.1}}%
      \put(792,3192){\makebox(0,0)[r]{\strut{}0.15}}
      \put(792,4096){\makebox(0,0)[r]{\strut{}0.2}}%
      \put(792,4999){\makebox(0,0)[r]{\strut{}0.25}}
      \put(792,5903){\makebox(0,0)[r]{\strut{}0.3}}%
      \put(792,6807){\makebox(0,0)[r]{\strut{}0.35}}
      \put(792,7711){\makebox(0,0)[r]{\strut{}0.4}}
      \put(3300,7900){\makebox(0,0)[c]{\strut{}Wahrscheinlichkeitsdichtefunktionen}}
      \put(936,240){\makebox(0,0){\strut{}-3}}
      \put(1708,240){\makebox(0,0){\strut{}-2}}
      \put(2480,240){\makebox(0,0){\strut{}-1}}
      \put(3252,240){\makebox(0,0){\strut{}0}}
      \put(4023,240){\makebox(0,0){\strut{}1}}%
      \put(4795,240){\makebox(0,0){\strut{}2}}
      \put(5567,240){\makebox(0,0){\strut{}3}}
      \put(4323,7528){\makebox(0,0)[l]{\strut{}$\mathcal{N}(0,1)$}}
      \put(4323,7288){\makebox(0,0)[l]{\strut{}t, $\nu = 1$}}%
      \put(4323,7048){\makebox(0,0)[l]{\strut{}t, $\nu = 3$}}
      \put(4323,6808){\makebox(0,0)[l]{\strut{}t, $\nu = 5$}}%
      \put(4323,6568){\makebox(0,0)[l]{\strut{}t, $\nu = 10$}}%
      \end{picture}%

        		\end{center}
     \end{column}
 \end{columns}}
}

\frame{\frametitle{Praktische Anwendung der Student'schen t-Verteilung}
\framesubtitle{}
\begin{columns}[t] 
     \begin{column}[T]{6cm} 
     	\begin{itemize}
     		\item t-Verteilung unterstellt ``breitere'' Wahrscheinlichkeitsdichte
		\item Sch\"atzwert der Stichprobenvarianz wird korrigiert
		\item Bestimmung Korrekturfaktor aus Konfidenz und Stichprobengr\"o{\ss}e $n$
		\item Freiheitsgrade $\nu = n - m$ f\"ur $m$ zu sch\"atzende Parameter
     	\end{itemize}
     \end{column}
     	\begin{column}[T]{5cm} 
         	\begin{center}
            		\setlength{\unitlength}{0.0250bp}%
		 
  \begin{picture}(6000.00,6000.00)(500,-1000)%
  	\scriptsize
  	\put(0,0){\includegraphics[scale=0.50]{cont4}}
      %\put(3000,5900){\makebox(0,0)[c]{\strut{}Wahrscheinlichkeitsdichtefunktionen}}
      \put(4323,5500){\makebox(0,0)[l]{\strut{}\color{blue!70!black}$\mathcal{N}(0,1)$}}
      \put(4323,5200){\makebox(0,0)[l]{\strut{}\color{green!70!black} t, $\nu = 1$}}%
      \put(4323,4900){\makebox(0,0)[l]{\strut{}\color{red!80!black} t, $\nu = 3$}}
      \end{picture}%

        		\end{center}
     \end{column}
 \end{columns}
 \small
 \hspace{.8cm}
\begin{tabular}{|c|c|c|c|c|c|c|c|c|c|}
\hline $\nu$ & 1 & 2 & 3 & 4 & 5 & 10 & 20 & 50 &$\infty$ \\ \hline
$k $ & 13{,}97 & 4{,}53 & 3{,}31 & 2{,}87 & 2{,}65& 2{,}28 & 2{,}13 & 2{,}05 & 2{,}00 \\ \hline
\end{tabular}
\begin{itemize}
		\item[] $k$-Werte zur Bestimmung der erweiterten Messunsicherheit $U_{95\%}$
		\end{itemize}}




\frame{\frametitle{Definitionen}
\framesubtitle{}
\begin{itemize}
\item Ma{\ss}: Bestimmung einer L\"ange
\item Nennma{\ss}: Zeichnungangabe
\item Istma{\ss}: tats\"achliches Ma{\ss}
\item Oberes/unteres Abma{\ss}: zul\"assige Abweichung - Achtung Vorzeichen!
\item Mindestma{\ss}: Nennma{\ss} - unteres Abma{\ss}
\item H\"ochstma{\ss}: Nennma{\ss} + oberes Abma{\ss}
\end{itemize}
\begin{center}
\includegraphics[width=0.7\textwidth]{Toleranzfeld}
\end{center}
}




%\frame{\frametitle{\"Ubung 3: Umgang mit Abweichungen}
%\framesubtitle{Ein Lieferant hat eine kleine St\"uckzahl an Teilen f\"ur das Gelenk gefertigt, es kommt zu Abweichungen, die in einem Messprotokoll festgehalten sind. Die Teile werden f\"ur Lieferungen ihrerseits dringend erwartet.}
%\begin{columns}[t] 
%     \begin{column}[T]{6cm} 
%     	\begin{enumerate}
%     		\item Formulieren Sie den Abweichungsantrag f\"ur den Lieferanten
%		\item Entscheiden Sie \"uber Annahme oder Ablehnung mit Begr\"undung.
%	\end{enumerate}
%     \end{column}
%     	\begin{column}[T]{5cm} 
%         	\begin{center}
%            		\includegraphics[width=0.95\textwidth]{Abweichungsantrag}
%        		\end{center}
%     \end{column}
% \end{columns}
% \nocite{pfeifer10}
%\nocite{hoischen}
%}

%Messunsicherheit
% !TEX root = ../SFV15001_MdQM_Fertigungsmesstechnik_Rev02.tex

\subsection{Messunsicherheit}
\label{Sec:Messunsicherheit}

\frame{\frametitle{Messergebnis und Messunsicherheit}
\framesubtitle{Der Vergleich einer Messgr\"o{\ss}e mit einer Einheit gelingt nicht fehlerfrei.}
\begin{columns}[t] 
     \begin{column}[T]{6cm} 
     	\begin{itemize}
	 
     		\item Messger\"ateabweichungen
		\item Instabilit\"at der Messgr\"o{\ss}e
		\item Umwelteinfl\"usse (z.B. Temperatur)
		\item Beobachtereinfl\"usse
		\item Es werden unterschieden:
		\begin{itemize}
			\item Systematische Messabweichungen
			\begin{itemize}
			\item bekannt o. unbekannt
			\end{itemize}
			\item Zuf\"allige Messabweichungen
		\end{itemize}
     	\end{itemize}
     \end{column}
     	\begin{column}[T]{5cm} 
		\vspace{.4cm}
         	\begin{center}
            		\includegraphics[width=0.75\textwidth]{Messfehler}
        		\end{center}
     \end{column}
 \end{columns}
  
 \begin{definition}[Messergebnis]
	Das Messergebnis ist der Sch\"atzwert des wahren Wertes einer Messgr\"o{\ss}e.
	\end{definition}
}

%\frame{\frametitle{Quellen von Messabweichungen}
%\framesubtitle{Quellen von Messabweichungen lassen sich gut im Ishikawa-Diagramm ermitteln.}
%\begin{center}
%            		\includegraphics[width=0.95\textwidth]{Ishikawa}
%        		\end{center}
%\begin{itemize}
%\item Aufteilung systematisch/zuf\"allig, bekannt/unbekannt, Ausrei{\ss}er
%\end{itemize}
%}

%%\frame{\frametitle{Auswirkungen der Messabweichungen}
%%\framesubtitle{Messabweichungen formen die Verteilung der Stichproben f\"ur den Sch\"atzer.}
%%\begin{columns}[t] 
%%     \begin{column}[T]{5cm}
%%     \hspace{.1cm}
%%          \scriptsize 
%%	\begin{tabular}{|l|c|l|}
%%	\hline
%%	Abweichung & Auswirkung & Einrechnung \\ \hline
%%	\multirow{3}{40 pt}{bekannt, systematisch} & $\hat{\mu} = \mu + \gamma$& 		\multirow{3}{50 pt}{korrigieren, nicht in $U$ ber\"ucksichtigen} \\ & & \\ & &  \\ 		\hline
%%	\multirow{2}{40 pt}{unbekannt, systematisch} & $\hat{\mu} = \mu + \gamma$& 
%%	\multirow{2}{50 pt}{In $U$ einrechnen} \\ & &   \\ 
%%	\hline 
%%	\multirow{2}{40 pt}{zuf\"allig} & $\hat{\sigma} \geq \sigma$&
%%	 \multirow{2}{50 pt}{In $U$ einrechnen}  \\ & &  \\ 	\hline
%%	 \multirow{2}{40 pt}{Ausrei{\ss}er} & $\hat{\mu} \neq \mu$&
%%	 \multirow{2}{50 pt}{nicht in $U$ einrechnen}  \\ & &  \\ 	\hline
%%	\end{tabular}
%%     \end{column}
%%     \begin{column}[T]{.3cm}
%%     \end{column}
%%     	\begin{column}[T]{6cm} 
%%	
%%         	\begin{center}
%%            		\setlength{\unitlength}{0.0200bp}%
%%\begin{picture}(8000.00,9000.00)%
%%\tiny
%% \put(0,0){\includegraphics[scale=0.4]{Error}}%
%%\put(1204,896){\makebox(0,0)[r]{\strut{}0}}%
%%            \put(1204,1992){\makebox(0,0)[r]{\strut{}0.05}}%
%%            \put(1204,3088){\makebox(0,0)[r]{\strut{}0.1}}%
%%      \put(1204,4184){\makebox(0,0)[r]{\strut{}0.15}}%
%%      
%%      \put(1204,5280){\makebox(0,0)[r]{\strut{}0.2}}%
%%      
%%      \put(1204,6375){\makebox(0,0)[r]{\strut{}0.25}}%
%%      
%%      \put(1204,7471){\makebox(0,0)[r]{\strut{}0.3}}%
%%      
%%      \put(1204,8567){\makebox(0,0)[r]{\strut{}0.35}}%
%%      
%%      \put(1204,9663){\makebox(0,0)[r]{\strut{}0.4}}%
%%      
%%      \put(1372,616){\makebox(0,0){\strut{}0}}%
%%      
%%      \put(2597,616){\makebox(0,0){\strut{}2}}%
%%      
%%      \put(3821,616){\makebox(0,0){\strut{}4}}%
%%      
%%      \put(5046,616){\makebox(0,0){\strut{}6}}%
%%      
%%      \put(6270,616){\makebox(0,0){\strut{}8}}%
%%      
%%      \put(7495,616){\makebox(0,0){\strut{}10}}%
%%      
%%      \put(224,5279){\rotatebox{90}{\makebox(0,0){\strut{}Rel. Haeufigkeit}}}%
%%      
%%      \put(4433,196){\makebox(0,0){\strut{}x}}%
%%    %}%
%%    %\gplgaddtomacro\gplfronttext{%
%%      
%%      \put(6000,9460){\makebox(0,0)[l]{\strut{}$\hat{\mu}\! =\! \mu$}}%
%%      
%%      \put(6000,9180){\makebox(0,0)[l]{\strut{}$\hat{\mu}\! =\! \mu\!+\!\gamma$}}%
%%      
%%      \put(6000,8900){\makebox(0,0)[l]{\strut{}$\hat{\sigma}\! \geq\! \sigma$}}%
%%    %}%
%%   % \%gplbacktext
%%   
%%    %\gplfronttext
%%  \end{picture}%
%%        		\end{center}
%%     \end{column}
%% \end{columns}
%%}
%
%\frame{\frametitle{Definition und Darstellung von Messunsicherheiten}
%\framesubtitle{}
%\begin{columns}[t] 
%     \begin{column}[T]{5.5cm} 
%     	\begin{itemize}
%     		\item Korrekte Angabe eines Messergebnisses:
%		\begin{itemize}
%		\item Merkmalsbezeichnung
%		\item Messresultat (Stellenzahl nicht gr\"o{\ss}er als Angabe der Messunsicherheit)
%		\item Messunsicherheit $U$ (gleiche Einheit)
%		\item Erweiterungsfaktor $k$ (f\"ur erweiterte Messunsicherheit)%\footnote{Erl\"auterung folgt}
%		\end{itemize}
%		\vspace{.5cm}
%		\item Beispiel:
%		\begin{equation*}
%		A1 = \left( 120{,}037 \pm 0{,}007 \right) \mathrm{mm}, \, k = 2
%		\end{equation*}
%     	\end{itemize}
%     \end{column}
%     	\begin{column}[T]{5.5cm} 
%	\scriptsize
%         	\begin{definition}[Messunsicherheit]
%		[...] ein dem Messergebnis zugeordneter Parameter, der die Streuung der Werte kennzeichnet, die der Messgr\"o{\ss}e zugeordnet werden k\"onnen.
%        		\end{definition}
%		\begin{definition}[Erweiterte Messunsicherheit]
%		[...] ein Kennwert, der einen Bereich um dass Messergebnis kennzeichnet, von dem erwartet wird, dass er einen gro{\ss}en Anteil der Werte umfasst, die der Messgr\"o{\ss}e zugeordnet werden k\"onnen.
%        		\end{definition}
%     \end{column}
% \end{columns}
%}


\subsubsection{Verfahren gemäß GUM}
\frame{\frametitle{Verfahren zur Absch\"atzung der Messunsicherheit}
\begin{columns}[t] 
     \begin{column}[T]{6cm} 
     \textbf{Verfahren A}
     	\begin{itemize}
     		\item Absch\"atzung aus vorliegender Stichprobe
		\item $n$ Messwerte liegen vor und k\"onnen statistisch ausgewertet werden
		\item Die Messwerte sind n\"aherungsweise normalverteilt
		\begin{equation*}
		u = \frac{s}{\sqrt{n}}
		\end{equation*}
		\item[] $s$: Standardabweichung der Stichprobe
     	\end{itemize}
     \end{column}
     	\begin{column}[T]{6cm} 
	\textbf{Verfahren B}
         	\begin{itemize}
     		\item Ermittlung der minimalen/maximalen Messabweichung $a$ durch Modellbildung
		\item Ermittlung von Grenzen und Form der Verteilung der Messabweichungen
		\begin{equation*}
		u = \frac{a}{k}
		\end{equation*}
		\item[] $k = \sqrt{4}$ f\"ur Normalverteilung
		\item[] $k = \sqrt{3}$ f\"ur Gleichverteilung
		\item[] $k = \sqrt{6}$ f\"ur Dreiecksverteilung
     	\end{itemize}
     \end{column}
 \end{columns}
}



\frame{\frametitle{Sch\"atzfunktionen}
\framesubtitle{Eine Sch\"atzfunktion (Sch\"atzer) dient zur Ermittlung eines Parameter-Sch\"atzwertes aus empirischen Daten}
\begin{columns}[t] 
     \begin{column}[T]{5cm} 
     	\begin{itemize}
     		\item Grundlage: endlich viele Beobachtungen (Stichprobe)
		\begin{itemize}
			\item Sch\"atzer selbst fehlerbehaftet
			\item H\"aufig Zufallsvariable
		\end{itemize}
		\item Schlu{\ss} auf Grundgesamtheit
		\item Sch\"atzen einzelner Parameter der Verteilung
		\begin{itemize}
			\item Mittelwert
			\item Median
			\item Standardabweichung
		\end{itemize}
     	\end{itemize}
     \end{column}
     	\begin{column}[T]{6cm} 
		\begin{definition}[Zufallsvariable]
		Als Zufallsvariable bezeichnet man eine messbare Funktion von einem Wahrscheinlichkeitsraum in einen Messraum.
		\end{definition}
		\begin{definition}[Sch\"atzfunktion]
		Eine Sch\"atzfunktion dient dazu, aufgrund von empirischen Daten einer Stichprobe einen Schätzwert zu ermitteln und dadurch Informationen über unbekannte Parameter einer Grundgesamtheit zu erhalten.
		\end{definition}
     \end{column}
 \end{columns}
}

\frame{\frametitle{Sch\"atzfunktionen und Eigenschaften}
\framesubtitle{G\"angige Sch\"atzfunktionen und w\"unschenswerte Eigenschaften}
\begin{columns}[t] 
     \begin{column}[T]{6cm} 
     	\begin{itemize}
     		\item Mittelwert
		\begin{eqnarray*}
		\bar{X} = \frac{1}{n} \sum_{i=1}^{n} X_{i} \\
		\hat{\mu} = \bar{x} = \frac{1}{n} \sum_{i=1}^{n} x_{i} 
		\end{eqnarray*}
		\item Varianz 
		\begin{eqnarray*}
		S_{n}^{2} = \frac{1}{n-1} \sum_{i=1}^{n}\left(X_{i} - \bar{X}\right)^{2} \\
		\hat{\sigma}^{2} = s_{n}^{2} = \frac{1}{n-1} \sum_{i=1}^{n}\left(x_{i} - \bar{x}\right)^{2}
		\end{eqnarray*}
     	\end{itemize}
     \end{column}
     	\begin{column}[T]{5cm} 
         	\begin{itemize}
     		\item Erwartungstreue:
		\begin{itemize}
		\item Erwartungswert der Sch\"atzfunktion gleich wahrem Parameter
		\item Kein systematischer Fehler (Bias).
		\end{itemize}
		\item Konsistenz:
		\begin{itemize}
		\item Unsicherheit des Sch\"atzers nimmt f\"ur $n \rightarrow \infty$ ab
		\end{itemize}
		\item Effizienz:
		\begin{itemize}
		\item Minimale Varianz des Sch\"atzers
		\end{itemize}
		\item BLUE: Best Linear Unbiased Estimator
     	\end{itemize}
     \end{column}
 \end{columns}
}

%\frame{\frametitle{Standardfehler}
%\framesubtitle{Die Standardabweichung des Mittelwerts nimmt mit der Stichprobengr\"o{\ss}e $\backsim \frac{1}{\sqrt{n}}$ an.}
%     	\begin{itemize}
%     		\item Der Standardfehler des arithmetischen Mittels einer Stichprobe des Umfangs $n$ ist
%		\begin{equation*}
%			\sigma\left(\bar{X}\right) = \frac{\sigma}{\sqrt{n}}
%		\end{equation*}
%		mit $\sigma$ der Standardabweichung einer Einzelmessung.
%		\item Herleitung: 
%		\begin{itemize}
%			\item Mittelwert einer Stichprobe: $\bar{x} = \frac{1}{n} \sum_{i=1}^{n} x_{i}$
%			\item Sch\"atzer: $\bar{X} = \frac{1}{n} \sum_{i=1}^{n} X_{i}$ mit $X_{i}$ unabh\"angig und identisch verteilt mit $\sigma^{2} < \infty$.
%			\item Damit folgt f\"ur die Varianz $\Var\left(\bar{X}\right) = \left(\sigma\left(\bar{X}\right)\right)^{2}$: \tiny
%			\begin{equation*}
%					\Var\left(\bar{X}\right) = \Var\left(\frac{1}{n}\sum_{i=1}^{n} X_{i}\right) = \frac{1}{n^{2}} \Var\left(\sum_{i=1}^{n} X_{i}\right) = \frac{1}{n^{2}} \sum_{i=1}^{n}\Var\left(X_{i}\right) = \frac{1}{n^{2}} n \sigma^{2} = \frac{\sigma^{2}}{n}
%			\end{equation*} \normalsize
%			\item Damit folgt $\sigma\left(\bar{X}\right) = \frac{\sigma}{\sqrt{n}}$
%		\end{itemize}
%     	\end{itemize}
%   }



\frame{\frametitle{Bestimmung der Messunsicherheit durch Modellbildung}
\begin{itemize}
\item Ermitteln der Einflussgr\"o{\ss}en:
\begin{itemize}
\item Ermittlung systematischer Abweichungen, ggf. kompensieren.
\item Einfluss des Normals
\item Wiederholpr\"azision (Messunsicherheit bei wiederholter Messung an einem Messobjekt)
\item Temperatureinfluss
\end{itemize}
\item Modellbildung Messgr\"o{\ss}e $y = f\left(x_{1}, x_{2}, \ldots, x_{n}\right)$
\item Bestimmung kombinierte Standardunsicherheit $u_{c}$
\begin{equation*}
u_{c} = \sqrt{\left(\frac{\partial f}{\partial x_{1}} u_{x_{1}} \right)^{2} + \left(\frac{\partial f}{\partial x_{2}} u_{x_{n}} \right)^{2} + \ldots + \left(\frac{\partial f}{\partial x_{n}} u_{x_{n}}\right)^{2}}
\end{equation*}
\item Vereinfachung f\"ur Linearit\"at von $f$ und gleichem Gewicht der $x_{i}$
\begin{equation*}
u_{c} = \sqrt{u^{2}_{x_{1}} + u^{2}_{x_{2}} + \ldots + u^{2}_{x_{n}}}
\end{equation*}
\end{itemize}
}


\subsubsection{Bestimmung der Messunsicherheit durch Monte Carlo Simulation, Faltungsintegrale}


%\offslide{Warum ein weiteres Verfahren?}

%\frame{\frametitle{Kombinierte Wahrscheinlichkeitsdichte}
%\framesubtitle{}
%\begin{columns}[t] 
%     \begin{column}[T]{6cm} 
%     	\begin{itemize}
%     		\item Experiment: Wurf mit zwei W\"urfeln
%		\item Einzelne W\"urfel gleichverteilt
%		\item Summe dreiecksverteilt
%     	\end{itemize}
%     \end{column}
%     	\begin{column}[T]{6cm} 
%         	\begin{center}
%		 %\pgfplotsset{width=3cm,compat=1.12}
%		\begin{tikzpicture}[scale = 0.5]
%            \begin{axis}[
%                ybar,
%                ymin=0, xmin = 1, xmax = 6
%            ]
%            \addplot +[
%                hist={
%                    bins=6,
%%                    data min=1,
%%                    data max=6
%                }   
%            ] table [y=W1] {Faltung.dat};
%            \end{axis}
%            %\draw[draw = red, ultra thick] (3.41,0) -- (3.41,5.7); 
%            \end{tikzpicture}
%            \begin{tikzpicture}[scale = 0.5]
%            \begin{axis}[
%                ybar,
%                ymin=0, xmin = 1, xmax = 6
%            ]
%            \addplot +[
%                hist={
%                    bins=6,
%%                    data min=1,
%%                    data max=6
%                }   
%            ] table [y=W2] {Faltung.dat};
%            \end{axis}
%            %\draw[draw = red, ultra thick] (3.41,0) -- (3.41,5.7); 
%            \end{tikzpicture}
%
%% \begin{tikzpicture}[scale = 0.8]
%%            \begin{axis}[
%%                ybar,
%%                ymin=0, xmin = 2, xmax = 12
%%            ]
%%            \addplot +[
%%                hist={
%%                    bins=11,
%%%                    data min=1,
%%%                    data max=6
%%                }   
%%            ] table [y=S] {Faltung.dat};
%%            \end{axis}
%%            %\draw[draw = red, ultra thick] (3.41,0) -- (3.41,5.7); 
%%            \end{tikzpicture}
%%
%%        		\end{center}
%%     \end{column}
%% \end{columns}
%%}
%%
%%\offslide{Faltungsintegral}

\frame{\frametitle{Monte-Carlo-Simulation: Motivation}
\framesubtitle{Monte-Carlo-Simulationen (MC-Simulationen) k\"onnen zur Ermittlung kombinierter Wahrscheinlichkeiten eingesetzt werden.}
\begin{columns}[t] 
     \begin{column}[T]{6cm} 
     	\begin{itemize}
     		\item Statistische Verfahren:
		\begin{itemize}
		\item Hohe Stichprobenzahlen f\"ur Validit\"at n\"otig
		\item Zeitaufw\"andig
		\item Teuer
		\end{itemize}
		\item Fehlerfortpflanzung:
		\begin{itemize}
		\item Analytisch teils komplex
		\item Konservativ bei nicht normalverteilten Zufallsvariablen:
		\begin{itemize}
		\item Ann\"aherung Gleichverteilung durch Normalverteilung
		\end{itemize}
		\end{itemize}
	     	\end{itemize}
     \end{column}
     	\begin{column}{6cm} 
         	\begin{center}
            		% GNUPLOT: LaTeX picture with Postscript
\begingroup
  \makeatletter
  \providecommand\color[2][]{%
    \GenericError{(gnuplot) \space\space\space\@spaces}{%
      Package color not loaded in conjunction with
      terminal option `colourtext'%
    }{See the gnuplot documentation for explanation.%
    }{Either use 'blacktext' in gnuplot or load the package
      color.sty in LaTeX.}%
    \renewcommand\color[2][]{}%
  }%
  \providecommand\includegraphics[2][]{%
    \GenericError{(gnuplot) \space\space\space\@spaces}{%
      Package graphicx or graphics not loaded%
    }{See the gnuplot documentation for explanation.%
    }{The gnuplot epslatex terminal needs graphicx.sty or graphics.sty.}%
    \renewcommand\includegraphics[2][]{}%
  }%
  \providecommand\rotatebox[2]{#2}%
  \@ifundefined{ifGPcolor}{%
    \newif\ifGPcolor
    \GPcolorfalse
  }{}%
  \@ifundefined{ifGPblacktext}{%
    \newif\ifGPblacktext
    \GPblacktexttrue
  }{}%
  % define a \g@addto@macro without @ in the name:
  \let\gplgaddtomacro\g@addto@macro
  % define empty templates for all commands taking text:
  \gdef\gplbacktext{}%
  \gdef\gplfronttext{}%
  \makeatother
  \ifGPblacktext
    % no textcolor at all
    \def\colorrgb#1{}%
    \def\colorgray#1{}%
  \else
    % gray or color?
    \ifGPcolor
      \def\colorrgb#1{\color[rgb]{#1}}%
      \def\colorgray#1{\color[gray]{#1}}%
      \expandafter\def\csname LTw\endcsname{\color{white}}%
      \expandafter\def\csname LTb\endcsname{\color{black}}%
      \expandafter\def\csname LTa\endcsname{\color{black}}%
      \expandafter\def\csname LT0\endcsname{\color[rgb]{1,0,0}}%
      \expandafter\def\csname LT1\endcsname{\color[rgb]{0,1,0}}%
      \expandafter\def\csname LT2\endcsname{\color[rgb]{0,0,1}}%
      \expandafter\def\csname LT3\endcsname{\color[rgb]{1,0,1}}%
      \expandafter\def\csname LT4\endcsname{\color[rgb]{0,1,1}}%
      \expandafter\def\csname LT5\endcsname{\color[rgb]{1,1,0}}%
      \expandafter\def\csname LT6\endcsname{\color[rgb]{0,0,0}}%
      \expandafter\def\csname LT7\endcsname{\color[rgb]{1,0.3,0}}%
      \expandafter\def\csname LT8\endcsname{\color[rgb]{0.5,0.5,0.5}}%
    \else
      % gray
      \def\colorrgb#1{\color{black}}%
      \def\colorgray#1{\color[gray]{#1}}%
      \expandafter\def\csname LTw\endcsname{\color{white}}%
      \expandafter\def\csname LTb\endcsname{\color{black}}%
      \expandafter\def\csname LTa\endcsname{\color{black}}%
      \expandafter\def\csname LT0\endcsname{\color{black}}%
      \expandafter\def\csname LT1\endcsname{\color{black}}%
      \expandafter\def\csname LT2\endcsname{\color{black}}%
      \expandafter\def\csname LT3\endcsname{\color{black}}%
      \expandafter\def\csname LT4\endcsname{\color{black}}%
      \expandafter\def\csname LT5\endcsname{\color{black}}%
      \expandafter\def\csname LT6\endcsname{\color{black}}%
      \expandafter\def\csname LT7\endcsname{\color{black}}%
      \expandafter\def\csname LT8\endcsname{\color{black}}%
    \fi
  \fi
  \setlength{\unitlength}{0.0500bp}%
  \begin{picture}(3400.00,3000.00)%
  \tiny
    \gplgaddtomacro\gplbacktext{%
      \colorrgb{0.00,0.00,0.00}%
      \put(322,2335){\makebox(0,0)[r]{\strut{}0}}%
      \colorrgb{0.00,0.00,0.00}%
      \put(322,2562){\makebox(0,0)[r]{\strut{}20}}%
      \colorrgb{0.00,0.00,0.00}%
      \put(322,2790){\makebox(0,0)[r]{\strut{}40}}%
      \colorrgb{0.00,0.00,0.00}%
      \put(322,3017){\makebox(0,0)[r]{\strut{}60}}%
      \colorrgb{0.00,0.00,0.00}%
      \put(322,3244){\makebox(0,0)[r]{\strut{}80}}%
      \colorrgb{0.00,0.00,0.00}%
      \put(322,3472){\makebox(0,0)[r]{\strut{}100}}%
      \colorrgb{0.00,0.00,0.00}%
      \put(322,3699){\makebox(0,0)[r]{\strut{}120}}%
      \colorrgb{0.00,0.00,0.00}%
      \put(442,2135){\makebox(0,0){\strut{}-0.8}}%
      \colorrgb{0.00,0.00,0.00}%
      %\put(584,2135){\makebox(0,0){\strut{}-0.6}}%
      %\colorrgb{0.00,0.00,0.00}%
      \put(726,2135){\makebox(0,0){\strut{}-0.4}}%
      \colorrgb{0.00,0.00,0.00}%
      %\put(868,2135){\makebox(0,0){\strut{}-0.2}}%
      \colorrgb{0.00,0.00,0.00}%
      \put(1011,2135){\makebox(0,0){\strut{}0}}%
      \colorrgb{0.00,0.00,0.00}%
      %\put(1153,2135){\makebox(0,0){\strut{}0.2}}%
      \colorrgb{0.00,0.00,0.00}%
      \put(1295,2135){\makebox(0,0){\strut{}0.4}}%
      \colorrgb{0.00,0.00,0.00}%
      %\put(1437,2135){\makebox(0,0){\strut{}0.6}}%
      \colorrgb{0.00,0.00,0.00}%
      \put(1579,2135){\makebox(0,0){\strut{}0.8}}%
    }%
    \gplgaddtomacro\gplfronttext{%
    }%
    \gplgaddtomacro\gplbacktext{%
      \colorrgb{0.00,0.00,0.00}%
      \put(1819,2335){\makebox(0,0)[r]{\strut{}0}}%
      \colorrgb{0.00,0.00,0.00}%
      \put(1819,2608){\makebox(0,0)[r]{\strut{}50}}%
      \colorrgb{0.00,0.00,0.00}%
      \put(1819,2881){\makebox(0,0)[r]{\strut{}100}}%
      \colorrgb{0.00,0.00,0.00}%
      \put(1819,3153){\makebox(0,0)[r]{\strut{}150}}%
      \colorrgb{0.00,0.00,0.00}%
      \put(1819,3426){\makebox(0,0)[r]{\strut{}200}}%
      \colorrgb{0.00,0.00,0.00}%
      \put(1819,3699){\makebox(0,0)[r]{\strut{}250}}%
      \colorrgb{0.00,0.00,0.00}%
      \put(1939,2135){\makebox(0,0){\strut{}-1.5}}%
      \colorrgb{0.00,0.00,0.00}%
      \put(2166,2135){\makebox(0,0){\strut{}-1}}%
      \colorrgb{0.00,0.00,0.00}%
      \put(2394,2135){\makebox(0,0){\strut{}-0.5}}%
      \colorrgb{0.00,0.00,0.00}%
      \put(2621,2135){\makebox(0,0){\strut{}0}}%
      \colorrgb{0.00,0.00,0.00}%
      \put(2849,2135){\makebox(0,0){\strut{}0.5}}%
      \colorrgb{0.00,0.00,0.00}%
      \put(3076,2135){\makebox(0,0){\strut{}1}}%
    }%
    \gplgaddtomacro\gplfronttext{%
    }%
    \gplgaddtomacro\gplbacktext{%
      \colorrgb{0.00,0.00,0.00}%
      \put(322,440){\makebox(0,0)[r]{\strut{}0}}%
      \colorrgb{0.00,0.00,0.00}%
      \put(322,713){\makebox(0,0)[r]{\strut{}50}}%
      \colorrgb{0.00,0.00,0.00}%
      \put(322,986){\makebox(0,0)[r]{\strut{}100}}%
      \colorrgb{0.00,0.00,0.00}%
      \put(322,1258){\makebox(0,0)[r]{\strut{}150}}%
      \colorrgb{0.00,0.00,0.00}%
      \put(322,1531){\makebox(0,0)[r]{\strut{}200}}%
      \colorrgb{0.00,0.00,0.00}%
      \put(322,1804){\makebox(0,0)[r]{\strut{}250}}%
      \colorrgb{0.00,0.00,0.00}%
      \put(442,240){\makebox(0,0){\strut{}-1.5}}%
      \colorrgb{0.00,0.00,0.00}%
      \put(881,240){\makebox(0,0){\strut{}-1}}%
      \colorrgb{0.00,0.00,0.00}%
      \put(1320,240){\makebox(0,0){\strut{}-0.5}}%
      \colorrgb{0.00,0.00,0.00}%
      \put(1759,240){\makebox(0,0){\strut{}0}}%
      \colorrgb{0.00,0.00,0.00}%
      \put(2198,240){\makebox(0,0){\strut{}0.5}}%
      \colorrgb{0.00,0.00,0.00}%
      \put(2637,240){\makebox(0,0){\strut{}1}}%
      \colorrgb{0.00,0.00,0.00}%
      \put(3076,240){\makebox(0,0){\strut{}1.5}}%
    }%
    \gplgaddtomacro\gplfronttext{%
    }%
    \gplbacktext
    \put(0,0){\includegraphics{MCIntro}}%
    \gplfronttext
  \end{picture}%
\endgroup

        		\end{center}
     \end{column}
 \end{columns}
}

\frame{\frametitle{Monte-Carlo-Simulation: Grundlagen}
\framesubtitle{MC-Simulationen basieren auf der h\"aufig wiederholten Simulation von Zufallsexperimenten.}
\begin{columns}[t] 
     \begin{column}[T]{6cm} 
     	\begin{itemize}
     		\item Grundlage:
		\begin{itemize}
		\item Gesetz der gro{\ss}en Zahlen
		\item Pseudozufallsverteilungen der Parameter erzeugen
		\item Resultierendes Ergebnis bilden
		\item Analyse der Lage- und Streuparameter
		\end{itemize}
		\item Vorteile:
		\begin{itemize}
		\item Einfache Modellierung
		\item Einfache Durchf\"uhrung
		\end{itemize}
		\item Nachteil:
		\begin{itemize}
		\item Korrektheit der L\"osungen nicht immer einfach zu beurteilen
		\end{itemize}
	     	\end{itemize}
     \end{column}
     	\begin{column}{6cm} 
         	\begin{center}
            		% GNUPLOT: LaTeX picture with Postscript
\begingroup
  \makeatletter
  \providecommand\color[2][]{%
    \GenericError{(gnuplot) \space\space\space\@spaces}{%
      Package color not loaded in conjunction with
      terminal option `colourtext'%
    }{See the gnuplot documentation for explanation.%
    }{Either use 'blacktext' in gnuplot or load the package
      color.sty in LaTeX.}%
    \renewcommand\color[2][]{}%
  }%
  \providecommand\includegraphics[2][]{%
    \GenericError{(gnuplot) \space\space\space\@spaces}{%
      Package graphicx or graphics not loaded%
    }{See the gnuplot documentation for explanation.%
    }{The gnuplot epslatex terminal needs graphicx.sty or graphics.sty.}%
    \renewcommand\includegraphics[2][]{}%
  }%
  \providecommand\rotatebox[2]{#2}%
  \@ifundefined{ifGPcolor}{%
    \newif\ifGPcolor
    \GPcolorfalse
  }{}%
  \@ifundefined{ifGPblacktext}{%
    \newif\ifGPblacktext
    \GPblacktexttrue
  }{}%
  % define a \g@addto@macro without @ in the name:
  \let\gplgaddtomacro\g@addto@macro
  % define empty templates for all commands taking text:
  \gdef\gplbacktext{}%
  \gdef\gplfronttext{}%
  \makeatother
  \ifGPblacktext
    % no textcolor at all
    \def\colorrgb#1{}%
    \def\colorgray#1{}%
  \else
    % gray or color?
    \ifGPcolor
      \def\colorrgb#1{\color[rgb]{#1}}%
      \def\colorgray#1{\color[gray]{#1}}%
      \expandafter\def\csname LTw\endcsname{\color{white}}%
      \expandafter\def\csname LTb\endcsname{\color{black}}%
      \expandafter\def\csname LTa\endcsname{\color{black}}%
      \expandafter\def\csname LT0\endcsname{\color[rgb]{1,0,0}}%
      \expandafter\def\csname LT1\endcsname{\color[rgb]{0,1,0}}%
      \expandafter\def\csname LT2\endcsname{\color[rgb]{0,0,1}}%
      \expandafter\def\csname LT3\endcsname{\color[rgb]{1,0,1}}%
      \expandafter\def\csname LT4\endcsname{\color[rgb]{0,1,1}}%
      \expandafter\def\csname LT5\endcsname{\color[rgb]{1,1,0}}%
      \expandafter\def\csname LT6\endcsname{\color[rgb]{0,0,0}}%
      \expandafter\def\csname LT7\endcsname{\color[rgb]{1,0.3,0}}%
      \expandafter\def\csname LT8\endcsname{\color[rgb]{0.5,0.5,0.5}}%
    \else
      % gray
      \def\colorrgb#1{\color{black}}%
      \def\colorgray#1{\color[gray]{#1}}%
      \expandafter\def\csname LTw\endcsname{\color{white}}%
      \expandafter\def\csname LTb\endcsname{\color{black}}%
      \expandafter\def\csname LTa\endcsname{\color{black}}%
      \expandafter\def\csname LT0\endcsname{\color{black}}%
      \expandafter\def\csname LT1\endcsname{\color{black}}%
      \expandafter\def\csname LT2\endcsname{\color{black}}%
      \expandafter\def\csname LT3\endcsname{\color{black}}%
      \expandafter\def\csname LT4\endcsname{\color{black}}%
      \expandafter\def\csname LT5\endcsname{\color{black}}%
      \expandafter\def\csname LT6\endcsname{\color{black}}%
      \expandafter\def\csname LT7\endcsname{\color{black}}%
      \expandafter\def\csname LT8\endcsname{\color{black}}%
    \fi
  \fi
  \setlength{\unitlength}{0.0500bp}%
  \begin{picture}(3400.00,3000.00)%
  \tiny
    \gplgaddtomacro\gplbacktext{%
      \colorrgb{0.00,0.00,0.00}%
      \put(322,2335){\makebox(0,0)[r]{\strut{}0}}%
      \colorrgb{0.00,0.00,0.00}%
      \put(322,2562){\makebox(0,0)[r]{\strut{}20}}%
      \colorrgb{0.00,0.00,0.00}%
      \put(322,2790){\makebox(0,0)[r]{\strut{}40}}%
      \colorrgb{0.00,0.00,0.00}%
      \put(322,3017){\makebox(0,0)[r]{\strut{}60}}%
      \colorrgb{0.00,0.00,0.00}%
      \put(322,3244){\makebox(0,0)[r]{\strut{}80}}%
      \colorrgb{0.00,0.00,0.00}%
      \put(322,3472){\makebox(0,0)[r]{\strut{}100}}%
      \colorrgb{0.00,0.00,0.00}%
      \put(322,3699){\makebox(0,0)[r]{\strut{}120}}%
      \colorrgb{0.00,0.00,0.00}%
      \put(442,2135){\makebox(0,0){\strut{}-0.8}}%
      \colorrgb{0.00,0.00,0.00}%
      %\put(584,2135){\makebox(0,0){\strut{}-0.6}}%
      %\colorrgb{0.00,0.00,0.00}%
      \put(726,2135){\makebox(0,0){\strut{}-0.4}}%
      \colorrgb{0.00,0.00,0.00}%
      %\put(868,2135){\makebox(0,0){\strut{}-0.2}}%
      \colorrgb{0.00,0.00,0.00}%
      \put(1011,2135){\makebox(0,0){\strut{}0}}%
      \colorrgb{0.00,0.00,0.00}%
      %\put(1153,2135){\makebox(0,0){\strut{}0.2}}%
      \colorrgb{0.00,0.00,0.00}%
      \put(1295,2135){\makebox(0,0){\strut{}0.4}}%
      \colorrgb{0.00,0.00,0.00}%
      %\put(1437,2135){\makebox(0,0){\strut{}0.6}}%
      \colorrgb{0.00,0.00,0.00}%
      \put(1579,2135){\makebox(0,0){\strut{}0.8}}%
    }%
    \gplgaddtomacro\gplfronttext{%
    }%
    \gplgaddtomacro\gplbacktext{%
      \colorrgb{0.00,0.00,0.00}%
      \put(1819,2335){\makebox(0,0)[r]{\strut{}0}}%
      \colorrgb{0.00,0.00,0.00}%
      \put(1819,2608){\makebox(0,0)[r]{\strut{}50}}%
      \colorrgb{0.00,0.00,0.00}%
      \put(1819,2881){\makebox(0,0)[r]{\strut{}100}}%
      \colorrgb{0.00,0.00,0.00}%
      \put(1819,3153){\makebox(0,0)[r]{\strut{}150}}%
      \colorrgb{0.00,0.00,0.00}%
      \put(1819,3426){\makebox(0,0)[r]{\strut{}200}}%
      \colorrgb{0.00,0.00,0.00}%
      \put(1819,3699){\makebox(0,0)[r]{\strut{}250}}%
      \colorrgb{0.00,0.00,0.00}%
      \put(1939,2135){\makebox(0,0){\strut{}-1.5}}%
      \colorrgb{0.00,0.00,0.00}%
      \put(2166,2135){\makebox(0,0){\strut{}-1}}%
      \colorrgb{0.00,0.00,0.00}%
      \put(2394,2135){\makebox(0,0){\strut{}-0.5}}%
      \colorrgb{0.00,0.00,0.00}%
      \put(2621,2135){\makebox(0,0){\strut{}0}}%
      \colorrgb{0.00,0.00,0.00}%
      \put(2849,2135){\makebox(0,0){\strut{}0.5}}%
      \colorrgb{0.00,0.00,0.00}%
      \put(3076,2135){\makebox(0,0){\strut{}1}}%
    }%
    \gplgaddtomacro\gplfronttext{%
    }%
    \gplgaddtomacro\gplbacktext{%
      \colorrgb{0.00,0.00,0.00}%
      \put(322,440){\makebox(0,0)[r]{\strut{}0}}%
      \colorrgb{0.00,0.00,0.00}%
      \put(322,713){\makebox(0,0)[r]{\strut{}50}}%
      \colorrgb{0.00,0.00,0.00}%
      \put(322,986){\makebox(0,0)[r]{\strut{}100}}%
      \colorrgb{0.00,0.00,0.00}%
      \put(322,1258){\makebox(0,0)[r]{\strut{}150}}%
      \colorrgb{0.00,0.00,0.00}%
      \put(322,1531){\makebox(0,0)[r]{\strut{}200}}%
      \colorrgb{0.00,0.00,0.00}%
      \put(322,1804){\makebox(0,0)[r]{\strut{}250}}%
      \colorrgb{0.00,0.00,0.00}%
      \put(442,240){\makebox(0,0){\strut{}-1.5}}%
      \colorrgb{0.00,0.00,0.00}%
      \put(881,240){\makebox(0,0){\strut{}-1}}%
      \colorrgb{0.00,0.00,0.00}%
      \put(1320,240){\makebox(0,0){\strut{}-0.5}}%
      \colorrgb{0.00,0.00,0.00}%
      \put(1759,240){\makebox(0,0){\strut{}0}}%
      \colorrgb{0.00,0.00,0.00}%
      \put(2198,240){\makebox(0,0){\strut{}0.5}}%
      \colorrgb{0.00,0.00,0.00}%
      \put(2637,240){\makebox(0,0){\strut{}1}}%
      \colorrgb{0.00,0.00,0.00}%
      \put(3076,240){\makebox(0,0){\strut{}1.5}}%
    }%
    \gplgaddtomacro\gplfronttext{%
    }%
    \gplbacktext
    \put(0,0){\includegraphics{MCIntro}}%
    \gplfronttext
  \end{picture}%
\endgroup

        		\end{center}
     \end{column}
 \end{columns}
}

\frame{\frametitle{Vergleich der Verfahren des GUM und MC-Simulationen}
\framesubtitle{}
\begin{columns}[t] 
     \begin{column}[T]{6cm} 
     \textbf{MC-Simulation}
     	\begin{itemize}
     		\item Eingangsgr\"o{\ss}en $x_{i}$ werden explizit W-Dichtefunktionen zugeordnet
		\item Keine partiellen Ableitungen ben\"otigt, k\"onnen bereitgestellt werden
		\item Keine Einschr\"ankung in Form der Verteilung von $y$
		\item Allgemeine, k\"urzest m\"ogliche, Konfidenzintervalle
     	\end{itemize}
     \end{column}
     	\begin{column}[T]{6cm} 
	\textbf{GUM Verfahren A/B}
         	\begin{itemize}
     		\item Eingangsgr\"o{\ss}en $x_{i}$ werden Sch\"atzwert und Standardunsicherheit zugeordnet
		\item Empfindlichkeitskoeffizienten und Taylor-Approximation ben\"otigt
		\item Beschr\"ankung auf Gau{\ss}- oder t-Verteilung
		\item Um den Sch\"atzwert symmetrische Konfidenzintervalle
     	\end{itemize}
     \end{column}
 \end{columns}
}


\frame{\frametitle{Monte-Carlo-Simulation: Beispiel}
\framesubtitle{Parameter eines Produktes variieren h\"aufig in Form von Gleich- oder Dreiecksverteilungen.}
\begin{columns}[t] 
     \begin{column}[T]{5cm} 
     \begin{center}
     \begin{picture}(120,30)(0,-40)
\thicklines
\scriptsize
{\color{gray!70!black}
\put(15,0){\oval[0](10,30)}
\put(12,18){$h_{1}$}
\put(45,0){\oval[0](50,6)}
\put(42,6){$l$}
\put(71.5,0){\oval[0](3,42)}
\put(68.5,24){$d$}
\put(80,0){\oval[0](64,42)}
\put(125,0){\oval[0](10,30)}
\put(122,18){$h_{2}$}
\put(116,0){\oval[0](8,6)}}


\put(75,20){\line(0,-1){40}}
\put(75,20){\line(1,-8){5}}
\put(80,-20){\line(1,8){5}}
\put(85,20){\line(1,-8){5}}
\put(90,-20){\line(1,8){5}}
\put(95,20){\line(1,-8){5}}
\put(100,-20){\line(1,8){5}}
\put(105,20){\line(1,-8){5}}
\put(110,-20){\line(0,1){40}}
\end{picture}
     \end{center}
     	\begin{itemize}
     		\item Kraft eines Federspeicherzylinders 
		\begin{equation*}
		\begin{split}
			\triangle F &= \\ &c\left(\triangle h_{1} + \triangle l + \triangle d + \triangle h_{2}  \right)
		\end{split}
		\end{equation*}
		\item MC-Simulation mit N = 10000
    	\end{itemize}
     \end{column}
     	\begin{column}{6cm} 
         	\begin{center}
         		\vspace{1cm}
            		% GNUPLOT: LaTeX picture with Postscript
\begingroup
  \makeatletter
  \providecommand\color[2][]{%
    \GenericError{(gnuplot) \space\space\space\@spaces}{%
      Package color not loaded in conjunction with
      terminal option `colourtext'%
    }{See the gnuplot documentation for explanation.%
    }{Either use 'blacktext' in gnuplot or load the package
      color.sty in LaTeX.}%
    \renewcommand\color[2][]{}%
  }%
  \providecommand\includegraphics[2][]{%
    \GenericError{(gnuplot) \space\space\space\@spaces}{%
      Package graphicx or graphics not loaded%
    }{See the gnuplot documentation for explanation.%
    }{The gnuplot epslatex terminal needs graphicx.sty or graphics.sty.}%
    \renewcommand\includegraphics[2][]{}%
  }%
  \providecommand\rotatebox[2]{#2}%
  \@ifundefined{ifGPcolor}{%
    \newif\ifGPcolor
    \GPcolorfalse
  }{}%
  \@ifundefined{ifGPblacktext}{%
    \newif\ifGPblacktext
    \GPblacktexttrue
  }{}%
  % define a \g@addto@macro without @ in the name:
  \let\gplgaddtomacro\g@addto@macro
  % define empty templates for all commands taking text:
  \gdef\gplbacktext{}%
  \gdef\gplfronttext{}%
  \makeatother
  \ifGPblacktext
    % no textcolor at all
    \def\colorrgb#1{}%
    \def\colorgray#1{}%
  \else
    % gray or color?
    \ifGPcolor
      \def\colorrgb#1{\color[rgb]{#1}}%
      \def\colorgray#1{\color[gray]{#1}}%
      \expandafter\def\csname LTw\endcsname{\color{white}}%
      \expandafter\def\csname LTb\endcsname{\color{black}}%
      \expandafter\def\csname LTa\endcsname{\color{black}}%
      \expandafter\def\csname LT0\endcsname{\color[rgb]{1,0,0}}%
      \expandafter\def\csname LT1\endcsname{\color[rgb]{0,1,0}}%
      \expandafter\def\csname LT2\endcsname{\color[rgb]{0,0,1}}%
      \expandafter\def\csname LT3\endcsname{\color[rgb]{1,0,1}}%
      \expandafter\def\csname LT4\endcsname{\color[rgb]{0,1,1}}%
      \expandafter\def\csname LT5\endcsname{\color[rgb]{1,1,0}}%
      \expandafter\def\csname LT6\endcsname{\color[rgb]{0,0,0}}%
      \expandafter\def\csname LT7\endcsname{\color[rgb]{1,0.3,0}}%
      \expandafter\def\csname LT8\endcsname{\color[rgb]{0.5,0.5,0.5}}%
    \else
      % gray
      \def\colorrgb#1{\color{black}}%
      \def\colorgray#1{\color[gray]{#1}}%
      \expandafter\def\csname LTw\endcsname{\color{white}}%
      \expandafter\def\csname LTb\endcsname{\color{black}}%
      \expandafter\def\csname LTa\endcsname{\color{black}}%
      \expandafter\def\csname LT0\endcsname{\color{black}}%
      \expandafter\def\csname LT1\endcsname{\color{black}}%
      \expandafter\def\csname LT2\endcsname{\color{black}}%
      \expandafter\def\csname LT3\endcsname{\color{black}}%
      \expandafter\def\csname LT4\endcsname{\color{black}}%
      \expandafter\def\csname LT5\endcsname{\color{black}}%
      \expandafter\def\csname LT6\endcsname{\color{black}}%
      \expandafter\def\csname LT7\endcsname{\color{black}}%
      \expandafter\def\csname LT8\endcsname{\color{black}}%
    \fi
  \fi
  \setlength{\unitlength}{0.0500bp}%
  \begin{picture}(3400.00,3000.00)%
  \tiny
    \gplgaddtomacro\gplbacktext{%
      \colorrgb{0.00,0.00,0.00}%
%      \put(322,3069){\makebox(0,0)[r]{\strut{}0}}%
%      \colorrgb{0.00,0.00,0.00}%
%      \put(322,3174){\makebox(0,0)[r]{\strut{}200}}%
%      \colorrgb{0.00,0.00,0.00}%
%      \put(322,3279){\makebox(0,0)[r]{\strut{}400}}%
%      \colorrgb{0.00,0.00,0.00}%
%      \put(322,3384){\makebox(0,0)[r]{\strut{}600}}%
%      \colorrgb{0.00,0.00,0.00}%
%      \put(322,3489){\makebox(0,0)[r]{\strut{}800}}%
%      \colorrgb{0.00,0.00,0.00}%
%      \put(322,3594){\makebox(0,0)[r]{\strut{}1000}}%
%      \colorrgb{0.00,0.00,0.00}%
%      \put(322,3699){\makebox(0,0)[r]{\strut{}1200}}%
      \colorrgb{0.00,0.00,0.00}%
      \put(442,2969){\makebox(0,0){\strut{}-0.2}}%
      \colorrgb{0.00,0.00,0.00}%
%      \put(584,2869){\makebox(0,0){\strut{}-0.15}}%
      \colorrgb{0.00,0.00,0.00}%
%      \put(726,2869){\makebox(0,0){\strut{}-0.1}}%
      \colorrgb{0.00,0.00,0.00}%
%      \put(868,2869){\makebox(0,0){\strut{}-0.05}}%
      \colorrgb{0.00,0.00,0.00}%
      \put(1011,2969){\makebox(0,0){\strut{}0}}%
%      \colorrgb{0.00,0.00,0.00}%
%      \put(1153,2869){\makebox(0,0){\strut{}0.05}}%
      \colorrgb{0.00,0.00,0.00}%
%      \put(1295,2869){\makebox(0,0){\strut{}0.1}}%
      \colorrgb{0.00,0.00,0.00}%
%      \put(1437,2869){\makebox(0,0){\strut{}0.15}}%
      \colorrgb{0.00,0.00,0.00}%
      \put(1579,2969){\makebox(0,0){\strut{}0.2}}%
      \csname LTb\endcsname%
      \put(1010,3800){\makebox(0,0){\strut{}Kolbenstange $l$}}%
    }%
    \gplgaddtomacro\gplfronttext{%
    }%
    \gplgaddtomacro\gplbacktext{%
%      \colorrgb{0.00,0.00,0.00}%
%      \put(1819,3069){\makebox(0,0)[r]{\strut{}0}}%
%      \colorrgb{0.00,0.00,0.00}%
%      \put(1819,3174){\makebox(0,0)[r]{\strut{}200}}%
%      \colorrgb{0.00,0.00,0.00}%
%      \put(1819,3279){\makebox(0,0)[r]{\strut{}400}}%
%      \colorrgb{0.00,0.00,0.00}%
%      \put(1819,3384){\makebox(0,0)[r]{\strut{}600}}%
%      \colorrgb{0.00,0.00,0.00}%
%      \put(1819,3489){\makebox(0,0)[r]{\strut{}800}}%
%      \colorrgb{0.00,0.00,0.00}%
%      \put(1819,3594){\makebox(0,0)[r]{\strut{}1000}}%
%      \colorrgb{0.00,0.00,0.00}%
%      \put(1819,3699){\makebox(0,0)[r]{\strut{}1200}}%
%      \colorrgb{0.00,0.00,0.00}%
      \put(1939,2969){\makebox(0,0){\strut{}0}}%
%      \colorrgb{0.00,0.00,0.00}%
%      \put(2166,2869){\makebox(0,0){\strut{}0.02}}%
      \colorrgb{0.00,0.00,0.00}%
%      \put(2394,2969){\makebox(0,0){\strut{}0.04}}%
      \colorrgb{0.00,0.00,0.00}%
%      \put(2621,2869){\makebox(0,0){\strut{}0.06}}%
      \colorrgb{0.00,0.00,0.00}%
%      \put(2849,2869){\makebox(0,0){\strut{}0.08}}%
%      \colorrgb{0.00,0.00,0.00}%
      \put(3076,2969){\makebox(0,0){\strut{}0.1}}%
      \csname LTb\endcsname%
      \put(2507,3800){\makebox(0,0){\strut{}Kolben $d$}}%
    }%
    \gplgaddtomacro\gplfronttext{%
    }%
    \gplgaddtomacro\gplbacktext{%
      \colorrgb{0.00,0.00,0.00}%
%      \put(322,2192){\makebox(0,0)[r]{\strut{}0}}%
%      \colorrgb{0.00,0.00,0.00}%
%      \put(322,2297){\makebox(0,0)[r]{\strut{}500}}%
%      \colorrgb{0.00,0.00,0.00}%
%      \put(322,2402){\makebox(0,0)[r]{\strut{}1000}}%
%      \colorrgb{0.00,0.00,0.00}%
%      \put(322,2507){\makebox(0,0)[r]{\strut{}1500}}%
%      \colorrgb{0.00,0.00,0.00}%
%      \put(322,2612){\makebox(0,0)[r]{\strut{}2000}}%
%      \colorrgb{0.00,0.00,0.00}%
%      \put(322,2717){\makebox(0,0)[r]{\strut{}2500}}%
%      \colorrgb{0.00,0.00,0.00}%
%      \put(322,2822){\makebox(0,0)[r]{\strut{}3000}}%
      \colorrgb{0.00,0.00,0.00}%
      \put(442,2092){\makebox(0,0){\strut{}-0.4}}%
%      \colorrgb{0.00,0.00,0.00}%
%      \put(584,1992){\makebox(0,0){\strut{}-0.3}}%
%      \colorrgb{0.00,0.00,0.00}%
%      \put(726,1992){\makebox(0,0){\strut{}-0.2}}%
%      \colorrgb{0.00,0.00,0.00}%
%      \put(868,1992){\makebox(0,0){\strut{}-0.1}}%
%      \colorrgb{0.00,0.00,0.00}%
      \put(1011,2092){\makebox(0,0){\strut{}0}}%
      \colorrgb{0.00,0.00,0.00}%
%      \put(1153,1992){\makebox(0,0){\strut{}0.1}}%
%      \colorrgb{0.00,0.00,0.00}%
%      \put(1295,1992){\makebox(0,0){\strut{}0.2}}%
%      \colorrgb{0.00,0.00,0.00}%
%      \put(1437,1992){\makebox(0,0){\strut{}0.3}}%
      \colorrgb{0.00,0.00,0.00}%
      \put(1579,2092){\makebox(0,0){\strut{}0.4}}%
      \csname LTb\endcsname%
      \put(1010,2880){\makebox(0,0){\strut{}Federrate $c$}}%
    }%
    \gplgaddtomacro\gplfronttext{%
    }%
    \gplgaddtomacro\gplbacktext{%
      \colorrgb{0.00,0.00,0.00}%
%      \put(1819,2192){\makebox(0,0)[r]{\strut{}0}}%
%      \colorrgb{0.00,0.00,0.00}%
%      \put(1819,2297){\makebox(0,0)[r]{\strut{}200}}%
%      \colorrgb{0.00,0.00,0.00}%
%      \put(1819,2402){\makebox(0,0)[r]{\strut{}400}}%
%      \colorrgb{0.00,0.00,0.00}%
%      \put(1819,2507){\makebox(0,0)[r]{\strut{}600}}%
%      \colorrgb{0.00,0.00,0.00}%
%      \put(1819,2612){\makebox(0,0)[r]{\strut{}800}}%
%      \colorrgb{0.00,0.00,0.00}%
%      \put(1819,2717){\makebox(0,0)[r]{\strut{}1000}}%
%      \colorrgb{0.00,0.00,0.00}%
%      \put(1819,2822){\makebox(0,0)[r]{\strut{}1200}}%
      \colorrgb{0.00,0.00,0.00}%
      \put(1939,2092){\makebox(0,0){\strut{}-0.1}}%
      \colorrgb{0.00,0.00,0.00}%
%      \put(2223,1992){\makebox(0,0){\strut{}-0.05}}%
      \colorrgb{0.00,0.00,0.00}%
      \put(2508,2092){\makebox(0,0){\strut{}0}}%
      \colorrgb{0.00,0.00,0.00}%
%      \put(2792,1992){\makebox(0,0){\strut{}0.05}}%
      \colorrgb{0.00,0.00,0.00}%
      \put(3076,2092){\makebox(0,0){\strut{}0.1}}%
      \csname LTb\endcsname%
      \put(2507,2880){\makebox(0,0){\strut{}Hammerkopf $h_{1,2}$}}%
    }%
    \gplgaddtomacro\gplfronttext{%
    }%
    \gplgaddtomacro\gplbacktext{%
      \colorrgb{0.00,0.00,0.00}%
      \put(422,440){\makebox(0,0)[r]{\strut{}0}}%
      \colorrgb{0.00,0.00,0.00}%
      \put(422,817){\makebox(0,0)[r]{\strut{}500}}%
      \colorrgb{0.00,0.00,0.00}%
      \put(422,1193){\makebox(0,0)[r]{\strut{}1000}}%
      \colorrgb{0.00,0.00,0.00}%
      \put(422,1570){\makebox(0,0)[r]{\strut{}1500}}%
      \colorrgb{0.00,0.00,0.00}%
      \put(422,1946){\makebox(0,0)[r]{\strut{}2000}}%
      \colorrgb{0.00,0.00,0.00}%
      \put(442,340){\makebox(0,0){\strut{}-4}}%
      \colorrgb{0.00,0.00,0.00}%
      \put(818,340){\makebox(0,0){\strut{}-3}}%
      \colorrgb{0.00,0.00,0.00}%
      \put(1195,340){\makebox(0,0){\strut{}-2}}%
      \colorrgb{0.00,0.00,0.00}%
      \put(1571,340){\makebox(0,0){\strut{}-1}}%
      \colorrgb{0.00,0.00,0.00}%
      \put(1947,340){\makebox(0,0){\strut{}0}}%
      \colorrgb{0.00,0.00,0.00}%
      \put(2323,340){\makebox(0,0){\strut{}1}}%
      \colorrgb{0.00,0.00,0.00}%
      \put(2700,340){\makebox(0,0){\strut{}2}}%
      \colorrgb{0.00,0.00,0.00}%
      \put(3076,340){\makebox(0,0){\strut{}3}}%
      \csname LTb\endcsname%
      \put(1759,2000){\makebox(0,0){\strut{}Kraftabweichung $\triangle F$/kN}}%
    }%
    \gplgaddtomacro\gplfronttext{%
    }%
    \gplbacktext
    \put(0,0){\includegraphics{MCExample}}%
    \gplfronttext
  \end{picture}%
\endgroup

        		\end{center}
		
     \end{column}
 \end{columns}
}




\frame{\frametitle{MC-Simulation zur Bestimmung der Messunsicherheit}
\framesubtitle{Nicht normatives Beiblatt 1 zu DIN V ENV 13005}
\begin{itemize}
\item MC-Simulation als Erg\"anzung zu Verfahren A und B des ``GUM'', z.B. falls
\begin{itemize}
		\item Linearisierung des Modells zu unangemessener Darstellung f\"uhrt, z.B. durch Sensibilit\"at
		\item Partielle Ableitungen schwierig oder unm\"oglich zu finden sind
		\item die Wahrscheinlichkeitsdichte merklich von Gau{\ss}- oder $t$-Verteilung abweicht , z.B. durch Asymmetrie
		\item Wiederholte Versuche zur Bestimmung der Messunsicherheit nicht m\"oglich sind
		\item Modell des Messprozesses nicht in explizite Form gebracht werden kann
		\item Unsicherheitsbeitr\"age nicht n\"aherungsweise von der gleichen Gr\"o{\ss}enordnung sind 
		\end{itemize}
	\item MC-Simulation ist im Einklang mit GUM
	\begin{itemize}
		\item Ermittlung von Erwartungswert, Standardunsicherheit und \"Uberdeckungsintervall der Messgr\"o{\ss}e $y$
		\end{itemize}
\end{itemize}
}


\frame{\frametitle{Ablauf Erstellung einer MC-Simulation}
\framesubtitle{}
     	\begin{enumerate}[1)]
     		\item Modellierung des Systems:
		\begin{enumerate}[a)]
		\item Ausgangsgr\"o{\ss}e $y$
		\item Eingangsgr\"o{\ss}en $\mathbf{x} = \left(x_{1}, \ldots, x_{i}\right)^{T}$
		\item Modell als Beziehung zwischen $y$ und $\mathbf{x}$, nicht zwingend explizit
		\item Zuordnung von Wahrscheinlichkeitsdichtefunktionen zu den $x_{i}$
		\begin{itemize}
		\item $x_{i}$ unabh\"angig: individuelle W-Verteilungen, z.B. Gau{\ss}-, Gleichverteilung etc.
		\item $x_{i}, x_{j}$ abh\"angig: Gemeinsame W-Verteilung
		\end{itemize}
		\end{enumerate}
		\item Fortpflanzung: Simulation des Systems
		\begin{enumerate}[a)]
		\item Bestimmung des Wertes $y$ aus den $x_{i}$
		\item Werte der $x_{i}$ als Pseudozufallszahlen
		\end{enumerate}
		\item Zusammenfassung:
		\begin{enumerate}[a)]
		\item Bestimmung des Erwartungswerts $\E(y)$\footnote{Nicht alle W-Verteilungen weisen einen Erwartungswert auf.}
		\item Bestimmung der Standardabweichung $\sigma_{y}$\footnote{Nicht alle W-Verteilungen weisen eine Standardabweichung auf.}
		\item Bestimmung des Konfidenzintervalls (\"Uberdeckungsintervalls), das $y$ mit einer festgelegten Wahrscheinlichlichkeit enth\"alt
		\end{enumerate}
     	\end{enumerate}
}

\frame{\frametitle{Bestimmung der Fortpflanzung}
\framesubtitle{Neben analytischen und statistischen Methoden sind MC-Simulationen zul\"assig und effizient. Verschiedene W-Verteilungen und nichtlineare Systemfunktionen machen sie notwendig.}
\begin{picture}(340,170)(-130,-85)
\thicklines
\put(-70,-15){
\put(-25,0){\vector(1,0){50}}
\put(0,0){\vector(0,1){30}}
\put(-15,0){\line(0,1){15}}
\put(15,0){\line(0,1){15}}
\put(-15,15){\line(1,0){30}}
\put(40,10){\vector(1,0){30}}
\put(-5, -8){$x_{2}$}
}
\put(-70,30){
\put(-25,0){\vector(1,0){50}}
\put(0,0){\vector(0,1){30}}
\put(-15,0){\line(1,1){15}}
\put(15,0){\line(-1,1){15}}
\put(40,10){\vector(1,0){30}}
\put(-5, -8){$x_{1}$}
}

\put(-70,-60){
\put(-25,0){\vector(1,0){50}}
\put(0,0){\vector(0,1){30}}
\qbezier(-20,0)(-12,2)(-8,10)
\qbezier(-8,10)(-4,21)(0,23)
\qbezier(20,0)(12,2)(8,10)
\qbezier(8,10)(4,21)(0,23)
\put(40,10){\vector(1,0){30}}
\put(-5, -8){$x_{3}$}
}

\put(150,-15){
\put(-40,0){\vector(1,0){80}}
\put(0,0){\vector(0,1){50}}
\qbezier(-20,0)(-10,0)(-8,10)
\qbezier(-8,10)(-3,40)(0,40)
\qbezier(40,0)(23,0)(20,10)
\qbezier(20,10)(10,40)(0,40)
\put(-70,10){\vector(1,0){30}}
\put(-3, -8){$y$}
}


\put(40,-5){\oval[3](60,120)}
\put(20,-8){$y = f\left(\mathbf{x}\right)$}
\end{picture}
}


\frame{\frametitle{W-Dichtefunktionen f\"ur die Eingangsgr\"o{\ss}en}
\framesubtitle{}
\begin{itemize}
\item Allgemein: $\mathbf{x}$ wird gemeinsame PDF zugeordnet
\item Unabh\"angigen $x_{i}$ werden einzelne PDF zugewiesen
\item Allgemein: Bestimmung PDF gem\"a{\ss} Bayes-Theorem oder Prinzip der maximalen Entropie
\item Praktisch h\"aufig auftretende F\"alle: \cite[S. 34f.]{en13005}
	\begin{itemize}
		\item Rechteckverteilung: Toleranzb\"ander
		\item $\arcsin$-Verteilung (U-f\"ormig): Sinusf\"ormige periodische Schwingungen
		\item Gau{\ss}-Verteilung: Schwankung durch viele Einflussfaktoren (Zentraler GWS)
		\item $t$-Verteilung: Modellierung aus endlich vielen Anzeigewerten
		\item Exponential-Verteilung: Linkssteile Verteilung f\"ur positive Gr\"o{\ss}en
	\end{itemize}
\end{itemize}
}


\frame{\frametitle{Voraussetzungen f\"ur Anwendbarkeit und G\"ultigkeit der MC-Simulation}
\framesubtitle{}
\begin{itemize}
\item Es ist $f$ stetig bez\"uglich der $x_{i}$ in der Nachbarschaft der besten Sch\"atzwerte $\hat{x}_{i}$
\item Die Verteilungsfunktion f\"ur $y$ ist streng wachsend
\item Die Wahrscheinlichkeitsdichtefunktion (PDF) f\"ur $y$ ist
	\begin{itemize}
		\item stetig \"uber dem Intervall, f\"ur das die PDF streng positiv ist
		\item unimodal
		\item streng wachsend (oder 0) links vom Modalwert und streng fallend (oder 0) rechts vom Modalwert
		\end{itemize}
	\item Es existieren $\E(y)$ und $\Var(y)$
	\begin{itemize}
		\item[$\rightarrow$] Gew\"ahrleistet stochastische Konvergenz mit steigendem $M$ 
	\end{itemize}
	\item Es ist die Zahl der Monte-Carlo-Versuche $M$ ausreichend gro{\ss} gew\"ahlt
	\begin{itemize}
		\item[$\rightarrow$] Gew\"ahrleistet Zuverl\"assigkeit der stochastischen Information
		\end{itemize}
\end{itemize}
}

%\offslide{Weitere Anwendungen der MC-Simulation}

\frame{\frametitle{Praktische Umsetzung einer MC-Simulation}
\framesubtitle{Die hohe Anzahl Versuche $M$ macht effiziente Software n\"otig.}
\begin{enumerate}
%\item Software: Hohe Anzahl Versuche $M$ n\"otig, daher Effizienz wichtig
\item $M$ w\"ahlen:
	\begin{itemize}
		\item Da MC-Simulationen Zufallsexperimente sind, kann kein festes $M$ Korrektheit des Algorithmus garantieren
		\item F\"ur Signifikanzniveau $\alpha$: $M \gg \frac{1}{1-\alpha}$, z.B. $M \geq 10^{4} \frac{1}{1-\alpha}$
		\item Alternativ: Adaptive Verfahren
	\end{itemize}
	\item Erzeugung von Zufallszahlen f\"ur $\mathbf{x}_{r}$ aus den gew\"ahlten PDF
	\begin{itemize}
		\item Pseudozufallszahlengenerator muss \"uber gewisse Eigenschaften verf\"ugen, z.B. Sequenzl\"ange
		\end{itemize}
	\item Bestimmung der Werte $y_{r} = f\left(\mathbf{x}_{r}\right)$ f\"ur $r = 1, \ldots, M$
	\item Diskrete Darstellung der Verteilungsfunktion f\"ur die Ausgangsgr\"o{\ss}e
	\item Sch\"atzung der Ausgangsr\"o{\ss}e und der beigeordneten Standardunsicherheit
	\begin{itemize}
		\item $\bar{y} = \frac{1}{M} \sum_{i = 1}^{M} y_{i}$
		\item $s^{2}\left(\bar{y}\right) = \frac{1}{M-1} \sum_{i = 1}^{M} \left(y_{i} - \bar{y}\right)^{2} $
	\end{itemize}
\end{enumerate}
}


%Messmethoden, -strategien, Lehrende Pr\"ufung
% !TEX root = ../SFV15001_MdQM_Fertigungsmesstechnik_Rev01.tex
%
%\frame{\frametitle{Messmethoden}
%\framesubtitle{Unterschied direkte und indirekte Messmethoden}
%\begin{columns}[t] 
%     \begin{column}[T]{6cm} 
%     	\begin{itemize}
%     		\item Direkt:   ohne Hilfsgr\"o{\ss}e die gesuchte Gr\"o{\ss}e
%		
%		\begin{itemize}
%			\item Vergleichend:   Direkter Vergleich mit bekannter Ma{\ss}verk\"orperung
%			\item Ausschlag:   Unmittelbare Anzeige des Gr\"o{\ss}enwertes
%		\end{itemize}
%		\item Indirekt:   Messgr\"o{\ss}e \"uber Hilfsgr\"o{\ss}e  
%		\begin{itemize}
%			\item Hilfsgr\"o{\ss}e: z.B. Widerstand, Volumenstrom
%		\end{itemize}
%     	\end{itemize}
%	\begin{center}
%	\hspace{2cm}\includegraphics[width=0.5\textwidth]{Indirekt}\\
%			\tiny Indirektes Messen eines Innendurchmessers
%			\end{center}
%     \end{column}
%     	\begin{column}[T]{7cm} 
%         	\begin{center}
%            		\includegraphics[width=0.95\textwidth]{DirektIndirekt}\\
%			\tiny Vergleichendes Messen\\
%			\includegraphics[width=0.7\textwidth]{Ausschlag}\\
%			\tiny Ausschlagsmethode
%        		\end{center}
%     \end{column}
% \end{columns}
%}
%
%\frame{\frametitle{Messstrategie}
%\framesubtitle{Neben Messverfahren und Messmethode bestimmt die Antaststrategie ma{\ss}geblich die Messstrategie.}
%     	\begin{itemize}
%     		\item 1-Punkt-Antastung  
%		\begin{itemize}
%			\item Gemeinsamer Bezug (Fl\"ache) zwischen Messmittel und Messgr\"o{\ss}e  
%			\item Beispiel: Tiefenma{\ss}, Durchmesserbestimmung auf Drehtisch  
%		\end{itemize}
%		\item 2-Punkt-Antastung  
%		\begin{itemize}
%			\item Antastung an zwei Fl\"achen, am Schnittpunkt einer Durchsto{\ss}ungsgeraden  
%			\item Beispiel: B\"ugelmessschraube, Messschieber  
%		\end{itemize}
%		\item 3-Punkt-Antastung  
%		\begin{itemize}
%			\item Durchmesserbestimmung: 3 Punkte am Umfang definieren Kreis eindeutig  
%			\item Beispiel: Durchmesserbestimmung mit Prisma
%		\end{itemize}
%     	\end{itemize}
%}
%
%
%\section{Lehrende Pr\"ufung}
%\label{Sec:Lehrend}
%
%\frame{\frametitle{Lehrende Pr\"ufung}
%\framesubtitle{Effiziente Pr\"ufung, ob ein Merkmal in der vorgegebenen Toleranz liegt.}
%\begin{columns}[t] 
%     \begin{column}[T]{6cm} 
%     	\begin{itemize}
%     		\item Schnell
%		\item Nur positiv/negativ Entscheidung
%		\item Ma{\ss}-, Form., Lage- und Sonderlehren:
%		\begin{itemize}
%		\item Sollma{\ss} und -form
%		\item Evtl. einseitiges Grenzma{\ss}
%		\end{itemize}
%		\item Grenzlehren stets zweiseitig oder zweiteilig
%		\begin{itemize}
%		\item Gutseite
%		\item Ausschussseite
%		\end{itemize}
%     	\end{itemize}
%     \end{column}
%     	\begin{column}[T]{5cm} 
%         	\begin{itemize}
%		\item Ma{\ss}lehrung
%		\begin{itemize}
%		\item Pr\"ufstifte, F\"uhlerlehre, Rachenlehre, Lehrdorn
%		\end{itemize}
%		\item Formlehrung
%		\begin{itemize}
%		\item Haarlineal, Profilformlehre
%		\end{itemize}
%		\item Lagelehrung
%		\begin{itemize}
%		\item Winkellehre, Bohrbild
%		\end{itemize}
%		\item Grenzlehrung
%		\begin{itemize}
%		\item Passungslehre
%		\end{itemize}
%		\end{itemize}
%     \end{column}
% \end{columns}
%}
%
%\frame{\frametitle{Taylorscher Grundsatz}
%\framesubtitle{}
%\begin{columns}[t] 
%     \begin{column}[T]{6cm} 
%     	\begin{alertblock}{\color{green!70!black}Gutlehre}
%		Die Gutlehre soll so ausgebildet sein, dass sie die zu pr\"ufende Form in ihrer Gesamtwirkung beurteilt.
%	\end{alertblock}
%	\begin{alertblock}{\color{red!80!black}Ausschusslehre}
%	Die Ausschusslehre soll nur einzelne Bestimmungsst\"ucke der geometrischen Form des Werkst\"ucks pr\"ufen.
%	\end{alertblock}
%     \end{column}
%     	\begin{column}[T]{5cm} 
%         	\begin{center}
%		\only<1>{\includegraphics[width=0.8\textwidth]{BohrungTol}  \begin{center} \tiny
%		Bohrung mit {\color{green!60!black}Kleinstma{\ss}} und {\color{blue!70!black}Toleranz} \end{center}}
%		\only<2>{\includegraphics[width=0.8\textwidth]{Gutlehre} \begin{center} \tiny
%		{\color{green!70!black}Gutlehre} \end{center}}
%		\only<3>{\includegraphics[width=0.8\textwidth]{Ausschusslehre} \begin{center} \tiny
%		{\color{red!80!black}Ausschusslehre} \end{center}}
%        		\end{center}
%     \end{column}
% \end{columns}
%}
%
%
%
%\frame{\frametitle{Ergebnisse der lehrenden Pr\"ufung}
%\framesubtitle{Ergebnisse der lehrenden Pr\"ufung sind mittels diskreter W-Verteilungen zu modellieren.}
%\begin{columns}[t] 
%     \begin{column}[T]{5cm} 
%     	\begin{itemize}
%     		\item Ergebnisse qualitativ, zwei m\"ogliche Ausg\"ange:
%		\begin{itemize}
%		\item Werkst\"uck i.O.
%		\item Werkst\"uck n.i.O.
%		\end{itemize}
%		\item 100\%-Pr\"ufung:
%		\begin{itemize}
%		\item Direkte Entscheidung
%		\end{itemize}
%		\item Stichprobe:  
%		\item Deskriptiv: 
%		\begin{itemize}
%		\item Diagramm
%		\item Quantitativ
%		\end{itemize}
%		\item Induktiv:
%		\begin{itemize}
%		\item Schluss auf Grundgesamtheit
%		\end{itemize}
%     	\end{itemize}
%     \end{column}
%     	\begin{column}[T]{6cm}
%		%\scriptsize 
%         	\textbf{Beispiel:}\\
%		Von einem Produkt mit $p^{\star} = 0{,}01$ Ausschussquote wird eine Stichprobe mit Gr\"o{\ss}e $n = 50$ untersucht. Es werden 
%		\begin{enumerate}
%		\item 1 defektes Teil
%		\item 3 defekte Teile 
%		\end{enumerate}
%		gefunden. Die Binomial-Verteilung
%		\begin{equation*}
%			p(k) = \binom{n}{k} \left(p^{\star}\right)^{k}\left(1 - p^{\star}\right)^{n-k}
%		\end{equation*}
%		ergibt Werte von
%		 $p(1) = 0{,}3056$ und $p(3) = 0{,}0122$.
%     \end{column}
% \end{columns}
%}
%
%\frame{\frametitle{Nomenklatur Hypothesentests und Pr\"ufpl\"ane \cite{kahle13}}
%\framesubtitle{}
%\begin{itemize}
%\item $c:$ Annahmezahl, $c \in \mathbb{N}$
%\item $n$: Stichprobenumfang, $n > c$, $n\geq1$, $n \in \mathbb{N}$
%\item $N$: Anzahl Teile im Los, $n < N$, $N \in \mathbb{N}$
%\item $M$: Ausschussteile im gesamten Los, $M \in \mathbb{N}$
%\item $p^{\star}$: Obere Schranke Ausschussanteil
%\item $p = \frac{M}{N}$: Anteil Ausschussteile im Los
%\item $X$: Anzahl Ausschussteile in Stichprobe, $X \in \mathbb{N}$
%\item Quality Levels:
%	\begin{itemize}
%		\item AQL: Acceptable Quality Level
%		\item IQL: Indifferent Quality Level
%		\item RQL: Rejectable Quality Level
%	\end{itemize}
%\item Risiken:
%\begin{itemize}
%		\item $\alpha$: Herstellerrisiko: Fehlerhafte Ablehnung einer konformen Charge 
%		\item $\beta$: Abnehmerrisiko: Fehlerhafte Annahme einer nicht konformen Charge
%		\end{itemize}
%\end{itemize}
%}
%
%\frame{\frametitle{Hypergeometrische Verteilung}
%\framesubtitle{}
%\begin{itemize}
%\item Dichotome (d.h. mit genau zwei Eigenschaften ausgestattete) Grundgesamtheit
%\begin{itemize}
%	\item z.B. Urne mit Kugeln in zwei Farben
%\end{itemize}
%\item Ziehen ohne Zur\"ucklegen
%\begin{itemize}
%	\item Wahrscheinlichkeit, Kugel einer Farbe zu ziehen ist nicht konstant
%\end{itemize}
%\item F\"ur nat\"urliche Zahlen 
%	\begin{itemize}
%		\item $N$: Gr\"o{\ss}e der Grundgesamtheit
%		\item $M \leq N$ Anzahl Elemente mit Eigenschaft $A$
%		\item $n \leq N$ Gr\"o{\ss}e der Stichprobe und 
%		\item $k\leq n$ Anzahl der Elemente mit Eigenschaft $A$ in der Stichprobe
%	\end{itemize}
%\begin{equation*}
%p(k) = \frac{\binom{M}{k}\binom{N-M}{n-k}}{\binom{N}{n}}
%\end{equation*}
%\item Hoher Rechenaufwand, daher h\"aufig Ann\"aherung durch Binomialverteilung f\"ur $n \ll N$
%\end{itemize}
%}
%
%
%\frame{\frametitle{Hypothesentest}
%\framesubtitle{Um \"uber Ablehnung oder Annahme einer Charge aufgrund einer Stichprobe zu entscheiden, werden Hypothesentests eingesetzt.}
%\begin{columns}[t] 
%     \begin{column}[T]{6cm}
%\begin{itemize}
%\item Nullhypothese $H_{0}$: 
%	\begin{itemize}
%		\item i.d.R. Hypothese, dass das Los konform ist, d.h.\\
%		$H_{0}:\, p \leq p^{\star}$
%		
%		\item f\"ur sicherheitskritische Teile auch Gegenteil
%	\end{itemize}
%\item Alternativhypothese $H_{1}$:
%	\begin{itemize}
%		\item H\"aufig Gegenhypothese zu $H_{0}$, d.h. \\
%		$H_{1}:\, p > p^{\star}$
%	\end{itemize}
%\item Festlegung Signifikanzniveau $\alpha$, typisch 5\% oder 10\%
%\end{itemize}
%\end{column}
%     	\begin{column}[T]{6cm} 
%	\scriptsize
%         	\textbf{Beispiel:}\\ F\"ur Daten wie oben gilt:
%		\begin{itemize}
%		\item $H_{0}$: Charge weist $\leq 1\%$ Ausschuss auf
%		\item $H_{1}$: $H_{0}$ negiert 
%		\item $\alpha = 5\%$
%		\end{itemize}
%		Die kumulierten Wahrscheinlichkeiten laut Binomialverteilung sind:\\ \vspace{0.1cm}
%		\begin{tabular}{|c|c|c|c|c|}
%		\hline
%		$k$ & 0 & 1 & 2& 3 \\ \hline
%		$p(k)$ & 0{,}605 & 0{,}910 & 0,986 & 0{,}988 \\\hline
%		\end{tabular} \\ \vspace{0.1cm}
%		Wegen $p(2) > 0{,}95 = 1-\alpha$ sollen nur Stichproben mit 0 oder 1 defekten Teilen vom Abnehmer angenommen werden, bei 2 oder mehr defekten Teilen wird die Charge abnehmerseitig verworfen.
%     \end{column}
% \end{columns}
%}
%\frame{\frametitle{Pr\"ufpl\"ane und ihre Operationcharakteristik }
%\framesubtitle{Die Operationscharakteristik (OC) stellt die Wahrscheinlichkeit dar, ein Los auf Grundlage einer Pr\"ufstrategie anzunehmen \cite[S. 274ff.]{kahle13}.}
%\begin{columns}[t] 
%     \begin{column}[T]{6.5cm} 
%     	\begin{itemize}
%     		\item OC-Funktion:
%		\begin{equation*}
%		\begin{split}
%			L\left(p, N, N, c \right) = \mathbb{P}\left\{X \leq c \right\} =\\
%			\sum_{i=0}^{c} p_{i}\left(N, M, n\right)
%		\end{split}
%		\end{equation*}
%		\begin{itemize}
%		\item $p_{i}$ mittels Hypergeometrischer Verteilung
%		\item Binomialverteilung als N\"aherung f\"ur $n\ll N$
%		\end{itemize}
%		\item Eigenschaften von $L$:
%		\begin{itemize}
%		\item $L\left(0, N, n, c \right) = 1$
%		\item $L\left(1, N, n, c \right) = 0$
%		\item $L\left(\cdot, N, n, c \right)$ monoton fallend
%		\end{itemize}
%	     	\end{itemize}
%     \end{column}
%     	\begin{column}[T]{6cm} 
%            		\include{OC-Funktion}
%     \end{column}
% \end{columns}
%}
%\frame{\frametitle{Operationscharakteristik}
%\framesubtitle{}
%\begin{center}
%% GNUPLOT: LaTeX picture with Postscript
\begingroup
  \makeatletter
  \providecommand\color[2][]{%
    \GenericError{(gnuplot) \space\space\space\@spaces}{%
      Package color not loaded in conjunction with
      terminal option `colourtext'%
    }{See the gnuplot documentation for explanation.%
    }{Either use 'blacktext' in gnuplot or load the package
      color.sty in LaTeX.}%
    \renewcommand\color[2][]{}%
  }%
  \providecommand\includegraphics[2][]{%
    \GenericError{(gnuplot) \space\space\space\@spaces}{%
      Package graphicx or graphics not loaded%
    }{See the gnuplot documentation for explanation.%
    }{The gnuplot epslatex terminal needs graphicx.sty or graphics.sty.}%
    \renewcommand\includegraphics[2][]{}%
  }%
  \providecommand\rotatebox[2]{#2}%
  \@ifundefined{ifGPcolor}{%
    \newif\ifGPcolor
    \GPcolorfalse
  }{}%
  \@ifundefined{ifGPblacktext}{%
    \newif\ifGPblacktext
    \GPblacktexttrue
  }{}%
  % define a \g@addto@macro without @ in the name:
  \let\gplgaddtomacro\g@addto@macro
  % define empty templates for all commands taking text:
  \gdef\gplbacktext{}%
  \gdef\gplfronttext{}%
  \makeatother
  \ifGPblacktext
    % no textcolor at all
    \def\colorrgb#1{}%
    \def\colorgray#1{}%
  \else
    % gray or color?
    \ifGPcolor
      \def\colorrgb#1{\color[rgb]{#1}}%
      \def\colorgray#1{\color[gray]{#1}}%
      \expandafter\def\csname LTw\endcsname{\color{white}}%
      \expandafter\def\csname LTb\endcsname{\color{black}}%
      \expandafter\def\csname LTa\endcsname{\color{black}}%
      \expandafter\def\csname LT0\endcsname{\color[rgb]{1,0,0}}%
      \expandafter\def\csname LT1\endcsname{\color[rgb]{0,1,0}}%
      \expandafter\def\csname LT2\endcsname{\color[rgb]{0,0,1}}%
      \expandafter\def\csname LT3\endcsname{\color[rgb]{1,0,1}}%
      \expandafter\def\csname LT4\endcsname{\color[rgb]{0,1,1}}%
      \expandafter\def\csname LT5\endcsname{\color[rgb]{1,1,0}}%
      \expandafter\def\csname LT6\endcsname{\color[rgb]{0,0,0}}%
      \expandafter\def\csname LT7\endcsname{\color[rgb]{1,0.3,0}}%
      \expandafter\def\csname LT8\endcsname{\color[rgb]{0.5,0.5,0.5}}%
    \else
      % gray
      \def\colorrgb#1{\color{black}}%
      \def\colorgray#1{\color[gray]{#1}}%
      \expandafter\def\csname LTw\endcsname{\color{white}}%
      \expandafter\def\csname LTb\endcsname{\color{black}}%
      \expandafter\def\csname LTa\endcsname{\color{black}}%
      \expandafter\def\csname LT0\endcsname{\color{black}}%
      \expandafter\def\csname LT1\endcsname{\color{black}}%
      \expandafter\def\csname LT2\endcsname{\color{black}}%
      \expandafter\def\csname LT3\endcsname{\color{black}}%
      \expandafter\def\csname LT4\endcsname{\color{black}}%
      \expandafter\def\csname LT5\endcsname{\color{black}}%
      \expandafter\def\csname LT6\endcsname{\color{black}}%
      \expandafter\def\csname LT7\endcsname{\color{black}}%
      \expandafter\def\csname LT8\endcsname{\color{black}}%
    \fi
  \fi
  \setlength{\unitlength}{0.0500bp}%
  \begin{picture}(7000.00,3500.00)%
    \gplgaddtomacro\gplbacktext{%
    \put(0,0){\includegraphics{OC-Funktion2}}
    \thicklines
      \colorrgb{0.00,0.00,0.00}%
      \put(740,640){\makebox(0,0)[r]{\strut{}0}}%
      \colorrgb{0.00,0.00,0.00}%
      \put(740,1264){\makebox(0,0)[r]{\strut{}0.2}}%
      \colorrgb{0.00,0.00,0.00}%
      \put(740,1888){\makebox(0,0)[r]{\strut{}0.4}}%
      \colorrgb{0.00,0.00,0.00}%
      \put(740,2511){\makebox(0,0)[r]{\strut{}0.6}}%
      \colorrgb{0.00,0.00,0.00}%
      \put(740,3135){\makebox(0,0)[r]{\strut{}0.8}}%
      \colorrgb{0.00,0.00,0.00}%
      %%%%%%%%%%
      
      
      \put(860,3759){\makebox(0,0)[r]{\strut{}1}}%
      \colorrgb{0.00,0.00,0.00}%
      \put(860,440){\makebox(0,0){\strut{}0}}%
      \colorrgb{0.00,0.00,0.00}%
      \put(2016,440){\makebox(0,0){\strut{}0.1}}%
      \colorrgb{0.00,0.00,0.00}%
      \put(3172,440){\makebox(0,0){\strut{}0.2}}%
      \colorrgb{0.00,0.00,0.00}%
      \put(4327,440){\makebox(0,0){\strut{}0.3}}%
      \colorrgb{0.00,0.00,0.00}%
      \put(5483,440){\makebox(0,0){\strut{}0.4}}%
      \colorrgb{0.00,0.00,0.00}%
      \put(6639,440){\makebox(0,0){\strut{}0.5}}%
      \colorrgb{0.00,0.00,0.00}%
      \put(160,2199){\rotatebox{90}{\makebox(0,0){\strut{}$L, p_i$}}}%
      \colorrgb{0.00,0.00,0.00}%
      \put(6000,440){\makebox(0,0){\strut{}$p$}}%
    }%
    \gplgaddtomacro\gplfronttext{%
      \colorrgb{0.00,0.00,0.00}%
      \put(6519,3596){\makebox(0,0)[r]{\strut{}$p_i(N, M, n)$}}%
      \colorrgb{0.00,0.00,0.00}%
      \put(6519,3396){\makebox(0,0)[r]{\strut{}$L(p, M, N, c)$}}%
    }%
    \gplbacktext
    %\put(0,0){\includegraphics{OC-Funktion2}}%
    \gplfronttext \pause
    \put(2030,3600){\color{red!80!black}
      \put(0,0){\line(-1,0){1180}}
      \put(0,0){\line(0,-1){2960}}
      \put(0,0){\circle*{80}}
      \put(50,0){\makebox(0,0)[l]{\strut{}Produzentenpunkt}}
      \put(-1220,0){\makebox(0,0)[r]{\strut{}$1-\alpha$}}
      \put(0,-3400 ){\makebox(0,0)[c]{\strut{}AQL}}} \pause
      %%%
      \put(3020,2200){\color{red!80!black}
      \put(0,0){\line(-1,0){2160}}
      \put(0,0){\line(0,-1){1550}}
      \put(0,0){\circle*{80}}
      %\put(50,0){\makebox(0,0)[l]{\strut{}Produzentenpunkt}}
      \put(-2200,0){\makebox(0,0)[r]{\strut{}$0{,}5$}}
      \put(0,-2000 ){\makebox(0,0)[c]{\strut{}IQL}}} \pause
      %%%
      \put(3850,1000){\color{red!80!black}
      \put(0,0){\line(-1,0){3000}}
      \put(0,0){\line(0,-1){350}}
      \put(0,0){\circle*{80}}
      \put(50,0){\makebox(0,0)[l]{\strut{}Konsumentenpunkt}}
      \put(-3040,0){\makebox(0,0)[r]{\strut{}$\beta$}}
      \put(0,-800 ){\makebox(0,0)[c]{\strut{}RQL}}}
  \end{picture}%
\endgroup

%\end{center}
%}
%
%\frame{\frametitle{Verhalten einfacher Pr\"ufpl\"ane}
%\framesubtitle{}
%\begin{columns}[t] 
%     \begin{column}[T]{5cm} 
%     	\begin{itemize}
%     		\item Zunehmendes $n$ reduziert IQL-Bereich
%		\item Minimales Risko f\"ur $n = N$
%		\item Mittlerer Pr\"ufaufwand bei abbrechender Kontrolle
%		\begin{equation*}
%		\begin{split}
%		\scriptsize
%		&E = n L\left(p, N, n, c \right) + \\
%		&(c+1) \sum_{l = c+1}^{n}p_{c+1}\left(N, M, l\right)
%		\end{split}
%		\end{equation*}
%
%     	\end{itemize}
%     \end{column}
%     	\begin{column}[T]{7cm} 
%         	\begin{center}
%            		% GNUPLOT: LaTeX picture with Postscript
\begingroup
  \makeatletter
  \providecommand\color[2][]{%
    \GenericError{(gnuplot) \space\space\space\@spaces}{%
      Package color not loaded in conjunction with
      terminal option `colourtext'%
    }{See the gnuplot documentation for explanation.%
    }{Either use 'blacktext' in gnuplot or load the package
      color.sty in LaTeX.}%
    \renewcommand\color[2][]{}%
  }%
  \providecommand\includegraphics[2][]{%
    \GenericError{(gnuplot) \space\space\space\@spaces}{%
      Package graphicx or graphics not loaded%
    }{See the gnuplot documentation for explanation.%
    }{The gnuplot epslatex terminal needs graphicx.sty or graphics.sty.}%
    \renewcommand\includegraphics[2][]{}%
  }%
  \providecommand\rotatebox[2]{#2}%
  \@ifundefined{ifGPcolor}{%
    \newif\ifGPcolor
    \GPcolorfalse
  }{}%
  \@ifundefined{ifGPblacktext}{%
    \newif\ifGPblacktext
    \GPblacktexttrue
  }{}%
  % define a \g@addto@macro without @ in the name:
  \let\gplgaddtomacro\g@addto@macro
  % define empty templates for all commands taking text:
  \gdef\gplbacktext{}%
  \gdef\gplfronttext{}%
  \makeatother
  \ifGPblacktext
    % no textcolor at all
    \def\colorrgb#1{}%
    \def\colorgray#1{}%
  \else
    % gray or color?
    \ifGPcolor
      \def\colorrgb#1{\color[rgb]{#1}}%
      \def\colorgray#1{\color[gray]{#1}}%
      \expandafter\def\csname LTw\endcsname{\color{white}}%
      \expandafter\def\csname LTb\endcsname{\color{black}}%
      \expandafter\def\csname LTa\endcsname{\color{black}}%
      \expandafter\def\csname LT0\endcsname{\color[rgb]{1,0,0}}%
      \expandafter\def\csname LT1\endcsname{\color[rgb]{0,1,0}}%
      \expandafter\def\csname LT2\endcsname{\color[rgb]{0,0,1}}%
      \expandafter\def\csname LT3\endcsname{\color[rgb]{1,0,1}}%
      \expandafter\def\csname LT4\endcsname{\color[rgb]{0,1,1}}%
      \expandafter\def\csname LT5\endcsname{\color[rgb]{1,1,0}}%
      \expandafter\def\csname LT6\endcsname{\color[rgb]{0,0,0}}%
      \expandafter\def\csname LT7\endcsname{\color[rgb]{1,0.3,0}}%
      \expandafter\def\csname LT8\endcsname{\color[rgb]{0.5,0.5,0.5}}%
    \else
      % gray
      \def\colorrgb#1{\color{black}}%
      \def\colorgray#1{\color[gray]{#1}}%
      \expandafter\def\csname LTw\endcsname{\color{white}}%
      \expandafter\def\csname LTb\endcsname{\color{black}}%
      \expandafter\def\csname LTa\endcsname{\color{black}}%
      \expandafter\def\csname LT0\endcsname{\color{black}}%
      \expandafter\def\csname LT1\endcsname{\color{black}}%
      \expandafter\def\csname LT2\endcsname{\color{black}}%
      \expandafter\def\csname LT3\endcsname{\color{black}}%
      \expandafter\def\csname LT4\endcsname{\color{black}}%
      \expandafter\def\csname LT5\endcsname{\color{black}}%
      \expandafter\def\csname LT6\endcsname{\color{black}}%
      \expandafter\def\csname LT7\endcsname{\color{black}}%
      \expandafter\def\csname LT8\endcsname{\color{black}}%
    \fi
  \fi
  \setlength{\unitlength}{0.0500bp}%
  \begin{picture}(4000.00,3500.00)%
    \gplgaddtomacro\gplbacktext{%
      \colorrgb{0.00,0.00,0.00}%
      \put(540,640){\makebox(0,0)[r]{\strut{}0}}%
      \colorrgb{0.00,0.00,0.00}%
      \put(540,1192){\makebox(0,0)[r]{\strut{}0.2}}%
      \colorrgb{0.00,0.00,0.00}%
      \put(540,1744){\makebox(0,0)[r]{\strut{}0.4}}%
      \colorrgb{0.00,0.00,0.00}%
      \put(540,2295){\makebox(0,0)[r]{\strut{}0.6}}%
      \colorrgb{0.00,0.00,0.00}%
      \put(540,2847){\makebox(0,0)[r]{\strut{}0.8}}%
      \colorrgb{0.00,0.00,0.00}%
      \put(540,3399){\makebox(0,0)[r]{\strut{}1}}%
      \colorrgb{0.00,0.00,0.00}%
      \put(660,440){\makebox(0,0){\strut{}0}}%
      \colorrgb{0.00,0.00,0.00}%
      \put(1157,440){\makebox(0,0){\strut{}0.05}}%
      \colorrgb{0.00,0.00,0.00}%
      \put(1653,440){\makebox(0,0){\strut{}0.1}}%
      \colorrgb{0.00,0.00,0.00}%
      \put(2150,440){\makebox(0,0){\strut{}0.15}}%
      \colorrgb{0.00,0.00,0.00}%
      \put(2646,440){\makebox(0,0){\strut{}0.2}}%
      \colorrgb{0.00,0.00,0.00}%
      \put(3143,440){\makebox(0,0){\strut{}0.25}}%
      \colorrgb{0.00,0.00,0.00}%
      \put(3639,440){\makebox(0,0){\strut{}0.3}}%
      \colorrgb{0.00,0.00,0.00}%
      \put(2149,140){\makebox(0,0){\strut{}$p$}}%
      \csname LTb\endcsname%
      \put(2149,3699){\makebox(0,0){\strut{}OC für $N = 1000$, $p^{\star} = 0{,}1$}}%
    }%
    \gplgaddtomacro\gplfronttext{%
      \colorrgb{0.00,0.00,0.00}%
      \put(3519,3236){\makebox(0,0)[r]{\strut{}$n= 50$}}%
      \colorrgb{0.00,0.00,0.00}%
      \put(3519,3036){\makebox(0,0)[r]{\strut{}$n= 100$}}%
      \colorrgb{0.00,0.00,0.00}%
      \put(3519,2836){\makebox(0,0)[r]{\strut{}$n= 500$}}%
      \colorrgb{0.00,0.00,0.00}%
      \put(3519,2636){\makebox(0,0)[r]{\strut{}$n= 1000$}}%
    }%
    \gplbacktext
    \put(0,0){\includegraphics{OC-Funktion3}}%
    \gplfronttext
  \end{picture}%
\endgroup

%        		\end{center}
%     \end{column}
% \end{columns}
%}

%\frame{\frametitle{Algorithmus zur Erstellung einfacher Pr\"ufpl\"ane}
%\framesubtitle{}
%\begin{columns}[t] 
%     \begin{column}[T]{5cm} 
%     	\begin{itemize}
%     		\item 
%     	\end{itemize}
%     \end{column}
%     	\begin{column}[T]{7cm} 
%         	\begin{picture}(200, 200)(-100,-200)
%	\thicklines
%	\scriptsize
%            		\put(0,0){\oval(100,16)[8]}
%			\put(0,0){\makebox(0,0)[c]{\strut{}Eingabe $\alpha$, $\beta$, $p_{\alpha}$, $p_{\beta}$, $N$}}
%			\put(0,-8){\vector(0,-1){8}}
%			\put(0,0){\oval(100,16)[8]}
%			\put(0,0){\makebox(0,0)[c]{\strut{}Eingabe $\alpha$, $\beta$, $p_{\alpha}$, $p_{\beta}$, $N$}}
%        		\end{picture}
%     \end{column}
% \end{columns}
%}

%Messmethoden, -strategien, Lehrende Pr\"ufung
% !TEX root = ../SFV15001_MdQM_Fertigungsmesstechnik_Rev01.tex
%% \section{Erforderliche Messgenauigkeit}
%\label{Sec:Messgenauigkeit}
%%\offslide{Erforderliche Messgenauigkeit}
%
%\frame{
%\frametitle{Erforderliche Messgenauigkeit}
%\framesubtitle{Die Einhaltung gewisser Intervalle kann nur mit einer gegebenen Wahrscheinlichkeit bestimmt werden.}{
%\begin{center}
%\begin{picture}(320,160)(0,0)
%\thicklines
%\put(76,23){\line(0,1){120}}
%\put(267,23){\line(0,1){120}}
%\put(10,10){Untere Toleranzgrenze (UT)}
%\put(190,10){Obere Toleranzgrenze (OT)}
%\put(171.4,115){\oval[0](191,15)}
%\put(130,112){Toleranzzone}
% 
%\put(0,0){\includegraphics[width=1.02\textwidth]{Toleranz}}
%\put(76,23){\line(0,1){120}}
%\put(267,23){\line(0,1){120}}
%\put(10,10){Untere Toleranzgrenze (UT)}
%\put(190,10){Obere Toleranzgrenze (OT)}
%\put(171.4,115){\oval[0](191,15)}
%\put(130,112){Toleranzzone}
% 
%\put(55,130){\vector(1,0){21.5}
%\put(21,0){\vector(-1,0){21.5}}
%\put(8,2){$U$}}
%\put(55,23){\line(0,1){120}}
%\put(76,130){\vector(1,0){21.5}
%\put(21,0){\vector(-1,0){21.5}}
%\put(8,2){$U$}}
%\put(97,23){\line(0,1){120}}
%
%\put(246,23){\line(0,1){120}}
%\put(246,130){\vector(1,0){21.5}
%\put(21,0){\vector(-1,0){21.5}}
%\put(8,2){$U$}}
%\put(288,23){\line(0,1){120}}
%\put(267,130){\vector(1,0){21.5}
%\put(21,0){\vector(-1,0){21.5}}
%\put(8,2){$U$}}  
%\put(171.5,97){\oval[0](149,15)}
%\put(130,92){\"Ubereinstimmung}  
%\put(294,30){\rotatebox{90}{Ablehnung durch Kunden}}
%\put(41,30){\rotatebox{90}{Ablehnung durch Kunden}}  
%\put(171.5,77){\oval[0](107,15)}
%\put(130,72){Nutzbare Toleranz} 
%\put(97,77){\vector(1,0){21.5}
%\put(21,0){\vector(-1,0){21.5}}
%\put(8,2){$U$}} 
%\put(225,77){\vector(1,0){21.5}
%\put(21,0){\vector(-1,0){21.5}}
%\put(8,2){$U$}} 
%
%\end{picture}
%\end{center}
%}}
%
%\frame{\frametitle{Nutzbare Toleranz}
%\framesubtitle{Die Messunsicherheit bestimmt die nutzbare Toleranz.}
%\begin{columns}[t] 
%     \begin{column}[T]{6cm} 
%     	\begin{itemize}
%     		\item Pr\"azisere Messtechnik erlaubt Nutzung breiterer Toleranzen
%		\item Kostenreduzierung in der Fertigung
%		\item Reduzierte Ausfallquoten
%		\item Faustregel \textit{(Goldene Regel)} \\
%		$U \leq \frac{1}{10} \left(\mathsf{OT} - \mathsf{UT}\right)$
%     	\end{itemize}
%     \end{column}
%     	\begin{column}[T]{6cm} 
%         	\begin{center}
%            		\begin{picture}(320,100)(0,0)
%			\setlength{\unitlength = 0.5pt}
%        %\thicklines
%        \tiny
%        \put(76,23){\line(0,1){120}}
%        \put(267,23){\line(0,1){120}}
%        \put(10,10){Untere Toleranzgrenze (UT)}
%        \put(190,10){Obere Toleranzgrenze (OT)}
%        \put(171.4,115){\oval[0](191,15)}
%        \put(130,112){Toleranzzone}
%        \put(0,0){\includegraphics[width=0.95\textwidth]{Toleranz}}
%        \put(76,23){\line(0,1){120}}
%        \put(267,23){\line(0,1){120}}
%        \put(10,10){Untere Toleranzgrenze (UT)}
%        \put(190,10){Obere Toleranzgrenze (OT)}
%        \put(171.4,115){\oval[0](191,15)}
%        \put(130,112){Toleranzzone}
%        \put(55,130){\vector(1,0){21.5}
%        \put(21,0){\vector(-1,0){21.5}}
%        \put(8,2){$U$}}
%        \put(55,23){\line(0,1){120}}
%        \put(76,130){\vector(1,0){21.5}
%        \put(21,0){\vector(-1,0){21.5}}
%        \put(8,2){$U$}}
%        \put(97,23){\line(0,1){120}}
%        
%        \put(246,23){\line(0,1){120}}
%        \put(246,130){\vector(1,0){21.5}
%        \put(21,0){\vector(-1,0){21.5}}
%        \put(8,2){$U$}}
%        \put(288,23){\line(0,1){120}}
%        \put(267,130){\vector(1,0){21.5}
%        \put(21,0){\vector(-1,0){21.5}}
%        \put(8,2){$U$}}  
%        \put(171.5,97){\oval[0](149,15)}
%        \put(130,92){\"Ubereinstimmung} 
%        \put(294,30){\rotatebox{90}{Ablehnung}}
%        \put(41,30){\rotatebox{90}{Ablehnung}} 
%        \put(171.5,77){\oval[0](107,15)}
%        \put(125,72){Nutzbare Toleranz} 
%        \put(97,77){\vector(1,0){21.5}
%        \put(21,0){\vector(-1,0){21.5}}
%        \put(8,2){$U$}} 
%        \put(225,77){\vector(1,0){21.5}
%        \put(21,0){\vector(-1,0){21.5}}
%        \put(8,2){$U$}} 
%\end{picture}
%
%        		\end{center}
%     \end{column}
% \end{columns}
%}
%
%\section{Pr\"ufprozesseignung}
%\frame{\frametitle{Pr\"ufprozesseignung}
%\framesubtitle{}
%\begin{columns}[t] 
%     \begin{column}[T]{6cm} 
%     	\begin{itemize}
%     		\item Ermittlung Teil des Pr\"ufmittelmanagements
%		\item Vornehmlich f\"ur quantitative Merkmale anwendbar
%		\item Typisch: Messprozesse
%		\item Teilgebiet: Pr\"ufmittelf\"ahigkeit
%		\item i.d.R. Voraussetzung: Normalverteilung
%		\begin{itemize}
%		\item Zentraler Grenzwertsatz
%		\end{itemize} 
%		\item Vier Methoden:
%		\begin{itemize}
%		\item Goldene Regel
%		\item Pr\"ufprozessf\"ahigkeit
%		\item R\&R-Studie
%		\item Konformit\"atspr\"ufung
%		\end{itemize}
%     	\end{itemize}
%     \end{column}
%     	\begin{column}[T]{6cm} 
%         	\begin{center}
%			\only<1>{
%            		\includegraphics[width=0.95\textwidth]{Prozessfahigkeit}\\
%			Beherrscht}
%			\only<2>{
%            		\includegraphics[width=0.95\textwidth]{Prozessfahigkeit2}\\
%			Mittelwertsdrift}
%			\only<3>{
%            		\includegraphics[width=0.95\textwidth]{Prozessfahigkeit3}\\
%			Ver\"anderliche Streuung}
%        		\end{center}
%     \end{column}
% \end{columns}
%}
%
%
%
%\frame{\frametitle{Prozessf\"ahigkeitsindex}
%\framesubtitle{Der \textit{Process Capability Index} gibt skalare Werte zum Vergleich der Messprozessf\"ahigkeit an.}
%    	\begin{itemize}
%		\item Allgemein
%     			\begin{equation*}
%				C_{p} = \frac{T}{x_{99{,}865\%} - x_{0{;}135\%}}
%			\end{equation*}
%			\begin{equation*}
%			C_{pk} = \min\left(\frac{OT - x_{50\%}}{x_{99{,}865\%} - x_{50\%}}, \, \frac{\left|UT - x_{50\%}\right|}{\left|x_{0{,}135\%} - x_{50\%}\right|}  \right)
%			\end{equation*}
%			\normalsize
%		\item F\"ur Normalverteilung
%			\begin{equation*}
%				C_{p} = \frac{T}{6 \sigma} \approx \frac{T}{6s}
%			\end{equation*}
%			\begin{equation*}
%			C_{pk} = \min\left(\frac{OT - \mu}{3\sigma}, \, \frac{\left|UT -\mu\right|}{3\sigma}  \right) \approx \min\left(\frac{OT - \mu}{3s}, \, \frac{\left|UT -\mu\right|}{3s}  \right)
%			\end{equation*}
%			\item $T$, $UT$, $OT$ bezogen auf Messprozess
%		\end{itemize} 
%}
%
%\frame{\frametitle{Prozessf\"ahigkeitsindex}
%\framesubtitle{}
%\begin{columns}[t] 
%     \begin{column}[T]{5cm} 
%     	\begin{itemize}
%		\item $C_{p}$: Beurteilung Streuung
%		\item $C_{pk}$: Beurteilung Mittelwertslage
%     		\item Geforderte Werte brachen- bzw. unternehmensabh\"angig
%		\item H\"aufig: $C_{p} \geq 1{,}33$ und $C_{pk} \geq 1{,}33$
%		\begin{itemize}
%		\item Entspricht 75\% Toleranzausnutzung
%		\end{itemize}
%     	\end{itemize}
%     \end{column}
%     	\begin{column}[T]{7cm} 
%         	\begin{itemize}
%		\item $C_{p} \geq 1{,}33$: 
%		\begin{itemize}
%		\item Fehlerfreie Fertigung unproblematisch
%		\end{itemize}
%		\item $ 1 \leq C_{p} \leq 1{,}33$: 
%		\begin{itemize}
%		\item Fehlerfreie Fertigung problematisch
%		\end{itemize}
%		\item $C_{p} < 1$: 
%		\begin{itemize}
%		\item Fehlerfreie Fertigung nicht m\"oglich
%		\item 100\% Sortierpr\"ufung veranlassen
%		\end{itemize}
%		\item $C_{pk} \geq 1{,}33$: 
%		\begin{itemize}
%		\item Fehlerfreie Fertigung unproblematisch
%		\end{itemize}
%		\item $C_{pk} < 1{,}33$: 
%		\begin{itemize}
%		\item Fehlerfreie Fertigung problematisch
%		\end{itemize}
%		\end{itemize}
%     \end{column}
% \end{columns}
%}
%
%\section{R\"uckf\"uhrbarkeit}
%\label{Sec:Rueckfuehr}
%\frame{\frametitle{Pr\"ufmittel\"uberwachung}
%\framesubtitle{}
%\begin{columns}[t] 
%     \begin{column}[T]{6cm} 
%     	\begin{itemize}
%     		\item Ab Verf\"ugbarkeit im Betrieb bis Ausmusterung
%		\item Notwendig z.B. wegen Verschlei{\ss}, Alterung oder Besch\"adigung
%		\item \"Uberwachung
%		\begin{itemize}
%		\item Pr\"ufmittelorientiert
%		\item Pr\"ufaufgabenorientiert
%		\end{itemize}
%		\item \"Uberwachung:
%		\begin{itemize}
%		\item Erstmalige Eignungspr\"ufung
%		\item \"Uberwachungspr\"ufung
%		\item Nach Justierung oder Reparatur
%		\end{itemize}
%		\item \"Uberwachungskennzeichen am Pr\"ufmittel
%     	\end{itemize}
%     \end{column}
%     	\begin{column}[T]{5cm} 
%         	\begin{definition}
%            		Die \textbf{Pr\"ufmittel\"uberwachung} umfasst alle Ma{\ss}nahmen, welche die Genauigkeit, Zuverl\"assigkeit und Einsatzf\"ahigkeit von  Pr\"ufmitteln gew\"ahrleisten. Die Qualit\"at und F\"ahigkeit eines Pr\"ufmittels ist anhand eines Pr\"ufplans oder einer Checkliste nachzuweisen.
%        		\end{definition}
%     \end{column}
% \end{columns}
%}
%
%\frame{\frametitle{R\"uckf\"uhrbarkeit}
%\framesubtitle{}
%\begin{columns}[t] 
%     \begin{column}[T]{6cm} 
%     	\begin{itemize}
%     		\item Zweck
%		\begin{itemize}
%		\item Weltweite Austauschbarkeit von Einzelzeilen
%		\item Weltweite Anerkennung von Messungen
%		\item Weltweite Vergleichbarkeit von Messger\"aten
%		\end{itemize}
%		\item Forderung ISO 9001: 
%		\begin{itemize}
%		\item Kalibrierung kann auf nationale oder internationale Normale zur\"uckgef\"uhrt werden
%		\end{itemize}
%		\item Ununterbrochene Kette von Vergleichsmessungen mit Messunsicherheiten, endend bei nationalen Normalen
%     	\end{itemize}
%     \end{column}
%     	\begin{column}[T]{5cm} 
%         	\begin{center}
%	\vspace{1cm}
%            		\includegraphics[width=0.9\textwidth]{Ruckfuhrbarkeit}
%        		\end{center}
%     \end{column}
% \end{columns}
%}
%
%\frame{\frametitle{R\"uckf\"uhrbarkeit}
%         	\begin{center}
%            		\includegraphics[width=0.95\textwidth]{Ruckfuhrbarkeit}
%		\rotatebox{90}{\tiny nach \cite[S. 442]{pfeifer10}}
%        		\end{center}
%		
%}



\section{Manufacturing data}
\begin{frame}\frametitle{Scenario}
    Please assume that you are working as a freelance quality consultant to a manufacturing company in the railway sector. Your client manufactures spring parking brake cylinders as depicted below.
    \vspace{1cm}
\begin{center}
\begin{picture}(120,30)(0,-40)
\thicklines
\scriptsize
{\color{gray!70!black}
\put(15,0){\oval[0](10,30)}
\put(12,18){$h_{1}$}
\put(45,0){\oval[0](50,6)}
\put(42,6){$l$}
\put(71.5,0){\oval[0](3,42)}
\put(68.5,24){$d$}
\put(80,0){\oval[0](64,42)}
\put(125,0){\oval[0](10,30)}
\put(122,18){$h_{2}$}
\put(116,0){\oval[0](8,6)}}


\put(75,20){\line(0,-1){40}}
\put(75,20){\line(1,-8){5}}
\put(80,-20){\line(1,8){5}}
\put(85,20){\line(1,-8){5}}
\put(90,-20){\line(1,8){5}}
\put(95,20){\line(1,-8){5}}
\put(100,-20){\line(1,8){5}}
\put(105,20){\line(1,-8){5}}
\put(110,-20){\line(0,1){40}}
\end{picture}
\end{center}
A spring parking brake is used to maintain a rail vehicle in a stationary position, for movements, the brake is released by applying compressed air to the cylinder. For this purpose, one of the determining parameters of a spring parking brake the force is the force applied to the so called hammerheads of the brake cylinder. This force has to maintain the vehicle stationary. For the product under consideration, the required minimum force is $18 \, \mathrm{kN}$.
\end{frame}

\begin{frame}
    \frametitle{Tasks}
    \framesubtitle{Data import and graphical representation}
        You receive a call from you client because in todays production shift, more than 10\% of the production had to be scrapped due to insufficient force. The client sends you a .csv file containing the quality records of the shift.

The file contains the following data:
\vspace{.5cm}

{\small
\begin{tabular}{|c|c|c|c|c|c|c|}

\hline
Time, Date & F/N & $\triangle d$/mm & $\triangle h_{1}$/mm & $ \triangle h_{2}$/mm & $\triangle l$/mm & $x$/mm \\ \hline
\end{tabular}}

\begin{equation*}
x = h_1 + h_2 + d + l
\end{equation*}
\vspace{.5cm}

In order to start your analysis, you import it and inspect it qualitatively for randomness and systematic influences. Your customer is rather fluent in Python, so you decide to exchange Jupyter Notebooks highlighting your findings.
\end{frame}

\begin{frame}
    \frametitle{Tasks}
    \framesubtitle{Optional: Monte-Carlo simulation}
    The client considers a systematic behaviour of the spring stiffness as a possible root cause of the force loss. Each cylinder contains 24 springs of three types, each being supplied with a $\pm 10 \, \%$ tolerance on the nominal value:
    \begin{itemize}
        \item 8 small springs: $c_{1, \text{nom}} = 100\, \mathrm{N/mm}$
        \item 8 medium springs: $c_{2, \text{nom}} = 150\, \mathrm{N/mm}$
        \item 8 large springs: $c_{3, \text{nom}} = 200\, \mathrm{N/mm}$
    \end{itemize} 
    
    Perform a Monte Carlo Simulation of the combined spring stiffness $c$ and its variation. Assume two cases:
    \begin{itemize}
        \item[$C_1$] The supplier ships springs equally distributed within the tolerance range.
        \item[$C_2$] The supplier ships large springs within tolerance, however the distribution is skewed towards the lower end of the spectrum, yielding forces with $c_{3} \in \left[180, 190\right]\, \mathrm{N/mm}$. All other springs are as described above.
    \end{itemize}

    Hint: remember to draw the individual springs of a set from an independent and identically distributed (i.i.d.) random variable.
\end{frame}

\begin{frame}
    \frametitle{Tasks}
    \framesubtitle{Using support vector machine to predict an unobserved product property}
    esting of the cylinders for their effective force is a costly task. The customer would prefer to sort the components such that only capable cylinders are assembled. They propose to use the hammerhead dimensions $h_{1}$, $h_{2}$ as well as the length $l$ to predict the force based on learned data.

For this purpose:
\begin{itemize}
	\item Import the existing dataset
	\item Scale the data using a standard scaler
	\item Train a support vector machine based on this data
	\item Use the new data set (unknown to the SVM) ``SpringPBDataValidation.csv'' to test the performance
	\begin{itemize}
		\item You may also generate test data using a Monte-Carlo simulation
	\end{itemize}
	\item Plot a confusion matrix of the results
\end{itemize}

How do you like the performance of your classifier? Can you manufacture without end of line testing? Is your data basis sufficient to exclude any non-conforming cylinders from being shipped to the customer?
\end{frame}
% \section{QM-Systeme}
% %%Requirements Engineering
% % !TEX root = ../17-MdQM-Vorlesung.tex
\subsection{Requirements Engineering}
%\sectionpage

\frame{\frametitle{Beispiel mechatronisches System: Gleitschutz}
\begin{center}
            		\includegraphics[width=\textwidth]{WSP}\source{}
        		\end{center}
}

\frame{\frametitle{Beispiel mechatronisches System}
\framesubtitle{}
\begin{columns}[t] 
     \begin{column}[T]{6cm} 
     	\begin{itemize}
     		\item Mechatronik nach VDI 2206:
		\begin{itemize}
		\item Synergetisches Zusammenwirken der Disziplinen:
		\begin{itemize}
		\item Maschinenbau
		\item Eletrotechnik
		\item Informationstechnik
		\end{itemize}
		\end{itemize}
		\item H\"aufiges Kennzeichen:
		\begin{itemize}
		\item Verschmelzung von Sensorik, Aktorik, Rechner und Mechanik
		\end{itemize}
		\item Typisch
		\begin{itemize}
		\item Hoher Aufwand w\"ahrend Inbetriebnahme
		\item Schnittstellenkl\"arung schwierig
		\end{itemize}
     	\end{itemize}
     \end{column}
     	\begin{column}[T]{6cm} 
         	\begin{center}
            		\includegraphics[width=\textwidth]{WSP}\source{}
        		\end{center}
     \end{column}
 \end{columns}
}

\frame{\frametitle{Warum Requirements Engineering (RE)?}
\framesubtitle{Requirements Engineering befasst sich mit dem systematischen Erfassen, Umsetzen und Pr\"ufen von Anforderungen im Entwicklungsprozess.}
\begin{itemize}
\item Qualit\"at: 
\begin{itemize}
		\item Grad, in dem ein Satz inhärenter Merkmale eines Objekts Anforderungen erfüllt \cite{iso9000}
		\item Qualit\"at ist das Ma{\ss} der Erf\"ullung der Anforderungen an ein Produkt
		\end{itemize}
\item Durch RE Reduzierung von:
\begin{itemize}
		\item Entwicklungs- und Garantiekosten
		\item Konstruktions\"anderungen
		\item Design lead time
		\item Fehlerrate in Entwicklung und Einsatz
		\end{itemize}
\item Kosten- und Termintreue
\item Einbindung der Stakeholder (Anspruchsteller)
\item Systematisierung der Beschaffung und der Produktentwicklung
\item Nachvollziehbare Entscheidungen und Risikomanagement
\end{itemize}
}

%\frame{\frametitle{Key-Aspects of Requirements Engineering}
%\framesubtitle{}
%\begin{itemize}
%\item Stakeholder Involvement
%\item Technical Reviews
%\item Traceability
%\end{itemize}
%}

%\offslide{Generisches Phasenmodell}{Modell einer beliebigen Phase eines Entwicklungsprozesses}

\frame{\frametitle{Erweitertes V-Modell mit Lebenszyklus}
\framesubtitle{}
\begin{center}
\includegraphics[width = 11cm]{VModel} \source{Quelle: \cite{SEGuide}}
\end{center}
}


\frame{\frametitle{Generisches Phasenmodell}
\framesubtitle{Modell einer beliebigen Phase eines Entwicklungsprozesses}
\begin{columns}[t] 
     \begin{column}[T]{6cm} 
     \textbf{F\"ur jede Phase festzulegen:}
     	\begin{itemize}
     		\item Purpose
		\item Inputs
		\item Entry Criteria
		\item Roles
		\item Verification steps
		\item Outputs
		\item Exit criteria
		\item Resources
		\item Management review activities
     	\end{itemize}
     \end{column}
     	\begin{column}[T]{5cm} 
         	\begin{center}
            		\includegraphics[width=1.0\textwidth]{Phase.png}
        		\end{center}
     \end{column}
 \end{columns}
}

\frame{\frametitle{System Requirements}
\framesubtitle{Obtain System Level Requirements}
\begin{columns}[t] 
     \begin{column}[T]{6cm} 
     \textbf{Questions:}
     	\begin{itemize}
		\item What are the stakeholders?
     		\item What is the system to do?
		\item How well it is to do it?
		\item Under what conditions?
     	\end{itemize}
	\textbf{Typical Milestone: Initial Design Review (IDR)}
     \end{column}
     	\begin{column}[T]{5cm} 
         	\begin{center}
            		\includegraphics[width=0.95\textwidth]{Phase}
        		\end{center}
     \end{column}
 \end{columns}
}

\frame{\frametitle{High Level Design}
\framesubtitle{Top Level Design: Architekture, Solutions, Subsystems}
\begin{columns}[t] 
     \begin{column}[T]{6cm} 
     \textbf{Questions:}
     	\begin{itemize}
		\item Is the required system feasible?
     		\item What are system and subsystem borders?
		\item What are associated costs/lead times/risks?
		\item How can the risk be reduced?
		\item Which system integration steps are necessary?
		\item What is the suitable subsystem structure?
     	\end{itemize}
	\textbf{Typical Milestone: Preliminary Design Review (PDR)}
     \end{column}
     	\begin{column}[T]{5cm} 
         	\begin{center}
            		\includegraphics[width=0.95\textwidth]{Phase}
        		\end{center}
     \end{column}
 \end{columns}
}

%\offslide{Erg\"anzen des generischen Phasenmodells}%{Durchf\"uhrung in der \"Ubung}

\frame{\frametitle{Subsystem Design}
\framesubtitle{``Build to Specifications'': Drawings, Electrical Schemes,...}
\begin{columns}[t] 
     \begin{column}[T]{6cm} 
     \textbf{Questions:}
     	\begin{itemize}
		\item What are the subsystem requirements?
		%\item Make or Buy?
		\item Which deliverables (e.g. documentation) are requested?
		%\item How can the module be realised efficiently?
		\item What are critical characteristics of the module and its parts?
		\item Can service proven modules be used or adapted?
     	\end{itemize}
	\textbf{Typical Milestone: Critical Design Review (CDR)}
     \end{column}
     	\begin{column}[T]{5cm} 
         	\begin{center}
            		\includegraphics[width=0.95\textwidth]{Phase}
        		\end{center}
     \end{column}
 \end{columns}
}

%\offslide{Erg\"anzen des generischen Phasenmodells}%{Durchf\"uhrung in der \"Ubung}


% %% QM-Systeme
% % !TEX root = ../16_MdQM_Vorlesung RevA.tex
\subsection{QM-Systeme}
%\sectionpage

%\frame{\frametitle{}
%\begin{center}
%            		\begin{tikzpicture}[limb/.style={line cap=round,line width=1.5mm,line join=bevel}]
%\draw[line width=2mm,rounded corners,fill=yellow] (-2,0) -- (0,-2) -- (2,0) -- (0,2) -- cycle;
%\fill (1.5mm,7mm) circle (1.5mm);
%\fill(0,-7.5mm) -- ++(10mm,0mm) -- ++(120:2mm)--++(100:1mm)--++(150:2mm) arc (70:170:2.5mm and 1mm);
%\draw[limb] (-7.5mm,-6.5mm)--++(70:4mm)--++(85:4mm) coordinate(a)--++(-45:5mm)--(-2.5mm,-6.5mm);
%\fill[rotate around={45:(a)}] ([shift={(-0.5mm,0.55mm)}]a) --++(0mm,-3mm)--++
%        (7mm,-0.5mm)coordinate(b)--++(0mm,4mm)coordinate(c)--cycle;
%\draw[limb] ([shift={(-0.6mm,-0.4mm)}]b) --++(-120:5mm) ([shift={(-0.5mm,-0.5mm)}]c) --++
%        (-3mm,0mm)--++(-100:3mm)coordinate (d);
%\draw[ultra thick] (d) -- ++(-45:1.25cm);
%\end{tikzpicture}
%        		\end{center}
%}

\frame{\frametitle{Effectivity}
\framesubtitle{}
\begin{columns}[t] 
     \begin{column}[T]{6cm} 
     	\begin{center}
     		A manager's task is to make the strengths of people effective and their weakness irrelevant - and that applies fully as much to the manager's boss as it applies to the manager's subordinates.
     	\end{center}
	\begin{flushright}
	\textit{Peter Drucker\\
	Managing for the Future: The 1990's and Beyond}
	\end{flushright}
     \end{column}
     	\begin{column}[T]{6cm} 
         	\begin{center}
            		\includegraphics[width=0.8\textwidth]{Drucker}\source{Quelle: Jeff McNeill}
        		\end{center}
     \end{column}
 \end{columns}
}

\frame{\frametitle{Disclaimer}
\framesubtitle{}
\begin{center}
\Large Normen werden im Rahmen der Vorlesung bewusst verk\"urzt und auszugsweise wiedergegeben, die Vorlesungsunterlagen k\"onnen daher nicht als Quelle f\"ur Arbeiten im Zusammenhang mit einem QMS dienen!
\end{center}
}


\frame{\frametitle{Qualit\"atsmanagementsystem (QMS) }
\framesubtitle{}
\begin{itemize}
\item Sammlung von Gesch\"aftsprozessen mit dem Ziel:
	\begin{itemize}
		\item Kundenanforderungen zu gen\"ugen
		\item Kundenzufriedenheit zu steigern
	\end{itemize}
\item Entwicklung:
	\begin{itemize}
		\item Massenfertigung/Austauschbarkeit
		\item Industrieller Prozess
		\item Teamwork
		\item Continuous Improvement
		\item Shareholder Value
		\item Sustainability 
		\item Transparency
	\end{itemize}
\item Bekannte Standards:
\begin{itemize}
		\item ISO 9001
		\item ISO/TS 16949
		\item IRIS
		\item CMMI
		\end{itemize}
\end{itemize}
}

\frame{\frametitle{Warum ein QMS einsetzen und nutzen?}
\framesubtitle{}
\begin{columns}[t] 
     \begin{column}[T]{6cm} 
     	\begin{itemize}
     		\item Muss gegen\"uber Marktforderungen (z.B. als Zulieferer)
		\item Strategisches Hilfsmittel zur Sicherung von
		\begin{itemize}
		\item Qualit\"at
		\item Abl\"aufen
		\item Produktqualit\"at
		\end{itemize}
		\item Baustein des gesamtbetrieblichen Managementsystems
	\item Wichtig bei Einf\"uhrung und Umsetzung:
	\begin{itemize}
		\item Einfach benutzbar
		\item Change Management
		\end{itemize}
	\end{itemize}
     \end{column}
     	\begin{column}[T]{6cm} 
         	\begin{center}
            		\begin{tikzpicture} \begin{axis}[
    x tick label style={
        /pgf/number format/1000 sep=},
    ylabel= Number of certificates,
    enlargelimits=0.15,
    %legend style={at={(0.5,-0.15)},
        %anchor=north,legend columns=-1},
		symbolic x coords={China, Italy, Germany, Japan, India, United Kingdom, Spain, United States},
		xtick=data,
		x tick label style={rotate=45,anchor=east},
    ybar,
    bar width=7pt,
]
\addplot
    coordinates {(China, 342800) (Italy, 168960)
         (Germany, 55363) (Japan, 45785) (India, 41016)
	(United Kingdom, 40200)	(Spain, 36005) (United States, 33008)};
\end{axis}
\end{tikzpicture}
        		\end{center}
     \end{column}
 \end{columns}
}

\frame{\frametitle{QM-Prozesslandkarte (1)}
\framesubtitle{}
\begin{center}
\includegraphics[width = 10 cm]{Kernprozess} \source{Quelle: Wikimedia/Benji und Markus B\"arlocher}
\end{center}
}

\frame{\frametitle{QM-Prozesslandkarte (2)}
\framesubtitle{}
\begin{center}
\includegraphics[width = 7 cm]{ProzessLK} \source{Quelle: Wikimedia/pz0151}
\end{center}
}



\frame{\frametitle{Elemente des Qualit\"atsmanagementsystem}
\framesubtitle{}
\begin{description}
\item[\textbf{Qualit\"atspolitik}] Absichten und Ausrichtung einer Organisation zu Qualit\"at (durch oberste Leitung ausgedr\"uckt).
\item[\textbf{Qualit\"atsplanung}] Teil des QM, der auf Festlegen der Qualit\"atsziele, Ausf\"uhrungsprozesse und Ressourcen zur Erf\"ullung der Ziele  ausgerichtet ist.
\item[\textbf{Qualit\"atslenkung}]  \"Uberwachung und Korrektur der Realisierung einer Einheit mit dem Ziel, die Qualit\"atsforderungen zu erf\"ullen.
\item[\textbf{Qualit\"atssicherung}] Teil des QM, der darauf ausgerichtet ist, Vertrauen zu erzeugen, dass de Qualit\"atsforderungen erf\"ullt werden.
\item[\textbf{Qualit\"atsverbesserung}]  alle Ma{\ss}nahmen zur Steigerung von Effektivit\"at und Effizienz in T\"atigkeiten und Prozessen.
\end{description}
}

\frame{\frametitle{Beispiel ISO 9000 - Prinzipien}
\framesubtitle{}
\begin{itemize}
\item Kundenorientierung
\item Verantwortlichkeit der F\"uhrung
\item Einbeziehung der beteiligten Personen
\item Prozessorientierter Ansatz
\item Systemorientierter Managementansatz
\item Kontinuierliche Verbesserung
\item Sachbezogener Entscheidungsfindungsansatz
\item Lieferanteneinbeziehung zum gegenseitigen Nutzen
\end{itemize}
}

\frame{\frametitle{ISO 9001 - Ziele}
\framesubtitle{}
\begin{itemize}
\item Vertrauen in Produkte und Dienstleistungen einer Organisation
\begin{itemize}
	\item Konformit\"at
	\item Zuverl\"assigkeit der Ausf\"uhrung
\end{itemize}
\item Kundenzufriedenheit im Fokus
\item Verbesserung der Steuerung der Prozesse
\item Verbesserung des Verst\"andnisses der Prozesse
\item Verbesserung der internen Kommunikation
\end{itemize}
}

\frame{\frametitle{ISO 9001 - Kontext der Organisation}
\framesubtitle{}
\begin{itemize}
\item Verstehen der Organisation und ihres Kontextes
	\begin{itemize}
		\item Bestimmung interner und externer Themen, die f\"ur ihren Zweck relevant sind 
	\end{itemize}
	\item Verstehen der Erfordernisse und Erwartungen interessierter Parteien
	\begin{itemize}
		\item Bestimmung interessierter Parteien f\"ur QMS und deren relevante Anforderungen
	\end{itemize}
	\item Festlegen des Anwendungsbereichs des QMS
	\begin{itemize}
		\item Grenzen und Anwendungsbereich des QMS
		\item Dokumentierte Information:
		\begin{itemize}
		\item Arten der Produkte und Dienstleistungen
		\item Begr\"undung f\"ur Ausschluss
		\end{itemize}
		\item Ausgeschlossene Themen d\"urfen nicht die F\"ahigkeit oder die Verantwortung der Organisation i.S. der Norm beeintr\"achtigen
		\end{itemize}
\end{itemize}
}

\frame{\frametitle{ISO 9001 - Qualit\"atsmanagementsystem und seine Prozesse}
\framesubtitle{}
\begin{itemize}
\item Organisation muss ein QMS aufbauen, aufrechterhalten und fortlaufend verbessern
\item F\"ur QMS ben\"otigte Prozesse bestimmen und Anwendung festlegen
\begin{itemize}
		\item Eingaben und Ergebnisse
		\item Abfolge und Wechselwirkung
		\item Kriterien und Verfahren (inkl. Leistungsindikatoren) f\"ur wirksame Durchf\"uhrung und Steuerung bestimmen und anwenden
		\item Ben\"otigte Ressourcen ermitteln und Verf\"ugbarkeit sicherstellen
		\item Verantwortlichkeiten und Befugnisse zuweisen
		\item Risiken und Chancen behandeln
		\item Prozesse und das QMS verbessern
		\end{itemize}
\item Die Organisation muss dokumentierte Informationen aufrechterhalten, um die Durchf\"uhrung ihrer Prozesse zu unterst\"utzen und die geplante Durchf\"uhrbarkeit der Prozesse zu erm\"oglichen
\end{itemize}
}


\frame[allowframebreaks]{\frametitle{ISO 9001 - Verantwortung der Leitung}
\framesubtitle{}
\begin{itemize}
\item F\"uhrung und Verpflichtung:
	\begin{itemize}
		\item Rechenschaftspflicht
		\item Qualit\"atspolitik und Qualit\"atsziele f\"ur das QMS festlegen (mit Strategie vereinbar)
		\item  Anforderungen des QMS werden in Gesch\"aftsprozesse der Organisation aufgenommen
		\item F\"ordert prozessorientierten Ansatz und risikobasiertes Denken
		\item Sicherstellung von Ressourcenverf\"ugbarkeit
		\item Bedeutung und Wichtigkeit des QMS vermitteln
		\item Personen einsetzen, anleiten und unterst\"utzen, damit diese zur Wirksamkeit des QMS beitragen
		\item Kundenorientierung
	\begin{itemize}
		\item Kunden- und gesetzliche sowie normative Anforderungen erden erf\"ullt
		\item Verbesserung der Kundenzufriedenheit im Fokus
		\item Risiken und Chancen, die Konformit\"at von Produkten und Dienstleistungen beeinflussen k\"onnen, bestimmen und behandeln
	\end{itemize}
	\end{itemize} \newpage
\item Qualit\"atspolitik:
\begin{itemize}
	\item Angemessene Qulit\"atspolitik festlegen (f\"ur Zweck und Kontext der Organisation)
	\item Bietet Rahmen f\"ur Festlegung von Qualit\"atszielen
	\item Verpflichtung zur Erf\"ullung zutreffender Anforderungen (aus ISO 9001) sowie zur fortlaufenden Verbesserung des QMS
	\item Qualit\"atspolitik muss bekanntgemacht werden (als dokumentierte Information)
\end{itemize}
\item Rollen, Verantwortlichkeiten und Befugnisse in der Organisation:
\begin{itemize}
	\item Konformit\"at des QMS mit ISO 9001
	\item Sicherstellung, dass Prozesse die beabsichtigten Ergebnisse liefern
	\item Berichte \"uber Leistung und Verbesserungsm\"oglichkeiten des QMS
	\item F\"orderung der Kundenorientierung
	\item Sicherstellung des QMS bei \"Anderungen
\end{itemize}
\end{itemize}
}

\frame{\frametitle{Planung}
\framesubtitle{}
\begin{columns}[t] 
     \begin{column}[T]{6cm} 
     	\begin{itemize}
     		\item Umgang mit Chancen und Risiken
		\begin{itemize}
		\item Erw\"unschte Auswirkungen verst\"arken
		\item Unerw\"unschte Auswirkungen verhindern
		\end{itemize}
		\item Qualit\"atsziele und deren Erreichung
		\item Planung von \"Anderungen am QMS
     	\end{itemize}
     \end{column}
     	\begin{column}[T]{6cm} 
         	\begin{center}
            		\includegraphics[width=0.8\textwidth]{Risikograf}\source{Quelle: Peter Wratil}
        		\end{center}
     \end{column}
 \end{columns}
}


\frame[allowframebreaks]{\frametitle{ISO 9001 - Unterst\"utzung}
\framesubtitle{}
\begin{itemize}
\item Erforderliche Ressourcen f\"ur Aufbau, Verwirklichung, Aufrechterhaltung und Verbesserung der QMS bestimmen und bereitstellen
\begin{itemize}
	\item F\"ahigkeiten und Beschr\"ankungen von internen Ressourcen
	\item Notwendigerweise extern zu beziehendes
\end{itemize}
\item Personen bestimmen und bereitstellen, die f\"ur Umsetzung des QMS und Betreiben und Steuern der Prozesse n\"otig sind
\item Prozessumgebung (Umgebung f\"ur Durchf\"uhrung der Prozesse und zur Erreichung der Konformit\"at von Produkten und Dienstleistungen):
\begin{itemize}
	\item Soziale, psychologische und physikalische Faktoren
\end{itemize} 
\item Ressourcen zur \"Uberwachung und Messung:
\begin{itemize}
		\item Ressourcen zur Sicherstellung g\"ultiger und zuverl\"assiger \"Uberwachungs- und Messergebnisse
		\item Geeignet f\"ur Art der T\"atigkeit
		\item Aufrechterhaltung der fortlaufenden Eignung
		\item Messtechnische R\"uckf\"uhrbarkeit:
		\begin{itemize}
		\item Kalibrierung gegen nationale oder internationale Normale
		\item Kennzeichnung
		\end{itemize}
		\end{itemize}
\item Wissen der Organisation
\begin{itemize}
		\item Bestimmung des notwendigen Wissens
		\item Aufrechterhaltung des Wissens der Organisation
		\item Anpassung des Wissens der Organisation an sich \"andernde Anforderungen und Entwicklungstendenzen
		\end{itemize}
		\item Kompetenz:
		\begin{itemize}
		\item Erforderliche Kompetenz f\"ur T\"atigkeiten im Zusammenhang mit dem QMS bestimmen
		\item Kompetenz sicherstellen oder erlangen
		\item Angemessene dokumentierte Informationen aufbewahren
		\end{itemize}
\item Bewusstsein der t\"atigen Personen \"uber Qualit\"atspolitik, Qualit\"atsziele, Beitrag zu QMS und Folgen der Nichterf\"ullung
\item Festlegung der Kommunikationswege (intern und extern)
\end{itemize}
}


\frame{\frametitle{ISO 9001 - Lenkung dokumentierter Informationen}
\framesubtitle{}
\begin{itemize}
\item Erstellung und Aktualisierung
\begin{itemize}
	\item Angemessene Kennzeichnung und Beschreibung
	\begin{itemize}
		\item z.B. Titel, Datum, Autor, Referenznummer
	\end{itemize}
	\item angemessenes Format und Medium
	\begin{itemize}
		\item z.B. Sprache, Softwareversion, Grafiken
	\end{itemize}
	\item Angemessene \"Uberpr\"ufung und Genehmigung im Hinblick auf Eignung und Angemessenheit
\end{itemize}
\item Lenkung der Informationen
\begin{itemize}
	\item Informationen sind verf\"ugbar und geeignet
	\item Informationen sind angemessen gesch\"utzt
\end{itemize}
\item Besondere Aufgaben der Dokumentenlenkung
\begin{itemize}
	\item Verteilung, Zugriff, Auffindung und Verwendung
	\item Ablage/Speicherung und Erhaltung (einschlie{\ss}lich Lesbarkeit)
	\item \"Uberwachung von \"Anderungen (z.B. Versionskontrolle)
	\item Aufbewahrung und Verf\"ugung \"uber den weiteren Verbleib
\end{itemize}
\end{itemize}
}

\frame[allowframebreaks]{\frametitle{ISO 9001 - Entwicklung von Produkten% und Dienstleistungen
}
\framesubtitle{}
\begin{itemize}
\item Entwicklungsprozess erarbeiten, umsetzen und aufrechterhalten
\item Entwicklungsplanung
\begin{itemize}
		\item Art, Dauer und Umfang der Entwicklungst\"atigkeiten
		\item Erforderliche Prozessphasen einschlie{\ss}lich \"Uberpr\"ufung der Entwicklung
		\item T\"atigkeiten zur Verifizierung und Validierung
		\item Verantwortlichkeiten und Befugnisse
		\item Internen und externen Ressourcenbedarf
		\item Notwendigkeit der Schnittstellensteuerung
		\item Notwendigkeit der Kunden- und Anwendereinbindung
		\item Anforderungen an Produktion
		\item Steuerungsebene
		\item Dokumentierte Informationen, die die Anforderungserf\"ullung best\"atigen
		\end{itemize}
\item Entwicklungseingaben (angemessen, vollst\"andig und eindeutig)
\begin{itemize}
		\item Funktions- und Leistungsanforderungen
		\item Lessons learned
		\item Gesetzliche und beh\"ordliche Anforderungen
		\item Normen, Standards oder Anleitungen f\"ur die Praxis
		\item M\"ogliche Konsequenzen aus Fehler aufgrund der Art der Produkte
		\item Dokumentierte Informationen \"uber Entwicklungseingaben m\"ussen aufbewahrt werden
		\end{itemize}
\item Steuerungsma{\ss}nahmen f\"ur die Entwicklung
\begin{itemize}
		\item Definition der zu erzielenden Ergebnisse
		\item \"Uberpr\"ufung auf Anforderungsabdeckung
		\item Verifikationsma{\ss}nahmen - Entwicklung er\"ullt Anforderungen aus Entwicklungseingaben
		\item Validierungsma{\ss}nahmen - Entwicklung erf\"ullt Anforderungen aus vorgesehener Anwendung oder dem beabsichtigten Gebrauch
		\item Notwendige Ma{\ss}nahmen aus \"Uberpr\"ufung, Verifikation und Validierung werden umgesetzt
		\item Dokumentierte Informationen \"uber diese T\"atigkeiten aufbewahrt werden
		\end{itemize}
\item Entwicklungs\"anderungen
\begin{itemize}
		\item \"Anderungen im Entwicklungs- oder Produktionsprozess in dem Umfang ermitteln, \"uberpr\"ufen und steuern, der sicherstellt, dass keine nachteilige Auswirkung auf die Konformit\"at mit den Anforderungen entsteht
		\end{itemize} Dokumentierte Informationen
		\begin{itemize}
		\item Entwicklungs\"anderungen
		\item Ergebnisse von \"Uberpr\"ufungen
		\item Autorisierung der \"Anderungen
		\item Eingeleitete Ma{\ss}nahmen zur Vorbeugung nachteiliger Auswirkungen
		\end{itemize}
\end{itemize}
}


\frame{\frametitle{QMS - Umsetzung in der Organisation}
\framesubtitle{}
\begin{columns}[t] 
     \begin{column}[T]{6cm} 
     	\begin{itemize}
     		\item QM-Handbuch: Grunds\"atze, Aufbau- und Ablauforganisation, betriebsweite Zusammenh\"ange, Verantwortungen und Befugnisse
		\item Verfahrensanweisungen: Detaillierte Beschreibung von Teilgebieten des QMS
		\item Arbeits-, Pr\"ufanweisungen etc.: Definition von Einzelt\"atigkeiten, Detailwanweisungen
     	\end{itemize}
     \end{column}
     	\begin{column}[T]{6cm} 
         	\begin{center}
            		\begin{tikzpicture}[scale = 0.8]
 				\draw[draw = none, fill = blue!70!black, opacity = 0.2] (0,0) -- (3,6) -- (6,0) -- (0,0);
				\draw[draw  = blue!70!black, opacity = 0.5, ->, ultra thick] (-0.2,6) -- (-0.2,0) node[pos = 0.5, rotate = 90, above] {Detailtiefe};
				\draw[draw = none, fill = red!70!black, opacity = 0.2] (0,6) -- (3,0) -- (6,6) -- (0,6);
				\draw[draw  = red!70!black, opacity = 0.5, ->, ultra thick] (6.2,0) -- (6.2,6) node[pos = 0.5, rotate = 90, below] {Geltungsbereich};
				\node at (3,1) {Arbeitsanweisungen};
				\node at (3,3) {Verfahrensanweisungen};
				\node at (3,5) {QM-Handbuch};
			\end{tikzpicture}
        		\end{center}
     \end{column}
 \end{columns}
}

\frame{\frametitle{Auditierung, Zertifizierung und Akkreditierung}
\framesubtitle{}
\begin{itemize}
\item QM-Audit:
\begin{itemize}
		\item Systematische, unabh\"angige Untersuchung
		\item Entsprechen die qualit\"atsbezogenen T\"atigkeiten den geplanten Anordnungen? 
		\item Sind Anordnungen geeignet, die Ziele zu verwirklichen?
		\item Unterscheidung: 
		\begin{itemize}
		\item intern/ extern
		\item System-, verfahrens-, produkt, dienstleistungsorientiert
		\end{itemize}
		\end{itemize}
\item Im Audit wird die Konformit\"at mit Norm festgestellt
\begin{itemize}
		\item Das QM-System wird zertifiziert
		\item Ablauf:
		\begin{itemize}
		\item Autditvorbereitung: Selbstbeurteilung $\rightarrow$ Reife wird beurteilt
		\item Pr\"ufung QM-Handbuch $\rightarrow$ Stichprobe konform?
		\item Audit im Unternehmen $\rightarrow$ Werden beschriebene Abl\"aufe gelebt?
		\item  Zertifikatserteilung
		\end{itemize}
\end{itemize}
\end{itemize}
}

\frame{\frametitle{Branchenspezifische QM-Systeme}
\framesubtitle{Branchenspezifische QMS basieren \"uberwiegend auf der ISO 9001}
\begin{itemize}
\item Automobil:
\begin{itemize}
		\item QS9000 (Nordamerika): branchen- und herstellerspezifische Forderungen
		\item VDA 6 (Deutschland): Erg\"anzend bspw. Prozessf\"ahigkeit, Verf\"ugbarkeit
		\item TS 16949: Vereinheitlichung der verschiedenen Automobil-QMS
\end{itemize}
\item Schienenfahrzeuge:
\begin{itemize}
		\item IRIS (International Railway Industry Standard): Soll Kundenaudits ersetzen, Punktevergabe
		\end{itemize}		
\end{itemize}
}

\frame{\frametitle{Ausblick: Alternative Managementsysteme}
\framesubtitle{Beispiel Capability Maturity Modell Integration (CMMI)}
\begin{columns}[t] 
     \begin{column}[T]{6cm} 
     	\begin{itemize}
     		\item Initial: keine Anforderungen
		\item Managed: Projekte werden gef\"uhrt, ein \"ahnliches Projekt kann erfolgreich wiederholt werden
		\item Defined: angepasster Standardprozess organisationsweite kontinuierliche Verbesserung
		\item Quantitatively Managed: Statistische Prozesskontrolle
		\item Optimizing: Arbeit und Arbeitsweise mittels statistischer Prozesskontrolle verbessert
     	\end{itemize}
     \end{column}
     	\begin{column}[T]{6cm} 
         	\begin{center}
            		\includegraphics[width=0.95\textwidth]{CMMI}\source{Quelle: NASA}
        		\end{center}
     \end{column}
 \end{columns}
}


\frame{\frametitle{Beispiel EG-Konformit\"atserkl\"arung f\"ur Bahnkomponenten \cite{tsilocpas}}
\framesubtitle{}
\begin{columns}[t] 
     \begin{column}[T]{7cm} 
     	\begin{itemize}
     		\item CA: Interne Fertigungskontrolle
		\item CA1: CA + Einzeluntersuchung
		\item CA2: CA + Stichproben
		\item CB: EG-Baumusterpr\"ufung
		\item CC: Bauart-Konformit\"at (interne Fertigungskontrolle)
		\item CD: Bauart-Konformit\"at (QMS f\"ur Produktion)
		\item CF: Konformit\"at (Produktpr\"ufung)
		\item CH: Konformit\"at (vollst\"andiges QMS)
		\item CH1: Konformit\"at (vollst\"andiges QMS) + Enwurfspr\"ufung
     	\end{itemize}
     \end{column}
     	\begin{column}[T]{5cm} 
         	\begin{center}
            		\includegraphics[width=0.95\textwidth]{EGKonformitaet}\source{Quelle: TSI Loc \& Pas}
        		\end{center}
     \end{column}
 \end{columns}
}

% %%QFD, FTA, FMEA
% \section{Methoden}
% % !TEX root = ../16_MdQM_Vorlesung RevB.tex

%\sectionpage

%\frame{\frametitle{}
%\begin{center}
%            		\begin{tikzpicture}[limb/.style={line cap=round,line width=1.5mm,line join=bevel}]
%\draw[line width=2mm,rounded corners,fill=yellow] (-2,0) -- (0,-2) -- (2,0) -- (0,2) -- cycle;
%\fill (1.5mm,7mm) circle (1.5mm);
%\fill(0,-7.5mm) -- ++(10mm,0mm) -- ++(120:2mm)--++(100:1mm)--++(150:2mm) arc (70:170:2.5mm and 1mm);
%\draw[limb] (-7.5mm,-6.5mm)--++(70:4mm)--++(85:4mm) coordinate(a)--++(-45:5mm)--(-2.5mm,-6.5mm);
%\fill[rotate around={45:(a)}] ([shift={(-0.5mm,0.55mm)}]a) --++(0mm,-3mm)--++
%        (7mm,-0.5mm)coordinate(b)--++(0mm,4mm)coordinate(c)--cycle;
%\draw[limb] ([shift={(-0.6mm,-0.4mm)}]b) --++(-120:5mm) ([shift={(-0.5mm,-0.5mm)}]c) --++
%        (-3mm,0mm)--++(-100:3mm)coordinate (d);
%\draw[ultra thick] (d) -- ++(-45:1.25cm);
%\end{tikzpicture}
%        		\end{center}
%}

\subsection{Exkurs: Vertragspr\"ufung}
\subsectionpage

\frame{\frametitle{Aufgaben der Vertragspr\"ufung}
\framesubtitle{}
\begin{itemize}
\item Angebotsphase
\begin{itemize}
	\item Pr\"ufung auf Vollst\"andigkeit
	\item Pr\"ufung auf Risiken
\end{itemize}
\item Vertragsabschlussphase
\begin{itemize}
	\item Pr\"ufung auf Vollst\"andigkeit
	\item Pr\"ufung auf Unstimmigkeit
	\item Pr\"ufung auf Widerspr\"uchlichkeit
\end{itemize}
\item Abwicklungsphase
\begin{itemize}
	\item Verfolgen von \"Anderungen
	\item Verfolgen von Abweichungen
\end{itemize}
\end{itemize}
}

\frame{\frametitle{Vorgehen in der Angebotsphase}
\framesubtitle{}
\begin{itemize}
\item Rechte und Pflichten der Vertragsparteien
\begin{itemize}
	\item Dokumentenhierarchie
	\item Liefer- und Leistungsumfang
\end{itemize}
\item Mitwirkungspflichten
\begin{itemize}
	\item Auftraggeber
	\item Auftragnehmer
\end{itemize}
\item Analyse der Regelungen u.a. zu
\begin{itemize}
	\item Vertragsstrafen (z.B. Gewichtsp\"onale, Lieferverzug,...)
	\item Abnahmen
	\item \"Anderungen
	\item Verz\"ogerungen
\end{itemize}
\item Beurteilen besonderer vertraglicher Risiken
\end{itemize}
\begin{enumerate}
\item Lesen der Dokumente
\item Herausforderungen erkennen
\item Ma{\ss}nahmen erarbeiten und umsetzen
\end{enumerate}
}

\frame[allowframebreaks]{\frametitle{Wichtige Aspekte bei der Vertragspr\"ufung}
\framesubtitle{}
\begin{itemize}
\item Anwendbares Recht, Gerichtsstand
\item Regelung von Folgesch\"aden
\item Verzeichnis der Vertragsdokumente (inkl. Ausgabestand)
\item Liefer- und Leistungsumfang
\item Preisstellung (DDP Oslo vs. EXW), Preiseskalation
\item Umgang mit Abweichungen, technischem Fortschritt
\item Technische Termine
\item Optionen
\item Teillieferungen
\item Versp\"atung bei Lieferung, Dokumentation, IBS und P\"onalen
\item Nichteinhalten der vertraglichen Leistungswerte (Qualit\"at, RAMS, LCC,...)
\item Force-Majeur-Klausel
\item Produktionsstandorte
\item Logistik, Verpackung und Konservierung
\item Pr\"ufungen und Tests
\item Schulungen (Kunde und Betreiber)
\item Zertifikate
\item Gew\"ahrleistung
\end{itemize}
}

\subsection{Exkurs: Kosten- und Aufwandssch\"atzung}
\subsectionpage

\frame{\frametitle{Warum Kosten- und Aufwandssch\"atzung?}
\framesubtitle{Kosten- und Aufwandssch\"atzung ist die Grundlage f\"ur erfolgreiche Projektbearbeitung.}
\begin{itemize}
\item Aufwandssch\"atzung (Gr\"o{\ss}e: Zeit)
\begin{itemize}
	\item Identifikation von Arbeitspaketen
	\item Input f\"ur Kostensch\"atzung
	\item Ressourcenplanung und -allokation
	\item Terminplanung (auch projekt\"ubergreifend)
\end{itemize}
\item Kostensch\"atzung (Gr\"o{\ss}e: Geld)
\begin{itemize}
	\item Bestimmung von:
	\begin{itemize}
		\item Einmalkosten \textit{non recurring cost (NRC)}
		\item St\"uckkosten \textit{recurring cost (RC)}
	\end{itemize}
	\item Identifikation von Investitionen
\end{itemize}
\item Entscheidungshilfe im Entwicklungsprozess
\item Bestimmung des Angebotspreises
\end{itemize}
}

\frame{\frametitle{Herausforderungen Aufwands- und Kostensch\"atzung}
\framesubtitle{}
\begin{itemize}
\item Informationen:
\begin{itemize}
	\item unvollst\"andig
	\item unsicher
	\item fehlerbehaftet 
	\item Daher: Sch\"atzung, d.h. wahrscheinlichste Vorhersage \"uber den wahren Aufwand %{\color{red!80!black} Was wird gesch\"atzt?}
\end{itemize}
\item Projektdefinition:
\begin{itemize}
	\item Anforderungen nicht final (``to be defined during design stage'')
	\item \"Anderungen m\"oglich
\end{itemize}
\item Projektablauf:
\begin{itemize}
	\item Beginn durch Angebotsrunden verz\"ogert
	\item Projektverlauf durch externe Einfl\"usse (teil-)gesteuert 
\end{itemize}
\item Projektressourcen:
\begin{itemize}
	\item Durch andere Projekte Ressourcen blockiert oder eingeschr\"ankt nutzbar 
\end{itemize}
\end{itemize}
}


\frame{\frametitle{Ans\"atze zur Aufwandssch\"atzung}
\framesubtitle{}
\begin{itemize}
\item Expertensch\"atzverfahren, z.B.:
\begin{itemize}
	\item Projektstrukturplan-basiert \textit{(WBS-based)}
	\item Gruppensch\"atzung
\end{itemize}
\item Formale Sch\"atzverfahren, z.B.:
\begin{itemize}
	\item Analogie-basiert (z.B. Bremszange wie ..., jedoch mit ...)
	\item Parametrische Modelle (z.B. E-Kupplung Verkabelung: 100 h)
	\item Gr\"o{\ss}enbasiert: (z.B. Anpasskonstruktion: 500 h)
\end{itemize}
\item Kombinierte Sch\"atzverfahren, z.B.:
\begin{itemize}
	\item Zerlegung mit WBS, parametrische Sch\"atzung der Pakete
\end{itemize}
\item Auswahl des Verfahrens:
\begin{itemize}
	\item Abh\"angig von der Organisation
	\item Formale Verfahren weniger ``lernf\"ahig''
	\item Expertensch\"atzverfahren anf\"allig f\"ur ``wishful thinking''
\end{itemize}
\item Psychologische Herausforderungen \textit{(Cognitive biases)}:
\begin{itemize}
	\item \textit{Planning fallacy, cognitive dissonance, anchoring, confirmation bias, wishful thinking}
\end{itemize}
\end{itemize}
}


\frame{\frametitle{Projektstrukturplan}
\framesubtitle{Work Breakdown Structure (WBS) f\"ur die Ermittlung von Arbeitspaketen}
\begin{columns}[t] 
     \begin{column}[T]{6cm} 
     	\begin{itemize}
		\item Dekomposition eines Projekts
		\begin{itemize}
		\item Hierarchisch
		\item Inkrementell
		\end{itemize}
		\item Baumstruktur
		\item Gliederung gem\"a{\ss} DIN 69900
		\begin{itemize}
		\item Funktionsorientiert
		\item \textbf{Objektorientiert}
		\item Zeitorientiert
		\end{itemize}
		\item Starke Abh\"angigkeit von Deliverables
		\item Erstellung \"ublicherweise Top-Down
		\item Nutzen: Vollst\"andige \"Ubersicht
		\item Hilfreich: ``Tickler list'' 
     	\end{itemize}
     \end{column}
     	\begin{column}[T]{6cm} 
         	\begin{center}
            		\includegraphics[width=1\textwidth]{WBS}
        		\end{center}
     \end{column}
 \end{columns}
}

\frame{\frametitle{Aufbereitung WBS f\"ur Projektplanung}
\framesubtitle{Die Identifikation der Arbeitspakete allein l\"asst keine Planung des Projekts zu.}
\begin{itemize}
\item Sch\"atzung des Aufwands
\begin{itemize}
	\item Besprechung: m\"oglicher Bias
	\item Alternative: Planning Poker
\end{itemize}
\item Abh\"angigkeit (Reihenfolge) der Projektbearbeitung
\item Externe Inputs oder Vorbedingungen f\"ur Arbeitspakete
\item Vertraglich zugesicherte Termine
\item Zuordnung zu:
\begin{itemize}
	\item Ressourcen
	\item Phasen
\end{itemize}
\item Zieldefinition (Definition of Done)
\end{itemize}
}

\frame{\frametitle{Kostensch\"atzung}
\framesubtitle{}
\begin{columns}[t] 
     \begin{column}[T]{6cm} 
     \textbf{NRC estimation}
     	\begin{itemize}
     		\item Basierend auf Aufwandssch\"atzung
		\item Erg\"anzend:
		\begin{itemize}
		\item Kundenbetreuung
		\item Reisekosten
		\item Externe Dienstleistungen (z.B. Tests, Abnahmen, ...)
		\item Prototypen, Muster
		\item Investitionen
		\end{itemize}
		\item Zu beachten:
		\begin{itemize}
		\item Stundens\"atze
		\item Kostenentwicklung
		\end{itemize}
		\item N\"utzlich: Checkliste
     	\end{itemize}
     \end{column}
     	\begin{column}[T]{6cm} 
	\textbf{RC estimation} (\cite{niazi05, pahlbeitz})
         \begin{itemize}
         	\item Intuitive Verfahren:
		\begin{itemize}
		\item Basierend auf Expertensch\"atzung
		\item Unterst\"utzt durch Regeln
		\end{itemize} 
         	\item Analogiebasierte Sch\"atzung
		\begin{itemize}
		\item \"Ahnlichkeit
		\item Komplexit\"at 
		\end{itemize} 
     		\item Parametrische Sch\"atzung:
		\begin{itemize}
		\item z.B. Gewicht, Material oder kombiniert %, z.B. \tiny
%		\begin{equation*} 
%		C = FC+\left(C_{co} N_{co} +  \frac{C_{rm}TF}{1-SC} \right)W
%		\end{equation*}
		\end{itemize}
		\item Analytische Verfahren, z.B.
		\begin{itemize}
		\item Bearbeitungssimulation
		\item Feature based cost estimation
		\end{itemize}
     	\end{itemize}
     \end{column}
 \end{columns}
}

\frame{\frametitle{Risiko}
\framesubtitle{}
\begin{columns}[t] 
     \begin{column}[T]{6cm} 
     	\begin{center}
     		\Large Wenn Sie alle Risiken vermeiden wollen, haben Sie bald keine Risiken mehr zu vermeiden, weil Sie nicht mehr im Geschäft sind.
     	\end{center}	
\begin{flushright}
 Josef Ackermann
\end{flushright}
     \end{column}
     	\begin{column}[T]{6cm} 
         	\begin{center}
            		\includegraphics[width=0.8\textwidth]{Ackermann}\source{Quelle: Agencia Brasil}
        		\end{center}
     \end{column}
 \end{columns}
}

\subsection{Ausgew\"ahlte Elementare Qualit\"atstools}
\frame{\frametitle{Ausgew\"ahlte elementare Qualit\"atstools}
\framesubtitle{}
\begin{columns}[t] 
     \begin{column}[T]{6cm} 
     	\begin{itemize}
     		\item Histogramm
		\begin{itemize}
		\item Diagramm relativer H\"aufigkeiten
		\end{itemize}
		\item 5 x Warum
		\begin{itemize}
		\item Wiederholtes Fragen nach der Ursache f\"uhrt zur Root-Cause
		\end{itemize}
		\item Paretoanalyse
		\begin{itemize}
		\item 80\% Fehlerh\"aufigkeit bei 20\% der Arten
		\end{itemize}
		\item Ishikawa-Diagramm
		\begin{itemize}
		\item Systematische Ursachensuche
		\end{itemize}
     	\end{itemize}
     \end{column}
     	\begin{column}[T]{6cm} 
         	\begin{center}
            		\includegraphics[width=0.8\textwidth]{Histogramm}\source{}
        		\end{center}
     \end{column}
 \end{columns}
}

\subsection{Quality Function Deployment (QFD)}
\subsectionpage

\subsection{Quality Function Deployment (QFD)}
\frame{\frametitle{Quality Function Deployment (QFD) - Einf\"uhrung}
\framesubtitle{``Copy the spririt, not the form.''}
\begin{columns}[t] 
     \begin{column}[T]{6cm} 
     	\begin{itemize}
     		\item Hauptidee: Korrelationen feststellen
		\item Zeilen: Was?
		\begin{itemize}
		\item Was braucht der Kunde?
		\end{itemize}
		\item Spalten: Wie?
		\begin{itemize}
		\item Wie bekommt der Kunde das Gew\"unschte?
		\end{itemize}
		\item Gewichtung der Merkmale
		\item Autokorrelation zeigt Widerspr\"uche auf
		\item Korrelationsmatrizen an jeder Schnittstelle denkbar
     	\end{itemize}
     \end{column}
     	\begin{column}[T]{6cm} 
         	\vspace{-.5cm}
         	\begin{center}
            		\includegraphics[width=0.8\textwidth]{QFD7}\source{}
        		\end{center}

     \end{column}
 \end{columns}
}

\frame{\frametitle{Quality Function Deployment (QFD) - Vor- und Nachteile}
\framesubtitle{``Copy the spririt, not the form.''}
\begin{columns}[t] 
     \begin{column}[T]{6cm} 
     \vspace{-.5cm}
     	\begin{itemize}
     		\item[-] Exponentielles Wachstum
		\item[-] Mangelnde Kunden- und Anwenderinformationen
		\item[-] Schnittstellen = Machtkonflikte
		\item[+] Qualit\"atssteigerung
		\item[+] Anforderungen und Zielkonflikte fr\"uhzeitig identifizieren
		\item[+] Optimale Anforderungsabdeckung
		\item[+] Abstraktion von L\"osungen 
		\begin{itemize}
		\item Kundenverst\"andnis
		\item Innovationsans\"atze
		\end{itemize}
		\item[+] Nachvollziehbarkeit der Entscheidungen 
     	\end{itemize}
     \end{column}
     	\begin{column}[T]{6cm} 
		\vspace{-.5cm}
         	\begin{center}
            		\includegraphics[width=0.8\textwidth]{QFD7}\source{}
        		\end{center}
     \end{column}
 \end{columns}
}

\frame{\frametitle{QFD in der Praxis}
\framesubtitle{}
\begin{itemize}
\item H\"aufig nur erste Planungsphase durchgef\"uhrt
\item Teamzusammensetzung
\begin{itemize}
	\item Abteilungs\"ubergreifend: Entwicklung, Konstruktion, Marketing, Vertrieb, Fertigung, QM, Einkauf
	\item Erfahrener Moderator
\end{itemize}
\item (M\"oglicher) Ablauf:
\begin{itemize}
	\item Ermittlung und Gewichtung der Kundenforderungen
	\item Ableitung korrelierender Merkmale
	\item Korrelationen bewerten
	\item Wechselwirkungen pr\"ufen
	\item Festlegung von Optimierungsrichtungen f\"ur die Merkmale
	\item Benchmarking
	\item Festlegung von Zielgr\"o{\ss}en f\"ur Merkmale
\end{itemize}
\end{itemize}
}

\frame{\frametitle{Vorgehen QFD}
\framesubtitle{Am Beispiel des Antriebssystems Railway Challenge}
\begin{columns}[t] 
     \begin{column}[T]{6cm} 
     	\begin{enumerate}
     		\item Kundenanforderungen sammeln und wichten
		\item Merkmale ableiten
		\item Optimierungsrichtung festlegen
		\item Korrelation Anforderungen/Merkmale
		\item Wechselwirkung der Merkmale
		\item Absolutes Gewicht bestimmen
		\item Benchmarking gegen Wettbewerb
		\item Weiterf\"uhren, z.B.:
		\begin{itemize}
		\item Festlegen Zielwerte
		\item Technische Bedeutung
		\end{itemize}
     	\end{enumerate}
     \end{column}
     	\begin{column}[T]{6cm} 
         	\begin{center}
			\vspace{-.5cm}
            		\only<1>{\includegraphics[width=0.8\textwidth]{QFD1}\source{}}
			\only<2>{\includegraphics[width=0.8\textwidth]{QFD2}\source{}}
			\only<3>{\includegraphics[width=0.8\textwidth]{QFD3}\source{}}
			\only<4>{\includegraphics[width=0.8\textwidth]{QFD4}\source{}}
			\only<5>{\includegraphics[width=0.8\textwidth]{QFD5}\source{}}
			\only<6>{\includegraphics[width=0.8\textwidth]{QFD6}\source{}}
			\only<7>{\includegraphics[width=0.8\textwidth]{QFD7}\source{}}
        		\end{center}
     \end{column}
 \end{columns}
}


\subsection{Fehlerbaumanalyse (FTA)}
\subsectionpage

\frame{\frametitle{Einf\"uhrung Fehlerbaumanalyse}
\framesubtitle{Das Tool FTA stammt urspr\"unglich aus den Bereichen Aerospace und Reaktortechnik.}
\begin{itemize}
\item Ermittlung der logischen Verkn\"upfungen von Komponenten- und Subsystemausf\"allen
\item Ausgangspunkt: Fehlerevent (Core Hazard, Top Event)
\item Ermittlung aller m\"oglichen Ausfallskombinationen
\item Darstellung als Fehlerbaum
\begin{itemize}
		\item Endlicher gerichteter Graph
		\item Endlich viele Eing\"ange
		\item Ein Ausgang (Core Hazard)
	\end{itemize} 
\item Ziele der FTA:
\begin{itemize}
		\item Systematische Identifikation aller m\"oglichen Ausfallkombinationen, die zum Core Hazard f\"uhren
		\item Ermittlung von Zuverl\"assigkeitskenngr\"o{\ss}en:
		\begin{itemize}
		\item Eintrittsh\"aufigkeit Ausfallkombinationen/Core Hazard
		\item Unverf\"ugbarkeit des Systems
		\end{itemize}
		\item Aufstellung eines grafischen Systemmodells
		\end{itemize}
\end{itemize}
}

\frame{\frametitle{Vorgehen FTA}
\framesubtitle{}
\begin{columns}[t] 
     \begin{column}[T]{6cm} 
     	\begin{itemize}
     		\item Systemanalyse
		\begin{itemize}
		\item Funktionen, Leistungsziele, Abweichungen
		\item Betriebszust\"ande
		\item Umgebungsbedingungen
		\item Hilfsquellen
		\item Systemkomponenten und Zusammenwirken
		\item Analyse Operation und Verhalten
		\end{itemize}
		\item Aufstellung Fehlerbaum
		\begin{itemize}
		\item Core Hazard
		\item Logisch 1 entspricht Ausfall
		\end{itemize}
     	\end{itemize}
     \end{column}
     	\begin{column}[T]{6cm} 
         	\begin{enumerate}
            		\item Systemuntersuchung
			\item Festlegung des unerw\"unschten Ereignisses, Ausfallkriterien
			\item Festlegung Zuverl\"assigkeitskenngr\"o{\ss}e, Zeitintervalle
			\item Ausfallarten der Komponenten
			\item Fehlerbaum aufstellen
			\item Bestimmung Eingangsgr\"o{\ss}en
			\item Auswertung Fehlerbaum, Bewertung
        		\end{enumerate}
     \end{column}
 \end{columns}
}

\frame{\frametitle{Blockdiagramm Antriebssystem}
\framesubtitle{}
\begin{center}
 \includegraphics[width = .9\textwidth]{RCBlockDiagram}
\end{center}
}

\frame{\frametitle{Resultierender Fehlerbaum}
\framesubtitle{}
\begin{center}
 \includegraphics[width = \textwidth]{FTA}
\end{center}
}


\subsection{Failure Mode and Effect Analysis (FMEA)}
\subsectionpage

\frame{\frametitle{FMEA - Einf\"uhrung}
\framesubtitle{}
\begin{itemize}
\item Arten:
\begin{itemize}
		\item Design-FMEA: Produkt-/Bauteilebene
		\item Prozess-FMEA
		\end{itemize}
\item Vorgehen:
\begin{itemize}
		\item Team-Bildung: interdisziplin\"ar
		\item System- und Funktionsanalyse: Funktionsstruktur
		\item Fehleranalyse
		\item Bewertung mittels Risikopriorit\"atszahl (RPZ):
		\begin{itemize}
		\item $A \in[1,10]$: Auftretenswahrscheinlichkeit
		\item $B \in[1,10$: Bedeutung des Fehlers f\"ur den Kunden
		\item $E \in[1,10$: Wahrscheinlichkeit der Entdeckung (vor der Auslieferung)
		\item $RPZ = A B E$
		\end{itemize}
		\end{itemize}	
\end{itemize}
}

\frame{\frametitle{FMEA - Auswertung}
\framesubtitle{}
\begin{columns}[t] 
     \begin{column}[T]{6cm} 
     	\begin{itemize}
     		\item Kritik: Multiplikation ordinal skalierter Merkmale nicht definiert
		\item Eindeutigkeit RPZ fraglich
		\item Optimierung h\"aufig nach:
		\begin{itemize}
		\item Bedeutung $B$
		\item Technische Kardinalit\"at $B A$
		\end{itemize}
		\item Vergleich mit akzeptierten Risiken oder \textit{ALARP}
		\item Wichtig: Nachverfolgung der Ma{\ss}nahmen
     	\end{itemize}
     \end{column}
     	\begin{column}[T]{6cm} 
         	\begin{center}
            		\includegraphics[width=0.8\textwidth]{Risikograph}\source{}
        		\end{center}
     \end{column}
 \end{columns}
}

%%%% Einf\"uhrung FMT


\frame[allowframebreaks]{\frametitle{Literatur}
\framesubtitle{}
\nocite{pfeifer10}
%\nocite{hoischen}
%\renewcommand{\bibname}{References}
%\renewcommand{\refname}{References}
%\addcontentsline{toc}{chapter}{References}
\bibliographystyle{abbrv}
\bibliography{../../../bib}
}


\end{document}
