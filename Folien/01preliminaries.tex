\begin{frame} % Cover slide
\titlepage
\end{frame}

\frame{\frametitle{Prof. Dr. Raphael Pfaff}
\framesubtitle{Lehr- und Forschungsgebiet Schienenfahrzeugtechnik}
\begin{columns}[t] 
     \begin{column}[T]{7cm} 
     	\begin{itemize}
		\item[] \includegraphics[width=0.4cm]{Email} \hspace{.1cm} pfaff@fh-aachen.de
		\item[] \includegraphics[width=0.4cm]{Twitter} \hspace{.1cm} @RailProfAC
		\item[] \includegraphics[width=0.4cm]{Wordpress} \hspace{.1cm} www.raphaelpfaff.net
		\item[] Prezume: \texttt{\url{http://goo.gl/iq6lhh}}
		\vspace{1cm}
		\item Raum 03207
		\item Sprechstunde nach Vereinbarung
     	\end{itemize}
	
     \end{column}
     	\begin{column}[T]{5cm} 
         	\begin{center}
            		\includegraphics[width=0.8\textwidth]{Profilklein}
        		\end{center}
     \end{column}
 \end{columns}
}

\frame{\frametitle{Anforderungen ``Second Cycle'' - Master}
\framesubtitle{Anforderungen gem\"a{\ss} Dublin Descriptors}
\begin{columns}[t] 
     \begin{column}[T]{8cm} 
     	\begin{itemize}
     		\item Knowledge and understanding founed upon and extends or enhances that typically associated with Bachelor's level:
		\begin{itemize}
		\item Provides basis or opportunity for originality in developing and applying ideas
		\item Often within a research context
		\end{itemize}
		\item Apply their knowledge and understanding and problem solving abilities in new or unfamiliar environments 
		\item Have the ability to integrate knowledge and handle complexity and formulate judgements with incomplete or limited information
		\item Have the learning skills to allow them to continue to study in a manner that may be largely self-directed or autonomous
     	\end{itemize}
     \end{column}
     	\begin{column}[T]{4cm} 
         	\begin{center}
	\vspace{1.5cm}
            		\includegraphics[width=0.8\textwidth]{GraduationHat}
        		\end{center}
     \end{column}
 \end{columns}
}

\frame{\frametitle{Anforderungen ``Niveau 7'' - Master}
\framesubtitle{Anforderungen gem\"a{\ss} Deutschem Qualifizierungsrahmen}
\begin{columns}[t] 
     \begin{column}[T]{7cm} 
     	\begin{itemize}
     		\item Umfassendes, detailliertes und spezialisiertes Wissen
		\begin{itemize}
		\item Auf dem neuesten Erkenntnisstand
		\item Erweitertes Wissen in angrenzenden Bereichen 
		\end{itemize}
		\item Spezialisierte fachliche oder konzeptionelle Fertigkeiten zur L�sung auch strategischer Probleme
		\begin{itemize}
		\item Neue L\"osungen erarbeiten und bewerten
		\end{itemize}
		\item Gruppen oder Organisationen* im Rahmen komplexer Aufgabenstellungen verantwortlich leiten und ihre Arbeitsergebnisse vertreten.
		\item F�r neue anwendungs- oder forschungsorientierte Aufgaben Ziele definieren
		\item Wissen eigenst\"andig erschlie{\ss}en
     	\end{itemize}
     \end{column}
     	\begin{column}[T]{5cm} 
         	\begin{center}
	\vspace{1cm}
            		\includegraphics[width=0.8\textwidth]{GraduationHat}
        		\end{center}
     \end{column}
 \end{columns}}
\frame{\frametitle{Rolle des Lehrenden}
\framesubtitle{}
\begin{columns}[t] 
     \begin{column}[T]{6cm} 
     \vspace{1cm}
     	\begin{quote}
     		A teacher is never a giver of truth; he is a guide, a pointer to the truth that each student must find for himself.
     	\end{quote}
	\flushright Bruce Lee
	
     \end{column}
     	\begin{column}[T]{6cm} 
         	\begin{center}
            		\includegraphics[width=0.8\textwidth]{Bruce}
        		\end{center}
     \end{column}
 \end{columns}
}

\frame{\frametitle{Themenplan}
\framesubtitle{Die angegebenen Kapitel aus \cite{keferstein}, \cite{hoffmann} und \cite{eberlin} sind Pflichtlekt\"ure. Alle sind als E-Book verf\"ugbar.}
\begin{center}
%\hspace{2cm}
\begin{tabular}{|c|l|l|l|}
\hline
KW & Thema & Kapitel \\ \hline
40 & Einf\"uhrung, Grundlagen  & \cite[Kap. 1.1, 2.1, 2.2]{keferstein} \\ \hline
41 & Pr\"ufdatenerfassung und -auswertung &  \\ \hline
42 & Fahrdynamik:  & 2.2.4-2.2.7 & Fahrzeugmassen z.B. EN 15663 (Cordes)\\  \hline
43 & Einf\"uhrung Spurf\"uhrung & 2.2.1-2.2.3 & - \\ \hline
44 &  Spurf\"uhrung: & 2.3 &  Verschiebung n\"otig\\ \hline
45 & Exkursion DB Krefeld & - & - \\ \hline
46 & Spurf\"uhrung: & 2.4 &  Pr\"ufung gem. EN 14363\\ \hline
47 & Zug- und Sto{\ss}einrichtungen: & 5.5 & Brandschutz gem. EN45545\\ 
48 & Bauarten und Anforderungen & & (B\"achler)\\ \hline
\end{tabular}
\end{center}
}


\frame{\frametitle{Themen\"ubersicht Fertigungsmesstechnik}
\framesubtitle{}
\begin{itemize}
\item Grundbegriffe der Fertigungsmesstechnik 
\item Grundlagen
\begin{itemize}
\item Ma{\ss}verk\"orperungen 
\item Messunsicherheit und Ma{\ss}abweichung 
\item Zeichnungseintragungen und Tolerierungen 
\end{itemize}
\item Pr\"ufdatenerfassung
\begin{itemize}
\item \"Ubersicht der Verfahren 
\item Werkstattpr\"ufmittel 
\item Messwertaufnehmer 
\item Lehrende Pr\"ufung
\end{itemize}
\item Pr\"ufdatenauswertung
\begin{itemize}
\item Statistische Grundlagen 
\item Pr\"ufprozesseignung 
\item Erforderliche Messgenauigkeit 
\end{itemize}
\item R\"uckf\"uhrbarkeit 
\item Monte-Carlo-Simulation und Faltungsintegrale
\end{itemize}
}
