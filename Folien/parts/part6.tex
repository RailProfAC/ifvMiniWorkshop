% !TEX root = ../SFV15001_MdQM_Fertigungsmesstechnik_Rev01.tex
%% \section{Erforderliche Messgenauigkeit}
%\label{Sec:Messgenauigkeit}
%%\offslide{Erforderliche Messgenauigkeit}
%
%\frame{
%\frametitle{Erforderliche Messgenauigkeit}
%\framesubtitle{Die Einhaltung gewisser Intervalle kann nur mit einer gegebenen Wahrscheinlichkeit bestimmt werden.}{
%\begin{center}
%\begin{picture}(320,160)(0,0)
%\thicklines
%\put(76,23){\line(0,1){120}}
%\put(267,23){\line(0,1){120}}
%\put(10,10){Untere Toleranzgrenze (UT)}
%\put(190,10){Obere Toleranzgrenze (OT)}
%\put(171.4,115){\oval[0](191,15)}
%\put(130,112){Toleranzzone}
% 
%\put(0,0){\includegraphics[width=1.02\textwidth]{Toleranz}}
%\put(76,23){\line(0,1){120}}
%\put(267,23){\line(0,1){120}}
%\put(10,10){Untere Toleranzgrenze (UT)}
%\put(190,10){Obere Toleranzgrenze (OT)}
%\put(171.4,115){\oval[0](191,15)}
%\put(130,112){Toleranzzone}
% 
%\put(55,130){\vector(1,0){21.5}
%\put(21,0){\vector(-1,0){21.5}}
%\put(8,2){$U$}}
%\put(55,23){\line(0,1){120}}
%\put(76,130){\vector(1,0){21.5}
%\put(21,0){\vector(-1,0){21.5}}
%\put(8,2){$U$}}
%\put(97,23){\line(0,1){120}}
%
%\put(246,23){\line(0,1){120}}
%\put(246,130){\vector(1,0){21.5}
%\put(21,0){\vector(-1,0){21.5}}
%\put(8,2){$U$}}
%\put(288,23){\line(0,1){120}}
%\put(267,130){\vector(1,0){21.5}
%\put(21,0){\vector(-1,0){21.5}}
%\put(8,2){$U$}}  
%\put(171.5,97){\oval[0](149,15)}
%\put(130,92){\"Ubereinstimmung}  
%\put(294,30){\rotatebox{90}{Ablehnung durch Kunden}}
%\put(41,30){\rotatebox{90}{Ablehnung durch Kunden}}  
%\put(171.5,77){\oval[0](107,15)}
%\put(130,72){Nutzbare Toleranz} 
%\put(97,77){\vector(1,0){21.5}
%\put(21,0){\vector(-1,0){21.5}}
%\put(8,2){$U$}} 
%\put(225,77){\vector(1,0){21.5}
%\put(21,0){\vector(-1,0){21.5}}
%\put(8,2){$U$}} 
%
%\end{picture}
%\end{center}
%}}
%
%\frame{\frametitle{Nutzbare Toleranz}
%\framesubtitle{Die Messunsicherheit bestimmt die nutzbare Toleranz.}
%\begin{columns}[t] 
%     \begin{column}[T]{6cm} 
%     	\begin{itemize}
%     		\item Pr\"azisere Messtechnik erlaubt Nutzung breiterer Toleranzen
%		\item Kostenreduzierung in der Fertigung
%		\item Reduzierte Ausfallquoten
%		\item Faustregel \textit{(Goldene Regel)} \\
%		$U \leq \frac{1}{10} \left(\mathsf{OT} - \mathsf{UT}\right)$
%     	\end{itemize}
%     \end{column}
%     	\begin{column}[T]{6cm} 
%         	\begin{center}
%            		\begin{picture}(320,100)(0,0)
%			\setlength{\unitlength = 0.5pt}
%        %\thicklines
%        \tiny
%        \put(76,23){\line(0,1){120}}
%        \put(267,23){\line(0,1){120}}
%        \put(10,10){Untere Toleranzgrenze (UT)}
%        \put(190,10){Obere Toleranzgrenze (OT)}
%        \put(171.4,115){\oval[0](191,15)}
%        \put(130,112){Toleranzzone}
%        \put(0,0){\includegraphics[width=0.95\textwidth]{Toleranz}}
%        \put(76,23){\line(0,1){120}}
%        \put(267,23){\line(0,1){120}}
%        \put(10,10){Untere Toleranzgrenze (UT)}
%        \put(190,10){Obere Toleranzgrenze (OT)}
%        \put(171.4,115){\oval[0](191,15)}
%        \put(130,112){Toleranzzone}
%        \put(55,130){\vector(1,0){21.5}
%        \put(21,0){\vector(-1,0){21.5}}
%        \put(8,2){$U$}}
%        \put(55,23){\line(0,1){120}}
%        \put(76,130){\vector(1,0){21.5}
%        \put(21,0){\vector(-1,0){21.5}}
%        \put(8,2){$U$}}
%        \put(97,23){\line(0,1){120}}
%        
%        \put(246,23){\line(0,1){120}}
%        \put(246,130){\vector(1,0){21.5}
%        \put(21,0){\vector(-1,0){21.5}}
%        \put(8,2){$U$}}
%        \put(288,23){\line(0,1){120}}
%        \put(267,130){\vector(1,0){21.5}
%        \put(21,0){\vector(-1,0){21.5}}
%        \put(8,2){$U$}}  
%        \put(171.5,97){\oval[0](149,15)}
%        \put(130,92){\"Ubereinstimmung} 
%        \put(294,30){\rotatebox{90}{Ablehnung}}
%        \put(41,30){\rotatebox{90}{Ablehnung}} 
%        \put(171.5,77){\oval[0](107,15)}
%        \put(125,72){Nutzbare Toleranz} 
%        \put(97,77){\vector(1,0){21.5}
%        \put(21,0){\vector(-1,0){21.5}}
%        \put(8,2){$U$}} 
%        \put(225,77){\vector(1,0){21.5}
%        \put(21,0){\vector(-1,0){21.5}}
%        \put(8,2){$U$}} 
%\end{picture}
%
%        		\end{center}
%     \end{column}
% \end{columns}
%}
%
%\section{Pr\"ufprozesseignung}
%\frame{\frametitle{Pr\"ufprozesseignung}
%\framesubtitle{}
%\begin{columns}[t] 
%     \begin{column}[T]{6cm} 
%     	\begin{itemize}
%     		\item Ermittlung Teil des Pr\"ufmittelmanagements
%		\item Vornehmlich f\"ur quantitative Merkmale anwendbar
%		\item Typisch: Messprozesse
%		\item Teilgebiet: Pr\"ufmittelf\"ahigkeit
%		\item i.d.R. Voraussetzung: Normalverteilung
%		\begin{itemize}
%		\item Zentraler Grenzwertsatz
%		\end{itemize} 
%		\item Vier Methoden:
%		\begin{itemize}
%		\item Goldene Regel
%		\item Pr\"ufprozessf\"ahigkeit
%		\item R\&R-Studie
%		\item Konformit\"atspr\"ufung
%		\end{itemize}
%     	\end{itemize}
%     \end{column}
%     	\begin{column}[T]{6cm} 
%         	\begin{center}
%			\only<1>{
%            		\includegraphics[width=0.95\textwidth]{Prozessfahigkeit}\\
%			Beherrscht}
%			\only<2>{
%            		\includegraphics[width=0.95\textwidth]{Prozessfahigkeit2}\\
%			Mittelwertsdrift}
%			\only<3>{
%            		\includegraphics[width=0.95\textwidth]{Prozessfahigkeit3}\\
%			Ver\"anderliche Streuung}
%        		\end{center}
%     \end{column}
% \end{columns}
%}
%
%
%
%\frame{\frametitle{Prozessf\"ahigkeitsindex}
%\framesubtitle{Der \textit{Process Capability Index} gibt skalare Werte zum Vergleich der Messprozessf\"ahigkeit an.}
%    	\begin{itemize}
%		\item Allgemein
%     			\begin{equation*}
%				C_{p} = \frac{T}{x_{99{,}865\%} - x_{0{;}135\%}}
%			\end{equation*}
%			\begin{equation*}
%			C_{pk} = \min\left(\frac{OT - x_{50\%}}{x_{99{,}865\%} - x_{50\%}}, \, \frac{\left|UT - x_{50\%}\right|}{\left|x_{0{,}135\%} - x_{50\%}\right|}  \right)
%			\end{equation*}
%			\normalsize
%		\item F\"ur Normalverteilung
%			\begin{equation*}
%				C_{p} = \frac{T}{6 \sigma} \approx \frac{T}{6s}
%			\end{equation*}
%			\begin{equation*}
%			C_{pk} = \min\left(\frac{OT - \mu}{3\sigma}, \, \frac{\left|UT -\mu\right|}{3\sigma}  \right) \approx \min\left(\frac{OT - \mu}{3s}, \, \frac{\left|UT -\mu\right|}{3s}  \right)
%			\end{equation*}
%			\item $T$, $UT$, $OT$ bezogen auf Messprozess
%		\end{itemize} 
%}
%
%\frame{\frametitle{Prozessf\"ahigkeitsindex}
%\framesubtitle{}
%\begin{columns}[t] 
%     \begin{column}[T]{5cm} 
%     	\begin{itemize}
%		\item $C_{p}$: Beurteilung Streuung
%		\item $C_{pk}$: Beurteilung Mittelwertslage
%     		\item Geforderte Werte brachen- bzw. unternehmensabh\"angig
%		\item H\"aufig: $C_{p} \geq 1{,}33$ und $C_{pk} \geq 1{,}33$
%		\begin{itemize}
%		\item Entspricht 75\% Toleranzausnutzung
%		\end{itemize}
%     	\end{itemize}
%     \end{column}
%     	\begin{column}[T]{7cm} 
%         	\begin{itemize}
%		\item $C_{p} \geq 1{,}33$: 
%		\begin{itemize}
%		\item Fehlerfreie Fertigung unproblematisch
%		\end{itemize}
%		\item $ 1 \leq C_{p} \leq 1{,}33$: 
%		\begin{itemize}
%		\item Fehlerfreie Fertigung problematisch
%		\end{itemize}
%		\item $C_{p} < 1$: 
%		\begin{itemize}
%		\item Fehlerfreie Fertigung nicht m\"oglich
%		\item 100\% Sortierpr\"ufung veranlassen
%		\end{itemize}
%		\item $C_{pk} \geq 1{,}33$: 
%		\begin{itemize}
%		\item Fehlerfreie Fertigung unproblematisch
%		\end{itemize}
%		\item $C_{pk} < 1{,}33$: 
%		\begin{itemize}
%		\item Fehlerfreie Fertigung problematisch
%		\end{itemize}
%		\end{itemize}
%     \end{column}
% \end{columns}
%}
%
%\section{R\"uckf\"uhrbarkeit}
%\label{Sec:Rueckfuehr}
%\frame{\frametitle{Pr\"ufmittel\"uberwachung}
%\framesubtitle{}
%\begin{columns}[t] 
%     \begin{column}[T]{6cm} 
%     	\begin{itemize}
%     		\item Ab Verf\"ugbarkeit im Betrieb bis Ausmusterung
%		\item Notwendig z.B. wegen Verschlei{\ss}, Alterung oder Besch\"adigung
%		\item \"Uberwachung
%		\begin{itemize}
%		\item Pr\"ufmittelorientiert
%		\item Pr\"ufaufgabenorientiert
%		\end{itemize}
%		\item \"Uberwachung:
%		\begin{itemize}
%		\item Erstmalige Eignungspr\"ufung
%		\item \"Uberwachungspr\"ufung
%		\item Nach Justierung oder Reparatur
%		\end{itemize}
%		\item \"Uberwachungskennzeichen am Pr\"ufmittel
%     	\end{itemize}
%     \end{column}
%     	\begin{column}[T]{5cm} 
%         	\begin{definition}
%            		Die \textbf{Pr\"ufmittel\"uberwachung} umfasst alle Ma{\ss}nahmen, welche die Genauigkeit, Zuverl\"assigkeit und Einsatzf\"ahigkeit von  Pr\"ufmitteln gew\"ahrleisten. Die Qualit\"at und F\"ahigkeit eines Pr\"ufmittels ist anhand eines Pr\"ufplans oder einer Checkliste nachzuweisen.
%        		\end{definition}
%     \end{column}
% \end{columns}
%}
%
%\frame{\frametitle{R\"uckf\"uhrbarkeit}
%\framesubtitle{}
%\begin{columns}[t] 
%     \begin{column}[T]{6cm} 
%     	\begin{itemize}
%     		\item Zweck
%		\begin{itemize}
%		\item Weltweite Austauschbarkeit von Einzelzeilen
%		\item Weltweite Anerkennung von Messungen
%		\item Weltweite Vergleichbarkeit von Messger\"aten
%		\end{itemize}
%		\item Forderung ISO 9001: 
%		\begin{itemize}
%		\item Kalibrierung kann auf nationale oder internationale Normale zur\"uckgef\"uhrt werden
%		\end{itemize}
%		\item Ununterbrochene Kette von Vergleichsmessungen mit Messunsicherheiten, endend bei nationalen Normalen
%     	\end{itemize}
%     \end{column}
%     	\begin{column}[T]{5cm} 
%         	\begin{center}
%	\vspace{1cm}
%            		\includegraphics[width=0.9\textwidth]{Ruckfuhrbarkeit}
%        		\end{center}
%     \end{column}
% \end{columns}
%}
%
%\frame{\frametitle{R\"uckf\"uhrbarkeit}
%         	\begin{center}
%            		\includegraphics[width=0.95\textwidth]{Ruckfuhrbarkeit}
%		\rotatebox{90}{\tiny nach \cite[S. 442]{pfeifer10}}
%        		\end{center}
%		
%}

