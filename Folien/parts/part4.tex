% !TEX root = ../SFV15001_MdQM_Fertigungsmesstechnik_Rev02.tex

\subsection{Messunsicherheit}
\label{Sec:Messunsicherheit}

\frame{\frametitle{Messergebnis und Messunsicherheit}
\framesubtitle{Der Vergleich einer Messgr\"o{\ss}e mit einer Einheit gelingt nicht fehlerfrei.}
\begin{columns}[t] 
     \begin{column}[T]{6cm} 
     	\begin{itemize}
	 
     		\item Messger\"ateabweichungen
		\item Instabilit\"at der Messgr\"o{\ss}e
		\item Umwelteinfl\"usse (z.B. Temperatur)
		\item Beobachtereinfl\"usse
		\item Es werden unterschieden:
		\begin{itemize}
			\item Systematische Messabweichungen
			\begin{itemize}
			\item bekannt o. unbekannt
			\end{itemize}
			\item Zuf\"allige Messabweichungen
		\end{itemize}
     	\end{itemize}
     \end{column}
     	\begin{column}[T]{5cm} 
		\vspace{.4cm}
         	\begin{center}
            		\includegraphics[width=0.75\textwidth]{Messfehler}
        		\end{center}
     \end{column}
 \end{columns}
  
 \begin{definition}[Messergebnis]
	Das Messergebnis ist der Sch\"atzwert des wahren Wertes einer Messgr\"o{\ss}e.
	\end{definition}
}

%\frame{\frametitle{Quellen von Messabweichungen}
%\framesubtitle{Quellen von Messabweichungen lassen sich gut im Ishikawa-Diagramm ermitteln.}
%\begin{center}
%            		\includegraphics[width=0.95\textwidth]{Ishikawa}
%        		\end{center}
%\begin{itemize}
%\item Aufteilung systematisch/zuf\"allig, bekannt/unbekannt, Ausrei{\ss}er
%\end{itemize}
%}

%%\frame{\frametitle{Auswirkungen der Messabweichungen}
%%\framesubtitle{Messabweichungen formen die Verteilung der Stichproben f\"ur den Sch\"atzer.}
%%\begin{columns}[t] 
%%     \begin{column}[T]{5cm}
%%     \hspace{.1cm}
%%          \scriptsize 
%%	\begin{tabular}{|l|c|l|}
%%	\hline
%%	Abweichung & Auswirkung & Einrechnung \\ \hline
%%	\multirow{3}{40 pt}{bekannt, systematisch} & $\hat{\mu} = \mu + \gamma$& 		\multirow{3}{50 pt}{korrigieren, nicht in $U$ ber\"ucksichtigen} \\ & & \\ & &  \\ 		\hline
%%	\multirow{2}{40 pt}{unbekannt, systematisch} & $\hat{\mu} = \mu + \gamma$& 
%%	\multirow{2}{50 pt}{In $U$ einrechnen} \\ & &   \\ 
%%	\hline 
%%	\multirow{2}{40 pt}{zuf\"allig} & $\hat{\sigma} \geq \sigma$&
%%	 \multirow{2}{50 pt}{In $U$ einrechnen}  \\ & &  \\ 	\hline
%%	 \multirow{2}{40 pt}{Ausrei{\ss}er} & $\hat{\mu} \neq \mu$&
%%	 \multirow{2}{50 pt}{nicht in $U$ einrechnen}  \\ & &  \\ 	\hline
%%	\end{tabular}
%%     \end{column}
%%     \begin{column}[T]{.3cm}
%%     \end{column}
%%     	\begin{column}[T]{6cm} 
%%	
%%         	\begin{center}
%%            		\setlength{\unitlength}{0.0200bp}%
%%\begin{picture}(8000.00,9000.00)%
%%\tiny
%% \put(0,0){\includegraphics[scale=0.4]{Error}}%
%%\put(1204,896){\makebox(0,0)[r]{\strut{}0}}%
%%            \put(1204,1992){\makebox(0,0)[r]{\strut{}0.05}}%
%%            \put(1204,3088){\makebox(0,0)[r]{\strut{}0.1}}%
%%      \put(1204,4184){\makebox(0,0)[r]{\strut{}0.15}}%
%%      
%%      \put(1204,5280){\makebox(0,0)[r]{\strut{}0.2}}%
%%      
%%      \put(1204,6375){\makebox(0,0)[r]{\strut{}0.25}}%
%%      
%%      \put(1204,7471){\makebox(0,0)[r]{\strut{}0.3}}%
%%      
%%      \put(1204,8567){\makebox(0,0)[r]{\strut{}0.35}}%
%%      
%%      \put(1204,9663){\makebox(0,0)[r]{\strut{}0.4}}%
%%      
%%      \put(1372,616){\makebox(0,0){\strut{}0}}%
%%      
%%      \put(2597,616){\makebox(0,0){\strut{}2}}%
%%      
%%      \put(3821,616){\makebox(0,0){\strut{}4}}%
%%      
%%      \put(5046,616){\makebox(0,0){\strut{}6}}%
%%      
%%      \put(6270,616){\makebox(0,0){\strut{}8}}%
%%      
%%      \put(7495,616){\makebox(0,0){\strut{}10}}%
%%      
%%      \put(224,5279){\rotatebox{90}{\makebox(0,0){\strut{}Rel. Haeufigkeit}}}%
%%      
%%      \put(4433,196){\makebox(0,0){\strut{}x}}%
%%    %}%
%%    %\gplgaddtomacro\gplfronttext{%
%%      
%%      \put(6000,9460){\makebox(0,0)[l]{\strut{}$\hat{\mu}\! =\! \mu$}}%
%%      
%%      \put(6000,9180){\makebox(0,0)[l]{\strut{}$\hat{\mu}\! =\! \mu\!+\!\gamma$}}%
%%      
%%      \put(6000,8900){\makebox(0,0)[l]{\strut{}$\hat{\sigma}\! \geq\! \sigma$}}%
%%    %}%
%%   % \%gplbacktext
%%   
%%    %\gplfronttext
%%  \end{picture}%
%%        		\end{center}
%%     \end{column}
%% \end{columns}
%%}
%
%\frame{\frametitle{Definition und Darstellung von Messunsicherheiten}
%\framesubtitle{}
%\begin{columns}[t] 
%     \begin{column}[T]{5.5cm} 
%     	\begin{itemize}
%     		\item Korrekte Angabe eines Messergebnisses:
%		\begin{itemize}
%		\item Merkmalsbezeichnung
%		\item Messresultat (Stellenzahl nicht gr\"o{\ss}er als Angabe der Messunsicherheit)
%		\item Messunsicherheit $U$ (gleiche Einheit)
%		\item Erweiterungsfaktor $k$ (f\"ur erweiterte Messunsicherheit)%\footnote{Erl\"auterung folgt}
%		\end{itemize}
%		\vspace{.5cm}
%		\item Beispiel:
%		\begin{equation*}
%		A1 = \left( 120{,}037 \pm 0{,}007 \right) \mathrm{mm}, \, k = 2
%		\end{equation*}
%     	\end{itemize}
%     \end{column}
%     	\begin{column}[T]{5.5cm} 
%	\scriptsize
%         	\begin{definition}[Messunsicherheit]
%		[...] ein dem Messergebnis zugeordneter Parameter, der die Streuung der Werte kennzeichnet, die der Messgr\"o{\ss}e zugeordnet werden k\"onnen.
%        		\end{definition}
%		\begin{definition}[Erweiterte Messunsicherheit]
%		[...] ein Kennwert, der einen Bereich um dass Messergebnis kennzeichnet, von dem erwartet wird, dass er einen gro{\ss}en Anteil der Werte umfasst, die der Messgr\"o{\ss}e zugeordnet werden k\"onnen.
%        		\end{definition}
%     \end{column}
% \end{columns}
%}


\subsubsection{Verfahren gemäß GUM}
\frame{\frametitle{Verfahren zur Absch\"atzung der Messunsicherheit}
\begin{columns}[t] 
     \begin{column}[T]{6cm} 
     \textbf{Verfahren A}
     	\begin{itemize}
     		\item Absch\"atzung aus vorliegender Stichprobe
		\item $n$ Messwerte liegen vor und k\"onnen statistisch ausgewertet werden
		\item Die Messwerte sind n\"aherungsweise normalverteilt
		\begin{equation*}
		u = \frac{s}{\sqrt{n}}
		\end{equation*}
		\item[] $s$: Standardabweichung der Stichprobe
     	\end{itemize}
     \end{column}
     	\begin{column}[T]{6cm} 
	\textbf{Verfahren B}
         	\begin{itemize}
     		\item Ermittlung der minimalen/maximalen Messabweichung $a$ durch Modellbildung
		\item Ermittlung von Grenzen und Form der Verteilung der Messabweichungen
		\begin{equation*}
		u = \frac{a}{k}
		\end{equation*}
		\item[] $k = \sqrt{4}$ f\"ur Normalverteilung
		\item[] $k = \sqrt{3}$ f\"ur Gleichverteilung
		\item[] $k = \sqrt{6}$ f\"ur Dreiecksverteilung
     	\end{itemize}
     \end{column}
 \end{columns}
}



\frame{\frametitle{Sch\"atzfunktionen}
\framesubtitle{Eine Sch\"atzfunktion (Sch\"atzer) dient zur Ermittlung eines Parameter-Sch\"atzwertes aus empirischen Daten}
\begin{columns}[t] 
     \begin{column}[T]{5cm} 
     	\begin{itemize}
     		\item Grundlage: endlich viele Beobachtungen (Stichprobe)
		\begin{itemize}
			\item Sch\"atzer selbst fehlerbehaftet
			\item H\"aufig Zufallsvariable
		\end{itemize}
		\item Schlu{\ss} auf Grundgesamtheit
		\item Sch\"atzen einzelner Parameter der Verteilung
		\begin{itemize}
			\item Mittelwert
			\item Median
			\item Standardabweichung
		\end{itemize}
     	\end{itemize}
     \end{column}
     	\begin{column}[T]{6cm} 
		\begin{definition}[Zufallsvariable]
		Als Zufallsvariable bezeichnet man eine messbare Funktion von einem Wahrscheinlichkeitsraum in einen Messraum.
		\end{definition}
		\begin{definition}[Sch\"atzfunktion]
		Eine Sch\"atzfunktion dient dazu, aufgrund von empirischen Daten einer Stichprobe einen Schätzwert zu ermitteln und dadurch Informationen über unbekannte Parameter einer Grundgesamtheit zu erhalten.
		\end{definition}
     \end{column}
 \end{columns}
}

\frame{\frametitle{Sch\"atzfunktionen und Eigenschaften}
\framesubtitle{G\"angige Sch\"atzfunktionen und w\"unschenswerte Eigenschaften}
\begin{columns}[t] 
     \begin{column}[T]{6cm} 
     	\begin{itemize}
     		\item Mittelwert
		\begin{eqnarray*}
		\bar{X} = \frac{1}{n} \sum_{i=1}^{n} X_{i} \\
		\hat{\mu} = \bar{x} = \frac{1}{n} \sum_{i=1}^{n} x_{i} 
		\end{eqnarray*}
		\item Varianz 
		\begin{eqnarray*}
		S_{n}^{2} = \frac{1}{n-1} \sum_{i=1}^{n}\left(X_{i} - \bar{X}\right)^{2} \\
		\hat{\sigma}^{2} = s_{n}^{2} = \frac{1}{n-1} \sum_{i=1}^{n}\left(x_{i} - \bar{x}\right)^{2}
		\end{eqnarray*}
     	\end{itemize}
     \end{column}
     	\begin{column}[T]{5cm} 
         	\begin{itemize}
     		\item Erwartungstreue:
		\begin{itemize}
		\item Erwartungswert der Sch\"atzfunktion gleich wahrem Parameter
		\item Kein systematischer Fehler (Bias).
		\end{itemize}
		\item Konsistenz:
		\begin{itemize}
		\item Unsicherheit des Sch\"atzers nimmt f\"ur $n \rightarrow \infty$ ab
		\end{itemize}
		\item Effizienz:
		\begin{itemize}
		\item Minimale Varianz des Sch\"atzers
		\end{itemize}
		\item BLUE: Best Linear Unbiased Estimator
     	\end{itemize}
     \end{column}
 \end{columns}
}

%\frame{\frametitle{Standardfehler}
%\framesubtitle{Die Standardabweichung des Mittelwerts nimmt mit der Stichprobengr\"o{\ss}e $\backsim \frac{1}{\sqrt{n}}$ an.}
%     	\begin{itemize}
%     		\item Der Standardfehler des arithmetischen Mittels einer Stichprobe des Umfangs $n$ ist
%		\begin{equation*}
%			\sigma\left(\bar{X}\right) = \frac{\sigma}{\sqrt{n}}
%		\end{equation*}
%		mit $\sigma$ der Standardabweichung einer Einzelmessung.
%		\item Herleitung: 
%		\begin{itemize}
%			\item Mittelwert einer Stichprobe: $\bar{x} = \frac{1}{n} \sum_{i=1}^{n} x_{i}$
%			\item Sch\"atzer: $\bar{X} = \frac{1}{n} \sum_{i=1}^{n} X_{i}$ mit $X_{i}$ unabh\"angig und identisch verteilt mit $\sigma^{2} < \infty$.
%			\item Damit folgt f\"ur die Varianz $\Var\left(\bar{X}\right) = \left(\sigma\left(\bar{X}\right)\right)^{2}$: \tiny
%			\begin{equation*}
%					\Var\left(\bar{X}\right) = \Var\left(\frac{1}{n}\sum_{i=1}^{n} X_{i}\right) = \frac{1}{n^{2}} \Var\left(\sum_{i=1}^{n} X_{i}\right) = \frac{1}{n^{2}} \sum_{i=1}^{n}\Var\left(X_{i}\right) = \frac{1}{n^{2}} n \sigma^{2} = \frac{\sigma^{2}}{n}
%			\end{equation*} \normalsize
%			\item Damit folgt $\sigma\left(\bar{X}\right) = \frac{\sigma}{\sqrt{n}}$
%		\end{itemize}
%     	\end{itemize}
%   }



\frame{\frametitle{Bestimmung der Messunsicherheit durch Modellbildung}
\begin{itemize}
\item Ermitteln der Einflussgr\"o{\ss}en:
\begin{itemize}
\item Ermittlung systematischer Abweichungen, ggf. kompensieren.
\item Einfluss des Normals
\item Wiederholpr\"azision (Messunsicherheit bei wiederholter Messung an einem Messobjekt)
\item Temperatureinfluss
\end{itemize}
\item Modellbildung Messgr\"o{\ss}e $y = f\left(x_{1}, x_{2}, \ldots, x_{n}\right)$
\item Bestimmung kombinierte Standardunsicherheit $u_{c}$
\begin{equation*}
u_{c} = \sqrt{\left(\frac{\partial f}{\partial x_{1}} u_{x_{1}} \right)^{2} + \left(\frac{\partial f}{\partial x_{2}} u_{x_{n}} \right)^{2} + \ldots + \left(\frac{\partial f}{\partial x_{n}} u_{x_{n}}\right)^{2}}
\end{equation*}
\item Vereinfachung f\"ur Linearit\"at von $f$ und gleichem Gewicht der $x_{i}$
\begin{equation*}
u_{c} = \sqrt{u^{2}_{x_{1}} + u^{2}_{x_{2}} + \ldots + u^{2}_{x_{n}}}
\end{equation*}
\end{itemize}
}


\subsubsection{Bestimmung der Messunsicherheit durch Monte Carlo Simulation, Faltungsintegrale}


%\offslide{Warum ein weiteres Verfahren?}

%\frame{\frametitle{Kombinierte Wahrscheinlichkeitsdichte}
%\framesubtitle{}
%\begin{columns}[t] 
%     \begin{column}[T]{6cm} 
%     	\begin{itemize}
%     		\item Experiment: Wurf mit zwei W\"urfeln
%		\item Einzelne W\"urfel gleichverteilt
%		\item Summe dreiecksverteilt
%     	\end{itemize}
%     \end{column}
%     	\begin{column}[T]{6cm} 
%         	\begin{center}
%		 %\pgfplotsset{width=3cm,compat=1.12}
%		\begin{tikzpicture}[scale = 0.5]
%            \begin{axis}[
%                ybar,
%                ymin=0, xmin = 1, xmax = 6
%            ]
%            \addplot +[
%                hist={
%                    bins=6,
%%                    data min=1,
%%                    data max=6
%                }   
%            ] table [y=W1] {Faltung.dat};
%            \end{axis}
%            %\draw[draw = red, ultra thick] (3.41,0) -- (3.41,5.7); 
%            \end{tikzpicture}
%            \begin{tikzpicture}[scale = 0.5]
%            \begin{axis}[
%                ybar,
%                ymin=0, xmin = 1, xmax = 6
%            ]
%            \addplot +[
%                hist={
%                    bins=6,
%%                    data min=1,
%%                    data max=6
%                }   
%            ] table [y=W2] {Faltung.dat};
%            \end{axis}
%            %\draw[draw = red, ultra thick] (3.41,0) -- (3.41,5.7); 
%            \end{tikzpicture}
%
%% \begin{tikzpicture}[scale = 0.8]
%%            \begin{axis}[
%%                ybar,
%%                ymin=0, xmin = 2, xmax = 12
%%            ]
%%            \addplot +[
%%                hist={
%%                    bins=11,
%%%                    data min=1,
%%%                    data max=6
%%                }   
%%            ] table [y=S] {Faltung.dat};
%%            \end{axis}
%%            %\draw[draw = red, ultra thick] (3.41,0) -- (3.41,5.7); 
%%            \end{tikzpicture}
%%
%%        		\end{center}
%%     \end{column}
%% \end{columns}
%%}
%%
%%\offslide{Faltungsintegral}

\frame{\frametitle{Monte-Carlo-Simulation: Motivation}
\framesubtitle{Monte-Carlo-Simulationen (MC-Simulationen) k\"onnen zur Ermittlung kombinierter Wahrscheinlichkeiten eingesetzt werden.}
\begin{columns}[t] 
     \begin{column}[T]{6cm} 
     	\begin{itemize}
     		\item Statistische Verfahren:
		\begin{itemize}
		\item Hohe Stichprobenzahlen f\"ur Validit\"at n\"otig
		\item Zeitaufw\"andig
		\item Teuer
		\end{itemize}
		\item Fehlerfortpflanzung:
		\begin{itemize}
		\item Analytisch teils komplex
		\item Konservativ bei nicht normalverteilten Zufallsvariablen:
		\begin{itemize}
		\item Ann\"aherung Gleichverteilung durch Normalverteilung
		\end{itemize}
		\end{itemize}
	     	\end{itemize}
     \end{column}
     	\begin{column}{6cm} 
         	\begin{center}
            		% GNUPLOT: LaTeX picture with Postscript
\begingroup
  \makeatletter
  \providecommand\color[2][]{%
    \GenericError{(gnuplot) \space\space\space\@spaces}{%
      Package color not loaded in conjunction with
      terminal option `colourtext'%
    }{See the gnuplot documentation for explanation.%
    }{Either use 'blacktext' in gnuplot or load the package
      color.sty in LaTeX.}%
    \renewcommand\color[2][]{}%
  }%
  \providecommand\includegraphics[2][]{%
    \GenericError{(gnuplot) \space\space\space\@spaces}{%
      Package graphicx or graphics not loaded%
    }{See the gnuplot documentation for explanation.%
    }{The gnuplot epslatex terminal needs graphicx.sty or graphics.sty.}%
    \renewcommand\includegraphics[2][]{}%
  }%
  \providecommand\rotatebox[2]{#2}%
  \@ifundefined{ifGPcolor}{%
    \newif\ifGPcolor
    \GPcolorfalse
  }{}%
  \@ifundefined{ifGPblacktext}{%
    \newif\ifGPblacktext
    \GPblacktexttrue
  }{}%
  % define a \g@addto@macro without @ in the name:
  \let\gplgaddtomacro\g@addto@macro
  % define empty templates for all commands taking text:
  \gdef\gplbacktext{}%
  \gdef\gplfronttext{}%
  \makeatother
  \ifGPblacktext
    % no textcolor at all
    \def\colorrgb#1{}%
    \def\colorgray#1{}%
  \else
    % gray or color?
    \ifGPcolor
      \def\colorrgb#1{\color[rgb]{#1}}%
      \def\colorgray#1{\color[gray]{#1}}%
      \expandafter\def\csname LTw\endcsname{\color{white}}%
      \expandafter\def\csname LTb\endcsname{\color{black}}%
      \expandafter\def\csname LTa\endcsname{\color{black}}%
      \expandafter\def\csname LT0\endcsname{\color[rgb]{1,0,0}}%
      \expandafter\def\csname LT1\endcsname{\color[rgb]{0,1,0}}%
      \expandafter\def\csname LT2\endcsname{\color[rgb]{0,0,1}}%
      \expandafter\def\csname LT3\endcsname{\color[rgb]{1,0,1}}%
      \expandafter\def\csname LT4\endcsname{\color[rgb]{0,1,1}}%
      \expandafter\def\csname LT5\endcsname{\color[rgb]{1,1,0}}%
      \expandafter\def\csname LT6\endcsname{\color[rgb]{0,0,0}}%
      \expandafter\def\csname LT7\endcsname{\color[rgb]{1,0.3,0}}%
      \expandafter\def\csname LT8\endcsname{\color[rgb]{0.5,0.5,0.5}}%
    \else
      % gray
      \def\colorrgb#1{\color{black}}%
      \def\colorgray#1{\color[gray]{#1}}%
      \expandafter\def\csname LTw\endcsname{\color{white}}%
      \expandafter\def\csname LTb\endcsname{\color{black}}%
      \expandafter\def\csname LTa\endcsname{\color{black}}%
      \expandafter\def\csname LT0\endcsname{\color{black}}%
      \expandafter\def\csname LT1\endcsname{\color{black}}%
      \expandafter\def\csname LT2\endcsname{\color{black}}%
      \expandafter\def\csname LT3\endcsname{\color{black}}%
      \expandafter\def\csname LT4\endcsname{\color{black}}%
      \expandafter\def\csname LT5\endcsname{\color{black}}%
      \expandafter\def\csname LT6\endcsname{\color{black}}%
      \expandafter\def\csname LT7\endcsname{\color{black}}%
      \expandafter\def\csname LT8\endcsname{\color{black}}%
    \fi
  \fi
  \setlength{\unitlength}{0.0500bp}%
  \begin{picture}(3400.00,3000.00)%
  \tiny
    \gplgaddtomacro\gplbacktext{%
      \colorrgb{0.00,0.00,0.00}%
      \put(322,2335){\makebox(0,0)[r]{\strut{}0}}%
      \colorrgb{0.00,0.00,0.00}%
      \put(322,2562){\makebox(0,0)[r]{\strut{}20}}%
      \colorrgb{0.00,0.00,0.00}%
      \put(322,2790){\makebox(0,0)[r]{\strut{}40}}%
      \colorrgb{0.00,0.00,0.00}%
      \put(322,3017){\makebox(0,0)[r]{\strut{}60}}%
      \colorrgb{0.00,0.00,0.00}%
      \put(322,3244){\makebox(0,0)[r]{\strut{}80}}%
      \colorrgb{0.00,0.00,0.00}%
      \put(322,3472){\makebox(0,0)[r]{\strut{}100}}%
      \colorrgb{0.00,0.00,0.00}%
      \put(322,3699){\makebox(0,0)[r]{\strut{}120}}%
      \colorrgb{0.00,0.00,0.00}%
      \put(442,2135){\makebox(0,0){\strut{}-0.8}}%
      \colorrgb{0.00,0.00,0.00}%
      %\put(584,2135){\makebox(0,0){\strut{}-0.6}}%
      %\colorrgb{0.00,0.00,0.00}%
      \put(726,2135){\makebox(0,0){\strut{}-0.4}}%
      \colorrgb{0.00,0.00,0.00}%
      %\put(868,2135){\makebox(0,0){\strut{}-0.2}}%
      \colorrgb{0.00,0.00,0.00}%
      \put(1011,2135){\makebox(0,0){\strut{}0}}%
      \colorrgb{0.00,0.00,0.00}%
      %\put(1153,2135){\makebox(0,0){\strut{}0.2}}%
      \colorrgb{0.00,0.00,0.00}%
      \put(1295,2135){\makebox(0,0){\strut{}0.4}}%
      \colorrgb{0.00,0.00,0.00}%
      %\put(1437,2135){\makebox(0,0){\strut{}0.6}}%
      \colorrgb{0.00,0.00,0.00}%
      \put(1579,2135){\makebox(0,0){\strut{}0.8}}%
    }%
    \gplgaddtomacro\gplfronttext{%
    }%
    \gplgaddtomacro\gplbacktext{%
      \colorrgb{0.00,0.00,0.00}%
      \put(1819,2335){\makebox(0,0)[r]{\strut{}0}}%
      \colorrgb{0.00,0.00,0.00}%
      \put(1819,2608){\makebox(0,0)[r]{\strut{}50}}%
      \colorrgb{0.00,0.00,0.00}%
      \put(1819,2881){\makebox(0,0)[r]{\strut{}100}}%
      \colorrgb{0.00,0.00,0.00}%
      \put(1819,3153){\makebox(0,0)[r]{\strut{}150}}%
      \colorrgb{0.00,0.00,0.00}%
      \put(1819,3426){\makebox(0,0)[r]{\strut{}200}}%
      \colorrgb{0.00,0.00,0.00}%
      \put(1819,3699){\makebox(0,0)[r]{\strut{}250}}%
      \colorrgb{0.00,0.00,0.00}%
      \put(1939,2135){\makebox(0,0){\strut{}-1.5}}%
      \colorrgb{0.00,0.00,0.00}%
      \put(2166,2135){\makebox(0,0){\strut{}-1}}%
      \colorrgb{0.00,0.00,0.00}%
      \put(2394,2135){\makebox(0,0){\strut{}-0.5}}%
      \colorrgb{0.00,0.00,0.00}%
      \put(2621,2135){\makebox(0,0){\strut{}0}}%
      \colorrgb{0.00,0.00,0.00}%
      \put(2849,2135){\makebox(0,0){\strut{}0.5}}%
      \colorrgb{0.00,0.00,0.00}%
      \put(3076,2135){\makebox(0,0){\strut{}1}}%
    }%
    \gplgaddtomacro\gplfronttext{%
    }%
    \gplgaddtomacro\gplbacktext{%
      \colorrgb{0.00,0.00,0.00}%
      \put(322,440){\makebox(0,0)[r]{\strut{}0}}%
      \colorrgb{0.00,0.00,0.00}%
      \put(322,713){\makebox(0,0)[r]{\strut{}50}}%
      \colorrgb{0.00,0.00,0.00}%
      \put(322,986){\makebox(0,0)[r]{\strut{}100}}%
      \colorrgb{0.00,0.00,0.00}%
      \put(322,1258){\makebox(0,0)[r]{\strut{}150}}%
      \colorrgb{0.00,0.00,0.00}%
      \put(322,1531){\makebox(0,0)[r]{\strut{}200}}%
      \colorrgb{0.00,0.00,0.00}%
      \put(322,1804){\makebox(0,0)[r]{\strut{}250}}%
      \colorrgb{0.00,0.00,0.00}%
      \put(442,240){\makebox(0,0){\strut{}-1.5}}%
      \colorrgb{0.00,0.00,0.00}%
      \put(881,240){\makebox(0,0){\strut{}-1}}%
      \colorrgb{0.00,0.00,0.00}%
      \put(1320,240){\makebox(0,0){\strut{}-0.5}}%
      \colorrgb{0.00,0.00,0.00}%
      \put(1759,240){\makebox(0,0){\strut{}0}}%
      \colorrgb{0.00,0.00,0.00}%
      \put(2198,240){\makebox(0,0){\strut{}0.5}}%
      \colorrgb{0.00,0.00,0.00}%
      \put(2637,240){\makebox(0,0){\strut{}1}}%
      \colorrgb{0.00,0.00,0.00}%
      \put(3076,240){\makebox(0,0){\strut{}1.5}}%
    }%
    \gplgaddtomacro\gplfronttext{%
    }%
    \gplbacktext
    \put(0,0){\includegraphics{MCIntro}}%
    \gplfronttext
  \end{picture}%
\endgroup

        		\end{center}
     \end{column}
 \end{columns}
}

\frame{\frametitle{Monte-Carlo-Simulation: Grundlagen}
\framesubtitle{MC-Simulationen basieren auf der h\"aufig wiederholten Simulation von Zufallsexperimenten.}
\begin{columns}[t] 
     \begin{column}[T]{6cm} 
     	\begin{itemize}
     		\item Grundlage:
		\begin{itemize}
		\item Gesetz der gro{\ss}en Zahlen
		\item Pseudozufallsverteilungen der Parameter erzeugen
		\item Resultierendes Ergebnis bilden
		\item Analyse der Lage- und Streuparameter
		\end{itemize}
		\item Vorteile:
		\begin{itemize}
		\item Einfache Modellierung
		\item Einfache Durchf\"uhrung
		\end{itemize}
		\item Nachteil:
		\begin{itemize}
		\item Korrektheit der L\"osungen nicht immer einfach zu beurteilen
		\end{itemize}
	     	\end{itemize}
     \end{column}
     	\begin{column}{6cm} 
         	\begin{center}
            		% GNUPLOT: LaTeX picture with Postscript
\begingroup
  \makeatletter
  \providecommand\color[2][]{%
    \GenericError{(gnuplot) \space\space\space\@spaces}{%
      Package color not loaded in conjunction with
      terminal option `colourtext'%
    }{See the gnuplot documentation for explanation.%
    }{Either use 'blacktext' in gnuplot or load the package
      color.sty in LaTeX.}%
    \renewcommand\color[2][]{}%
  }%
  \providecommand\includegraphics[2][]{%
    \GenericError{(gnuplot) \space\space\space\@spaces}{%
      Package graphicx or graphics not loaded%
    }{See the gnuplot documentation for explanation.%
    }{The gnuplot epslatex terminal needs graphicx.sty or graphics.sty.}%
    \renewcommand\includegraphics[2][]{}%
  }%
  \providecommand\rotatebox[2]{#2}%
  \@ifundefined{ifGPcolor}{%
    \newif\ifGPcolor
    \GPcolorfalse
  }{}%
  \@ifundefined{ifGPblacktext}{%
    \newif\ifGPblacktext
    \GPblacktexttrue
  }{}%
  % define a \g@addto@macro without @ in the name:
  \let\gplgaddtomacro\g@addto@macro
  % define empty templates for all commands taking text:
  \gdef\gplbacktext{}%
  \gdef\gplfronttext{}%
  \makeatother
  \ifGPblacktext
    % no textcolor at all
    \def\colorrgb#1{}%
    \def\colorgray#1{}%
  \else
    % gray or color?
    \ifGPcolor
      \def\colorrgb#1{\color[rgb]{#1}}%
      \def\colorgray#1{\color[gray]{#1}}%
      \expandafter\def\csname LTw\endcsname{\color{white}}%
      \expandafter\def\csname LTb\endcsname{\color{black}}%
      \expandafter\def\csname LTa\endcsname{\color{black}}%
      \expandafter\def\csname LT0\endcsname{\color[rgb]{1,0,0}}%
      \expandafter\def\csname LT1\endcsname{\color[rgb]{0,1,0}}%
      \expandafter\def\csname LT2\endcsname{\color[rgb]{0,0,1}}%
      \expandafter\def\csname LT3\endcsname{\color[rgb]{1,0,1}}%
      \expandafter\def\csname LT4\endcsname{\color[rgb]{0,1,1}}%
      \expandafter\def\csname LT5\endcsname{\color[rgb]{1,1,0}}%
      \expandafter\def\csname LT6\endcsname{\color[rgb]{0,0,0}}%
      \expandafter\def\csname LT7\endcsname{\color[rgb]{1,0.3,0}}%
      \expandafter\def\csname LT8\endcsname{\color[rgb]{0.5,0.5,0.5}}%
    \else
      % gray
      \def\colorrgb#1{\color{black}}%
      \def\colorgray#1{\color[gray]{#1}}%
      \expandafter\def\csname LTw\endcsname{\color{white}}%
      \expandafter\def\csname LTb\endcsname{\color{black}}%
      \expandafter\def\csname LTa\endcsname{\color{black}}%
      \expandafter\def\csname LT0\endcsname{\color{black}}%
      \expandafter\def\csname LT1\endcsname{\color{black}}%
      \expandafter\def\csname LT2\endcsname{\color{black}}%
      \expandafter\def\csname LT3\endcsname{\color{black}}%
      \expandafter\def\csname LT4\endcsname{\color{black}}%
      \expandafter\def\csname LT5\endcsname{\color{black}}%
      \expandafter\def\csname LT6\endcsname{\color{black}}%
      \expandafter\def\csname LT7\endcsname{\color{black}}%
      \expandafter\def\csname LT8\endcsname{\color{black}}%
    \fi
  \fi
  \setlength{\unitlength}{0.0500bp}%
  \begin{picture}(3400.00,3000.00)%
  \tiny
    \gplgaddtomacro\gplbacktext{%
      \colorrgb{0.00,0.00,0.00}%
      \put(322,2335){\makebox(0,0)[r]{\strut{}0}}%
      \colorrgb{0.00,0.00,0.00}%
      \put(322,2562){\makebox(0,0)[r]{\strut{}20}}%
      \colorrgb{0.00,0.00,0.00}%
      \put(322,2790){\makebox(0,0)[r]{\strut{}40}}%
      \colorrgb{0.00,0.00,0.00}%
      \put(322,3017){\makebox(0,0)[r]{\strut{}60}}%
      \colorrgb{0.00,0.00,0.00}%
      \put(322,3244){\makebox(0,0)[r]{\strut{}80}}%
      \colorrgb{0.00,0.00,0.00}%
      \put(322,3472){\makebox(0,0)[r]{\strut{}100}}%
      \colorrgb{0.00,0.00,0.00}%
      \put(322,3699){\makebox(0,0)[r]{\strut{}120}}%
      \colorrgb{0.00,0.00,0.00}%
      \put(442,2135){\makebox(0,0){\strut{}-0.8}}%
      \colorrgb{0.00,0.00,0.00}%
      %\put(584,2135){\makebox(0,0){\strut{}-0.6}}%
      %\colorrgb{0.00,0.00,0.00}%
      \put(726,2135){\makebox(0,0){\strut{}-0.4}}%
      \colorrgb{0.00,0.00,0.00}%
      %\put(868,2135){\makebox(0,0){\strut{}-0.2}}%
      \colorrgb{0.00,0.00,0.00}%
      \put(1011,2135){\makebox(0,0){\strut{}0}}%
      \colorrgb{0.00,0.00,0.00}%
      %\put(1153,2135){\makebox(0,0){\strut{}0.2}}%
      \colorrgb{0.00,0.00,0.00}%
      \put(1295,2135){\makebox(0,0){\strut{}0.4}}%
      \colorrgb{0.00,0.00,0.00}%
      %\put(1437,2135){\makebox(0,0){\strut{}0.6}}%
      \colorrgb{0.00,0.00,0.00}%
      \put(1579,2135){\makebox(0,0){\strut{}0.8}}%
    }%
    \gplgaddtomacro\gplfronttext{%
    }%
    \gplgaddtomacro\gplbacktext{%
      \colorrgb{0.00,0.00,0.00}%
      \put(1819,2335){\makebox(0,0)[r]{\strut{}0}}%
      \colorrgb{0.00,0.00,0.00}%
      \put(1819,2608){\makebox(0,0)[r]{\strut{}50}}%
      \colorrgb{0.00,0.00,0.00}%
      \put(1819,2881){\makebox(0,0)[r]{\strut{}100}}%
      \colorrgb{0.00,0.00,0.00}%
      \put(1819,3153){\makebox(0,0)[r]{\strut{}150}}%
      \colorrgb{0.00,0.00,0.00}%
      \put(1819,3426){\makebox(0,0)[r]{\strut{}200}}%
      \colorrgb{0.00,0.00,0.00}%
      \put(1819,3699){\makebox(0,0)[r]{\strut{}250}}%
      \colorrgb{0.00,0.00,0.00}%
      \put(1939,2135){\makebox(0,0){\strut{}-1.5}}%
      \colorrgb{0.00,0.00,0.00}%
      \put(2166,2135){\makebox(0,0){\strut{}-1}}%
      \colorrgb{0.00,0.00,0.00}%
      \put(2394,2135){\makebox(0,0){\strut{}-0.5}}%
      \colorrgb{0.00,0.00,0.00}%
      \put(2621,2135){\makebox(0,0){\strut{}0}}%
      \colorrgb{0.00,0.00,0.00}%
      \put(2849,2135){\makebox(0,0){\strut{}0.5}}%
      \colorrgb{0.00,0.00,0.00}%
      \put(3076,2135){\makebox(0,0){\strut{}1}}%
    }%
    \gplgaddtomacro\gplfronttext{%
    }%
    \gplgaddtomacro\gplbacktext{%
      \colorrgb{0.00,0.00,0.00}%
      \put(322,440){\makebox(0,0)[r]{\strut{}0}}%
      \colorrgb{0.00,0.00,0.00}%
      \put(322,713){\makebox(0,0)[r]{\strut{}50}}%
      \colorrgb{0.00,0.00,0.00}%
      \put(322,986){\makebox(0,0)[r]{\strut{}100}}%
      \colorrgb{0.00,0.00,0.00}%
      \put(322,1258){\makebox(0,0)[r]{\strut{}150}}%
      \colorrgb{0.00,0.00,0.00}%
      \put(322,1531){\makebox(0,0)[r]{\strut{}200}}%
      \colorrgb{0.00,0.00,0.00}%
      \put(322,1804){\makebox(0,0)[r]{\strut{}250}}%
      \colorrgb{0.00,0.00,0.00}%
      \put(442,240){\makebox(0,0){\strut{}-1.5}}%
      \colorrgb{0.00,0.00,0.00}%
      \put(881,240){\makebox(0,0){\strut{}-1}}%
      \colorrgb{0.00,0.00,0.00}%
      \put(1320,240){\makebox(0,0){\strut{}-0.5}}%
      \colorrgb{0.00,0.00,0.00}%
      \put(1759,240){\makebox(0,0){\strut{}0}}%
      \colorrgb{0.00,0.00,0.00}%
      \put(2198,240){\makebox(0,0){\strut{}0.5}}%
      \colorrgb{0.00,0.00,0.00}%
      \put(2637,240){\makebox(0,0){\strut{}1}}%
      \colorrgb{0.00,0.00,0.00}%
      \put(3076,240){\makebox(0,0){\strut{}1.5}}%
    }%
    \gplgaddtomacro\gplfronttext{%
    }%
    \gplbacktext
    \put(0,0){\includegraphics{MCIntro}}%
    \gplfronttext
  \end{picture}%
\endgroup

        		\end{center}
     \end{column}
 \end{columns}
}

\frame{\frametitle{Vergleich der Verfahren des GUM und MC-Simulationen}
\framesubtitle{}
\begin{columns}[t] 
     \begin{column}[T]{6cm} 
     \textbf{MC-Simulation}
     	\begin{itemize}
     		\item Eingangsgr\"o{\ss}en $x_{i}$ werden explizit W-Dichtefunktionen zugeordnet
		\item Keine partiellen Ableitungen ben\"otigt, k\"onnen bereitgestellt werden
		\item Keine Einschr\"ankung in Form der Verteilung von $y$
		\item Allgemeine, k\"urzest m\"ogliche, Konfidenzintervalle
     	\end{itemize}
     \end{column}
     	\begin{column}[T]{6cm} 
	\textbf{GUM Verfahren A/B}
         	\begin{itemize}
     		\item Eingangsgr\"o{\ss}en $x_{i}$ werden Sch\"atzwert und Standardunsicherheit zugeordnet
		\item Empfindlichkeitskoeffizienten und Taylor-Approximation ben\"otigt
		\item Beschr\"ankung auf Gau{\ss}- oder t-Verteilung
		\item Um den Sch\"atzwert symmetrische Konfidenzintervalle
     	\end{itemize}
     \end{column}
 \end{columns}
}


\frame{\frametitle{Monte-Carlo-Simulation: Beispiel}
\framesubtitle{Parameter eines Produktes variieren h\"aufig in Form von Gleich- oder Dreiecksverteilungen.}
\begin{columns}[t] 
     \begin{column}[T]{5cm} 
     \begin{center}
     \begin{picture}(120,30)(0,-40)
\thicklines
\scriptsize
{\color{gray!70!black}
\put(15,0){\oval[0](10,30)}
\put(12,18){$h_{1}$}
\put(45,0){\oval[0](50,6)}
\put(42,6){$l$}
\put(71.5,0){\oval[0](3,42)}
\put(68.5,24){$d$}
\put(80,0){\oval[0](64,42)}
\put(125,0){\oval[0](10,30)}
\put(122,18){$h_{2}$}
\put(116,0){\oval[0](8,6)}}


\put(75,20){\line(0,-1){40}}
\put(75,20){\line(1,-8){5}}
\put(80,-20){\line(1,8){5}}
\put(85,20){\line(1,-8){5}}
\put(90,-20){\line(1,8){5}}
\put(95,20){\line(1,-8){5}}
\put(100,-20){\line(1,8){5}}
\put(105,20){\line(1,-8){5}}
\put(110,-20){\line(0,1){40}}
\end{picture}
     \end{center}
     	\begin{itemize}
     		\item Kraft eines Federspeicherzylinders 
		\begin{equation*}
		\begin{split}
			\triangle F &= \\ &c\left(\triangle h_{1} + \triangle l + \triangle d + \triangle h_{2}  \right)
		\end{split}
		\end{equation*}
		\item MC-Simulation mit N = 10000
    	\end{itemize}
     \end{column}
     	\begin{column}{6cm} 
         	\begin{center}
         		\vspace{1cm}
            		% GNUPLOT: LaTeX picture with Postscript
\begingroup
  \makeatletter
  \providecommand\color[2][]{%
    \GenericError{(gnuplot) \space\space\space\@spaces}{%
      Package color not loaded in conjunction with
      terminal option `colourtext'%
    }{See the gnuplot documentation for explanation.%
    }{Either use 'blacktext' in gnuplot or load the package
      color.sty in LaTeX.}%
    \renewcommand\color[2][]{}%
  }%
  \providecommand\includegraphics[2][]{%
    \GenericError{(gnuplot) \space\space\space\@spaces}{%
      Package graphicx or graphics not loaded%
    }{See the gnuplot documentation for explanation.%
    }{The gnuplot epslatex terminal needs graphicx.sty or graphics.sty.}%
    \renewcommand\includegraphics[2][]{}%
  }%
  \providecommand\rotatebox[2]{#2}%
  \@ifundefined{ifGPcolor}{%
    \newif\ifGPcolor
    \GPcolorfalse
  }{}%
  \@ifundefined{ifGPblacktext}{%
    \newif\ifGPblacktext
    \GPblacktexttrue
  }{}%
  % define a \g@addto@macro without @ in the name:
  \let\gplgaddtomacro\g@addto@macro
  % define empty templates for all commands taking text:
  \gdef\gplbacktext{}%
  \gdef\gplfronttext{}%
  \makeatother
  \ifGPblacktext
    % no textcolor at all
    \def\colorrgb#1{}%
    \def\colorgray#1{}%
  \else
    % gray or color?
    \ifGPcolor
      \def\colorrgb#1{\color[rgb]{#1}}%
      \def\colorgray#1{\color[gray]{#1}}%
      \expandafter\def\csname LTw\endcsname{\color{white}}%
      \expandafter\def\csname LTb\endcsname{\color{black}}%
      \expandafter\def\csname LTa\endcsname{\color{black}}%
      \expandafter\def\csname LT0\endcsname{\color[rgb]{1,0,0}}%
      \expandafter\def\csname LT1\endcsname{\color[rgb]{0,1,0}}%
      \expandafter\def\csname LT2\endcsname{\color[rgb]{0,0,1}}%
      \expandafter\def\csname LT3\endcsname{\color[rgb]{1,0,1}}%
      \expandafter\def\csname LT4\endcsname{\color[rgb]{0,1,1}}%
      \expandafter\def\csname LT5\endcsname{\color[rgb]{1,1,0}}%
      \expandafter\def\csname LT6\endcsname{\color[rgb]{0,0,0}}%
      \expandafter\def\csname LT7\endcsname{\color[rgb]{1,0.3,0}}%
      \expandafter\def\csname LT8\endcsname{\color[rgb]{0.5,0.5,0.5}}%
    \else
      % gray
      \def\colorrgb#1{\color{black}}%
      \def\colorgray#1{\color[gray]{#1}}%
      \expandafter\def\csname LTw\endcsname{\color{white}}%
      \expandafter\def\csname LTb\endcsname{\color{black}}%
      \expandafter\def\csname LTa\endcsname{\color{black}}%
      \expandafter\def\csname LT0\endcsname{\color{black}}%
      \expandafter\def\csname LT1\endcsname{\color{black}}%
      \expandafter\def\csname LT2\endcsname{\color{black}}%
      \expandafter\def\csname LT3\endcsname{\color{black}}%
      \expandafter\def\csname LT4\endcsname{\color{black}}%
      \expandafter\def\csname LT5\endcsname{\color{black}}%
      \expandafter\def\csname LT6\endcsname{\color{black}}%
      \expandafter\def\csname LT7\endcsname{\color{black}}%
      \expandafter\def\csname LT8\endcsname{\color{black}}%
    \fi
  \fi
  \setlength{\unitlength}{0.0500bp}%
  \begin{picture}(3400.00,3000.00)%
  \tiny
    \gplgaddtomacro\gplbacktext{%
      \colorrgb{0.00,0.00,0.00}%
%      \put(322,3069){\makebox(0,0)[r]{\strut{}0}}%
%      \colorrgb{0.00,0.00,0.00}%
%      \put(322,3174){\makebox(0,0)[r]{\strut{}200}}%
%      \colorrgb{0.00,0.00,0.00}%
%      \put(322,3279){\makebox(0,0)[r]{\strut{}400}}%
%      \colorrgb{0.00,0.00,0.00}%
%      \put(322,3384){\makebox(0,0)[r]{\strut{}600}}%
%      \colorrgb{0.00,0.00,0.00}%
%      \put(322,3489){\makebox(0,0)[r]{\strut{}800}}%
%      \colorrgb{0.00,0.00,0.00}%
%      \put(322,3594){\makebox(0,0)[r]{\strut{}1000}}%
%      \colorrgb{0.00,0.00,0.00}%
%      \put(322,3699){\makebox(0,0)[r]{\strut{}1200}}%
      \colorrgb{0.00,0.00,0.00}%
      \put(442,2969){\makebox(0,0){\strut{}-0.2}}%
      \colorrgb{0.00,0.00,0.00}%
%      \put(584,2869){\makebox(0,0){\strut{}-0.15}}%
      \colorrgb{0.00,0.00,0.00}%
%      \put(726,2869){\makebox(0,0){\strut{}-0.1}}%
      \colorrgb{0.00,0.00,0.00}%
%      \put(868,2869){\makebox(0,0){\strut{}-0.05}}%
      \colorrgb{0.00,0.00,0.00}%
      \put(1011,2969){\makebox(0,0){\strut{}0}}%
%      \colorrgb{0.00,0.00,0.00}%
%      \put(1153,2869){\makebox(0,0){\strut{}0.05}}%
      \colorrgb{0.00,0.00,0.00}%
%      \put(1295,2869){\makebox(0,0){\strut{}0.1}}%
      \colorrgb{0.00,0.00,0.00}%
%      \put(1437,2869){\makebox(0,0){\strut{}0.15}}%
      \colorrgb{0.00,0.00,0.00}%
      \put(1579,2969){\makebox(0,0){\strut{}0.2}}%
      \csname LTb\endcsname%
      \put(1010,3800){\makebox(0,0){\strut{}Kolbenstange $l$}}%
    }%
    \gplgaddtomacro\gplfronttext{%
    }%
    \gplgaddtomacro\gplbacktext{%
%      \colorrgb{0.00,0.00,0.00}%
%      \put(1819,3069){\makebox(0,0)[r]{\strut{}0}}%
%      \colorrgb{0.00,0.00,0.00}%
%      \put(1819,3174){\makebox(0,0)[r]{\strut{}200}}%
%      \colorrgb{0.00,0.00,0.00}%
%      \put(1819,3279){\makebox(0,0)[r]{\strut{}400}}%
%      \colorrgb{0.00,0.00,0.00}%
%      \put(1819,3384){\makebox(0,0)[r]{\strut{}600}}%
%      \colorrgb{0.00,0.00,0.00}%
%      \put(1819,3489){\makebox(0,0)[r]{\strut{}800}}%
%      \colorrgb{0.00,0.00,0.00}%
%      \put(1819,3594){\makebox(0,0)[r]{\strut{}1000}}%
%      \colorrgb{0.00,0.00,0.00}%
%      \put(1819,3699){\makebox(0,0)[r]{\strut{}1200}}%
%      \colorrgb{0.00,0.00,0.00}%
      \put(1939,2969){\makebox(0,0){\strut{}0}}%
%      \colorrgb{0.00,0.00,0.00}%
%      \put(2166,2869){\makebox(0,0){\strut{}0.02}}%
      \colorrgb{0.00,0.00,0.00}%
%      \put(2394,2969){\makebox(0,0){\strut{}0.04}}%
      \colorrgb{0.00,0.00,0.00}%
%      \put(2621,2869){\makebox(0,0){\strut{}0.06}}%
      \colorrgb{0.00,0.00,0.00}%
%      \put(2849,2869){\makebox(0,0){\strut{}0.08}}%
%      \colorrgb{0.00,0.00,0.00}%
      \put(3076,2969){\makebox(0,0){\strut{}0.1}}%
      \csname LTb\endcsname%
      \put(2507,3800){\makebox(0,0){\strut{}Kolben $d$}}%
    }%
    \gplgaddtomacro\gplfronttext{%
    }%
    \gplgaddtomacro\gplbacktext{%
      \colorrgb{0.00,0.00,0.00}%
%      \put(322,2192){\makebox(0,0)[r]{\strut{}0}}%
%      \colorrgb{0.00,0.00,0.00}%
%      \put(322,2297){\makebox(0,0)[r]{\strut{}500}}%
%      \colorrgb{0.00,0.00,0.00}%
%      \put(322,2402){\makebox(0,0)[r]{\strut{}1000}}%
%      \colorrgb{0.00,0.00,0.00}%
%      \put(322,2507){\makebox(0,0)[r]{\strut{}1500}}%
%      \colorrgb{0.00,0.00,0.00}%
%      \put(322,2612){\makebox(0,0)[r]{\strut{}2000}}%
%      \colorrgb{0.00,0.00,0.00}%
%      \put(322,2717){\makebox(0,0)[r]{\strut{}2500}}%
%      \colorrgb{0.00,0.00,0.00}%
%      \put(322,2822){\makebox(0,0)[r]{\strut{}3000}}%
      \colorrgb{0.00,0.00,0.00}%
      \put(442,2092){\makebox(0,0){\strut{}-0.4}}%
%      \colorrgb{0.00,0.00,0.00}%
%      \put(584,1992){\makebox(0,0){\strut{}-0.3}}%
%      \colorrgb{0.00,0.00,0.00}%
%      \put(726,1992){\makebox(0,0){\strut{}-0.2}}%
%      \colorrgb{0.00,0.00,0.00}%
%      \put(868,1992){\makebox(0,0){\strut{}-0.1}}%
%      \colorrgb{0.00,0.00,0.00}%
      \put(1011,2092){\makebox(0,0){\strut{}0}}%
      \colorrgb{0.00,0.00,0.00}%
%      \put(1153,1992){\makebox(0,0){\strut{}0.1}}%
%      \colorrgb{0.00,0.00,0.00}%
%      \put(1295,1992){\makebox(0,0){\strut{}0.2}}%
%      \colorrgb{0.00,0.00,0.00}%
%      \put(1437,1992){\makebox(0,0){\strut{}0.3}}%
      \colorrgb{0.00,0.00,0.00}%
      \put(1579,2092){\makebox(0,0){\strut{}0.4}}%
      \csname LTb\endcsname%
      \put(1010,2880){\makebox(0,0){\strut{}Federrate $c$}}%
    }%
    \gplgaddtomacro\gplfronttext{%
    }%
    \gplgaddtomacro\gplbacktext{%
      \colorrgb{0.00,0.00,0.00}%
%      \put(1819,2192){\makebox(0,0)[r]{\strut{}0}}%
%      \colorrgb{0.00,0.00,0.00}%
%      \put(1819,2297){\makebox(0,0)[r]{\strut{}200}}%
%      \colorrgb{0.00,0.00,0.00}%
%      \put(1819,2402){\makebox(0,0)[r]{\strut{}400}}%
%      \colorrgb{0.00,0.00,0.00}%
%      \put(1819,2507){\makebox(0,0)[r]{\strut{}600}}%
%      \colorrgb{0.00,0.00,0.00}%
%      \put(1819,2612){\makebox(0,0)[r]{\strut{}800}}%
%      \colorrgb{0.00,0.00,0.00}%
%      \put(1819,2717){\makebox(0,0)[r]{\strut{}1000}}%
%      \colorrgb{0.00,0.00,0.00}%
%      \put(1819,2822){\makebox(0,0)[r]{\strut{}1200}}%
      \colorrgb{0.00,0.00,0.00}%
      \put(1939,2092){\makebox(0,0){\strut{}-0.1}}%
      \colorrgb{0.00,0.00,0.00}%
%      \put(2223,1992){\makebox(0,0){\strut{}-0.05}}%
      \colorrgb{0.00,0.00,0.00}%
      \put(2508,2092){\makebox(0,0){\strut{}0}}%
      \colorrgb{0.00,0.00,0.00}%
%      \put(2792,1992){\makebox(0,0){\strut{}0.05}}%
      \colorrgb{0.00,0.00,0.00}%
      \put(3076,2092){\makebox(0,0){\strut{}0.1}}%
      \csname LTb\endcsname%
      \put(2507,2880){\makebox(0,0){\strut{}Hammerkopf $h_{1,2}$}}%
    }%
    \gplgaddtomacro\gplfronttext{%
    }%
    \gplgaddtomacro\gplbacktext{%
      \colorrgb{0.00,0.00,0.00}%
      \put(422,440){\makebox(0,0)[r]{\strut{}0}}%
      \colorrgb{0.00,0.00,0.00}%
      \put(422,817){\makebox(0,0)[r]{\strut{}500}}%
      \colorrgb{0.00,0.00,0.00}%
      \put(422,1193){\makebox(0,0)[r]{\strut{}1000}}%
      \colorrgb{0.00,0.00,0.00}%
      \put(422,1570){\makebox(0,0)[r]{\strut{}1500}}%
      \colorrgb{0.00,0.00,0.00}%
      \put(422,1946){\makebox(0,0)[r]{\strut{}2000}}%
      \colorrgb{0.00,0.00,0.00}%
      \put(442,340){\makebox(0,0){\strut{}-4}}%
      \colorrgb{0.00,0.00,0.00}%
      \put(818,340){\makebox(0,0){\strut{}-3}}%
      \colorrgb{0.00,0.00,0.00}%
      \put(1195,340){\makebox(0,0){\strut{}-2}}%
      \colorrgb{0.00,0.00,0.00}%
      \put(1571,340){\makebox(0,0){\strut{}-1}}%
      \colorrgb{0.00,0.00,0.00}%
      \put(1947,340){\makebox(0,0){\strut{}0}}%
      \colorrgb{0.00,0.00,0.00}%
      \put(2323,340){\makebox(0,0){\strut{}1}}%
      \colorrgb{0.00,0.00,0.00}%
      \put(2700,340){\makebox(0,0){\strut{}2}}%
      \colorrgb{0.00,0.00,0.00}%
      \put(3076,340){\makebox(0,0){\strut{}3}}%
      \csname LTb\endcsname%
      \put(1759,2000){\makebox(0,0){\strut{}Kraftabweichung $\triangle F$/kN}}%
    }%
    \gplgaddtomacro\gplfronttext{%
    }%
    \gplbacktext
    \put(0,0){\includegraphics{MCExample}}%
    \gplfronttext
  \end{picture}%
\endgroup

        		\end{center}
		
     \end{column}
 \end{columns}
}




\frame{\frametitle{MC-Simulation zur Bestimmung der Messunsicherheit}
\framesubtitle{Nicht normatives Beiblatt 1 zu DIN V ENV 13005}
\begin{itemize}
\item MC-Simulation als Erg\"anzung zu Verfahren A und B des ``GUM'', z.B. falls
\begin{itemize}
		\item Linearisierung des Modells zu unangemessener Darstellung f\"uhrt, z.B. durch Sensibilit\"at
		\item Partielle Ableitungen schwierig oder unm\"oglich zu finden sind
		\item die Wahrscheinlichkeitsdichte merklich von Gau{\ss}- oder $t$-Verteilung abweicht , z.B. durch Asymmetrie
		\item Wiederholte Versuche zur Bestimmung der Messunsicherheit nicht m\"oglich sind
		\item Modell des Messprozesses nicht in explizite Form gebracht werden kann
		\item Unsicherheitsbeitr\"age nicht n\"aherungsweise von der gleichen Gr\"o{\ss}enordnung sind 
		\end{itemize}
	\item MC-Simulation ist im Einklang mit GUM
	\begin{itemize}
		\item Ermittlung von Erwartungswert, Standardunsicherheit und \"Uberdeckungsintervall der Messgr\"o{\ss}e $y$
		\end{itemize}
\end{itemize}
}


\frame{\frametitle{Ablauf Erstellung einer MC-Simulation}
\framesubtitle{}
     	\begin{enumerate}[1)]
     		\item Modellierung des Systems:
		\begin{enumerate}[a)]
		\item Ausgangsgr\"o{\ss}e $y$
		\item Eingangsgr\"o{\ss}en $\mathbf{x} = \left(x_{1}, \ldots, x_{i}\right)^{T}$
		\item Modell als Beziehung zwischen $y$ und $\mathbf{x}$, nicht zwingend explizit
		\item Zuordnung von Wahrscheinlichkeitsdichtefunktionen zu den $x_{i}$
		\begin{itemize}
		\item $x_{i}$ unabh\"angig: individuelle W-Verteilungen, z.B. Gau{\ss}-, Gleichverteilung etc.
		\item $x_{i}, x_{j}$ abh\"angig: Gemeinsame W-Verteilung
		\end{itemize}
		\end{enumerate}
		\item Fortpflanzung: Simulation des Systems
		\begin{enumerate}[a)]
		\item Bestimmung des Wertes $y$ aus den $x_{i}$
		\item Werte der $x_{i}$ als Pseudozufallszahlen
		\end{enumerate}
		\item Zusammenfassung:
		\begin{enumerate}[a)]
		\item Bestimmung des Erwartungswerts $\E(y)$\footnote{Nicht alle W-Verteilungen weisen einen Erwartungswert auf.}
		\item Bestimmung der Standardabweichung $\sigma_{y}$\footnote{Nicht alle W-Verteilungen weisen eine Standardabweichung auf.}
		\item Bestimmung des Konfidenzintervalls (\"Uberdeckungsintervalls), das $y$ mit einer festgelegten Wahrscheinlichlichkeit enth\"alt
		\end{enumerate}
     	\end{enumerate}
}

\frame{\frametitle{Bestimmung der Fortpflanzung}
\framesubtitle{Neben analytischen und statistischen Methoden sind MC-Simulationen zul\"assig und effizient. Verschiedene W-Verteilungen und nichtlineare Systemfunktionen machen sie notwendig.}
\begin{picture}(340,170)(-130,-85)
\thicklines
\put(-70,-15){
\put(-25,0){\vector(1,0){50}}
\put(0,0){\vector(0,1){30}}
\put(-15,0){\line(0,1){15}}
\put(15,0){\line(0,1){15}}
\put(-15,15){\line(1,0){30}}
\put(40,10){\vector(1,0){30}}
\put(-5, -8){$x_{2}$}
}
\put(-70,30){
\put(-25,0){\vector(1,0){50}}
\put(0,0){\vector(0,1){30}}
\put(-15,0){\line(1,1){15}}
\put(15,0){\line(-1,1){15}}
\put(40,10){\vector(1,0){30}}
\put(-5, -8){$x_{1}$}
}

\put(-70,-60){
\put(-25,0){\vector(1,0){50}}
\put(0,0){\vector(0,1){30}}
\qbezier(-20,0)(-12,2)(-8,10)
\qbezier(-8,10)(-4,21)(0,23)
\qbezier(20,0)(12,2)(8,10)
\qbezier(8,10)(4,21)(0,23)
\put(40,10){\vector(1,0){30}}
\put(-5, -8){$x_{3}$}
}

\put(150,-15){
\put(-40,0){\vector(1,0){80}}
\put(0,0){\vector(0,1){50}}
\qbezier(-20,0)(-10,0)(-8,10)
\qbezier(-8,10)(-3,40)(0,40)
\qbezier(40,0)(23,0)(20,10)
\qbezier(20,10)(10,40)(0,40)
\put(-70,10){\vector(1,0){30}}
\put(-3, -8){$y$}
}


\put(40,-5){\oval[3](60,120)}
\put(20,-8){$y = f\left(\mathbf{x}\right)$}
\end{picture}
}


\frame{\frametitle{W-Dichtefunktionen f\"ur die Eingangsgr\"o{\ss}en}
\framesubtitle{}
\begin{itemize}
\item Allgemein: $\mathbf{x}$ wird gemeinsame PDF zugeordnet
\item Unabh\"angigen $x_{i}$ werden einzelne PDF zugewiesen
\item Allgemein: Bestimmung PDF gem\"a{\ss} Bayes-Theorem oder Prinzip der maximalen Entropie
\item Praktisch h\"aufig auftretende F\"alle: \cite[S. 34f.]{en13005}
	\begin{itemize}
		\item Rechteckverteilung: Toleranzb\"ander
		\item $\arcsin$-Verteilung (U-f\"ormig): Sinusf\"ormige periodische Schwingungen
		\item Gau{\ss}-Verteilung: Schwankung durch viele Einflussfaktoren (Zentraler GWS)
		\item $t$-Verteilung: Modellierung aus endlich vielen Anzeigewerten
		\item Exponential-Verteilung: Linkssteile Verteilung f\"ur positive Gr\"o{\ss}en
	\end{itemize}
\end{itemize}
}


\frame{\frametitle{Voraussetzungen f\"ur Anwendbarkeit und G\"ultigkeit der MC-Simulation}
\framesubtitle{}
\begin{itemize}
\item Es ist $f$ stetig bez\"uglich der $x_{i}$ in der Nachbarschaft der besten Sch\"atzwerte $\hat{x}_{i}$
\item Die Verteilungsfunktion f\"ur $y$ ist streng wachsend
\item Die Wahrscheinlichkeitsdichtefunktion (PDF) f\"ur $y$ ist
	\begin{itemize}
		\item stetig \"uber dem Intervall, f\"ur das die PDF streng positiv ist
		\item unimodal
		\item streng wachsend (oder 0) links vom Modalwert und streng fallend (oder 0) rechts vom Modalwert
		\end{itemize}
	\item Es existieren $\E(y)$ und $\Var(y)$
	\begin{itemize}
		\item[$\rightarrow$] Gew\"ahrleistet stochastische Konvergenz mit steigendem $M$ 
	\end{itemize}
	\item Es ist die Zahl der Monte-Carlo-Versuche $M$ ausreichend gro{\ss} gew\"ahlt
	\begin{itemize}
		\item[$\rightarrow$] Gew\"ahrleistet Zuverl\"assigkeit der stochastischen Information
		\end{itemize}
\end{itemize}
}

%\offslide{Weitere Anwendungen der MC-Simulation}

\frame{\frametitle{Praktische Umsetzung einer MC-Simulation}
\framesubtitle{Die hohe Anzahl Versuche $M$ macht effiziente Software n\"otig.}
\begin{enumerate}
%\item Software: Hohe Anzahl Versuche $M$ n\"otig, daher Effizienz wichtig
\item $M$ w\"ahlen:
	\begin{itemize}
		\item Da MC-Simulationen Zufallsexperimente sind, kann kein festes $M$ Korrektheit des Algorithmus garantieren
		\item F\"ur Signifikanzniveau $\alpha$: $M \gg \frac{1}{1-\alpha}$, z.B. $M \geq 10^{4} \frac{1}{1-\alpha}$
		\item Alternativ: Adaptive Verfahren
	\end{itemize}
	\item Erzeugung von Zufallszahlen f\"ur $\mathbf{x}_{r}$ aus den gew\"ahlten PDF
	\begin{itemize}
		\item Pseudozufallszahlengenerator muss \"uber gewisse Eigenschaften verf\"ugen, z.B. Sequenzl\"ange
		\end{itemize}
	\item Bestimmung der Werte $y_{r} = f\left(\mathbf{x}_{r}\right)$ f\"ur $r = 1, \ldots, M$
	\item Diskrete Darstellung der Verteilungsfunktion f\"ur die Ausgangsgr\"o{\ss}e
	\item Sch\"atzung der Ausgangsr\"o{\ss}e und der beigeordneten Standardunsicherheit
	\begin{itemize}
		\item $\bar{y} = \frac{1}{M} \sum_{i = 1}^{M} y_{i}$
		\item $s^{2}\left(\bar{y}\right) = \frac{1}{M-1} \sum_{i = 1}^{M} \left(y_{i} - \bar{y}\right)^{2} $
	\end{itemize}
\end{enumerate}
}

