% !TEX root = ../17-MdQM-Vorlesung.tex
%\section*{Einführung}
\label{Sec:Einfuehrung}

\frame{\frametitle{Prof. Dr. Raphael Pfaff}
\framesubtitle{Lehr- und Forschungsgebiet Schienenfahrzeugtechnik}
\begin{columns}[t] 
     \begin{column}[T]{7cm} 
     	\begin{itemize}
		\item[] \includegraphics[width=0.4cm]{Email} \hspace{.1cm} pfaff@fh-aachen.de
		\item[] \includegraphics[width=0.4cm]{Twitter} \hspace{.1cm} @RailProfAC
		\item[] \includegraphics[width=0.4cm]{Wordpress} \hspace{.1cm} www.raphaelpfaff.net
		\item[] Prezume: \texttt{\url{http://goo.gl/iq6lhh}}
		\vspace{1cm}
		\item Raum 02305
		\item Sprechstunde nach Vereinbarung
     	\end{itemize}
	
     \end{column}
     	\begin{column}[T]{5cm} 
         	\begin{center}
            		\includegraphics[width=0.8\textwidth]{Profilklein}
        		\end{center}
     \end{column}
 \end{columns}
}

\frame{\frametitle{Curriculum Vitae}
\framesubtitle{}
\begin{center}
\includegraphics[width = 12 cm]{CVGraphic}
\end{center}
}

\frame{\frametitle{Selected publications}
\framesubtitle{}
\scriptsize
\begin{itemize}
%\item Enning, M.; Pfaff, R.: Güterwagen 4.0 -- Ausgewählte technologische Ansätze für einen wettbewerbsfähigen Schienengüterverkehr. Mai 2016: Bahntechnik-Symposium 2016 der IFV-Bahntechnik. 
\item Pfaff, R.; Schmidt, B.D.: \textbf{Daten in der Cloud und dann?} - Vorhersage der Subsystemverfügbarkeit aus beobachteten Fehlerraten der Teilfunktionen. Deine Bahn 5/2016, Bahnfachverlag, Berlin.
\item Pfaff, R.: \textbf{Aktive, mitdenkende Güterwagenbremse.} VDI-Expertenforum Automatisierung für Schienenverkehrssysteme – Der Weg zum Güterwagen 4.0. Aachen, 01./02.09.2016
%\item Enning, M.; Pfaff, R.: Titel noch offen. Combinet-Konferenz. Wien, 10.11.2016
%\item Pfaff, R.; Enning, M.: Zeitgemäße Automatisierung am Güterwagen: das Potential des Güterwagen 4.0 in der Logistik der Industrie 4.0. November 2016: Keynote zum Fachsymposium Rail-IT des IFV-Bahntechnik.
\item Enning, M.; Pfaff, R.: \textbf{Digitalisierung bringt mehr Güter auf die Schiene}. ATZ Automobiltechnische Zeitschrift, Dezember 2016, Springer, Wiesbaden.
\item Moshiri, A.; Pfaff, R.; Reich, A.; Gäbel, M.: \textbf{Modellierung der Adhäsionsfläche im Rad-Schiene-Kontakt unter Einsatz von reibwertverbessernden Mitteln.} 15. International Schienenfahrzeugtagung, März 2017, Dresden. 
\item Pfaff, R.; Enning, M.: \textbf{Towards Inclusion of the Freight Rail System in the Industrial Internet of Things - Wagon 4.0.} The Stephenson Conference 2017: Research for Railways, April 2017, London. 
\item Shahidi, P; Pfaff, R.; Enning, M.: \textbf{The connected wagon - a concept for the integration of vehicle side sensors and actors with cyber physical representation for condition based maintenance.} First International Conference on Rail Transportation, July 2017, Chengdu.
\item Pfaff, R.; Moshiri, A.; Reich, A.; Gäbel, M.: \textbf{Modelling of the effect of sanding on the wheel rail contact area.} First International Conference on Rail Transportation, July 2017, Chengdu. 
\item Pfaff, R.: \textbf{Analysis of Big Data Streams to obtain Braking Reliability Information for Train Protection systems.} Asia-Pacific Conference of the Prognostics and Health Management Society, July 2017, Jeju, Korea
\end{itemize}
}

\frame{\frametitle{Vorstellungsrunde und Erwartungen}
\framesubtitle{}
\begin{columns}[t] 
     \begin{column}[T]{6cm} 
     	\begin{itemize}
     		\item Ausbildung?
		\item Berufserfahrung?
		\item Undergraduate?
		\item Warum MPE?
		\item QM-Vorwissen?
		\item Was muss passieren, damit ich MdQM hassen werde?
		\item Was muss passieren, damit ich MdQM lieben werde?
     	\end{itemize}
     \end{column}
     	\begin{column}[T]{6cm} 
         	\begin{center}
            		\includegraphics[width=0.95\textwidth]{HelloSticker}\source{}
        		\end{center}
     \end{column}
 \end{columns}
}

\frame{\frametitle{Anforderungen ``Second Cycle'' - Master}
\framesubtitle{Anforderungen gem\"a{\ss} Dublin Descriptors}
\begin{columns}[t] 
     \begin{column}[T]{8cm} 
     	\begin{itemize}
     		\item Knowledge and understanding founded upon and extends or enhances that typically associated with Bachelor's level:
		\begin{itemize}
		\item Provides basis or opportunity for originality in developing and applying ideas
		\item Often within a research context
		\end{itemize}
		\item Apply their knowledge and understanding and problem solving abilities in new or unfamiliar environments 
		\item Have the ability to integrate knowledge and handle complexity and formulate judgements with incomplete or limited information
		\item Have the learning skills to allow them to continue to study in a manner that may be largely self-directed or autonomous
     	\end{itemize}
     \end{column}
     	\begin{column}[T]{4cm} 
         	\begin{center}
	\vspace{1.5cm}
            		\includegraphics[width=0.8\textwidth]{GraduationHat}
        		\end{center}
     \end{column}
 \end{columns}
}

\frame{\frametitle{Anforderungen ``Niveau 7'' - Master}
\framesubtitle{Anforderungen gem\"a{\ss} Deutschem Qualifizierungsrahmen}
\begin{columns}[t] 
     \begin{column}[T]{7cm} 
     	\begin{itemize}
     		\item Umfassendes, detailliertes und spezialisiertes Wissen
		\begin{itemize}
		\item Auf dem neuesten Erkenntnisstand
		\item Erweitertes Wissen in angrenzenden Bereichen 
		\end{itemize}
		\item Spezialisierte fachliche oder konzeptionelle Fertigkeiten zur Lösung auch strategischer Probleme
		\begin{itemize}
		\item Neue L\"osungen erarbeiten und bewerten
		\end{itemize}
		\item Gruppen oder Organisationen im Rahmen komplexer Aufgabenstellungen verantwortlich leiten und ihre Arbeitsergebnisse vertreten.
		\item Für neue anwendungs- oder forschungsorientierte Aufgaben Ziele definieren
		\item Wissen eigenst\"andig erschlie{\ss}en
     	\end{itemize}
     \end{column}
     	\begin{column}[T]{5cm} 
         	\begin{center}
	\vspace{1cm}
            		\includegraphics[width=0.8\textwidth]{GraduationHat}
        		\end{center}
     \end{column}
 \end{columns}}
\frame{\frametitle{Rolle des Lehrenden}
\framesubtitle{}
\begin{columns}[t] 
     \begin{column}[T]{6cm} 
     \vspace{1cm}
     	\begin{quote}
     		A teacher is never a giver of truth; he is a guide, a pointer to the truth that each student must find for himself.
     	\end{quote}
	\flushright Bruce Lee
	
     \end{column}
     	\begin{column}[T]{6cm} 
         	\begin{center}
            		\includegraphics[width=0.8\textwidth]{Bruce}
        		\end{center}
     \end{column}
 \end{columns}
}

\frame{\frametitle{Was bedeutet Methoden des QM?}
\framesubtitle{}
\centering
\tikz [small mindmap, every node/.style=concept, concept color=black!20, grow cyclic,
	level 1/.append style={level distance=3cm,sibling angle=72},
	level 2/.append style={level distance=3cm,sibling angle=45}] 
	\node [root concept] {Methoden des Qualit\"ats-managements} % root
	child { node {Lebenszyklus}}
	child { node {QM-Systeme}} 
	child { node {Mechatronik}}
	child { node {Methoden}}
	child { node {Data Science}}
	;
}


\frame{\frametitle{Themenplan}
\framesubtitle{Die angegebenen Kapitel sind dringend empfohlene Begleitlekt\"ure. Alle sind als E-Book verf\"ugbar.}
\begin{center}
\footnotesize
\vspace{-.8cm}
\begin{tabular}{|p{1cm}|p{6cm}|p{3cm}|l|}
\hline
Datum & Thema & Literatur \\ \hline
4. 10 & Einf\"uhrung, Einf\"uhrung Data Science & \cite{tebeka2017python}\\ \hline
5. 11.  & Recap Fertigungsmesstechnik  & \cite[Kap. 1.1, 2.1, 2.2]{keferstein} \\ 
  & Pr\"ufdatenerfassung und -auswertung, Messunsicherheit &  \cite[Kap. 1.1, 2.1, 2.2, 4.1, 4.2, 5.5.1]{keferstein} \\ \hline
 12. 11 & [V+P] Data Science: Tools and Methods &  \cite{tebeka2017python}\\ \hline
19. 11.& [V+P] Data Science: Tools and Methods &  \cite{tebeka2017python}\\ \hline
26. 11. & \hyperref[Sec:Zuverlaessig]{Einführung Zuverlässigkeitstechnik} & \cite[Kap. 2, 3, 10.5]{eberlin} \\ \hline
3. 12. & {Prüfdatenerfassung und -auswertung elektrischer Komponenten} & \cite[Kap. 3, 6.1]{muehl}   \\  \hline
17. 12. & \hyperref[Sec:Software]{Software-QS} & \cite[Kap. 1, 2, 4]{hoffmann} \\ \hline
7. 1. & Requirements Engineering, QM-Systeme & \cite[Kap. 1, 5]{bruggemann2012grundlagen} \\ \hline
 14. 1. & QFD, FTA, FMEA & \cite[Kap. 3.1, 3.2, 3.3]{bruggemann2012grundlagen} \\ \hline
%& Reserve / Q\&A & \\ \hline
21. 1. & Pitch-Event  &  \\ \hline
\end{tabular}
\end{center}
}


\frame[allowframebreaks]{\frametitle{Themen\"ubersicht}
\framesubtitle{}
%\begin{itemize}
%\item Grundbegriffe der Fertigungsmesstechnik 
%\item Grundlagen
%\begin{itemize}
%\item Ma{\ss}verk\"orperungen 
%\item Messunsicherheit und Ma{\ss}abweichung 
%\item Zeichnungseintragungen und Tolerierungen 
%\end{itemize}
%\item Pr\"ufdatenerfassung
%\begin{itemize}
%\item \"Ubersicht der Verfahren 
%\item Werkstattpr\"ufmittel 
%\item Messwertaufnehmer 
%\item Lehrende Pr\"ufung
%\end{itemize}
%\item Pr\"ufdatenauswertung
%\begin{itemize}
%\item Statistische Grundlagen 
%\item Pr\"ufprozesseignung 
%\item Erforderliche Messgenauigkeit 
%\end{itemize}
%\item R\"uckf\"uhrbarkeit 
%\item Monte-Carlo-Simulation und Faltungsintegrale
%\end{itemize}
\tableofcontents
}

\frame{\frametitle{Lernziele}
\framesubtitle{}
Die Studierenden k\"onnen nach erfolgreicher Bearbeitung des Moduls:
\begin{itemize}
\item QM-Methoden im Produktlebenszyklus anwenden und bewerten, um sie im Rahmen der Produktentwicklung einzusetzen
\item Produkte innerhalb eines QM-Systems entwickeln
\item Mess- und Pr\"ufverfahren anwenden und die Ergebnisse bewerten und kommunizieren
\item Angemessene Verfahren der Pr\"ufdatenerfassung ausw\"ahlen und bestehende beurteilen
\item Daten visualisieren und interpretieren, um Entscheidungen zu untermauern
\item Methoden und Werkzeuge der Datenwissenschaft nutzen, um Vorhersagen zu treffen
\item Produktdaten nutzen, um Fehlerbilder zu identifizieren
\item Grundlegende Pr\"ufdaten elektrischer Komponenten bestimmen und auswerten
\item Zuverl\"assigkeit von Systemen und Komponenten bewerten
\item Verfahren zur Qualit\"atssicherung in Software-Komponenten bewerten und anwenden
\end{itemize}
}

%\frame{\frametitle{Praktika und Pr\"ufung}
%\framesubtitle{Zwei verpflichtende Praktikumstermine [V+P], semesterbegleitende Coursework}
%\begin{columns}[t] 
%     \begin{column}[T]{6cm} 
%     	\begin{enumerate}
%     		\item Werkstattpr\"ufmittel (ca. KW 45)
%		\begin{itemize}
%		\item Erstellung eines Berichts (15\% Modulnote)
%		\item Formular zum Download auf meiner Homepage
%		\end{itemize}
%		\item Messen Ohmscher Widerst\"ande (ca. KW 47)
%		\begin{itemize}
%		\item Erstellung eines Berichts (15\% Modulnote)
%		\item Formular zum Download auf meiner Homepage
%		\end{itemize}
%		\item Coursework inkl. Pitch (70\% Modulnote)
%     	\end{enumerate}
%     \end{column}
%     	\begin{column}[T]{6cm} 
%         	\begin{center}
%            		\includegraphics[width=0.8\textwidth]{TechReport}\source{}
%        		\end{center}
%     \end{column}
% \end{columns}
%}



\frame{\frametitle{Praktikum/Coursework}
\framesubtitle{Zwei verpflichtende Praktikumstermine [V+P], semesterbegleitende Coursework}
\begin{columns}[t] 
     \begin{column}[T]{7cm} 
\begin{itemize}
\item Gewichtung: 100\% der Modulnote durch Seminararbeit
\end{itemize}
\begin{itemize}
\item Recommendations:
\begin{itemize}
	\item Start early!
	\item Join the slack group at mdqmws1819.slack.com
\end{itemize}
\item Requirements:
\begin{itemize}
	\item Use Jupyter and hand in a Jupyter Notebook for each question, containing fully readable documents.
	\item Exchange with your fellow students, however hand in an individual solution.
	\item Deliver a pitch
	\begin{itemize}
		\item Not more than five minutes
		\item Why do you provide the best solution?
		\end{itemize} 
\end{itemize}
\end{itemize}
\end{column}
     	\begin{column}[T]{6cm} 
	\begin{center}
            		\includegraphics[width=0.8\textwidth]{MdQMSlack1819} 
        		\end{center}
     \end{column}
 \end{columns}
}

\frame{\frametitle{Keine Angst!}
\framesubtitle{Evaluationsergebnisse WS 17/18}
         	\begin{center}
            		\includegraphics[width=0.95\textwidth]{EvaMdQM17}\source{}
        		\end{center}
}

