% !TEX root = ../SFV15001_MdQM_Fertigungsmesstechnik_Rev02.tex

%\label{Sec:Pruefdaten}
%\subsection{Anzeigende Messger\"ate}
%\frame{\frametitle{Werkstattpr\"ufmittel}
%\framesubtitle{}
%\begin{itemize}
%\item Werkstattpr\"ufmittel dienen der Aufnahme eindimensionaler Merkmale:
%	\begin{itemize}
%		\item Au{\ss}en-, Innen- und Absatzma{\ss}e
%		\item Durchmesser, Breite und Dicke
%		\item H\"ohe und Tiefe
%		\item Winkel
%	\end{itemize}
%\item Auch als \textit{Messzeuge} bezeichnet
%\item Gro{\ss}e Bedeutung f\"ur fertigungsnahes Messen
%\begin{itemize}
%		\item Universell einsetzbar
%		\item Leicht handhabbar
%	\end{itemize}
%	\item Heute h\"aufig elektronische Anzeigen
%	\begin{itemize}
%		\item Vereinfachte Ablesbarkeit
%		\item Erweiterte Funktionalit\"at
%	\end{itemize}
%	\item Mechanische Messzeuge robust und kosteng\"unstig 
%\end{itemize}
%}
%
%\frame{\frametitle{Abbesches Komparatorprinzip}
%\framesubtitle{}
%\begin{columns}[t] 
%     \begin{column}[T]{6cm} 
%     	\begin{itemize}
%     		\item Fluchtende Ausrichtung von Ma{\ss}verk\"orperung und Messgr\"o{\ss}e
%		\item Vermeidung von Fehlern durch Kippen des Messger\"ates zum Messobjekt $\rightarrow$ Fehler 1. Ordnung
%		\item Bei Ber\"ucksichtigung nur Verkippen des Messobjektes zur Ebene m\"oglich $\rightarrow$ Fehler 2. Ordnung
%     	\end{itemize}
%     \end{column}
%     	\begin{column}[T]{6cm} 
%         	\begin{center}
%            		\includegraphics[width=0.5\textwidth]{Abbe}\source{Quelle: wikimedia/ArtMechanic}
%        		\end{center}
%     \end{column}
% \end{columns}
%}
%
%\frame{\frametitle{Messschieber}
%\framesubtitle{}
%\begin{columns}[t] 
%     \begin{column}[T]{6cm} 
%     	\begin{itemize}
%     		\item Gebr\"auchlichstes Handmessmittel
%		\begin{itemize}
%		\item Au{\ss}enmessung (1)
%		\item Innenmessung (2)
%		\item Tiefenmessung (3)
%		\end{itemize}
%		\item Ausf\"uhrung nach DIN 862
%     	\end{itemize}
%     \end{column}
%     	\begin{column}[T]{6cm} 
%         	\begin{itemize}
%		\item Ausf\"uhrungen bis 3000 mm (2000 mm nach Norm)
%		\item Ablesung am Nonius
%		\item Alternativ:
%		\begin{itemize}
%		\item Digital
%		\item Rundskala
%		\end{itemize}
%		\end{itemize}
%     \end{column}
% \end{columns}
% \begin{center}
%            		\includegraphics[width=0.85\textwidth]{Messchieber}\\
%		\begin{itemize}
%		\item {\color{red!80!black}Abbesches Komparatorprinzip nicht eingehalten!}
%		\end{itemize}
%        		\end{center}
%}
%
%\frame{\frametitle{H\"ohenmessger\"ate}
%\framesubtitle{}
%     	\begin{itemize}
%     		\item Verwandt mit Messschieber
%		\item 1-Punkt-Antastung
%		\begin{itemize}
%		\item Verwendung i.d.R. auf Messtisch
%		\end{itemize}
%     	\end{itemize}
%         	\begin{center}
%            		\includegraphics[width=0.8\textwidth]{Hohenmess}
%        		\end{center}
%}
%
%\frame{\frametitle{Messschrauben}
%\framesubtitle{}
%\begin{columns}[t] 
%     \begin{column}[T]{6cm} 
%     	\begin{itemize}
%     		\item Ma{\ss}verk\"orperung \begin{itemize}
%		\item fluchtend mit Messstrecke
%		\item Gewindesteigung
%		\end{itemize}
%		\item B\"ugelmessschraube: DIN 863
%		\item Rutschkupplung zur Begrenzung der Messkraft
%     	\end{itemize}
%     \end{column}
%     	\begin{column}[T]{6cm} 
%        	\begin{itemize}
%		\item G\"angige Ausf\"uhrungen:
%		\begin{itemize}
%		\item Au{\ss}en-,
%		\item Innen-,
%		\item Tiefenmessschraube
%		\end{itemize}
%		\item Angepasste Messfl\"achen, z.B. f\"ur Gewinde
%		\end{itemize}
%     \end{column}
% \end{columns}
% \begin{center}
%            		\includegraphics[width=0.8\textwidth]{Micrometer}
%        		\end{center}
%}
%
%\frame{\frametitle{Anzeigende Aufnehmer mit mechanischer \"Ubersetzung}
%\framesubtitle{}
%\begin{columns}[t] 
%     \begin{column}[T]{6cm} 
%     	\begin{itemize}
%     		\item 1-Punkt-Antastung
%		\begin{itemize}
%		\item Messuhren
%		\item Feinzeiger
%		\item F\"uhlhebel
%		\end{itemize}
%		\item Messuhr: 
%		\begin{itemize}
%		\item Anzeigebereich 3 oder 10 mm
%		\item Skalenteilung $10 \mum$ 
%		\end{itemize} 
%		\item Feinzeiger:
%		\begin{itemize}
%		\item Anzeigendes L\"angenmessger\"at
%		\item Winkelausschlag $< 360^{\circ}$
%		\end{itemize}
%		\item F\"uhlhebelmessger\"at:
%		\begin{itemize}
%		\item Winkelausschlag
%		\item Kleine Messspanne
%		\end{itemize}
%     	\end{itemize}
%     \end{column}
%     	\begin{column}[T]{5cm} 
%         	\begin{center}
%            		\includegraphics[width=0.7\textwidth]{Messuhr}
%        		\end{center}
%     \end{column}
% \end{columns}
%}
%
%\subsection{Messwertaufnehmer}
%\frame{\frametitle{Potenziometer}
%\framesubtitle{}
%\begin{columns}[t] 
%     \begin{column}[T]{6cm} 
%     	\begin{itemize}
%     		\item Einfach
%		\item Kosteng\"unstig
%		\item F\"ur Widerstand $R_{0}$ und Gesamtl\"ange $S_{max}$
%		\end{itemize}
%		\begin{equation*}
%		R = \frac{s}{s_{max}} R_{0}
%		\end{equation*}
%     	%\end{itemize}
%     \end{column}
%     	\begin{column}[T]{5cm} 
%	 \begin{itemize}
%		\item Betrieb als Spannungsteiler
%		\end{itemize}
%		\begin{equation*}
%		\frac{U_{A}}{U_{0}} = \frac{\frac{s}{s_{max}}}{1+\frac{R_{0} s}{R_{B}s_{max}} \left(1 - \frac{s}{s_{max}}\right)}
%		\end{equation*}
%         	     \end{column}
% \end{columns}
% \begin{center}
%            		\includegraphics[width=0.95\textwidth]{Potentiometer}
%        	\end{center}
%
%}
%\frame{\frametitle{Induktive Sensoren}
%\framesubtitle{}
%\begin{columns}[t] 
%     \begin{column}[T]{6cm} 
%     	\begin{itemize}
%     		\item Grundlage:
%		\begin{equation*}
%			U = -L \dot{I}
%		\end{equation*}
%			\item Bauarten:
%			\begin{itemize}
%		\item Tauchankeraufnehmer
%		\item Querankeraufnehmer
%		\item Wirbelstromaufnehmer
%		\end{itemize}
%		\item Kleine bis mittlere Wege
%		\item Dynamisch (Weg\"anderungen)
%		\item Nichtlinearer Zusammenhang:
%		\begin{equation*}
%			\frac{L}{L_{max}} = \frac{1}{1+ \frac{\frac{s_{Luft}}{\mu_{Luft}}}{\frac{s_{Eisen}}{\mu_{Eisen}}}}
%		\end{equation*}
%     	\end{itemize}
%     \end{column}
%     	\begin{column}[T]{6cm} 
%         	\begin{center}
%            		\includegraphics[width=0.9\textwidth]{InduktiverTaster}
%        		\end{center}
%     \end{column}
% \end{columns}
%}
%
%\frame{\frametitle{Lasertriangulation}
%\framesubtitle{}
%\begin{columns}[t] 
%     \begin{column}[T]{6cm} 
%     	\begin{itemize}
%		\item Abstandsmessung
%     		\item Weit verbreitet
%		\item Verschiebung Antastpunkt f\"uhrt zu Verschiebung Bildpunkt auf Sensor
%		\item Messunsicherheit h\"angt ab von Oberfl\"ache des Messobjekts (Streuung) 
%     	\end{itemize}
%	\begin{equation*}
%		\triangle h = b_{0} \frac{\sin\left(\phi\right)}{\sin\left(\delta -\phi\right)}
%	\end{equation*}
%
%     \end{column}
%     	\begin{column}[T]{5cm} 
%         	\begin{center}
%            		\includegraphics[width=0.8\textwidth]{Triangulation}
%        		\end{center}
%     \end{column}
% \end{columns}
%}
%
%\frame{\frametitle{Streifenprojektionsverfahren}
%\framesubtitle{}
%\begin{picture}(500,300)(-250,-150)
%\thicklines
%\put(-230,-80){\includegraphics[width=0.4\textwidth]{Zylinder}}
%\put(-150,10){{\color{red!80!black}\vector(1,1){40}}}
%\put(-150,45){\includegraphics[width=0.5\textwidth]{Projektion}}
%\put(-100,-80){\includegraphics[width=0.5\textwidth]{Streifen}}
%\put(-30,50){{\color{red!80!black}\vector(1,-1){40}}}
%\end{picture}
%}
%
%\frame{\frametitle{Streifenprojektionsverfahren}
%\framesubtitle{}
%\begin{columns}[t] 
%     \begin{column}[T]{6cm} 
%     	\begin{itemize}
%     		\item Fl\"achenerfassungsverfahren
%		\item Statisch: eine Aufnahme
%		\begin{itemize}
%		\item Aufl\"osung ca. 1/20 Streifenbreite
%		\end{itemize}
%		\item Dynamisch: Bilderserie
%		\begin{itemize}
%		\item Aufl\"osung bis zu 1/100 Streifenbreite
%		\end{itemize}
%		\item Streifenverschiebung nicht eindeutig
%		\begin{itemize}
%		\item Kontinuierliche Formen g\"unstiger
%		\end{itemize}
%		\item Phasenlage durch Sch\"atzen der Sinusfunktion
%     	\end{itemize}
%     \end{column}
%     	\begin{column}[T]{6cm} 
%         	\begin{center}
%            		\includegraphics[width=1.0\textwidth]{Streifen}
%        		\end{center}
%     \end{column}
% \end{columns}
%}
%
%
%
%
%\subsection{Ma{\ss}verk\"orperungen}
%
%\frame{\frametitle{Ma{\ss}verk\"orperungen}
%\framesubtitle{}
%\begin{itemize}
%\item Unterscheidung:  
%\begin{itemize}
%\item Materiell:   Stellt Ma{\ss} durch geometrische Gestalt dar, z.B. Endma{\ss}  
%\item Immateriell:   Stellt Ma{\ss} durch eigenes Merkmal dar, z.B. Lichtgeschwindigkeit als L\"angennormal  
%\end{itemize}
%\begin{definition}[Ma{\ss}verk\"orperung, \cite{pfeifer10}]
%Unter einer Ma{\ss}verk\"orperung ist allgemein ein fa{\ss}bares Objekt oder auch ein Naturph\"anomen zu verstehen, das durch ein bestimmtes unver\"anderliches Merkmal das zu verk\"orpernde Ma{\ss} darstellt. Ma{\ss}verk\"orperungen haben keine w\"ahrend der Messung beweglichen Teile.
%\end{definition} 
%\item Alternative Definition: Ger\"at, ``mit dem in gleichbleibender Weise w\"arend seines Gebrauchs ein oder mehrere Werte einer Gr\"o{\ss}e wiedergegeben oder geliefert werden sollen. \cite{Din94}''
%\end{itemize}
%}
%
%\frame{\frametitle{Parallelendma{\ss}e}
%\framesubtitle{Praxisnahe Grundlage f\"ur industrielles Messen und Pr\"ufen}
%\begin{columns}[t] 
%     \begin{column}[T]{6cm} 
%     	\begin{itemize}
%     		\item L\"angenverk\"orperung durch Abstand zwei paralleler Fl\"achen
%		\item Bereich $\left(0{,}5 \ldots 3000\right) \mathsf{mm}$
%		\item Verschlei{\ss}fest, formbest\"andiger Werkstoff: Hartmetall, Keramik
%		\item Oberfl\"ache fehlerfrei
%		\item Verbindung durch Ansprengen m\"oglich: Viele Kombinationen mit wenigen Elementen m\"oglich.
%		\item Zusammenstellung des Endma{\ss}satzes bestimmt Me{\ss}bereich und kleinste Abstufung
%     	\end{itemize}
%     \end{column}
%     	\begin{column}[T]{5cm} 
%         	\begin{center}
%            		\includegraphics[width=0.95\textwidth]{Endmass}\\ \vspace{0.2cm}
%		\scriptsize
%		\begin{tabular}{c|c}
%		Klasse & M\"oglicher Einsatz \\ \hline
%		 K & Kalibriergrad \\ \hline
%		0 & Betriebliche Kontrolle \\
%		& von Endma{\ss}en \\ \hline
%		I & Lehrenkontrolle, Einstellung \\
%		& von Me{\ss}gera\"aten \\ \hline
%		II &  Messen und Pr\"ufen 
%		\end{tabular}
%		\end{center}
%     \end{column}
% \end{columns}
%}
%
%\frame{\frametitle{Besondere Geometrieverk\"orperungen}
%\framesubtitle{Vorwiegend zu Kalibrierzwecken eingesetzte Geometrieverk\"orperungen}
%\begin{columns}[t] 
%     \begin{column}[T]{6cm} 
%     	\begin{itemize}
%     		\item Bidirektionales Stufenendma{\ss}:
%		\begin{itemize}
%		\item Verwirklicht Innen-, Au{\ss}en-, vorderes und hinteres Stufenma{\ss}, Mittelpunktsabst\"ande
%		\item \"Uberwachung von Koordinatenmessger\"aten
%		\item Stufen zur Entdeckung kurz- und langperiodischer Fehler
%		\end{itemize}
%		\item Sinuslineal
%		\begin{itemize}
%		\item Feste L\"ange
%		\item H\"ohenunterschied durch Endma{\ss}e
%		\item Winkel \"uber Sinusbeziehung
%		\end{itemize}
%     	\end{itemize}
%     \end{column}
%     	\begin{column}[T]{5cm} 
%         	\begin{center}
%			%Bidirektionales Stufenendma{\ss}
%            		\includegraphics[width=0.45\textwidth]{Stufenendmass}\\
%			\includegraphics[width=0.6\textwidth]{Sinuslineal}		
%			%Sinuslineal
%        		\end{center}
%     \end{column}
% \end{columns}
%}
%
%\frame{\frametitle{Inkrementale Ma{\ss}verk\"orperungen}
%\framesubtitle{}
%\begin{itemize}
%\item L\"angenerfassung durch Schrittz\"ahlung
%	\begin{itemize}
%		\item Multiplikation mit Schrittweite ergibt Gesamtverschiebung
%		\item Richtungssignal ben\"otigt
%		\item Absolutwert durch Referenzmarke 
%	\end{itemize}
%\item Vorteil: Ben\"otigt nur eine Spur zur Codierung der L\"ange
%\item H\"aufiger Einsatz: NC-Maschinen und Koordinatenmessger\"ate
%\item Beispiele:
%	\begin{itemize}
%		\item Strichma{\ss}st\"abe
%		\item Polygonspiegel
%		\item Interferometer
%	\end{itemize}
%\end{itemize}
%}
%
%\frame{\frametitle{Strichma{\ss}st\"abe}
%\framesubtitle{}
%\begin{columns}[t] 
%     \begin{column}[T]{6cm} 
%     	\begin{itemize}
%     		\item Optisch: Glasma{\ss}stab
%		\begin{itemize}
%		\item Feste und bewegliche Glasplatte
%		\item Verk\"orperung durch Streifen
%		\item Verschiebungsmessung durch Lichtimpulse
%		\item ggf. Moir\'e-Verfahren
%		\end{itemize}
%		\item Elektrisch:
%		\begin{itemize}
%		\item Induktiv:
%		\begin{itemize}
%		\item Inductosyn: Verschiebung Prim\"ar- gegen Sekund\"arspule
%		\item Accupin: Verschiebung ferromagnetischer Zapfen im $b$-Feld
%		\end{itemize}
%		\item Magnetisch
%		\item Kapazitiv
%		\end{itemize}
%     	\end{itemize}
%     \end{column}
%     	\begin{column}[T]{5cm} 
%         	\begin{center}
%            		\includegraphics[width=0.95\textwidth]{Glasmass}\rotatebox{90}{{\tiny \color{gray} Quelle: wikimedia/MatthiasDD}} \vspace{.2cm}
%			%\includegraphics[width=0.9\textwidth]{ElektrInk}\hspace{.1cm}\rotatebox{90}{{\tiny \color{gray} \cite[S. 42]{pfeifer10}}}
%        		\end{center}
%     \end{column}
% \end{columns}
%}
%
%
%
%\frame{\frametitle{Inkrementale Ma{\ss}verk\"orperung, optisch}
%\framesubtitle{}
%\begin{columns}[t] 
%     \begin{column}[T]{6.2cm} 
%		\begin{itemize}
%		\item Polygonspiegel:
%		\begin{itemize}
%		\item Winkelma{\ss}verk\"orperung, $(4 \ldots 72)$ Fl\"achen
%		\item Antastung mittel Autokollimationsfernrohr
%		\end{itemize}
%		\item Interferenz
%		\begin{itemize}
%		\item Immaterielle Ma{\ss}verk\"orperung: Einsatz der Lichtwellenl\"ange als Ma{\ss}verk\"orperung
%		\item Auswertung mittels Michelson-Interferometer
%		\item Messung geringer Unterschiede zwischen $l_{1}$ und $l_{2}$
%		\item Aufl\"osung: $\frac{\lambda}{2}$
%		\end{itemize}
%		\end{itemize}
%     \end{column}
%     	\begin{column}[T]{5cm} 
%         	\begin{center}
%	\setlength{\unitlength}{0.75pt}
%            		\begin{picture}(300, 150)(-100,-60)
%\thicklines
%\scriptsize
%\put(-100,3){\vector(1,0){95}}
%\put(-100,-3){\vector(1,0){95}}
%% Prolong here
%\put(5,3){\vector(1,0){75}}
%\put(80,-3){\vector(-1,0){75}}
%%Til here
%\put(3,5){\vector(0,1){75}}
%\put(-3,80){\vector(0,-1){75}}
%\put(-20,-20){\framebox(40,40){}}
%\put(-20,-20){\line(1,1){40}}
%\put(-3,-5){\vector(0,-1){35}}
%\put(3,-5){\vector(0,-1){35}}
%%Mirrors
%\put(-30,80){\framebox(60,5){}}
%\put(80,-30){\framebox(5,60){}}
%\put(-30,-50){\framebox(60,10){Beobachter}}
%\put(40,80){Spiegel $S_1$}
%\put(60,-40){Spiegel $S_2$}
%\put(-90,10){Strahlteiler}
%\put(0, 30){\vector(1,0){80}}
%\put(10, 30){\vector(-1,0){10}}
%\put(55, 33){$l_2$}
%\put(30, 0){\vector(0,1){80}}
%\put(30, 10){\vector(0,-1){10}}
%\put(33, 55){$l_1$}
%\end{picture}
%\scriptsize Michelson-Interferometer zur L\"angendifferenzmessung
%        		\end{center}
%     \end{column}
% \end{columns}
%}
%
%\frame{\frametitle{Inkrementale Ma{\ss}verk\"orperungen, mechanisch}
%\framesubtitle{}
%\begin{columns}[t] 
%     \begin{column}[T]{5cm} 
%     	\begin{itemize}
%     		\item Teilung von Ritzel oder Zahnstange
%		\item Funktionsintegration: Ma{\ss}verk\"orperung und Kraft\"ubertragung
%		\item z.B. Portalfr\"asmaschine, St\"anderbohrmaschine
%		\item Antastung durch Abrollen des Ritzels bzw. der Zahnstange 
%     	\end{itemize}
%     \end{column}
%     	\begin{column}[T]{7cm} 
%         	\begin{center}
%            		\includegraphics[width=0.9\textwidth]{Mechanisch} \hspace{.1cm}\rotatebox{90}{{\tiny \color{gray} \cite[S. 44]{pfeifer10}}}
%        		\end{center}
%     \end{column}
% \end{columns}
%}
%
%\frame{\frametitle{Absolut codierte Ma{\ss}verk\"orperungen}
%\framesubtitle{}
%\begin{columns}[t] 
%     \begin{column}[T]{5.5cm} 
%     	\begin{itemize}
%     		\item Eindeutige Zuordnung zwischen Position und Anzeige
%		\item Einsetzbar f\"ur L\"angen und Winkel
%		\item H\"ohere Anzahl Spuren und Detektoren n\"otig:
%		\begin{equation*}
%			n = \log_{2}\left(\frac{L}{a}\right)
%		\end{equation*}
%		\begin{itemize}
%		\item[] $L$: L\"ange des Ma{\ss}stabs
%		\item[] $a$: Aufl\"osung der Ma{\ss}verk\"orperung 
%		\end{itemize}
%		\item Keine Interpolation m\"oglich
%     	\end{itemize}
%     \end{column}
%     	\begin{column}[T]{6cm} 
%	\begin{center}
%		\setlength{\unitlength}{2pt}
%         	\begin{picture}(200,200)(0,-200)
%		%Inkremental
%		\multiput(0,0)(10,0){8}{
%			\put(0,0){\line(1,0){10}}
%			\put(0,-5){\rule{10pt}{10pt}}
%			\put(0,-5){\line(1,0){10}}}
%			\put(80,0){\line(0,-1){5}}
%			\put(10,-10){Inkrementaler Ma{\ss}stab}
%            	%Bin\"ar
%		\multiput(0,-20)(10,0){8}{
%			\put(0,0){\line(0,-1){5}}
%			\put(0,0){\line(1,0){10}}
%			\put(0,-5){\rule{10pt}{10pt}}
%			\put(0,-5){\line(1,0){10}}
%			\put(10,0){\line(0,-1){5}}}
%		\multiput(0,-25)(20,0){4}{
%			\put(0,0){\line(1,0){20}}
%			\put(0,-5){\line(1,0){20}}
%			\multiput(0,0)(5,0){5}{
%			\put(0,0){\line(0,-1){5}}}
%			\put(10,-5){\rule{20pt}{10pt}}
%			}
%		\multiput(0,-30)(40,0){2}{
%			\put(0,0){\line(1,0){40}}
%			\put(0,-5){\line(1,0){40}}
%			\multiput(0,0)(5,0){9}{
%			\put(0,0){\line(0,-1){5}}}
%			\put(20,-5){\rule{40pt}{10pt}}
%			}
%		\multiput(0,-35)(80,0){1}{
%			\put(0,0){\line(1,0){80}}
%			\put(0,-5){\line(1,0){80}}
%			\multiput(0,0)(5,0){17}{
%			\put(0,0){\line(0,-1){5}}}
%			\put(40,-5){\rule{80pt}{10pt}}
%			}		
%			\put(10,-45){Bin\"ar codierter Ma{\ss}stab}		
%		\multiput(0,-55)(80,0){1}{
%			\put(0,0){\line(1,0){80}}
%			\put(0,-5){\line(1,0){80}}
%			\multiput(0,0)(5,0){17}{
%			\put(0,0){\line(0,-1){5}}}
%			\multiput(0,0)(20,0){4}{
%			\put(5,-5){\rule{20pt}{10pt}}}
%			}
%		\multiput(0,-60)(80,0){1}{
%			\put(0,0){\line(1,0){80}}
%			\put(0,-5){\line(1,0){80}}
%			\multiput(0,0)(5,0){17}{
%			\put(0,0){\line(0,-1){5}}}
%			\multiput(0,0)(40,0){2}{
%			\put(10,-5){\rule{40pt}{10pt}}}
%			}
%		\multiput(0,-65)(80,0){1}{
%			\put(0,0){\line(1,0){80}}
%			\put(0,-5){\line(1,0){80}}
%			\multiput(0,0)(5,0){17}{
%			\put(0,0){\line(0,-1){5}}}
%			\put(20,-5){\rule{80pt}{10pt}}
%			}
%		\multiput(0,-70)(80,0){1}{
%			\put(0,0){\line(1,0){80}}
%			\put(0,-5){\line(1,0){80}}
%			\multiput(0,0)(5,0){17}{
%			\put(0,0){\line(0,-1){5}}}
%			\put(40,-5){\rule{80pt}{10pt}}
%			}
%		\put(10,-80){Gray codierter Ma{\ss}stab}	
%        		\end{picture}
%		\end{center}
%     \end{column}
% \end{columns}
%}
%
%\frame{\frametitle{Pr\"ufdatenauswertung - Darstellung}
%\framesubtitle{Quantitative Messdaten k\"onnen als Verteilung dargestellt werden.}
%\begin{columns}[t] 
%     \begin{column}[T]{6cm} 
%     	\begin{itemize}
%		\item Diskrete Verteilungen:
%			\begin{itemize}
%				\item H\"aufigkeiten der Merkmalsauspr\"agung 
%			\end{itemize}
%     		\item Kontinuierliche Verteilungen:
%		\begin{itemize}
%			\item Einteilung in Klassen
%			\item G\"unstig: $\sqrt{n}$ Klassen
%			\item H\"aufigkeit der Klassenzugeh\"origkeit
%			\item Darstellung im Histogramm
%		\end{itemize}
%     	\end{itemize}
%     \end{column}
%     	\begin{column}[T]{5cm} 
%         	\begin{center}
%            		\includegraphics[width=0.95\textwidth]{hist}\\
%		\tiny Pseudo-Zufallsverteilung eines Ma{\ss}es $x = 50$
%        		\end{center}
%     \end{column}
% \end{columns}
%}

\subsection{Pr\"ufdatenauswertung}
\frame{\frametitle{Beschreibung}
\framesubtitle{Beschreibung der Verteilung durch Lage und Streuungsparameter verdichtet die information.}
\begin{columns}[t] 
	\begin{column}[T]{5cm} 
         	\begin{center}
            		\includegraphics[width=0.9\textwidth]{hist2}
		% Spannweite
		{\thicklines \color{black} \put(-96,9){\vector(1,0){93}}\put(-96,9){\vector(-1,0){20}} \put(-80, 13){Spannweite}} 
		% Quartilsabstand
		{\thicklines \color{black} \put(-80,59){\vector(1,0){46}}\put(-80,59){\vector(-1,0){4}} \put(-94, 63){Quartilsabstand}}
		% Standardabweichung
		{\thicklines \color{black} \put(-63,39){\vector(1,0){30}}\put(-63,39){\vector(-1,0){30}} \put(-64, 43){s} \put(-33,0){\line(0,1){120}} \put(-93,0){\line(0,1){120}}}
		%Arithmetisches Mittel
		{\thicklines \color{green!50!black}\put(-66,170){\line(0,-1){160}} \put(-137,160){$\bar{x} = \frac{1}{n} \sum_{i=1}^{n}x_{i}$}} 
		% Modalwert
		{\thicklines \color{blue}\put(-90,130){\vector(1,1){20}} \put(-140,128){Modalwert}}
        		\end{center}
     \end{column}
     \begin{column}[T]{6cm} 
     	\begin{itemize}
		\item Lageparameter:  
		\begin{itemize}
			\item Arithmetisches Mittel $\bar{x}$  
			\item Modalwert: am h\"aufigsten angenommene Klasse 			\item Median: je 50\% der Messwerte gr\"o{\ss}er bzw. kleiner
		\end{itemize}
		\item Streuungsparameter
		\begin{itemize}
			\item Spannweite  
			\item Quartilsabstand: Spannweite der zentralen 50\% der Messwerte 
			\item Standardabweichung $s$
			\item Quantile: Merkmalsauspr\"agung, f\"ur die ein Anteil  $\alpha$ kleiner ist 
		\end{itemize}
	\end{itemize}
     \end{column}
 \end{columns}
}

\frame{\frametitle{Verteilungen}
\framesubtitle{Zufallsmodelle k\"onnen helfen, beobachtete Ph\"anomene zu beschreiben und \"uber die Modellbildung vorhersagen zu treffen.}
\begin{columns}[t] 
     \begin{column}[T]{6cm} 
     	\begin{itemize}
     		\item N\"aherungsweise Beschreibung 
		\item I.d.R. Konvergenz f\"ur gro{\ss}e Stichproben
		\item Wahrscheinlichkeit kann als relative H\"aufigkeit interpretiert werden
		\item Verteilung stetiger und diskreter Merkmale unterschiedlich zu modellieren
     	\end{itemize}
     \end{column}
     	\begin{column}[T]{5cm} 
         	\begin{center}
            		\includegraphics[width=0.95\textwidth]{cont}
        		\end{center}
     \end{column}
 \end{columns}
}

\frame{\frametitle{Verteilungen diskreter Merkmale}
\framesubtitle{}
\begin{columns}[t] 
     \begin{column}[T]{6cm} 
     	\begin{itemize}
     		\item Poisson-Verteilung:
		\begin{equation*}
			p_{P}(k) = \frac{\lambda^{k}}{k!} \exp\left(-\lambda\right)
		\end{equation*}
		\item[$\rightarrow$] Geringe Wahrscheinlichkeiten, z.B. $p\leq0,1$
		\item Binomial-Verteilung:
		\begin{equation*}
			p_{B}(k) = \binom{n}{k} p^{k} \left(1-p\right)^{n-k}
		\end{equation*}
		\item[$\rightarrow$] H\"ohere Wahrscheinlichkeiten, z.B. $p>0,1$
     	\end{itemize}
     \end{column}
     	\begin{column}[T]{5cm} 
         	\begin{center}
            		\includegraphics[width=0.95\textwidth]{poisson}
        		\end{center}
     \end{column}
 \end{columns}
}

\frame{\frametitle{Verteilungen kontinuierlicher Merkmale}
\framesubtitle{}
\begin{columns}[t] 
     \begin{column}[T]{6cm} 
     	\begin{itemize}
     		\item Normalverteilung:
		\begin{equation*}
			p_{N}(x) = \frac{1}{\sigma \sqrt{2\pi}}\exp\left(- \frac{\left(x-\mu\right)^{2}}{2\sigma^{2}} \right)
		\end{equation*}
		\item[$\rightarrow$] Faltung vieler Verteilungen
		\item Gleichverteilung
		\item Dreieckverteilung
		\item[$\rightarrow$] Faltung zweier Rechteckverteilungen
     	\end{itemize}
     \end{column}
     	\begin{column}[T]{5cm} 
         	\begin{center}
            		\includegraphics[width=0.95\textwidth]{cont}
        		\end{center}
     \end{column}
 \end{columns}
}

\frame{\frametitle{Zentraler Grenzwertsatz}
\framesubtitle{Konvergenz der Summe von Zufallsvariablen gegen die Standardnormalverteilung.}
\begin{columns}[t] 
     \begin{column}[T]{6cm} 
     	\begin{itemize}
	\scriptsize
     		\item Sei $X_{1}, X_{2}, X_{3},\ldots$ eine Folge von Zufallsvariable, die auf demselben Wahrscheinlichkeitsraum unabh\"angig und identisch verteilt sind.
		\item Sei weiterhin $S_{n} = X_{1} + X_{2} + \ldots + X_{n}$ die $n$-te Teilsumme, eine Zufallsvariable mit $E\left(S_{n}\right) = n \mu$ und $\Var\left(S_{n}\right) = n \sigma^{2}$
		\item Dann konvergiert die Verteilungsfunktion der standardisierten Zufallsvariablen
		\begin{equation*}
		Z_{n} = \frac{S_{n} - n\mu}{\sigma \sqrt{n}}
		\end{equation*}
		f\"ur $n \rightarrow \infty$ punktweise gegen die Standardnormalverteilung $\mathcal{N}(0,1)$.
     	\end{itemize}
     \end{column}
     	\begin{column}[T]{6.5cm} 
         	\begin{center}
            		\setlength{\unitlength}{0.0500bp}%
  \begin{picture}(3800.00,3000.00)%
  	\scriptsize
    { \put(400,2127){\makebox(0,0)[r]{\strut{}0}}%
    %\put(400,2235){\makebox(0,0)[r]{\strut{}20}}%
      
      \put(400,2343){\makebox(0,0)[r]{\strut{}40}}%
      
      %\put(400,2451){\makebox(0,0)[r]{\strut{}60}}%
      
      \put(400,2558){\makebox(0,0)[r]{\strut{}80}}%
      
      %\put(400,2666){\makebox(0,0)[r]{\strut{}100}}%
      
      \put(400,2774){\makebox(0,0)[r]{\strut{}120}}%
      
      %\put(520,2000){\makebox(0,0){\strut{}0}}%
      
      %\put(788,2000){\makebox(0,0){\strut{}0.2}}%
      
      %\put(1055,2000){\makebox(0,0){\strut{}0.4}}%
      
      %\put(1323,2000){\makebox(0,0){\strut{}0.6}}%
      
      %\put(1590,2000){\makebox(0,0){\strut{}0.8}}%
      
      %\put(1858,1927){\makebox(0,0){\strut{}1}}%
      \csname LTb\endcsname%
      \put(1189,2900){\makebox(0,0){\strut{}$n = 1$}}%
    }%
    {%
      
      \put(2161,2127){\makebox(0,0)[r]{\strut{}0}}%
      
      %\put(2161,2235){\makebox(0,0)[r]{\strut{}20}}%
      
      \put(2161,2343){\makebox(0,0)[r]{\strut{}40}}%
      
      %\put(2161,2451){\makebox(0,0)[r]{\strut{}60}}%
      
      \put(2161,2558){\makebox(0,0)[r]{\strut{}80}}%
      
      %\put(2161,2666){\makebox(0,0)[r]{\strut{}100}}%
      
      \put(2161,2774){\makebox(0,0)[r]{\strut{}120}}%
      
      \put(2950,2900){\makebox(0,0){\strut{}$n = 5$}}%
    }%
    {%
      
      \put(400,1228){\makebox(0,0)[r]{\strut{}0}}%
      
      %\put(400,1336){\makebox(0,0)[r]{\strut{}20}}%
      
      \put(400,1444){\makebox(0,0)[r]{\strut{}40}}%
      
      %\put(400,1552){\makebox(0,0)[r]{\strut{}60}}%
      
      \put(400,1659){\makebox(0,0)[r]{\strut{}80}}%
      
      %\put(400,1767){\makebox(0,0)[r]{\strut{}100}}%
      
      \put(400,1875){\makebox(0,0)[r]{\strut{}120}}%
      
      %\put(520,1028){\makebox(0,0){\strut{}0}}%
      
      %\put(788,1028){\makebox(0,0){\strut{}0.2}}%
      
      %\put(1055,1028){\makebox(0,0){\strut{}0.4}}%
      
      %\put(1323,1028){\makebox(0,0){\strut{}0.6}}%
      
      %\put(1590,1028){\makebox(0,0){\strut{}0.8}}%
      
      %\put(1858,1028){\makebox(0,0){\strut{}1}}%
      %\csname LTb\endcsname%
      \put(1189,2000){\makebox(0,0){\strut{}$n = 9$}}%
    }%
    {%
      
      \put(2161,1228){\makebox(0,0)[r]{\strut{}0}}%
      
      %\put(2161,1336){\makebox(0,0)[r]{\strut{}20}}%
      
      \put(2161,1444){\makebox(0,0)[r]{\strut{}40}}%
      
      %\put(2161,1552){\makebox(0,0)[r]{\strut{}60}}%
      
      \put(2161,1659){\makebox(0,0)[r]{\strut{}80}}%
      
      %\put(2161,1767){\makebox(0,0)[r]{\strut{}100}}%
      
      \put(2161,1875){\makebox(0,0)[r]{\strut{}120}}%
       \put(2950,2000){\makebox(0,0){\strut{}$n = 13$}}%
    }%
    {%
      
      \put(400,330){\makebox(0,0)[r]{\strut{}0}}%
      
      %\put(400,438){\makebox(0,0)[r]{\strut{}20}}%
      
      \put(400,545){\makebox(0,0)[r]{\strut{}40}}%
      
      %\put(400,653){\makebox(0,0)[r]{\strut{}60}}%
      
      \put(400,761){\makebox(0,0)[r]{\strut{}80}}%
      
      %\put(400,868){\makebox(0,0)[r]{\strut{}100}}%
      
      \put(400,976){\makebox(0,0)[r]{\strut{}120}}%
      
      \put(520,130){\makebox(0,0){\strut{}0}}%
      
      \put(788,130){\makebox(0,0){\strut{}0.2}}%
      
      \put(1055,130){\makebox(0,0){\strut{}0.4}}%
      
      \put(1323,130){\makebox(0,0){\strut{}0.6}}%
      
      \put(1590,130){\makebox(0,0){\strut{}0.8}}%
      
      \put(1858,130){\makebox(0,0){\strut{}1}}%
      %\csname LTb\endcsname%
      \put(1189,1100){\makebox(0,0){\strut{}$n = 17$}}%
    }{%
      
      \put(2161,330){\makebox(0,0)[r]{\strut{}0}}%
      
      %\put(2161,438){\makebox(0,0)[r]{\strut{}20}}%
      
      \put(2161,545){\makebox(0,0)[r]{\strut{}40}}%
      
      %\put(2161,653){\makebox(0,0)[r]{\strut{}60}}%
      
      \put(2161,761){\makebox(0,0)[r]{\strut{}80}}%
      
      %\put(2161,868){\makebox(0,0)[r]{\strut{}100}}%
      
      \put(2161,976){\makebox(0,0)[r]{\strut{}120}}%
      
      \put(2281,130){\makebox(0,0){\strut{}0}}%
      
      \put(2549,130){\makebox(0,0){\strut{}0.2}}%
      
      \put(2816,130){\makebox(0,0){\strut{}0.4}}%
      
      \put(3084,130){\makebox(0,0){\strut{}0.6}}%
      
      \put(3351,130){\makebox(0,0){\strut{}0.8}}%
      
      \put(3619,130){\makebox(0,0){\strut{}1}}%
      \csname LTb\endcsname%
      \put(2950,1100){\makebox(0,0){\strut{}$n = 21$}}%
    }
    \put(0,0){\includegraphics{ZentralerGWS}}%
  \end{picture}
  \tiny Summe aus $n$ gleichverteilten Zufallsvariablen
        		\end{center}
     \end{column}
 \end{columns}
}

\frame{\frametitle{Eigenschaften der Normalverteilung}
\framesubtitle{Viele Ph\"anomene in Technik und Naturwissenschaft lassen sich durch eine Normalverteilung ann\"ahern.}
\begin{columns}[t]
	\begin{column}[T]{5cm} 
         	\begin{center}
            		\includegraphics[width=0.95\textwidth]{gaussian}
%		\begin{tikzpicture} 
%		\begin{axis}[xmin = -5, xmax = 5, ymin = 0,
%                    ]
%                    % density of Normal distribution:
%                    \addplot+[name path = A,
%                       red,
%                       domain=-5:5,
%                       samples=201,
%                    ]
%                       {exp(-x^2 / (2)) / (1 * sqrt(2*pi))};
%		
%		\path[name path=B] (-5,0) -- (5,0);
%		\addplot[red!30] fill between[of=A and B,
%        			soft clip= {domain=0:3}];
%                    \end{axis}
%                   \end{tikzpicture}
        		\end{center}
     \end{column} 
     \begin{column}[T]{6cm} 
     	\begin{itemize}
     		\item Mittelwert $\mu$
		\item Standardabweichung $\sigma$
		\item $\sigma$-Umgebungen:\\
		\begin{tabular}{|c|c|}
		\hline
		$k$ & \% der Realisierungen \\ \hline
		1 & 68{,}3 \\ \hline
		2 & 95{,}5 \\ \hline
		3 & 99{,}4 \\ \hline
		%6 & 99{,}99966 \\ \hline
		\end{tabular}
		\item Quantile:\\
		\begin{tabular}{|c|c|}
		\hline
		\% der Realisierungen & $k$  \\ \hline
		50 & 0{,}675 \\ \hline
		90 & 1{,}65 \\ \hline
		95 & 1{,}96 \\ \hline
		99 & 2{,}58 \\ \hline
		\end{tabular}
     	\end{itemize}
     \end{column}
 \end{columns}
}


\frame{\frametitle{Vertiefung: Konfidenzintervall, Bestimmung $u$}
\framesubtitle{Konfidenzintervalle sind Intervalle in denen der wahre Wert einer Merkmalsauspr\"agung mit Wahrscheinlichkeit $p$ liegt.}
%\only<1> {\begin{center}\includegraphics[scale=0.15]{Off} \end{center}}
%\only<2>
{
\begin{columns}[t] 
     \begin{column}[T]{6cm} 
     	\begin{itemize}
     		\item Gr\"o{\ss}e und Lage abh\"abh\"angig von
		\begin{itemize}
     		\item Verteilung
		\item Konfidenzniveau (z.B. 95\%)
		\end{itemize}
		\item F\"ur Normalverteilung:\\
		\scriptsize
		\begin{tabular}{|c|c|}
		\hline 
		$k$ & \% der Realisierungen \\ \hline
		1 & 68{,}3 \\ \hline
		2 & 95{,}5 \\ \hline
		3 & 99{,}4 \\ \hline
		%6 & 99{,}99966 \\ \hline
		\end{tabular}
		\normalsize
		\item Fertigungsmesstechnik: 95\%-Konfidenzintervall $U_{95\%}$ f\"ur $k=2$
     	\end{itemize}
     \end{column}
     	\begin{column}[T]{5cm} 
         	  \begin{picture}(80,60)(30,15)
    \put(0,0){\includegraphics[scale=0.401]{cont2}}
    \thicklines
    \scriptsize
    \put(85,11){{\line(0,1){55}}}
    \put(85,49){\color{red}{\vector(1,0){32}}}
    \put(117,49){\color{red}{\vector(-1,0){32}}}
    \put(117,11){\color{red}{\line(0,1){40}}}
    \put(88,65){$\mu$}
    \put(95,52){\color{red}$a_{\triangle}$}
    %%
    \put(62,15){\color{green!60!black}{\vector(1,0){22}}}
    \put(84,15){\color{green!60!black}{\vector(-1,0){22}}}
    \put(67,18){\color{green!60!black}$a_{\square}$}
    %%
    \put(90,34){\color{blue!60!black}$a_{\mathcal{N}}$}
    \put(112,2){\color{blue!60!black}{\line(0,1){35}}}
    \put(58,2){\color{blue!60!black}{\line(0,1){15}}}
    \put(85,30){\color{blue!60!black}{\vector(1,0){27}}}
    \put(85,2){\color{blue!60!black}{\vector(1,0){27}}}
    \put(85,2){\color{blue!60!black}{\vector(-1,0){27}}}
    \put(85,5){\color{blue!60!black}{$U_{95\%}$}}
  \end{picture}
  \vspace{.5cm}
			\begin{equation*}
		u = \frac{a}{K}
		\end{equation*}
		Normalverteilung: $k = \sqrt{4}$, $a$  Abma{\ss}e von $U_{95\%}$\\
		Gleichverteilung: $k = \sqrt{3}$\\
		Dreiecksverteilung: $k = \sqrt{6}$ 
     \end{column}
 \end{columns}}
}

\frame{\frametitle{Vertiefung: Konfidenzintervall, Bestimmung $u$}
\framesubtitle{}
\begin{picture}(160,160)(00,0)
	\setlength{\unitlength}{2pt}
    \put(0,0){\includegraphics[scale=0.8]{cont2}}
    \thicklines
    %\scriptsize
    \put(85,11){{\line(0,1){55}}}
    \put(85,49){\color{red}{\vector(1,0){32}}}
    \put(117,49){\color{red}{\vector(-1,0){32}}}
    \put(117,11){\color{red}{\line(0,1){40}}}
    \put(88,65){$\mu$}
    \put(95,52){\color{red}$a_{\triangle}$}
    %%
    \put(62,15){\color{green!60!black}{\vector(1,0){22}}}
    \put(84,15){\color{green!60!black}{\vector(-1,0){22}}}
    \put(67,18){\color{green!60!black}$a_{\square}$}
    %%
    \put(90,34){\color{blue!60!black}$a_{\mathcal{N}}$}
    \put(112,2){\color{blue!60!black}{\line(0,1){35}}}
    \put(58,2){\color{blue!60!black}{\line(0,1){15}}}
    \put(85,30){\color{blue!60!black}{\vector(1,0){27}}}
    \put(85,2){\color{blue!60!black}{\vector(1,0){27}}}
    \put(85,2){\color{blue!60!black}{\vector(-1,0){27}}}
    \put(85,5){\color{blue!60!black}{$U_{95\%}$}}
    \put(110,60){\framebox{\huge $u = \frac{a}{K}$}}
  \end{picture}
		\begin{itemize}
		\item Normalverteilung: $k = \sqrt{4}$, $a$  Abma{\ss}e von $U_{95\%}$\\
		\item Gleichverteilung: $k = \sqrt{3}$\\
		\item Dreiecksverteilung: $k = \sqrt{6}$ 
		\end{itemize}
}

\frame{\frametitle{Student'sche t-Verteilung}
%\framesubtitle{Bei kleinem Stichprobenumfang}
%\only<1> {\begin{center}\includegraphics[scale=0.15]{Off} \end{center}}
%\only<2>
{
\begin{columns}[t] 
     \begin{column}[T]{6cm} 
     	\begin{itemize}
     		\item Normalverteilung (Hypothese):
		\begin{itemize}
		\item Endliche Stichprobe
		\item Wahrscheinlichkeit, ``seltene'' Werte zu realisieren?
		\end{itemize}
		\item Stichprobenvarianz f\"allt zu klein aus
		\item ``schlanke'' Normalverteilung nicht konservativ
		\item Abhilfe: Student'sche t-Verteilung:
		\begin{equation*}
		p_{n}(x) = \frac{\Gamma\left(\frac{n+1}{2} \right)}{\sqrt{n \pi}\Gamma\left(\frac{n}{2} \right)} \left( 1 + \frac{x^{2}}{2}\right)^{\frac{-n+1}{2}} 
		\end{equation*}
		\begin{itemize}
		\item f\"ur $n$ Freiheitsgrade
		\end{itemize}
     	\end{itemize}
     \end{column}
     	\begin{column}[T]{5cm} 
         	\begin{center}
            		\setlength{\unitlength}{0.0280bp}%
		 
  \begin{picture}(6000.00,8000.00)(500,-1000)%
  	\scriptsize
  	\put(0,0){\includegraphics[scale=0.56]{cont3}}
      \put(792,480){\makebox(0,0)[r]{\strut{}0}}%
      \put(792,1384){\makebox(0,0)[r]{\strut{}0.05}}
      \put(792,2288){\makebox(0,0)[r]{\strut{}0.1}}%
      \put(792,3192){\makebox(0,0)[r]{\strut{}0.15}}
      \put(792,4096){\makebox(0,0)[r]{\strut{}0.2}}%
      \put(792,4999){\makebox(0,0)[r]{\strut{}0.25}}
      \put(792,5903){\makebox(0,0)[r]{\strut{}0.3}}%
      \put(792,6807){\makebox(0,0)[r]{\strut{}0.35}}
      \put(792,7711){\makebox(0,0)[r]{\strut{}0.4}}
      \put(3300,7900){\makebox(0,0)[c]{\strut{}Wahrscheinlichkeitsdichtefunktionen}}
      \put(936,240){\makebox(0,0){\strut{}-3}}
      \put(1708,240){\makebox(0,0){\strut{}-2}}
      \put(2480,240){\makebox(0,0){\strut{}-1}}
      \put(3252,240){\makebox(0,0){\strut{}0}}
      \put(4023,240){\makebox(0,0){\strut{}1}}%
      \put(4795,240){\makebox(0,0){\strut{}2}}
      \put(5567,240){\makebox(0,0){\strut{}3}}
      \put(4323,7528){\makebox(0,0)[l]{\strut{}$\mathcal{N}(0,1)$}}
      \put(4323,7288){\makebox(0,0)[l]{\strut{}t, $\nu = 1$}}%
      \put(4323,7048){\makebox(0,0)[l]{\strut{}t, $\nu = 3$}}
      \put(4323,6808){\makebox(0,0)[l]{\strut{}t, $\nu = 5$}}%
      \put(4323,6568){\makebox(0,0)[l]{\strut{}t, $\nu = 10$}}%
      \end{picture}%

        		\end{center}
     \end{column}
 \end{columns}}
}

\frame{\frametitle{Praktische Anwendung der Student'schen t-Verteilung}
\framesubtitle{}
\begin{columns}[t] 
     \begin{column}[T]{6cm} 
     	\begin{itemize}
     		\item t-Verteilung unterstellt ``breitere'' Wahrscheinlichkeitsdichte
		\item Sch\"atzwert der Stichprobenvarianz wird korrigiert
		\item Bestimmung Korrekturfaktor aus Konfidenz und Stichprobengr\"o{\ss}e $n$
		\item Freiheitsgrade $\nu = n - m$ f\"ur $m$ zu sch\"atzende Parameter
     	\end{itemize}
     \end{column}
     	\begin{column}[T]{5cm} 
         	\begin{center}
            		\setlength{\unitlength}{0.0250bp}%
		 
  \begin{picture}(6000.00,6000.00)(500,-1000)%
  	\scriptsize
  	\put(0,0){\includegraphics[scale=0.50]{cont4}}
      %\put(3000,5900){\makebox(0,0)[c]{\strut{}Wahrscheinlichkeitsdichtefunktionen}}
      \put(4323,5500){\makebox(0,0)[l]{\strut{}\color{blue!70!black}$\mathcal{N}(0,1)$}}
      \put(4323,5200){\makebox(0,0)[l]{\strut{}\color{green!70!black} t, $\nu = 1$}}%
      \put(4323,4900){\makebox(0,0)[l]{\strut{}\color{red!80!black} t, $\nu = 3$}}
      \end{picture}%

        		\end{center}
     \end{column}
 \end{columns}
 \small
 \hspace{.8cm}
\begin{tabular}{|c|c|c|c|c|c|c|c|c|c|}
\hline $\nu$ & 1 & 2 & 3 & 4 & 5 & 10 & 20 & 50 &$\infty$ \\ \hline
$k $ & 13{,}97 & 4{,}53 & 3{,}31 & 2{,}87 & 2{,}65& 2{,}28 & 2{,}13 & 2{,}05 & 2{,}00 \\ \hline
\end{tabular}
\begin{itemize}
		\item[] $k$-Werte zur Bestimmung der erweiterten Messunsicherheit $U_{95\%}$
		\end{itemize}}




\frame{\frametitle{Definitionen}
\framesubtitle{}
\begin{itemize}
\item Ma{\ss}: Bestimmung einer L\"ange
\item Nennma{\ss}: Zeichnungangabe
\item Istma{\ss}: tats\"achliches Ma{\ss}
\item Oberes/unteres Abma{\ss}: zul\"assige Abweichung - Achtung Vorzeichen!
\item Mindestma{\ss}: Nennma{\ss} - unteres Abma{\ss}
\item H\"ochstma{\ss}: Nennma{\ss} + oberes Abma{\ss}
\end{itemize}
\begin{center}
\includegraphics[width=0.7\textwidth]{Toleranzfeld}
\end{center}
}




%\frame{\frametitle{\"Ubung 3: Umgang mit Abweichungen}
%\framesubtitle{Ein Lieferant hat eine kleine St\"uckzahl an Teilen f\"ur das Gelenk gefertigt, es kommt zu Abweichungen, die in einem Messprotokoll festgehalten sind. Die Teile werden f\"ur Lieferungen ihrerseits dringend erwartet.}
%\begin{columns}[t] 
%     \begin{column}[T]{6cm} 
%     	\begin{enumerate}
%     		\item Formulieren Sie den Abweichungsantrag f\"ur den Lieferanten
%		\item Entscheiden Sie \"uber Annahme oder Ablehnung mit Begr\"undung.
%	\end{enumerate}
%     \end{column}
%     	\begin{column}[T]{5cm} 
%         	\begin{center}
%            		\includegraphics[width=0.95\textwidth]{Abweichungsantrag}
%        		\end{center}
%     \end{column}
% \end{columns}
% \nocite{pfeifer10}
%\nocite{hoischen}
%}
