% !TEX root = ../SFV15001_MdQM_Fertigungsmesstechnik_Rev03.tex
% \section{Prüfdatenerfassung und -auswertung elektrischer Komponenten}
\label{Sec:Elektrisch}

\frame{\frametitle{Motivation}
\framesubtitle{Warum elektrische Komponenten in MdQM aufnehmen?}
\begin{itemize}
\item Zunahme mechatronischer Komponenten
	\begin{itemize}
		\item Integrationsleistung steigt
		\item Engineering/Fertigung h\"aufig extern
		\item Aber Qualit\"atsmanagement?
	\end{itemize}
\item Fehlereingrenzung
	\begin{itemize}
		\item H\"aufig im elektrischen Subsystem
	\end{itemize}
\item Indirekte Messung
	\begin{itemize}
		\item Kraft, Weg \"uber Widerstand
		\item Schichtdicke \"uber Induktivit\"at
		\item ...
	\end{itemize}
\end{itemize}
}


\subsection{Eigenschaften elektrischer Messger\"ate}
\frame{\subsectionpage}

\frame{\frametitle{Statisches Verhalten}
\framesubtitle{}
\begin{columns}[t] 
     \begin{column}[T]{6cm} 
     	\begin{itemize}
     		\item Statischer Zustand:
		\begin{itemize}
		\item Eingangsgr\"o{\ss}e konstant
		\item Ausgangsgr\"o{\ss}e eingeschwungen
		\end{itemize}
		\item Kennlinie:
		\begin{itemize}
		\item Eindeutige Zuordung Eingang zu Ausgang
		\item Ideal: Linear
		\item Real: Nichtlinear
		\end{itemize}
		\item Empfindlichkeit:
		\begin{itemize}
		\item $E = \frac{\delta x_{a}}{\delta x_{a}} \approx \frac{x_{a}}{x_{e}}$
		\end{itemize}
     	\end{itemize}
     \end{column}
     	\begin{column}[T]{6cm} 
         	\begin{center}	
		\begin{tikzpicture}
 			\node[rectangle, draw, thick, align= center] (m) at (2,5) {Mess-\\ger\"at};
			\draw[->, thick] (0,5) -- (m) node[pos = 0.2, above] {$x_{e}$};
			\draw[->, thick] (m) -- (4,5)node[pos = 0.8, above] {$x_{a}$};
			
			\begin{axis}[
			axis lines=left,
			xmin = 0, xmax = 1, ymin = 0, ymax = 1,
			ytick = \empty, xtick = \empty,
			xlabel = $x_{e}$, ylabel = $x_{a}$,
			every axis x label/.style={
  				  at={(ticklabel* cs:1.05)},
 				   anchor=west,
				},
					every axis y label/.style={
				    at={(ticklabel* cs:1.05)},
				    anchor=south,
				},]
                            \addplot[smooth, blue] coordinates {
                                (0,   0)
                                (0.7,  0.75)
                                (0.9,  0.85)
                                (1,  0.87)
                                };
				\addplot[smooth, red] coordinates {
                                (0,   0)
                                (1,  1)
                                };
                            \end{axis}
		\end{tikzpicture}
        		\end{center}
     \end{column}
 \end{columns}
}

\frame{\frametitle{Anzeige- und Messbereich}
\framesubtitle{}
     	\begin{itemize}
     		\item Anzeigebereich
		\begin{itemize}
		\item Werte werden angezeigt
		\end{itemize}
		\item Messbereich
		\begin{itemize}
		\item Werte werden erfasst
		\item Genauigkeitsangabe gilt
		\end{itemize}
		\item Aufl\"osung
		\begin{itemize}
		\item Kleinste darstellbare \"Anderung von $x_{a}$
		\item Nicht mit Genauigkeit verwechseln!
		\end{itemize}
     	\end{itemize}
}

\frame{\frametitle{Dynamisches Verhalten}
\framesubtitle{}
\begin{columns}[t] 
     \begin{column}[T]{6cm} 
     	\begin{itemize}
     		\item Messeinrichtung folgt nicht beliebig schnell der Eingangsgr\"o{\ss}e
		\item Dynamische Messabweichung:
		\begin{equation*}
			e_{dyn} = x(t) - x_{w}(t)
		\end{equation*}
		\item Einschwingzeit
		\begin{itemize}
		\item Zeit, die nach Sprung auf $x_{e}$ gewartet werden muss
		\end{itemize}
     	\end{itemize}
     \end{column}
     	\begin{column}[T]{6cm} 
         	\begin{center}
            		\begin{tikzpicture}
 			\node[rectangle, draw, thick, align= center] (m) at (2,5) {Mess-\\ger\"at};
			\draw[->, thick] (0,5) -- (m) node[pos = 0.2, above] {$x_{e}$};
			\draw[->, thick] (m) -- (4,5)node[pos = 0.8, above] {$x_{a}$};
			
			\begin{axis}[
			axis lines=left,
			%xmin = 0, xmax = 1, ymin = 0, ymax = 1,
			ytick = \empty, xtick = \empty,
			xlabel = $t$, ylabel = $x$,
			legend entries = {$x_{e}$, $x_{a}$},
			every axis x label/.style={
  				  at={(ticklabel* cs:1.05)},
 				   anchor=west,
				},
					every axis y label/.style={
				    at={(ticklabel* cs:1.05)},
				    anchor=south,
				},
				every axis legend/.append style={
        				at={(0.7,0.8)},
        				anchor=south west}]
				  \addplot[thick, draw = blue!70!black] table[x=t,y=u, 
				  ] {Output.dat}; 
  
				  \addplot[thick, draw = red!70!black] table[x=t,y=y, 
				  ] {Output.dat}; 
    
\end{axis}
		\end{tikzpicture}
        		\end{center}
     \end{column}
 \end{columns}
}

\frame{\frametitle{Genauigkeitsangaben elektrischer Messger\"ate}
\framesubtitle{}
\begin{itemize}
\item Fehlergrenze:
\begin{itemize}
		\item Maximal zul\"assige Messabweichung $G = \left\| e \right\|_{max} = \left\| x-x_{w} \right\|_{max}$
		\item Interpretation: F\"ur Messwert $x_{m}$ gilt: $x_{w} \in \left[ x_{w}-G; x_{w} + G \right]$
		\end{itemize}
		\item Eigenabweichung:
		\begin{itemize}
		\item Grunds\"atzliche Messabweichung unter Normbedingungen, an einem Punkt im Messbereich
		\end{itemize}
		\item Betriebsmessabweichung:
		\begin{itemize}
		\item Im Betrieb erzielte Messabweichung, unter tats\"achlichen Bedingungen
		\end{itemize}
		\item Vereinbart: Abweichungen mit $U_{95\%}$-Konfidenz
		\item Angabe Fehlergrenzen:
		\begin{itemize}
		\item Absolut: $\pm 1 \mathrm{V}$
		\item Relativ: $\pm 0{,}5 \%$ des Anzeigewerts
		\item Kombiniert: $\pm\left(0{,}5 \%+ 1 \mathrm{V}\right)$
		\end{itemize}
\end{itemize}
}

\frame{\frametitle{Grunds\"atzliche Bauarten}
\framesubtitle{}
\begin{columns}[t] 
     \begin{column}[T]{6cm} 
     	\begin{itemize}
     		\item Elektromechanisch:
		\begin{itemize}
		\item Drehspulmesswerk
		\item Dreheisenmesswerk
		\item Elektrodynamisches Messwerk
		\item Elektrostatisches Messwerk
		\item und weitere
		\end{itemize}
		\item Digitale Messger\"ate
     	\end{itemize}
     \end{column}
     	\begin{column}[T]{6cm} 
         	\begin{center}
            		\includegraphics[width=0.5\textwidth]{Drehspul}\source{Quelle: wikimedia/S\o ren Peo Pedersen}
        		\end{center}
     \end{column}
 \end{columns}
}

\frame{\frametitle{Effekte digitaler Messger\"ate}
\framesubtitle{}
\begin{columns}[t] 
     \begin{column}[T]{6cm} 
     	\begin{itemize}
     		\item Abtastung:
		\begin{itemize}
		\item Zeitdiskretion der Anzeigewerte
		\item Abtastrate $f_{a} = \frac{1}{T_{a}}$
		\item Sampling-Theorem (Shannon-Nyquist): 
		\begin{itemize}
		\item $f_{a} > 2 f_{max}$
		\item i.d.R. h\"oher
		\end{itemize}
		\end{itemize}
		\item Quantisierung:
		\begin{itemize}
		\item Umwandlung in Wert mit endlicher Aufl\"osung
		\item F\"ur $N$ Bit:
			$\triangle U = \frac{U_{max}}{2^N-1}$
		\item Abweichung durch Quantisierung $\pm \frac{\triangle U}{2}$
		\end{itemize}
     	\end{itemize}
     \end{column}
     	\begin{column}[T]{6cm} 
         	\begin{center}
            		\only<1>{\includegraphics[width=0.6\textwidth]{Multimeter}\source{Quelle: wikimedia/Harke}}
			\only<2>{\includegraphics[width=0.75\textwidth]{ShannonNyquist}\source{Quelle: wikimedia/Peterpall}}
        		\end{center}
     \end{column}
 \end{columns}
}

\subsubsection{Strom- und Spannungsmessung}
\frame{\subsectionpage}
\offslide{Gleichstrommessung}

\offslide{Gleichspannungsmessung}

\subsubsection{Messung Ohmscher Widerst\"ande}
\frame{\subsectionpage}

\frame{\frametitle{Stromrichtige Messung}
\framesubtitle{}
\begin{columns}[t] 
     \begin{column}[T]{6cm} 
     	\begin{equation*}
		R = \frac{U}{I} = \frac{I \left(R_{x} + R_{I} \right)}{I} = R_{X} + R_{I}
	\end{equation*}
	\begin{equation*}
		R_{korr} = \frac{U}{I} - R_{I}
	\end{equation*}
     \end{column}
     	\begin{column}[T]{6cm} 
         	\begin{center}
            		\includegraphics[width=0.9\textwidth]{Stromrichtig}\source{}
        		\end{center}
     \end{column}
 \end{columns}
}

\frame{\frametitle{Spannungsrichtige Messung}
\framesubtitle{}
\begin{columns}[t] 
     \begin{column}[T]{6cm} 
     	\begin{equation*}
		R = \frac{U}{I} = R_{X} \| R_{U} = \frac{R_{X} R_{U}}{R_{X}+R_{U}}
	\end{equation*}
	\begin{equation*}
		R_{korr} = \frac{U}{I - \frac{U}{R_{U}}}
	\end{equation*}
     \end{column}
     	\begin{column}[T]{6cm} 
         	\begin{center}
            		\includegraphics[width=0.9\textwidth]{Spannungsrichtig}\source{}
        		\end{center}
     \end{column}
 \end{columns}
}

\frame{\frametitle{Wheatstone-Messbr\"ucke}
\framesubtitle{}
\begin{columns}[t] 
     \begin{column}[T]{6cm} 
     Spannungsteiler:
     	\begin{equation*}
		U_{1} = U_{0}\frac{R_{1}}{R_{1} + R_{2}}\quad U_{3} = U_{0}\frac{R_{3}}{R_{3} + R_{4}}
	\end{equation*}
	Maschenumlauf oben:
	\begin{equation*}
	\begin{split}
	U_{B} &= U_{3} - U_{1} = U_{0}\left(\frac{R_{3}}{R_{3} + R_{4}} - \frac{R_{1}}{R_{1} + R_{2}} \right)\\
	&= U_{0} \left(\frac{R_{2}R_{3} - R_{1}R_{4}}{\left(R_{1} + R_{2} \right)\left(R_{3} + R_{4} \right)} \right)
	\end{split}
	\end{equation*}
	Abgleich: $U_{B} = 0$
	\begin{equation*}
	R_{2} = R_{1}\frac{R_{4}}{R_{3}}
	\end{equation*}
     \end{column}
     	\begin{column}[T]{6cm} 
         	\begin{center}
            		\includegraphics[width=0.9\textwidth]{Wheatstone}\source{}
        		\end{center}
     \end{column}
 \end{columns}
}
