% !TEX root = ../17-MdQM-Vorlesung.tex
\subsection{Requirements Engineering}
%\sectionpage

\frame{\frametitle{Beispiel mechatronisches System: Gleitschutz}
\begin{center}
            		\includegraphics[width=\textwidth]{WSP}\source{}
        		\end{center}
}

\frame{\frametitle{Beispiel mechatronisches System}
\framesubtitle{}
\begin{columns}[t] 
     \begin{column}[T]{6cm} 
     	\begin{itemize}
     		\item Mechatronik nach VDI 2206:
		\begin{itemize}
		\item Synergetisches Zusammenwirken der Disziplinen:
		\begin{itemize}
		\item Maschinenbau
		\item Eletrotechnik
		\item Informationstechnik
		\end{itemize}
		\end{itemize}
		\item H\"aufiges Kennzeichen:
		\begin{itemize}
		\item Verschmelzung von Sensorik, Aktorik, Rechner und Mechanik
		\end{itemize}
		\item Typisch
		\begin{itemize}
		\item Hoher Aufwand w\"ahrend Inbetriebnahme
		\item Schnittstellenkl\"arung schwierig
		\end{itemize}
     	\end{itemize}
     \end{column}
     	\begin{column}[T]{6cm} 
         	\begin{center}
            		\includegraphics[width=\textwidth]{WSP}\source{}
        		\end{center}
     \end{column}
 \end{columns}
}

\frame{\frametitle{Warum Requirements Engineering (RE)?}
\framesubtitle{Requirements Engineering befasst sich mit dem systematischen Erfassen, Umsetzen und Pr\"ufen von Anforderungen im Entwicklungsprozess.}
\begin{itemize}
\item Qualit\"at: 
\begin{itemize}
		\item Grad, in dem ein Satz inhärenter Merkmale eines Objekts Anforderungen erfüllt \cite{iso9000}
		\item Qualit\"at ist das Ma{\ss} der Erf\"ullung der Anforderungen an ein Produkt
		\end{itemize}
\item Durch RE Reduzierung von:
\begin{itemize}
		\item Entwicklungs- und Garantiekosten
		\item Konstruktions\"anderungen
		\item Design lead time
		\item Fehlerrate in Entwicklung und Einsatz
		\end{itemize}
\item Kosten- und Termintreue
\item Einbindung der Stakeholder (Anspruchsteller)
\item Systematisierung der Beschaffung und der Produktentwicklung
\item Nachvollziehbare Entscheidungen und Risikomanagement
\end{itemize}
}

%\frame{\frametitle{Key-Aspects of Requirements Engineering}
%\framesubtitle{}
%\begin{itemize}
%\item Stakeholder Involvement
%\item Technical Reviews
%\item Traceability
%\end{itemize}
%}

%\offslide{Generisches Phasenmodell}{Modell einer beliebigen Phase eines Entwicklungsprozesses}

\frame{\frametitle{Erweitertes V-Modell mit Lebenszyklus}
\framesubtitle{}
\begin{center}
\includegraphics[width = 11cm]{VModel} \source{Quelle: \cite{SEGuide}}
\end{center}
}


\frame{\frametitle{Generisches Phasenmodell}
\framesubtitle{Modell einer beliebigen Phase eines Entwicklungsprozesses}
\begin{columns}[t] 
     \begin{column}[T]{6cm} 
     \textbf{F\"ur jede Phase festzulegen:}
     	\begin{itemize}
     		\item Purpose
		\item Inputs
		\item Entry Criteria
		\item Roles
		\item Verification steps
		\item Outputs
		\item Exit criteria
		\item Resources
		\item Management review activities
     	\end{itemize}
     \end{column}
     	\begin{column}[T]{5cm} 
         	\begin{center}
            		\includegraphics[width=1.0\textwidth]{Phase.png}
        		\end{center}
     \end{column}
 \end{columns}
}

\frame{\frametitle{System Requirements}
\framesubtitle{Obtain System Level Requirements}
\begin{columns}[t] 
     \begin{column}[T]{6cm} 
     \textbf{Questions:}
     	\begin{itemize}
		\item What are the stakeholders?
     		\item What is the system to do?
		\item How well it is to do it?
		\item Under what conditions?
     	\end{itemize}
	\textbf{Typical Milestone: Initial Design Review (IDR)}
     \end{column}
     	\begin{column}[T]{5cm} 
         	\begin{center}
            		\includegraphics[width=0.95\textwidth]{Phase}
        		\end{center}
     \end{column}
 \end{columns}
}

\frame{\frametitle{High Level Design}
\framesubtitle{Top Level Design: Architekture, Solutions, Subsystems}
\begin{columns}[t] 
     \begin{column}[T]{6cm} 
     \textbf{Questions:}
     	\begin{itemize}
		\item Is the required system feasible?
     		\item What are system and subsystem borders?
		\item What are associated costs/lead times/risks?
		\item How can the risk be reduced?
		\item Which system integration steps are necessary?
		\item What is the suitable subsystem structure?
     	\end{itemize}
	\textbf{Typical Milestone: Preliminary Design Review (PDR)}
     \end{column}
     	\begin{column}[T]{5cm} 
         	\begin{center}
            		\includegraphics[width=0.95\textwidth]{Phase}
        		\end{center}
     \end{column}
 \end{columns}
}

%\offslide{Erg\"anzen des generischen Phasenmodells}%{Durchf\"uhrung in der \"Ubung}

\frame{\frametitle{Subsystem Design}
\framesubtitle{``Build to Specifications'': Drawings, Electrical Schemes,...}
\begin{columns}[t] 
     \begin{column}[T]{6cm} 
     \textbf{Questions:}
     	\begin{itemize}
		\item What are the subsystem requirements?
		%\item Make or Buy?
		\item Which deliverables (e.g. documentation) are requested?
		%\item How can the module be realised efficiently?
		\item What are critical characteristics of the module and its parts?
		\item Can service proven modules be used or adapted?
     	\end{itemize}
	\textbf{Typical Milestone: Critical Design Review (CDR)}
     \end{column}
     	\begin{column}[T]{5cm} 
         	\begin{center}
            		\includegraphics[width=0.95\textwidth]{Phase}
        		\end{center}
     \end{column}
 \end{columns}
}

%\offslide{Erg\"anzen des generischen Phasenmodells}%{Durchf\"uhrung in der \"Ubung}

