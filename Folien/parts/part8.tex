% !TEX root = ../SFV15001_MdQM_Fertigungsmesstechnik_Rev04.tex

\label{Sec:Zuverlaessig}

\frame{\frametitle{Motivation}
\framesubtitle{Ein Gro{\ss}teil der Kosten technischer Systeme entsteht h\"aufig erst im Betrieb.}
\begin{itemize}
\item Kunde:
\begin{itemize}
	\item Verf\"ugbarkeit des Systems
	\item Ersatzteilbedarf
	\item Wartungsplanung
\end{itemize}
\item Lieferant:
\begin{itemize}
	\item Erwartungswert Gew\"ahrleistungskosten
	\item Absch\"atzung Obsolenszenz 
	\item Erf\"ullung vertraglicher Forderungen
\end{itemize}
\end{itemize}
}

\frame{\frametitle{Zuverl\"assigkeit}
\framesubtitle{}
\begin{itemize}
\item Zuverl\"assigkeit ist die F\"ahigkeit eines Objekts, unter gegebenen Bedingungen f\"ur eine bestimmte Zeit eine geforderte Funktion auszu\"uben.
\begin{itemize}
	\item Anwendbar f\"ur Systeme ohne Reparatur oder Wartung 
\end{itemize}
\item Verf\"ugbarkeit ist die Wahrscheinlichkeit, dass sich ein System zu einem beliebigen Zeitpunkt in einem Zustand befindet, in dem es seine Funktion erf\"ullen kann.
\begin{itemize}
		\item Auch anwendbar f\"ur reparierte und gewartete Systeme
		\end{itemize}
\end{itemize}
}


\frame{\frametitle{Fehler, Fehlertypen und Fehlerrate}
\framesubtitle{}
\begin{itemize}
\item Fehler:
\begin{itemize}
	\item System kann seine geforderten Eigenschaften nicht erf\"ullen
	\item Trotz ordnungsgem\"a{\ss}em Betrieb
\end{itemize}
\item Fehlertypen:
\begin{itemize}
	\item Fr\"uhe Fehler: nahezu ausschlie{\ss}lich am Anfang des Lebenszyklus auftretend
	\item Zuf\"allige Fehler: Auftreten unerwartet und unabh\"angig von Alter oder Nutzung einer Komponente
	\item Verschlei{\ss}fehler: zunehmende Fehler mit steigendem Alter der Komponente
\end{itemize}
\item Fehlerrate $\lambda$:
\begin{itemize}
	\item Fehler pro Zeiteinheit
	\item Basis: tats\"achliche Betriebsdauer
	\item Einheit f\"ur Elektrokomponenten h\"aufig FIT $\left(1 \mathrm{FIT} = 10^{-9} \mathrm{h}^{-1} \right)$ 
\end{itemize}
\end{itemize}
}

 \frame{\frametitle{Praktisches Verhalten der Ausfallrate}
\framesubtitle{}
\begin{center}
			 \pgfplotsset{width=12cm, height = 7cm}
            		\begin{tikzpicture}
			\only<3->{\fill[fill = yellow!50!black, opacity = 0.3] (0,0) rectangle (1.4,5.4);
			\node[rotate = 90] at (0.7,3.5) {\Large Fr\"uhausf\"alle};}
			\only<4->{\fill[fill = green!70!black, opacity = 0.3] (1.4,0) rectangle (6.8,5.4);
			\node[rotate = 90] at (4,3.5) {\Large Nutzbare Zeit};}
			\only<5->{\fill[fill = red!70!black, opacity = 0.3] (6.8,0) rectangle (10.4,5.4);
			\node[rotate = 90] at (9,3.5) {\Large Unzuverl\"assigkeit};}
			\only<6-6>{\draw[thick, draw = red!90!black] (6.8,0.4) circle [radius=0.2];
			\node[align = center] at (6,1){Wie l\"asst sich der \\ \"Ubergang erkennen?};}
			 
			\begin{scope}
				[spy using outlines={circle, black,
				magnification=2, size=3.5cm, connect spies}]
			 
				\begin{axis}[stack plots=y, no marks,
				xmax=14, xmin = 0,
				 ymin = 0, ymax = 1, %grid=both,
				 xtick = \empty, ytick = \empty,
				%axis lines=middle, 
				%ylabel = {$p$},
				%xlabel = {$t/\mathrm{a}$},
				%legend entries={Zufallssch\"aden, Zeitabh\"angigkeit},
				legend style = {at={(0.5,0.8)},
      				  anchor=north}]
  
  				\only<1->{\addplot[fill = red!70!black, opacity = 0.5, thick, draw = red!70!black] table[x=t,y=h, %row sep=crcr
				] {Exponential.dat} \closedcycle; }
				\only<1-2>{\addlegendentry{Zufallssch\"aden};}
  
  				\only<2->{\addplot[fill = blue!70!black, opacity = 0.5, thick, draw = blue!70!black] table[x=t,y=h, %row sep=crcr
				] {Weibull.dat} 
				coordinate [pos=.6] (A)
   				coordinate [pos=.7]  (B)
				\closedcycle ; }
				
				\only<1-2>{\addlegendentry{Zeitabh\"angigkeit};}
				\only<7->{\fill[thick, draw = black, fill = black] (A) -| (B)
				 node [align = center, pos=2.5,anchor=west]
    				 {Wie entwickelt \\ sich die \\Zuverl\"assigkeit?};}

	\end{axis}
	\only<7->{ \spy on (6.8,0.5) in node (zoom) [left] at (8.5,3.5);}
	\end{scope}
	
	\end{tikzpicture}
\end{center}
}


\frame{\frametitle{Mathematische Beschreibung des Ausfallverhaltens}
\framesubtitle{Durch mathematische Beschreibung lassen sich zu verschiedenen Zwecken Daten gewinnen.}
\begin{columns}[t] 
     \begin{column}[T]{6cm} 
     	\begin{itemize}
		\only<1-3>{\item Zuverl\"assigkeitsfunktion $R(t)$
		\begin{itemize}
		\item Anteil funktionsf\"ahiger Komponenten zum Zeitpunkt $t$
		\end{itemize}
		\item Hazard rate $h$
		\begin{itemize}
		\item Anteil ausfallender Komponenten zum Zeitpunkt $t$
		\end{itemize}}
     		\only<2-2>{\item Konstante Fehlerrate:
		\begin{itemize}
			\item Exponentialverteilung
			\begin{eqnarray*}
			R(t) &=& e^{-\left(\lambda t \right)}\\
			h &=& \lambda
			\end{eqnarray*}
		\end{itemize}}
		\only<3-3>{\item Zeitabh\"angige Fehlerraten:
		\begin{itemize}
			\item (Misch-)Weibull-Verteilung
			\begin{eqnarray*}
			R(t) &=& e^{-\left(\lambda t \right)^{\beta}}\\
			h&=& \beta \lambda \left(\lambda t \right)^{\beta-1}
			\end{eqnarray*}
			\end{itemize}}
     	\end{itemize}
     \end{column}
     	\begin{column}[T]{5cm} 
         	\begin{center}
			% Pgf plots
            		\begin{tikzpicture}
				\begin{axis}[no marks,
				xmax=14, xmin = 0,
				 ymin = 0, ymax = 1, grid=both,
				%axis lines=middle, 
				%ylabel = {$p$},
				xlabel = {$t/\mathrm{a}$},
				legend entries={\small Survival rate $\bar{F}$, \small Hazard rate $h$, PDF $f$},
				legend style = {at={(0.5,1.28)},
      				  anchor=north}]
				
				\only<1-2>{\addplot[thick, draw = blue!70!black] table[x=t,y=Fq, %row sep=crcr
				] {Exponential.dat}; 
  
  				\addplot[thick, draw = red!70!black] table[x=t,y=h, %row sep=crcr
				] {Exponential.dat}; }

 				 \only<3-3>{\addplot[thick, draw = blue!70!black] table[x=t,y=Fq, %row sep=crcr
				 ] {Weibull.dat}; 
  
  				\addplot[thick, draw = red!70!black] table[x=t,y=h, %row sep=crcr
				] {Weibull.dat}; }
  
  %\addplot[thick, draw = green!70!black] table[x=t,y=f, row sep=crcr] {Weibull.dat}; 
  
\end{axis}
\end{tikzpicture}

        		\end{center}
     \end{column}
 \end{columns}
}

\frame{\frametitle{Zuverl\"assigkeitswerte bei bekanntem Ausfallverhalten}
\framesubtitle{Definition von Vergleichsgr\"o{\ss}en}
\begin{itemize}
\item Mittlere Lebensdauer $T$:
\begin{itemize}
	\item Erwartungswert der Zeit bis zum Ausfall
\end{itemize}
\item MTBF (Mean Time Between Failures) (auch MDBF)
\begin{itemize}
	\item Erwartungswert der Zeit zwischen zwei Ausf\"allen
\end{itemize}
\end{itemize}
}

%\offslide{Herleitung mittlere Lebensdauer und MTBF}

%\offslide{\small Fehlerrate und MTBF f\"ur Systeme aus mehreren Komponenten}

\frame{\frametitle{Beispielsystem f\"ur Zuverl\"assigkeitsanalyse}
\framesubtitle{}
		\begin{itemize}
		\item Elektronische F\"uhrerbremsventilanlage
		\begin{itemize}
		\item Bremssteuerger\"at (BSG) $R_{B}(t), \lambda_{B}$
		\item F\"uhrerbremsventil (FbrV) $R_{F}(t), \lambda_{F}$
		\item Elektropneumatische R\"uckfallebene (BU) $R_{R}(t), \lambda_{R}$
		\end{itemize}
		\item Aufgabe: Umwandlung Sollwert (SW) in HL-Druck (HL)
		\end{itemize}
         	\begin{center}
            		\tikz 	{
			\node (sw) [rectangle, fill = gray!30, minimum width = 1.2cm] at (-1,0) {SW};
			\node[minimum width = 0cm] (a) at (0.5,1) {};
			\node (bsg) [rectangle, fill = gray!30, minimum width = 1.2cm] at (2,1) {BSG}; 
			\node (fbrv) [rectangle, fill = gray!30, minimum width = 1.2cm] at (4,1) {FbrV}; 
			\node (b) at (5.5,1) {};
			\node (bu) [rectangle, fill = gray!30, minimum width = 1.2cm] at (3,-1) {BU}; 
			\node (hl) [rectangle, fill = gray!30, minimum width = 1.2cm] at (7,0) {HL};
 			 %\graph { (sw) -> (bsg) -> (fbrv) -> (hl) 
			%(sw) -> (bu) -> (hl)};
			\draw[thick] (sw) -| (0.5,1);
			\draw[thick, ->] (0.5,1) -> (bsg);
			\draw[thick, ->] (bsg) -> (fbrv);
			\draw[thick]  (fbrv) -| (5.5,0);
			\draw[thick, ->] (5.5,0) -> (hl);
			\draw[thick] (sw) -| (0.5,-1);
			\draw[thick, ->] (0.5,-1) -> (bu);
			\draw[thick]  (bu) -| (5.5,0);
			}
        		\end{center}
}

%\offslide{Analytische Bestimmung der Zuverl\"assigkeit}

\subsection{Verfahren mittels Weibull-Verteilung und MCMC-Simulation}
\frame{\frametitle{Betriebszust\"ande der elektronischen F\"uhrerbremsventilanlage}
\framesubtitle{Die elektronische F\"uhrerbremsventilanlage kann verschiedene Betriebszust\"ande annehmen.}
\begin{columns}[t] 
     \begin{column}[T]{6cm} 
     	\begin{itemize}
     		\item $S_{0}$: Ausgeschaltet
		\begin{itemize}
		\item Ca. 3600 Betriebsstunden/Jahr
		\end{itemize}
		\item $S_{1}$: Normaler Betrieb (BSG und FbrV)
		\begin{itemize}
		\item  \"Ubergang zur Nutzung R\"uckfallebene: Wahrscheinlichkeit $p_{1}$
		\end{itemize}
		\item $S_{2}$: Betrieb mit R\"uckfallebene
		\begin{itemize}
		\item \"Ubergang zur St\"orung: Wahrscheinlichkeit $p_{2}$
		\end{itemize}
		\item $S_{3}$: St\"orung
		\begin{itemize}
		\item Fahrzeugst\"orung ``Stop on line''
		\end{itemize}
     	\end{itemize}
     \end{column}
     	\begin{column}[T]{6cm} 
         	\begin{center}
            		\only<1>{\begin{tikzpicture}[->,>=stealth',shorten >=1pt,auto,node distance=2.8cm,
                    semithick]
 		 \tikzstyle{every state}=[fill=gray!30,draw=none]

 		 \node[state] (s0)    at (0,0)                {$S_{0}$};
 		 \node[state]         (s1)  at (-2,-2) {$S_{1}$};
		  \node[state]         (s2) at (2,-2) {$S_{2}$};
		  \node[state]         (s3) at (0,-4)     {$S_{3}$};

 		 \path (s0) edge [bend left]  node {$p_{01}$} (s1)
		 edge [loop right] node{$p_{00}$} (s0)
           	(s1) edge [bend right]             node {$p_{12}$} (s2)
		edge [bend left] node{$p_{10}$} (s0)
		 edge [loop below] node{$p_{11}$} (s1)
		%(s2) edge              node {$p_{3}$} (s0)
        		(s2) edge              node {$p_{23}$} (s3)
		edge[loop above] node{$p_{22}$} (s1)
		edge[bend right] node{$p_{20}$} (s0)
		(s3) edge[loop left] node{$p_{33}$} (s3);
		\end{tikzpicture}}
		\only<2>{
		\begin{equation*}
		P = 
		\left( \begin{matrix}
 		p_{00} & p_{01} & 0 &0 \\
		p_{10} & p_{11} & p_{12} & 0 \\
		p_{20} & 0 & p_{22} & p_{23} \\
		0 & 0 & 0 & p_{33} \\
		\end{matrix} \right)
		\end{equation*}
		}
        		\end{center}
     \end{column}
 \end{columns}
}

\frame{\frametitle{Ausfallzeitpunkte Bremssteuerger\"at}
\framesubtitle{}
\begin{columns}[t] 
     \begin{column}[T]{6cm} 
     	\begin{itemize}
     		\item Annahme:
		\begin{itemize}
		\item Elektronikkomponente
		\item Weibullverteilt:
		\begin{itemize}
		\item $\beta = 5$
		\item $\lambda = 12$
		\end{itemize}
		\item Flottengr\"o{\ss}e $N = 100$
		\end{itemize}
		\item Ergebnis:
		\begin{itemize}
		\item $\hat{\beta} = 4.5$
		\item $\hat{\lambda} = 12.2$
		\end{itemize}
     	\end{itemize}
     \end{column}
     	\begin{column}[T]{6cm} 
	\vspace{-.8cm}
         	\begin{center}
            		\begin{tabular}{|c|c|}
		\hline
			Nr. & TTF \\ \hline
    			1 & 4.4 a \\ \hline  
    			2 & 5.6 a  \\ \hline
    			3 & 5.7 a \\ \hline  
   			4 & 5.7 a \\ \hline 
    			5 & 6.3 a \\ \hline 
    			6 & 6.3 a \\ \hline 
    			7 & 6.7 a \\ \hline 
    			8 & 6.9 a \\ \hline 
    			9 & 7.1 a \\ \hline 
    			10 & 7.6 a \\ \hline 
    			11 & 7.7 a \\ \hline
    			12 & 7.8 a \\ \hline 
    			13 & 7.8 a \\ \hline 
    			14 & 7.9 a \\ \hline 
    			16 & 8.0 a \\ \hline 
			\end{tabular}
        		\end{center}
     \end{column}
 \end{columns}
}

\frame{\frametitle{Ausfallzeitpunkte F\"uhrerbremsventilanlage}
\framesubtitle{}
\begin{columns}[t] 
     \begin{column}[T]{6cm} 
     	\begin{itemize}
     		\item Annahme:
		\begin{itemize}
		\item Elektropneumatik-Komponente
		\item Exponentialverteilt:
		\begin{itemize}
		\item $\lambda = 50$
		\end{itemize}
		\item Flottengr\"o{\ss}e $N = 100$
		\end{itemize}
		\item Ergebnis:
		\begin{itemize}
		\item $\hat{\lambda} = 55.6$
		\end{itemize}
     	\end{itemize}
     \end{column}
     	\begin{column}[T]{6cm} 
	%\vspace{-.8cm}
         	\begin{center}
            		\begin{tabular}{|c|c|}
			\hline
			Nr. & TTF \\ \hline
    			1 & 0.24 a \\ \hline  
    			2 & 0.30 a  \\ \hline
    			3 & 0.35 a \\ \hline  
   			4 & 0.53 a \\ \hline 
    			5 & 2.71 a \\ \hline 
    			6 & 3.08 a \\ \hline 
    			7 & 3.44 a \\ \hline 
    			8 & 3.46 a \\ \hline 
    			9 & 5.05 a \\ \hline 
    			10 & 7.44 a \\ \hline 
    			\end{tabular}
        		\end{center}
     \end{column}
 \end{columns}
}

\frame{\frametitle{Identifikation Verschleissdaten Bremssteuerger\"at}
\framesubtitle{}
\begin{center}
 \pgfplotsset{width=12cm, height = 7cm}
\begin{tikzpicture}
\begin{axis}[no marks,
xmax=15, xmin = 0,
 ymin = 0, ymax = 1, grid=both,
%axis lines=middle, 
%ylabel = {$p$},
xlabel = {$t/\mathrm{a}$},
%legend entries={$\bar{F}$, $h$, $\bar{F}_{\text{sim}}$, $\hat{F}$, $\hat{h}$}, 
legend style = {at={(0.05,0.5)},
        anchor=west}]

  \only<1,5>{\addplot[draw = blue!70!black] table[x=t,y=Fq, %row sep=crcr
  ] {BSG.dat}; \addlegendentry{$\bar{F}$}

 \addplot[draw = orange!70!black] table[x=t,y=h, %row sep=crcr
 ] {BSG.dat}; 
 \addlegendentry{$h$}
 }

\only<2-4>{ \addplot[thick, draw = red!70!black] table[x=t,y=Fqest, %row sep=crcr
] {BSG2.dat}; \addlegendentry{$\bar{F}_{\text{sim}}$}}
    
  
\only<3-5>{  \addplot[thick, draw = green!70!black] table[x=t,y=Fqsim,% row sep=crcr
] {BSG.dat}; \addlegendentry{$\hat{F}$}}
  
  
 \only<4-5>{ \addplot[thick, draw = gray] table[x=t,y=hest, %row sep=crcr
 ] {BSG.dat}
 coordinate [pos=0.53] (A)
   coordinate [pos=0.6]  (B)
 ; \addlegendentry{$\hat{h}$}
 \fill[fill = gray!70, opacity = 0.8] (A) -| (B)
 node [pos=0.75,anchor=west]
     {$\triangle h = 0.048$};}
 \end{axis}
\end{tikzpicture}
%%
\end{center}
}



\frame{\frametitle{Identifikation Verschleissdaten F\"uhrerbremsventilanlage}
\framesubtitle{}
\begin{center}
 \pgfplotsset{width=12cm, height = 7cm}
\begin{tikzpicture}
\begin{axis}[no marks,
xmax=15, xmin = 0,
 ymin = 0, ymax = 1, grid=both,
%axis lines=middle, 
%ylabel = {$p$},
xlabel = {$t/\mathrm{a}$},
%legend entries={$\bar{F}$, $h$, $\bar{F}_{\text{sim}}$, $\hat{F}$, $\hat{h}$}, 
legend style = {at={(0.05,0.5)},
        anchor=west}]

  \only<1,5>{\addplot[draw = blue!70!black] table[x=t,y=Fq, %row sep=crcr
  ] {FbrVA.dat}; \addlegendentry{$\bar{F}$}

 \addplot[draw = orange!70!black] table[x=t,y=h, %row sep=crcr
 ] {FbrVA.dat}; 
 \addlegendentry{$h$}
 }

\only<2-4>{ \addplot[thick, draw = red!70!black] table[x=t,y=Fqest, %row sep=crcr
] {FbrVA2.dat}; \addlegendentry{$\bar{F}_{\text{sim}}$}}
    
  
\only<3-5>{  \addplot[thick, draw = green!70!black] table[x=t,y=Fqsim, %row sep=crcr
] {FbrVA.dat}; \addlegendentry{$\hat{F}$}}
  
  
 \only<4-5>{ \addplot[thick, draw = gray] table[x=t,y=hest, %row sep=crcr
 ] {FbrVA.dat}
 coordinate [pos=0.53] (A)
   coordinate [pos=0.6]  (B)
 ; \addlegendentry{$\hat{h}$}}
 \end{axis}
\end{tikzpicture}
%%
\end{center}
}

\frame{\frametitle{Simulationsergebnisse}
\framesubtitle{Die wiederholte Simulation der Markov-Kette eines Betriebstages liefert die Wahrscheinlichkeitsdichte der Betriebszust\"ande.}
\begin{columns}[t] 
     \begin{column}[T]{6cm} 
     	\begin{itemize}
     		\item Markov-Chain-Monte-Carlo-Simulation
		\begin{itemize}
		\item Wiederholte Ausf\"uhrung einer Markov Kette
		\item $N = 10^6$ Betriebstage
		\end{itemize}
		\item Zust\"ande:
		\begin{itemize}
		\item $S_{0}$: 39{,}8\%
		\item $S_{1}$: 60{,}2\%
		\item $S_{2}$: 0{,}014\%
		\item $S_{3}$: 0\%
		\end{itemize}
     	\end{itemize}
     \end{column}
     	\begin{column}[T]{6cm} 
         	\begin{center}
            		\begin{tikzpicture}[->,>=stealth',shorten >=1pt,auto,node distance=2.8cm,
                    semithick]
 		 \tikzstyle{every state}=[fill=gray!30,draw=none]

 		 \node[state] (s0)    at (0,0)                {$S_{0}$};
 		 \node[state]         (s1)  at (-2,-2) {$S_{1}$};
		  \node[state]         (s2) at (2,-2) {$S_{2}$};
		  \node[state]         (s3) at (0,-4)     {$S_{3}$};
		  
		  \node [red!70!black,above] at (s0.north) {39{,}8\%};
		  \node [red!70!black,below] at (s1.south) {60{,}2\%};
		  \node [red!70!black,below] at (s2.south) {0{,}014\%};
		  \node [red!70!black,below] at (s3.south) {0\%}; 

 		 \path (s0) edge [bend left]  (s1)
		 edge [loop right]  (s0)
           	(s1) edge [bend right]  (s2)
		edge [bend left]  (s0)
		 edge [loop above]  (s1)
        		(s2) edge (s3)
		edge[loop above]  (s1)
		edge[bend right] (s0)
		(s3) edge[loop left]  (s3);
		\end{tikzpicture}
        		\end{center}
     \end{column}
 \end{columns}
}

\frame{\frametitle{Nutzung der Zuverl\"assigkeitsdaten}
\framesubtitle{Erkenntnisse \"uber das Lebensdauerverhalten lassen sich zur B\"undelung von Aktivit\"aten oder zur Fristverl\"angerung nutzen.}
\begin{columns}[t] 
     \begin{column}[T]{6cm} 
     	\begin{itemize}
     		\item B\"undelung von Aktivit\"aten
		\begin{itemize}
		\item Vorhersage \"uber Ausfallwahrscheinlichkeit im n\"achsten Intervall
		\item Reduzierung von reaktiven T\"atigkeiten
		\end{itemize}
		\item Verl\"angerung von Fristen
		\begin{itemize}
		\item Eskalation \"uber Instandhaltungsentwicklung
		\item Wirtschaftlichkeitspr\"ufung
		\item \"Anderung Wartungsanweisung
		\end{itemize}
		\item Vorhersage Ersatzteilbedarf
		\item Unterst\"utzung Fehlerdiagnose
     	\end{itemize}
     \end{column}
     	\begin{column}[T]{6cm} 
         	\begin{center}
		\tikzstyle{decision} = [diamond, fill=blue!20, 
	    text width=6em, text badly centered, inner sep=0pt, yshift = 1.5cm]
                    \tikzstyle{block} = [rectangle, fill=blue!20, 
                        text width=8em, text centered, rounded corners, minimum height=4em, yshift = 1.5cm]
                    \tikzstyle{line} = [thick, draw, -latex']
                    \tikzstyle{cloud} = [fill=red!20,  text width=8em, node distance=5cm, text centered,
                        minimum height=2em]
                     \vspace{-1cm}
                     \resizebox{!}{7cm}{   
                    \begin{tikzpicture}[node distance =  4cm, auto]%[node distance = 2cm,auto]
                        % Place nodes
                        \node [block] (init) {Komponente w\"ahlen};
                        \node [cloud, right of=init] (expert) {Expertenwissen};
                        \node [decision, below of=init] (decide) {Optimierung};
                        \node [block, below of = decide] (daten1) {Lebensdauerkurve identifizieren};
                        \node [block, below of = daten1] (predict1) {Vorhersage Zuverl\"assigkeit};
                        \node [block, below of = predict1] (decide2) {Entscheidung Eingriff};
                        
                       \node [block, right of = daten1] (daten2) {Lebensdauerkurve identifizieren};
                       \node [block, below of = daten2] (predict2) {Vorhersage Risiko};
                       \node [block, below of = predict2] (assess) {Bewertung Risiko};
                       \node [block, below of = assess] (do2) {Umsetzung};
                       
                        % Draw edges
                        \path [line] (init) -- (decide);
                        \path [line,dashed] (expert) -- (init);
                        \path [line] (decide) -| node [above, near start] {Verl\"angerung} (daten2);
                        \path [line] (decide) -- node {B\"undelung} (daten1);
                         \path [line] (daten1) -- (predict1);
                          \path [line] (predict1) -- (decide2);
                          
                           \path [line] (daten2) -- (predict2);
                            \path [line] (predict2) -- (assess);
                             \path [line] (assess) -- (do2);
                    \end{tikzpicture}}
        		\end{center}
     \end{column}
 \end{columns}
}

%\subsection{Importance Sampling}
%\frame{\frametitle{Importance sampling}
%\framesubtitle{Importance sampling (IS) increases the probability of ``desired'' outcomes in Monte-Carlo-Simulations.}
%\begin{itemize}
%\item Typical IS approaches:
%\begin{itemize}
%		\item Stratification: select only relevant strata of the sampling range
%		\item Scaling: Scale random variable
%		\item Translation: Move random variable to more relevant part of sampling space
%		\item Change of random variable: Replace random variable by one more likely to produce outcomes in the relevant range
%		\item Adaptive approaches
%		\end{itemize}
%\item Effect: higher number of samples in region of interest
%\item Correction factor: Likelihood ratio $L(y) = \frac{f(y)}{\tilde{f}(y)}$
%\end{itemize}
%}
%
%\frame{\frametitle{Example: Application of IS to braking curves}
%\framesubtitle{Change identified random variables, in the case at hand $\mu_{B}$}
%\begin{center}
%\includegraphics[width = .9\textwidth]{hist}
%\end{center}
%}
%
%\frame{\frametitle{Example: Application of IS to braking curves}
%\framesubtitle{Step 3: Analyse for rare events, here braking distances in excess of 1100 m. $N = 5 \cdot 10^7$}
%\only<1>{\begin{center}
%\includegraphics[width = .9\textwidth]{rare}
%\end{center}}
%\only<2>{
%%\hspace{-.62cm}
%\begin{tabular}{|c|c|c|c|c|c|c|}
%\hline
%$s$ & $n_{\mathcal{U}}$ & $p_{\mathcal{U}}$ & $n_{IS, 1}$ & $p_{IS, 1}$ & $n_{IS, 2}$ & $p_{IS, 2}$ \\ \hline
%1000 & 24400 & 4.89$\cdot 10^{-3}$ & 2.27$\cdot 10^6$ & 1.14$\cdot 10^{-2}$ & 3.11$\cdot 10^5$ & 1.77$\cdot 10^{-3}$ \\ \hline
%1050 & 2 & 4$\cdot 10^{-7}$ & 6.66$\cdot 10^4$ & 2.02$\cdot 10^{-4}$ & 1.48$\cdot 10^4$ & 1.59$\cdot 10^{-4}$ \\ \hline
%1100 & 0 & 0 & 115 & 2.04$\cdot 10^{-7}$ & 419 & 7.50$\cdot 10^{-6}$ \\ \hline
%1150 & 0 & 0 & 0 & 0 & 15 & 3.88$\cdot 10^{-7}$ \\ \hline
%1160 & 0 & 0 & 0 & 0 & 7 & 2.05$\cdot 10^{-7}$ \\ \hline
%1170 & 0 & 0 & 0 & 0 & 5 & 1.46$\cdot 10^{-7}$ \\ \hline
%1180 & 0 & 0 & 0 & 0 & 4 & 1.16$\cdot 10^{-7}$ \\ \hline
%1190 & 0 & 0 & 0 & 0 & 1 & 2.90$\cdot 10^{-8}$ \\ \hline
%\end{tabular}
%}
%}


\subsection{Weibull-Verteilung und Analyse}
\frame{\frametitle{Formulierung der Weibull-Verteilung}
\framesubtitle{}
\begin{columns}[t] 
     \begin{column}[T]{6cm} 
     \textbf{Fr\"uhausf\"alle}
     	\begin{itemize}
     		\item $R(t) = e^{- \left( \frac{t}{\eta}\right)^\beta}$
		\item $\lambda(t) = \frac{\beta}{\eta} \left( \frac{t}{\eta} \right)^{\beta-1}$
		\item $T = \eta \Gamma\left( 1 + \frac{1}{\beta} \right) $
		\item $\beta > 1$, $t > 0$
     	\end{itemize}
     \end{column}
     	\begin{column}[T]{6cm} 
	\textbf{Alterung}
         \begin{itemize}
     		\item $R(t) = e^{- \left( \frac{t-t_{2}}{\eta}\right)^\beta}$
		\item $\lambda(t) = \frac{\beta}{\eta} \left( \frac{t - t_{2}}{\eta} \right)^{\beta-1}$
		\item $T = \eta \Gamma\left( 1 + \frac{1}{\beta} \right)  $
		\item $\beta > 1$, $t > t_{2}$
     	\end{itemize}
     \end{column}
 \end{columns}
}

%\begin{frame}[fragile]
%\frametitle{Weibull-Plot}
%\framesubtitle{}
%\begin{center}
%\begin{tikzpicture}
%\begin{axis}[scale only axis,
%        only marks,
%         xmin=10,xmax = 1000,
%         ymin=0.001, ymax=0.999,
%         grid=major,
%         xtick={10, 20, 50, 100, 200, 500, 1000},
%         ytick={0.001,0.05,0.4,0.99},
%         yticklabels={0.1,5,40,99},
%        xmode=log,
%        log ticks with fixed point,
%y coord trafo/.code=\pgfmathparse{ln(-ln(1-#1)))},
%y coord inv trafo/.code=\pgfmathparse{exp(-exp(-#1-1))}
%         ]
%%
%\addplot [domain=10:1000, sharp plot] {1-exp(-(x/820)^1.96)};
%\addplot [mark=o, mark size=1.5, black] table {
%96  0.02
%147 0.04
%202 0.0604
%232 0.0813
%277 0.1022
%278 0.1231
%325 0.1439
%346 0.1648
%354 0.1857
%367 0.2066
%414 0.228
%440 0.2501
%471 0.2721
%525 0.2942
%541 0.3163
%542 0.3383
%574 0.3604
%608 0.3824
%631 0.4045
%643 0.4265
%661 0.4504
%669 0.4743
%706 0.4982
%720 0.5221
%722 0.546
%738 0.5712
%747 0.598
%770 0.6248
%802 0.6537
%841 0.6825
%850 0.7114
%884 0.7403
%893 0.7691
%901 0.798
%};
%\end{axis} 
%\end{tikzpicture}
%
%\end{center}
%\end{frame}
