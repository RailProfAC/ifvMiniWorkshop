% !TEX root = ../SFV15001_MdQM_Fertigungsmesstechnik_Rev05.tex
 \subsection{Software-Qualität}
\label{Sec:Software}
\subsubsection{Einf\"uhrung}
\frame{\subsectionpage}
\frame{\frametitle{Motivation}
\framesubtitle{Zunehmende Allgegenw\"artigkeit von Computersystemen in Alltag und Industrie erh\"oht die Bedeutung von qualitativ hochwertiger Software.}
\begin{itemize}
\item Trends:
\begin{itemize}
	\item Industrie 4.0 / Internet of Things: 
	\begin{itemize}
		\item Selbststeuerung
		\item M2M-Kommunikation
		\item Embedded Systems
	\end{itemize}
	\item Zunehmende Wertsch\"opfung durch Software:
	\begin{itemize}
		\item Siemens einer der gr\"o{\ss}ten Software-Hersteller der Welt (nach Code-Menge)
		\item H\"aufig wird Software-Anteil im Maschinen- und Anlagenbau vorausgesetzt
		\item Aufstieg zur h\"aufigsten Fehlerursache
	\end{itemize}
	\item Code-Gr\"o{\ss}e je SW-Produkt steigt exponentiell	
\end{itemize}
	\item Software-Engineering junge Ingenieursdisziplin
	\begin{itemize}
		\item Detail-Spezifikationen n\"otig
		\item Experimentierfreudige Entwickler
		\item Wachsende Teamgr\"o{\ss}en
	\end{itemize}
\end{itemize}
}

\frame{\frametitle{Defintion und Aspekte von Software-Qualit\"at}
\framesubtitle{}
\begin{definition}[Software-Qualit\"at]
Software-Qualit\"at ist die Gesamtheit der Merkmale und Merkmalswerte eines Software-Produkts, die sich auf die Eignung beziehen, festgelegt Erfordernisse zu erf\"ullen.
\end{definition}
\begin{itemize}
\item Funktionalit\"at \textit{Functionality, Capability}
\item Laufzeit \textit{Performance}
\item Zuverl\"assigkeit \textit{Reliability}
\item Benutzbarkeit \textit{Usability}
\item Wartbarkeit \textit{Maintainability}
\item Transparenz \textit{Transparency}
\item \"Ubertragbarkeit \textit{Portability}
\item Testbarkeit \textit{Testability}
\end{itemize}}

\frame{\frametitle{Gr\"unde f\"ur mangelnde SW-Qualit\"at}
\framesubtitle{}
\begin{itemize}
\item H\"andisch erstellter Code: Fehlerdichte $>0$
\item Komplexit\"atswachstum exponentiell
\item Spezialisierung der SW-Ingenieure in junger Disziplin
\begin{itemize}
	\item Vergleichbar mit fr\"uher Phase des Maschinenbaus
\end{itemize}
\item Dokumentation des Codes im Code oder im Mitarbeiter
\begin{itemize}
	\item Fluktuation f\"uhrt zu Problemen
\end{itemize}
\item Inkrementelles Wachstum des SW-Produkts
\begin{itemize}
	\item Kompromissl\"osungen f\"ur Abw\"artskompatibilit\"at n\"otig
\end{itemize}
\item Unzureichende Code-Metriken
\end{itemize}
}

\frame{\frametitle{Ma{\ss}nahmen zur Quali\"atssicherung von Software}
\framesubtitle{}
     	\begin{itemize}
     		\item Produktqualit\"at
		\begin{itemize}
		\item Konstruktive Qualit\"atssicherung
		\only<1>{
		\begin{itemize}
		\item Software-Richtlinien
		\item Typisierung
		\item Vertragsbasierte Programmierung
		\item Fehlertolerante Programmierung
		\item Portabilit\"at
		\item Dokumentation
		\end{itemize}}
		\item Analytische Qualit\"atssicherung
		\only<2>{
		\begin{itemize}
		\item Software-Test: Black-Box, White-Box, Test-Metriken
		\item Statistische Analyse: Software-Metriken, Konformit\"atspr\"ufung, Exploit-Analyse, Anomalienanalyse, Manuelle Software-Pr\"ufung
		\item Software-Verifikation
		\end{itemize}}
		\end{itemize}
		\item Prozessqualit\"at
		\only<3>{
		\begin{itemize}
		\item Software-Infrastruktur
		\begin{itemize}
		\item Konfigurationsmanagement
		\item Build-Automatisierung
		\item Test-Automatisierung
		\item Defektmanagement
		\end{itemize}
		\item Managementprozess
		\begin{itemize}
		\item Vorgehensmodelle
		\item Reifegradmodelle
		\end{itemize}
		\end{itemize}
		}
     	\end{itemize}
}

\frame{\frametitle{Arten von Software-Fehlern}
\framesubtitle{}
\begin{columns}[t] 
     \begin{column}[T]{6cm} 
     	\begin{itemize}
     		\item Lexikalische und syntaktische Fehlerquellen
		\only<1>{\begin{itemize}
		\item Nutzung reservierter Ausdr\"ucke
		\item Fehlerhafte Syntax
		\end{itemize}}
		\item Semantische Fehlerquellen
		\only<2>{\begin{itemize}
		\item Fehlerhafte Nutzung der Bedeutung der Befehle
		\item Auch bei Kommunikation, z.B. \"uber Bus
		\end{itemize}}
		\item Parallelit\"at
		\only<3>{\begin{itemize}
		\item Bei Multitasking scheduling der Prozesse abh\"angig von Last
		\end{itemize}}
		\item Numerische Fehlerquellen:
		\only<4>{\begin{itemize}
		\item \"Uberlauf
		\item Rundung
		\end{itemize}}
		\item Portabilit\"at
		\only<5>{\begin{itemize}
		\item HW-abh\"angiges Verhalten
		\end{itemize}}
		\item Spezifikationsfehler
		\only<6>{\begin{itemize}
		\item Gefordertes Verhalten fehlerhaft
		\end{itemize}}
     	\end{itemize}
     \end{column}
     	\begin{column}[T]{6cm} 
		\scriptsize
         	\lstinputlisting[language = Fortran]{Mercury.txt}
     \end{column}
 \end{columns}
}
\subsubsection{Beispiel konstruktiver Software-QS}
\frame{\subsectionpage}
\frame{\frametitle{Coding conventions}
\framesubtitle{Mittels coding conventions erh\"alt der Quellcode ein standardisiertes Erscheinungsbild.}
\begin{columns}[t] 
     \begin{column}[T]{6cm} 
     	\begin{itemize}
     		\item Definition:
		\begin{itemize}
		\item Darstellun Schachtelung
		\item Einr\"uckung
		\end{itemize}
		\item Einschr\"ankung Sprachkonstrukte
		\item etc.
		\item Beispiel: MISRA-C
		\begin{itemize}
		\item 127 Regeln
		\end{itemize}
		\item Nutzung automatischer Pr\"ufwerkzeuge
     	\end{itemize}
     \end{column}
     	\begin{column}[T]{6cm} 
         	\begin{center}
			\vspace{-1cm}
            		\scriptsize
         	\lstinputlisting[language = C]{CodingConventions.txt}
        		\end{center}
     \end{column}
 \end{columns}
}


\subsubsection{Analytische Qualit\"atssicherung}
\frame{\subsectionpage}

%\offslide{V-Modell in der Software-Entwicklung}

\offslide{Branch-Modell in der Software-Entwicklung}

\frame{\frametitle{Integrationsstrategien}
\framesubtitle{}
\begin{itemize}
\item Big-Bang-Integration: gleichzeitige Zusammenf\"uhrung aller Module
\item Strukturorientiert:
\begin{itemize}
		\item Bottom-Up: beginnend mit Basiskomponenten
		\item Top-Down: beginneng mit oberster SW-Schicht
		\item Outside-In: Integration gleichzeitig von Top und Bottom
		\item Inside-Out: Integration beginnend mit der Zwischenschicht
\end{itemize}
\item Funktionsgetrieben:
\begin{itemize}
		\item Termingetrieben
		\item Risikogetrieben
		\item Testgetrieben
		\item Anwendungsgetrieben
		\end{itemize}
\end{itemize}
}


\frame{\frametitle{Klassifizierung von Tests}
\framesubtitle{}
\begin{itemize}
\item Pr\"ufebene:
\begin{itemize}
		\item Unit-Test
		\item Integrationstest
		\item Systemtest
		\item Abnahmetest
		\end{itemize}
\item Pr\"ufkriterien:
	\begin{itemize}
		\item Funktionale Tests
		\item Operationale Tests
		\item Temporale Tests
		\end{itemize}
\item Pr\"uftechniken:
\begin{itemize}
		\item Black-Box-Test
		\item White-Box-Test
		\item Grey-Box-Test
		\end{itemize}
\end{itemize}
}

\frame{\frametitle{Tests nach Pr\"ufebene}
\framesubtitle{}
\begin{itemize}
\item Unit-Test:
	\begin{itemize}
		\item Kleinste testbare Einheit
		\item z.B. Funktion, Klasse, etc.
		\item Fokus auf Schnittstellen
	\end{itemize}
\item Integrationstest:
	\begin{itemize}
		\item Zusammenwirken der einzelnen Module wird gepr\"uft
		\item Abh\"angig von Integrationsstrategie
	\end{itemize}
\item Systemtest:
\begin{itemize}
		\item Nach vollst\"andiger Integration
		\item Funktionalit\"at nach Pflichtenheft
		\end{itemize}
\item Abnahmetest:
\begin{itemize}
		\item Unter Kundenbeteiligung
		\item Nach Kundenvorgaben
		\item In Einsatzumgebung
		\end{itemize}
\end{itemize}
}

\frame{\frametitle{Tests nach Pr\"ufkriterium}
\framesubtitle{}
\begin{itemize}
\item Funktionale Tests:
	\begin{itemize}
		\item Funktionstest
		\item Trivialtest
		\item Crashtest
		\item Kompatibilit\"atstests
		\item Zufallstests
	\end{itemize}
\item Operationale Tests:
\begin{itemize}
		\item Installationstests
		\item Ergonomietests
		\item Sicherheitstests
		\end{itemize}
\item Temporale Tests:
\begin{itemize}
		\item Komplexit\"atstests
		\item Laufzeittests
		\item Lasttests
		\item Stresstest (\"Uberlasttest)
		\end{itemize}
\end{itemize}
}

\frame[allowframebreaks]{
\frametitle{Black-Box-Testtechniken}
\framesubtitle{Testen nur durch Bewertung des Eingabe- und Ausgabeverhaltens.}
\begin{itemize}
\item \"Aquivalenzklassentest:
\begin{itemize}
	\item Testen aller Zahlenwerte zu aufwendig
	\item \"Aquivalenzklassen bilden Wertebereich ab, f\"ur die gleiches Verhalten erwartet wird
	\item Test mit Zufallszahl aus \"Aquivalenzklasse
	\item Bei mehrdimensionalen Funktionen Kombination der individuellen \"Aquivalenzklassen
	\item Einschlie{\ss}lich oder ausschlie{\ss}lich ung\"ultiger Kombinationen
\end{itemize}
\item Grenzwertbetrachtung:
\begin{itemize}
		\item Nutzung von unterem und oberen Randwert, je innerhalb und au{\ss}erhalb
		\item Komination f\"ur mehrdimensionale Funktionen
		\end{itemize}
		\newpage
\item Zustandsbasierter Softwaretest
\begin{itemize}
		\item F\"ur ged\"achtnisbehaftete Funktionen
		\item Nutzung endlicher Automaten
		\item Pr\"ufung jedes \"Ubergangs 
		\end{itemize}
\item Use-Case-Test: 
\begin{itemize}
		\item \"Uberpr\"ufung einzelner Anwendungsf\"alle
		\end{itemize}
\item Entscheidungstabellenbasierter Test
\begin{itemize}
		\item Pr\"ufung aller Kombination (insbesonderer diskrete) Werte
		\item Reduzierung der Tabelle m\"oglich
		\end{itemize}
\item Paarweises Testen:
\begin{itemize}
		\item Reduzierung auf Paare fehlerhafter Eingaben
		\end{itemize}
\item Diversifizierende Verfahren:
\begin{itemize}
		\item Back-to-Back-Test
		\item Regressionstest
		\end{itemize}
\end{itemize}
}


\frame{\frametitle{ \includegraphics[scale=0.01] {Off}\hspace{1.5mm} Entwicklung Kontrollflussgraph}
\framesubtitle{Entwicklung am Beispiel der Manhattan-Distanz (Taxi-Cab-Norm): $d\left( \left( x_1, y_1 \right), \left( x_2, y_2 \right) \right) = \left| x_1 - x_2 \right| + \left| y_1 - y_2 \right|$}
\begin{center}
         	\lstinputlisting[language = C]{Manhattan.txt}
\end{center}

}


\frame[allowframebreaks]{
\frametitle{White-Box-Testtechniken}
\framesubtitle{Testen unter Kenntnis der Quellodes der Software.}
\begin{itemize}
\item Kontrollflussorientierte Tests
\begin{itemize}
		\item Modellierung des Kontrolflusses \"uber gerichtete Graphen 
		\item Anweisungs\"uberdeckung ($C_{0}$-Test): alle Knoten des Kontrollflussgraphen einmal durchlaufen
		\item Zweig\"uberdeckung ($C_{1}$-Test): alle Kanten des Kontrollflussgraphen einmal durchlaufen
		\item Pfad\"uberdeckung: alle Pfade des Kontrollflussgraphen einmal durchlaufen
		\item Bedingungs\"uberdeckung:
		\begin{itemize}
		\item Einfach: alle atomaren Pr\"adikate m\"ussen beide Wahrheitswerte annehmen
		\item Minimal Mehrfach: alle atomaren und zusammengesetzten Pr\"adikate m\"ussen beide Wahrheitswerte annehmen
		\item Mehrfach: alle Kombinationen der atomaren Pr\"adikate wurden getestet
		\end{itemize}
		\item McCabe-\"Uberdeckung: Reduzierung auf Elementarpfade
		\end{itemize}
\item Datenflussorientierte Tests
\begin{itemize}
		\item Defs-Uses-\"Uberdeckung
		\item Required-$k$-Tupel
		\end{itemize}
\item \"Ublicher Ablauf:
\begin{itemize}
		\item Strukturanalyse: Erstellung Kontrollflussgraph
		\item Testkonstruktion
		\item Testdurchf\"uhrung
		\end{itemize}
\end{itemize}
}

\frame{\frametitle{Testmetriken}
\framesubtitle{}
\begin{itemize}
\item \"Uberdeckungsmetrik:
\begin{itemize}
		\item Verh\"altnis \"uberdeckter Knoten zu Anzahl Knoten
		\item Ab 80\% gilt Software als gut getestet
		\end{itemize}
\end{itemize}
}

\frame{\frametitle{Grenzen des Software-Tests}
\framesubtitle{}
\begin{itemize}
\item Unklare oder fehlende Anforderungen
\begin{itemize}
		\item Insbesondere bei funktionalen Tests
		\end{itemize}
\item Programmkomplexit\"at
\begin{itemize}
		\item Starkes Wachstum der n\"otigen Testf\"alle
		\end{itemize}
\item Fehlende Unterst\"utzung durch das Management
\begin{itemize}
		\item Fehlende Ressourcen oder Zeit
		\end{itemize}
\item Ausbildungs- und Fortbildungsdefizite
\begin{itemize}
		\item Thema Testing vernachl\"assigt
		\end{itemize}
\end{itemize}
}

