% !TEX root = ../16_MdQM_Vorlesung RevC.tex
%\frame{\frametitle{Beispiel komplexe Komponente}
%\label{Sec:Grundlagen}
%\framesubtitle{}
%\begin{center}
%\includegraphics[width=.7\textwidth]{WSP} \source{ „189 038-3 C-AKv-Kupplung“ von Christian Liebscher}
%\end{center}
%}
\subsection{Recap Fertigungsmesstechnik}
\frame{\frametitle{Definition Fertigungsmesstechnik}
\framesubtitle{}
\begin{columns}[t] 
\begin{column}[T]{6cm} 
            		\begin{quotation}
				[...] Oberbegriff f\"ur alle mit Mess- und Pr\"ufaufgaben verbundenen T\"atigkeiten, die beim industriellen Entstehungsprozess eines Produktes zu erbringen sind. \cite[S. 1]{pfeifer10}
			\end{quotation} 
     \end{column}
     \begin{column}[T]{5cm} 
     	\begin{itemize}
	 
     		\item Im Entstehungsprozess
		 
		\item Messen und Pr\"ufen
		 
		\item Aspekte der Definition:
		\begin{itemize}
			\item Geringe Fertigungstiefe
			\item Automatisierung
			\item gestiegene Qualit\"atsforderungen
		\end{itemize}
		 
		\item Von Kontrollinstanz zu Komponente des QM
     	\end{itemize}
     \end{column}
     	 \end{columns}
}

\frame{\frametitle{Fr\"uhe Entwicklung und Rolle der Fertigungsmesstechnik}
\framesubtitle{}
\begin{itemize}
\item Messtechnik seit ca. 4000 v. Chr.: Vergleich mit nat\"urlichen Ma{\ss}en
\item Festlegung Urmeter 1799
\item Beschleunigt durch Austauschbau und Massenfertigung
\item Elektronische Messtechnik ca. seit 1970
\end{itemize}
\begin{columns}[t] 
     \begin{column}[T]{5cm} 
\begin{itemize}
\item Rolle der Messtechnik:
\begin{itemize}
\item 20er Jahre: Sortierung
\item 30er Jahre: Prozess\"uberwachung und Regelung
\item 80er Jahre: Planung zur Fehlervermeidung
\item 90er Jahre: Gesamtheitliches Qualit\"atsdenken
\end{itemize}
\end{itemize}
\end{column}
     	\begin{column}[T]{6cm} 
		\includegraphics[width=.9\textwidth]{Fehlerbehebung}
          \end{column}
 \end{columns}

}


\frame{\frametitle{Aufgaben und Ziele der Fertigungsmesstechnik}
\framesubtitle{}
\begin{itemize}
\item Erfassung von Qualit\"atsmerkmalen an Messobjekt
 
	\begin{itemize}
	\item Werkstoffeigenschaften (z.B. Gef\"uge, H\"arte) 
	\begin{itemize}
		\item Als Eingangspr\"ufung
		\item Nach thermischer Behandlung
		\item auch f\"ur Schwei{\ss}nahtg\"ute
	\end{itemize}
	\item Geometrie (z.B. Ma{\ss}, Form, Lage)  
	\begin{itemize}
		\item Dominierende Pr\"ufung
		\item Gestalt des Werkst\"ucks
		\item Oberfl\"acheneigenschaft
	\end{itemize}
	\item Funktion (z.B. Kraft, Geschwindigkeit)  
	\begin{itemize}
		\item Produkt oder Baugruppe
		\item Von Sichtpr\"ufung bis zur vollautomatisierten Funktionspr\"ufung
	\end{itemize}
	\end{itemize}
\end{itemize}
}

%\frame{\frametitle{Aufgaben im Entwicklungs- und Fertigungsprozess}
%\framesubtitle{}
%\begin{center}
%\includegraphics[width=.9\textwidth]{VModell}
%\end{center}
%}

%\frame{\frametitle{Geometrische Eigenschaften und Toleranzen}
%\framesubtitle{Das exakte Herstellen eines bestimmten Ma{\ss}es ist nicht m\"oglich.}
%\begin{columns}[t] 
%     \begin{column}[T]{6cm} 
%     	\begin{itemize}
%     		\item Abma{\ss}e
%		\item Form
%		\item Lage
%		\item Explizit oder \"uber Allgemeintoleranzen
%		\item Abgeleitet aus technischen Anforderungen
%		\item i.A. sind tolerierte Ma{\ss}e die intensiv \"uberpr\"uften
%     	\end{itemize}
%     \end{column}
%     	\begin{column}[T]{5cm} 
%         	\begin{center}
%            		\includegraphics[width=0.95\textwidth]{Toleranzen}
%        		\end{center}
%     \end{column}
% \end{columns}
%}
%
%\frame{\frametitle{Geometrische Eigenschaften und Toleranzen}
%\framesubtitle{Das exakte Herstellen eines bestimmten Ma{\ss}es ist nicht m\"oglich.}
%\begin{columns}[t] 
%     	\begin{column}[T]{5cm} 
%         	\begin{center}
%            		\includegraphics[width=0.95\textwidth]{Toleranzen}
%        		\end{center}
%     \end{column}
%     \begin{column}[T]{6cm} 
%           \begin{itemize}
%     		\item Abma{\ss}e
%		\begin{itemize}
%			\item[{\color{red}\put(0,2){\vector(-3,0){120}}}] Allgemeintoleranz (Schriftfeld, ugs. Freima{\ss}) 
%			\item[{\color{red}\put(0,2){\vector(-3,0){110}}}] Passungen 
%			\item[{\color{red}\put(0,2){\vector(-5,-2){62}}}] Spezifische Toleranzen
%		\end{itemize}
%		\item[{\color{red}\put(0,2){\vector(-5,-2){60}}}] Form
%		\item[{\color{red}\put(0,2){\vector(-4,2){85}}}] Pr\"ufma{\ss}e
%		\item Lagetoleranzen
%     	\end{itemize}
%     \end{column}
% \end{columns}
%}



%\frame{\frametitle{SI-Einheitensystem}
%\framesubtitle{Internationales Einheitensystem erm\"oglicht Vergleich der Messgr\"o{\ss}en}
%\begin{columns}[t] 
%     \begin{column}[T]{6cm} 
%     	\begin{itemize}
%     		\item International anerkannt
%		\item Beruht auf 7 physikalischen Gr\"o{\ss}en
%		\item Weitere Gr\"o{\ss}en abgeleitet, z.B.
%		\begin{equation*}
%			1 \mathrm{N} = 1 \frac{\mathrm{kg \, m}}{\mathrm{s^{2}}}
%		\end{equation*}
%		\item Koh\"arent: auschlie{\ss}lich Faktor 1
%     	\end{itemize}
%     \end{column}
%     	\begin{column}[T]{5cm} 
%         	\begin{center}
%	\vspace{0.5cm}
%            		\begin{table}
%%\begin{center}
%\scriptsize 
%\begin{tabular}{|c|c|c|}
%\hline
%Gr\"o{\ss}e & SI-Einheit & Einheiten-\\
% & & zeichen \\ \hline
% L\"ange & Meter & m \\ \hline
% Masse & Kilogramm & kg \\ \hline
% Zeit & Sekunde & s\\ \hline
% Temperatur & Kelvin & K \\ \hline
% Stromst\"arke & Ampere & A \\ \hline
% Stoffmenge & Mol & mol \\ \hline
% Lichtst\"arke & Candela & cd \\ \hline
%\end{tabular}
% \caption{Basiseinheiten des SI Systems}
%\end{center}
%\label{Tab:SIUnits}
%\end{table}%
%
%        		\end{center}
%     \end{column}
% \end{columns}
% \normalsize
%}
\frame{\frametitle{Gr\"o{\ss}enordnungen und Definitionen}
\framesubtitle{}
\begin{columns}[t] 
     \begin{column}[T]{.4cm} 
     	    \end{column}
     	\begin{column}[T]{11cm} 
         	\begin{itemize}
     		\item Messgr\"o{\ss}e L\"ange: \"ublicherweise $\left(10^{-9}...10^{2}\right) \mathrm{m}$
		 \item Toleranzen zunehmend enger
     	\end{itemize}
	 
	\begin{definition}[Messgr\"o{\ss}e] Die Messgr\"o{\ss}e ist die physikalische Gr\"o{\ss}e, der die Messung gilt.
	\end{definition}
	 
	\begin{definition}[Messen] Messen ist das Ausf\"uhren von geplanten T\"atigkeiten zum Vergleich der Messgr\"o{\ss}e mit einer Einheit.
	\end{definition}
	 
	\begin{definition}[Pr\"ufen] Pr\"ufen hei{\ss}t feststellen, inwieweit ein Pr\"ufobjekt eine Forderung erf\"ullt.
	\end{definition}
     \end{column}
 \end{columns}
}
% \frame{\frametitle{Pr\"ufplanung}
% \framesubtitle{}
% \begin{columns}[t] 
%      \begin{column}[T]{6cm} 
%      	\begin{itemize}
%      		\item Festlegung von Art und Umfang der Qualit\"atspr\"ufung
% 		\item M\"oglichst fr\"uhzeitig
% 		\item Pr\"ufdaten gem\"a{\ss}:
% 		\begin{itemize}
% 			\item Technik
% 			\item Prozess-FMEA
% 			\item Fertigungsunsicherheiten
% 			\item Vertragsforderungen
% 			\item ... 
% 		\end{itemize}
% 		\item Trade-Off: Kosten vs. Qualit\"at
% 		\item Pr\"ufzeitpunkt: Ausschuss nicht weiterverarbeiten
% 		\item Pr\"ufart: Lehren oder Messen
%      	\end{itemize}
%      \end{column}
%      	\begin{column}[T]{5cm} 
%          	\begin{center}
%             		\includegraphics[width=0.95\textwidth]{Pruefplan}
%         		\end{center}
%      \end{column}
%  \end{columns}
% }

%\frame{\frametitle{\"Ubung 2: Erstellen eines Pr\"ufplans}
%\framesubtitle{}
%\begin{columns}[t] 
%     \begin{column}[T]{6cm} 
%     	\begin{enumerate}
%     		\item Erstellen Sie Pr\"ufpl\"ane f\"ur die Komponenten des Gelenks gem\"a{\ss} SFU-14008 und SFU-14009 sowie f\"ur die Lagerscheibe (Zukaufteil).
%		\item Schlagen Sie f\"ur alle drei Pr\"ufpl\"ane Pr\"ufzeitpunkte vor. Gehen Sie dazu von Eigenfertigung basierend auf
%		\begin{enumerate}
%		\item Halbzeugen
%		\item Stahlgussteilen aus.
%	\end{enumerate} 
%     	\end{enumerate}
%     \end{column}
%     	\begin{column}[T]{5cm} 
%         	\begin{center}
%            		\includegraphics[width=0.8\textwidth]{SFU-14008}\\
%			\includegraphics[width=0.8\textwidth]{SFU-14009}
%        		\end{center}
%     \end{column}
% \end{columns}
%}

% \frame{\frametitle{Umgang mit Abweichungen}
% \framesubtitle{Abh\"angig von St\"uckzahl und Marktposition sind Abweichungen durch die Lieferanten mehr oder weniger h\"aufig der Fall.}
% \begin{columns}[t] 
%      \begin{column}[T]{6cm} 
%      	\begin{itemize}
%      		\item Abweichungen werden dem Kunden vom Lieferanten angezeigt.
% 		\item Verwendbarkeit der fehlerhaften Teile wird von betroffenen Abteilungen gepr\"uft.
% 		\item Akzeptanz und Ablehnung bis auf Bauteilebene, d.h. einzelne Teile einer Serie werden betrachtet. 
%      	\end{itemize}
%      \end{column}
%      	\begin{column}[T]{5cm} 
%          	\begin{center}
%             		\includegraphics[width=0.95\textwidth]{Abweichungsantrag}
%         		\end{center}
%      \end{column}
%  \end{columns}
% }

