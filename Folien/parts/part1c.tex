% !TEX root = ../16_MdQM_Vorlesung RevB.tex

%\sectionpage

%\frame{\frametitle{}
%\begin{center}
%            		\begin{tikzpicture}[limb/.style={line cap=round,line width=1.5mm,line join=bevel}]
%\draw[line width=2mm,rounded corners,fill=yellow] (-2,0) -- (0,-2) -- (2,0) -- (0,2) -- cycle;
%\fill (1.5mm,7mm) circle (1.5mm);
%\fill(0,-7.5mm) -- ++(10mm,0mm) -- ++(120:2mm)--++(100:1mm)--++(150:2mm) arc (70:170:2.5mm and 1mm);
%\draw[limb] (-7.5mm,-6.5mm)--++(70:4mm)--++(85:4mm) coordinate(a)--++(-45:5mm)--(-2.5mm,-6.5mm);
%\fill[rotate around={45:(a)}] ([shift={(-0.5mm,0.55mm)}]a) --++(0mm,-3mm)--++
%        (7mm,-0.5mm)coordinate(b)--++(0mm,4mm)coordinate(c)--cycle;
%\draw[limb] ([shift={(-0.6mm,-0.4mm)}]b) --++(-120:5mm) ([shift={(-0.5mm,-0.5mm)}]c) --++
%        (-3mm,0mm)--++(-100:3mm)coordinate (d);
%\draw[ultra thick] (d) -- ++(-45:1.25cm);
%\end{tikzpicture}
%        		\end{center}
%}

\subsection{Exkurs: Vertragspr\"ufung}
\subsectionpage

\frame{\frametitle{Aufgaben der Vertragspr\"ufung}
\framesubtitle{}
\begin{itemize}
\item Angebotsphase
\begin{itemize}
	\item Pr\"ufung auf Vollst\"andigkeit
	\item Pr\"ufung auf Risiken
\end{itemize}
\item Vertragsabschlussphase
\begin{itemize}
	\item Pr\"ufung auf Vollst\"andigkeit
	\item Pr\"ufung auf Unstimmigkeit
	\item Pr\"ufung auf Widerspr\"uchlichkeit
\end{itemize}
\item Abwicklungsphase
\begin{itemize}
	\item Verfolgen von \"Anderungen
	\item Verfolgen von Abweichungen
\end{itemize}
\end{itemize}
}

\frame{\frametitle{Vorgehen in der Angebotsphase}
\framesubtitle{}
\begin{itemize}
\item Rechte und Pflichten der Vertragsparteien
\begin{itemize}
	\item Dokumentenhierarchie
	\item Liefer- und Leistungsumfang
\end{itemize}
\item Mitwirkungspflichten
\begin{itemize}
	\item Auftraggeber
	\item Auftragnehmer
\end{itemize}
\item Analyse der Regelungen u.a. zu
\begin{itemize}
	\item Vertragsstrafen (z.B. Gewichtsp\"onale, Lieferverzug,...)
	\item Abnahmen
	\item \"Anderungen
	\item Verz\"ogerungen
\end{itemize}
\item Beurteilen besonderer vertraglicher Risiken
\end{itemize}
\begin{enumerate}
\item Lesen der Dokumente
\item Herausforderungen erkennen
\item Ma{\ss}nahmen erarbeiten und umsetzen
\end{enumerate}
}

\frame[allowframebreaks]{\frametitle{Wichtige Aspekte bei der Vertragspr\"ufung}
\framesubtitle{}
\begin{itemize}
\item Anwendbares Recht, Gerichtsstand
\item Regelung von Folgesch\"aden
\item Verzeichnis der Vertragsdokumente (inkl. Ausgabestand)
\item Liefer- und Leistungsumfang
\item Preisstellung (DDP Oslo vs. EXW), Preiseskalation
\item Umgang mit Abweichungen, technischem Fortschritt
\item Technische Termine
\item Optionen
\item Teillieferungen
\item Versp\"atung bei Lieferung, Dokumentation, IBS und P\"onalen
\item Nichteinhalten der vertraglichen Leistungswerte (Qualit\"at, RAMS, LCC,...)
\item Force-Majeur-Klausel
\item Produktionsstandorte
\item Logistik, Verpackung und Konservierung
\item Pr\"ufungen und Tests
\item Schulungen (Kunde und Betreiber)
\item Zertifikate
\item Gew\"ahrleistung
\end{itemize}
}

\subsection{Exkurs: Kosten- und Aufwandssch\"atzung}
\subsectionpage

\frame{\frametitle{Warum Kosten- und Aufwandssch\"atzung?}
\framesubtitle{Kosten- und Aufwandssch\"atzung ist die Grundlage f\"ur erfolgreiche Projektbearbeitung.}
\begin{itemize}
\item Aufwandssch\"atzung (Gr\"o{\ss}e: Zeit)
\begin{itemize}
	\item Identifikation von Arbeitspaketen
	\item Input f\"ur Kostensch\"atzung
	\item Ressourcenplanung und -allokation
	\item Terminplanung (auch projekt\"ubergreifend)
\end{itemize}
\item Kostensch\"atzung (Gr\"o{\ss}e: Geld)
\begin{itemize}
	\item Bestimmung von:
	\begin{itemize}
		\item Einmalkosten \textit{non recurring cost (NRC)}
		\item St\"uckkosten \textit{recurring cost (RC)}
	\end{itemize}
	\item Identifikation von Investitionen
\end{itemize}
\item Entscheidungshilfe im Entwicklungsprozess
\item Bestimmung des Angebotspreises
\end{itemize}
}

\frame{\frametitle{Herausforderungen Aufwands- und Kostensch\"atzung}
\framesubtitle{}
\begin{itemize}
\item Informationen:
\begin{itemize}
	\item unvollst\"andig
	\item unsicher
	\item fehlerbehaftet 
	\item Daher: Sch\"atzung, d.h. wahrscheinlichste Vorhersage \"uber den wahren Aufwand %{\color{red!80!black} Was wird gesch\"atzt?}
\end{itemize}
\item Projektdefinition:
\begin{itemize}
	\item Anforderungen nicht final (``to be defined during design stage'')
	\item \"Anderungen m\"oglich
\end{itemize}
\item Projektablauf:
\begin{itemize}
	\item Beginn durch Angebotsrunden verz\"ogert
	\item Projektverlauf durch externe Einfl\"usse (teil-)gesteuert 
\end{itemize}
\item Projektressourcen:
\begin{itemize}
	\item Durch andere Projekte Ressourcen blockiert oder eingeschr\"ankt nutzbar 
\end{itemize}
\end{itemize}
}


\frame{\frametitle{Ans\"atze zur Aufwandssch\"atzung}
\framesubtitle{}
\begin{itemize}
\item Expertensch\"atzverfahren, z.B.:
\begin{itemize}
	\item Projektstrukturplan-basiert \textit{(WBS-based)}
	\item Gruppensch\"atzung
\end{itemize}
\item Formale Sch\"atzverfahren, z.B.:
\begin{itemize}
	\item Analogie-basiert (z.B. Bremszange wie ..., jedoch mit ...)
	\item Parametrische Modelle (z.B. E-Kupplung Verkabelung: 100 h)
	\item Gr\"o{\ss}enbasiert: (z.B. Anpasskonstruktion: 500 h)
\end{itemize}
\item Kombinierte Sch\"atzverfahren, z.B.:
\begin{itemize}
	\item Zerlegung mit WBS, parametrische Sch\"atzung der Pakete
\end{itemize}
\item Auswahl des Verfahrens:
\begin{itemize}
	\item Abh\"angig von der Organisation
	\item Formale Verfahren weniger ``lernf\"ahig''
	\item Expertensch\"atzverfahren anf\"allig f\"ur ``wishful thinking''
\end{itemize}
\item Psychologische Herausforderungen \textit{(Cognitive biases)}:
\begin{itemize}
	\item \textit{Planning fallacy, cognitive dissonance, anchoring, confirmation bias, wishful thinking}
\end{itemize}
\end{itemize}
}


\frame{\frametitle{Projektstrukturplan}
\framesubtitle{Work Breakdown Structure (WBS) f\"ur die Ermittlung von Arbeitspaketen}
\begin{columns}[t] 
     \begin{column}[T]{6cm} 
     	\begin{itemize}
		\item Dekomposition eines Projekts
		\begin{itemize}
		\item Hierarchisch
		\item Inkrementell
		\end{itemize}
		\item Baumstruktur
		\item Gliederung gem\"a{\ss} DIN 69900
		\begin{itemize}
		\item Funktionsorientiert
		\item \textbf{Objektorientiert}
		\item Zeitorientiert
		\end{itemize}
		\item Starke Abh\"angigkeit von Deliverables
		\item Erstellung \"ublicherweise Top-Down
		\item Nutzen: Vollst\"andige \"Ubersicht
		\item Hilfreich: ``Tickler list'' 
     	\end{itemize}
     \end{column}
     	\begin{column}[T]{6cm} 
         	\begin{center}
            		\includegraphics[width=1\textwidth]{WBS}
        		\end{center}
     \end{column}
 \end{columns}
}

\frame{\frametitle{Aufbereitung WBS f\"ur Projektplanung}
\framesubtitle{Die Identifikation der Arbeitspakete allein l\"asst keine Planung des Projekts zu.}
\begin{itemize}
\item Sch\"atzung des Aufwands
\begin{itemize}
	\item Besprechung: m\"oglicher Bias
	\item Alternative: Planning Poker
\end{itemize}
\item Abh\"angigkeit (Reihenfolge) der Projektbearbeitung
\item Externe Inputs oder Vorbedingungen f\"ur Arbeitspakete
\item Vertraglich zugesicherte Termine
\item Zuordnung zu:
\begin{itemize}
	\item Ressourcen
	\item Phasen
\end{itemize}
\item Zieldefinition (Definition of Done)
\end{itemize}
}

\frame{\frametitle{Kostensch\"atzung}
\framesubtitle{}
\begin{columns}[t] 
     \begin{column}[T]{6cm} 
     \textbf{NRC estimation}
     	\begin{itemize}
     		\item Basierend auf Aufwandssch\"atzung
		\item Erg\"anzend:
		\begin{itemize}
		\item Kundenbetreuung
		\item Reisekosten
		\item Externe Dienstleistungen (z.B. Tests, Abnahmen, ...)
		\item Prototypen, Muster
		\item Investitionen
		\end{itemize}
		\item Zu beachten:
		\begin{itemize}
		\item Stundens\"atze
		\item Kostenentwicklung
		\end{itemize}
		\item N\"utzlich: Checkliste
     	\end{itemize}
     \end{column}
     	\begin{column}[T]{6cm} 
	\textbf{RC estimation} (\cite{niazi05, pahlbeitz})
         \begin{itemize}
         	\item Intuitive Verfahren:
		\begin{itemize}
		\item Basierend auf Expertensch\"atzung
		\item Unterst\"utzt durch Regeln
		\end{itemize} 
         	\item Analogiebasierte Sch\"atzung
		\begin{itemize}
		\item \"Ahnlichkeit
		\item Komplexit\"at 
		\end{itemize} 
     		\item Parametrische Sch\"atzung:
		\begin{itemize}
		\item z.B. Gewicht, Material oder kombiniert %, z.B. \tiny
%		\begin{equation*} 
%		C = FC+\left(C_{co} N_{co} +  \frac{C_{rm}TF}{1-SC} \right)W
%		\end{equation*}
		\end{itemize}
		\item Analytische Verfahren, z.B.
		\begin{itemize}
		\item Bearbeitungssimulation
		\item Feature based cost estimation
		\end{itemize}
     	\end{itemize}
     \end{column}
 \end{columns}
}

\frame{\frametitle{Risiko}
\framesubtitle{}
\begin{columns}[t] 
     \begin{column}[T]{6cm} 
     	\begin{center}
     		\Large Wenn Sie alle Risiken vermeiden wollen, haben Sie bald keine Risiken mehr zu vermeiden, weil Sie nicht mehr im Geschäft sind.
     	\end{center}	
\begin{flushright}
 Josef Ackermann
\end{flushright}
     \end{column}
     	\begin{column}[T]{6cm} 
         	\begin{center}
            		\includegraphics[width=0.8\textwidth]{Ackermann}\source{Quelle: Agencia Brasil}
        		\end{center}
     \end{column}
 \end{columns}
}

\subsection{Ausgew\"ahlte Elementare Qualit\"atstools}
\frame{\frametitle{Ausgew\"ahlte elementare Qualit\"atstools}
\framesubtitle{}
\begin{columns}[t] 
     \begin{column}[T]{6cm} 
     	\begin{itemize}
     		\item Histogramm
		\begin{itemize}
		\item Diagramm relativer H\"aufigkeiten
		\end{itemize}
		\item 5 x Warum
		\begin{itemize}
		\item Wiederholtes Fragen nach der Ursache f\"uhrt zur Root-Cause
		\end{itemize}
		\item Paretoanalyse
		\begin{itemize}
		\item 80\% Fehlerh\"aufigkeit bei 20\% der Arten
		\end{itemize}
		\item Ishikawa-Diagramm
		\begin{itemize}
		\item Systematische Ursachensuche
		\end{itemize}
     	\end{itemize}
     \end{column}
     	\begin{column}[T]{6cm} 
         	\begin{center}
            		\includegraphics[width=0.8\textwidth]{Histogramm}\source{}
        		\end{center}
     \end{column}
 \end{columns}
}

\subsection{Quality Function Deployment (QFD)}
\subsectionpage

\subsection{Quality Function Deployment (QFD)}
\frame{\frametitle{Quality Function Deployment (QFD) - Einf\"uhrung}
\framesubtitle{``Copy the spririt, not the form.''}
\begin{columns}[t] 
     \begin{column}[T]{6cm} 
     	\begin{itemize}
     		\item Hauptidee: Korrelationen feststellen
		\item Zeilen: Was?
		\begin{itemize}
		\item Was braucht der Kunde?
		\end{itemize}
		\item Spalten: Wie?
		\begin{itemize}
		\item Wie bekommt der Kunde das Gew\"unschte?
		\end{itemize}
		\item Gewichtung der Merkmale
		\item Autokorrelation zeigt Widerspr\"uche auf
		\item Korrelationsmatrizen an jeder Schnittstelle denkbar
     	\end{itemize}
     \end{column}
     	\begin{column}[T]{6cm} 
         	\vspace{-.5cm}
         	\begin{center}
            		\includegraphics[width=0.8\textwidth]{QFD7}\source{}
        		\end{center}

     \end{column}
 \end{columns}
}

\frame{\frametitle{Quality Function Deployment (QFD) - Vor- und Nachteile}
\framesubtitle{``Copy the spririt, not the form.''}
\begin{columns}[t] 
     \begin{column}[T]{6cm} 
     \vspace{-.5cm}
     	\begin{itemize}
     		\item[-] Exponentielles Wachstum
		\item[-] Mangelnde Kunden- und Anwenderinformationen
		\item[-] Schnittstellen = Machtkonflikte
		\item[+] Qualit\"atssteigerung
		\item[+] Anforderungen und Zielkonflikte fr\"uhzeitig identifizieren
		\item[+] Optimale Anforderungsabdeckung
		\item[+] Abstraktion von L\"osungen 
		\begin{itemize}
		\item Kundenverst\"andnis
		\item Innovationsans\"atze
		\end{itemize}
		\item[+] Nachvollziehbarkeit der Entscheidungen 
     	\end{itemize}
     \end{column}
     	\begin{column}[T]{6cm} 
		\vspace{-.5cm}
         	\begin{center}
            		\includegraphics[width=0.8\textwidth]{QFD7}\source{}
        		\end{center}
     \end{column}
 \end{columns}
}

\frame{\frametitle{QFD in der Praxis}
\framesubtitle{}
\begin{itemize}
\item H\"aufig nur erste Planungsphase durchgef\"uhrt
\item Teamzusammensetzung
\begin{itemize}
	\item Abteilungs\"ubergreifend: Entwicklung, Konstruktion, Marketing, Vertrieb, Fertigung, QM, Einkauf
	\item Erfahrener Moderator
\end{itemize}
\item (M\"oglicher) Ablauf:
\begin{itemize}
	\item Ermittlung und Gewichtung der Kundenforderungen
	\item Ableitung korrelierender Merkmale
	\item Korrelationen bewerten
	\item Wechselwirkungen pr\"ufen
	\item Festlegung von Optimierungsrichtungen f\"ur die Merkmale
	\item Benchmarking
	\item Festlegung von Zielgr\"o{\ss}en f\"ur Merkmale
\end{itemize}
\end{itemize}
}

\frame{\frametitle{Vorgehen QFD}
\framesubtitle{Am Beispiel des Antriebssystems Railway Challenge}
\begin{columns}[t] 
     \begin{column}[T]{6cm} 
     	\begin{enumerate}
     		\item Kundenanforderungen sammeln und wichten
		\item Merkmale ableiten
		\item Optimierungsrichtung festlegen
		\item Korrelation Anforderungen/Merkmale
		\item Wechselwirkung der Merkmale
		\item Absolutes Gewicht bestimmen
		\item Benchmarking gegen Wettbewerb
		\item Weiterf\"uhren, z.B.:
		\begin{itemize}
		\item Festlegen Zielwerte
		\item Technische Bedeutung
		\end{itemize}
     	\end{enumerate}
     \end{column}
     	\begin{column}[T]{6cm} 
         	\begin{center}
			\vspace{-.5cm}
            		\only<1>{\includegraphics[width=0.8\textwidth]{QFD1}\source{}}
			\only<2>{\includegraphics[width=0.8\textwidth]{QFD2}\source{}}
			\only<3>{\includegraphics[width=0.8\textwidth]{QFD3}\source{}}
			\only<4>{\includegraphics[width=0.8\textwidth]{QFD4}\source{}}
			\only<5>{\includegraphics[width=0.8\textwidth]{QFD5}\source{}}
			\only<6>{\includegraphics[width=0.8\textwidth]{QFD6}\source{}}
			\only<7>{\includegraphics[width=0.8\textwidth]{QFD7}\source{}}
        		\end{center}
     \end{column}
 \end{columns}
}


\subsection{Fehlerbaumanalyse (FTA)}
\subsectionpage

\frame{\frametitle{Einf\"uhrung Fehlerbaumanalyse}
\framesubtitle{Das Tool FTA stammt urspr\"unglich aus den Bereichen Aerospace und Reaktortechnik.}
\begin{itemize}
\item Ermittlung der logischen Verkn\"upfungen von Komponenten- und Subsystemausf\"allen
\item Ausgangspunkt: Fehlerevent (Core Hazard, Top Event)
\item Ermittlung aller m\"oglichen Ausfallskombinationen
\item Darstellung als Fehlerbaum
\begin{itemize}
		\item Endlicher gerichteter Graph
		\item Endlich viele Eing\"ange
		\item Ein Ausgang (Core Hazard)
	\end{itemize} 
\item Ziele der FTA:
\begin{itemize}
		\item Systematische Identifikation aller m\"oglichen Ausfallkombinationen, die zum Core Hazard f\"uhren
		\item Ermittlung von Zuverl\"assigkeitskenngr\"o{\ss}en:
		\begin{itemize}
		\item Eintrittsh\"aufigkeit Ausfallkombinationen/Core Hazard
		\item Unverf\"ugbarkeit des Systems
		\end{itemize}
		\item Aufstellung eines grafischen Systemmodells
		\end{itemize}
\end{itemize}
}

\frame{\frametitle{Vorgehen FTA}
\framesubtitle{}
\begin{columns}[t] 
     \begin{column}[T]{6cm} 
     	\begin{itemize}
     		\item Systemanalyse
		\begin{itemize}
		\item Funktionen, Leistungsziele, Abweichungen
		\item Betriebszust\"ande
		\item Umgebungsbedingungen
		\item Hilfsquellen
		\item Systemkomponenten und Zusammenwirken
		\item Analyse Operation und Verhalten
		\end{itemize}
		\item Aufstellung Fehlerbaum
		\begin{itemize}
		\item Core Hazard
		\item Logisch 1 entspricht Ausfall
		\end{itemize}
     	\end{itemize}
     \end{column}
     	\begin{column}[T]{6cm} 
         	\begin{enumerate}
            		\item Systemuntersuchung
			\item Festlegung des unerw\"unschten Ereignisses, Ausfallkriterien
			\item Festlegung Zuverl\"assigkeitskenngr\"o{\ss}e, Zeitintervalle
			\item Ausfallarten der Komponenten
			\item Fehlerbaum aufstellen
			\item Bestimmung Eingangsgr\"o{\ss}en
			\item Auswertung Fehlerbaum, Bewertung
        		\end{enumerate}
     \end{column}
 \end{columns}
}

\frame{\frametitle{Blockdiagramm Antriebssystem}
\framesubtitle{}
\begin{center}
 \includegraphics[width = .9\textwidth]{RCBlockDiagram}
\end{center}
}

\frame{\frametitle{Resultierender Fehlerbaum}
\framesubtitle{}
\begin{center}
 \includegraphics[width = \textwidth]{FTA}
\end{center}
}


\subsection{Failure Mode and Effect Analysis (FMEA)}
\subsectionpage

\frame{\frametitle{FMEA - Einf\"uhrung}
\framesubtitle{}
\begin{itemize}
\item Arten:
\begin{itemize}
		\item Design-FMEA: Produkt-/Bauteilebene
		\item Prozess-FMEA
		\end{itemize}
\item Vorgehen:
\begin{itemize}
		\item Team-Bildung: interdisziplin\"ar
		\item System- und Funktionsanalyse: Funktionsstruktur
		\item Fehleranalyse
		\item Bewertung mittels Risikopriorit\"atszahl (RPZ):
		\begin{itemize}
		\item $A \in[1,10]$: Auftretenswahrscheinlichkeit
		\item $B \in[1,10$: Bedeutung des Fehlers f\"ur den Kunden
		\item $E \in[1,10$: Wahrscheinlichkeit der Entdeckung (vor der Auslieferung)
		\item $RPZ = A B E$
		\end{itemize}
		\end{itemize}	
\end{itemize}
}

\frame{\frametitle{FMEA - Auswertung}
\framesubtitle{}
\begin{columns}[t] 
     \begin{column}[T]{6cm} 
     	\begin{itemize}
     		\item Kritik: Multiplikation ordinal skalierter Merkmale nicht definiert
		\item Eindeutigkeit RPZ fraglich
		\item Optimierung h\"aufig nach:
		\begin{itemize}
		\item Bedeutung $B$
		\item Technische Kardinalit\"at $B A$
		\end{itemize}
		\item Vergleich mit akzeptierten Risiken oder \textit{ALARP}
		\item Wichtig: Nachverfolgung der Ma{\ss}nahmen
     	\end{itemize}
     \end{column}
     	\begin{column}[T]{6cm} 
         	\begin{center}
            		\includegraphics[width=0.8\textwidth]{Risikograph}\source{}
        		\end{center}
     \end{column}
 \end{columns}
}
